\section{逼真渲染和光线追踪算法}\label{sec:逼真渲染和光线追踪算法}

逼真渲染的目标是创建3D场景的图像且与同一场景的照片难以区分。
在我们介绍渲染流程之前要重点理解的是,
此处的{\itshape 难以区分}一词不是精确说法,
因为它涉及人类观察者,
不同观察者对同一图像的感知可能是不同的。
尽管本书会涉及少量感知问题,
但明确给出观察者的精确特性是非常困难且远未解决的问题。
绝大多数时候,我们都对针对光及其与物质相互作用的物理仿真感到满意,
并以我们对显示技术的理解尽可能向观察者展示最好的图像。

几乎所有逼真渲染系统都基于光线追踪算法。
光线追踪算法其实很简单;
它跟随光线\sidenote{译者注:原文a ray of light。
      此外会按个人理解把ray译作“光线”或“射线”,把light译作“光”或“光线”甚至“光源”。}路径
穿过场景与环境中的物体相互作用并反射。
虽然编写光线追踪器的方法有很多,
但所有这些系统都必须模拟至少一项以下对象和现象:
\begin{itemize}
      \item \keyindex{相机}{camera}{}: 相机模型决定了从哪里、怎样观察场景,
            包括场景的图像是怎样记录到传感器上的。
            许多渲染系统从相机处开始生成视线并追踪到场景中。
      \item \keyindex{光线-物体相交}{ray-object intersection}{}: 此外,我们需要确定
            交点处物体的特定属性,例如曲面法线或材质。
            多数光线追踪器都有测试光线与多个物体相交的功能,
            典型的例如沿光线返回最近交点。
      \item \keyindex{光源}{light source}{}: 没有光,渲染场景就没有意义。
            光线追踪器必须对整个场景的光分布建模,
            不仅包括灯光自身的位置,还包括它们向整个场景发散能量的方式。
      \item \keyindex{可见性}{visibility}{}:为了知道给定光是否在表面上一点积累能量,
            必须确认从该点到光源是否存在一条不中断的路径。
            幸运的是,在光线追踪器中这个问题很容易回答,
            因为我们可以构造从表面到光源\sidenote{译者注:原文为light,我按个人理解译作“光源”。}的射线,
            寻找最近的光线-物体交点,
            并比较交点距离和光源距离。
      \item \keyindex{表面散射}{surface scattering}{}:每个物体都必须提供外观描述,
            包括光如何与物体表面相互作用等信息,
            以及再辐射\sidenote{译者注:原文reradiated。}(或散射\sidenote{译者注:原文scattered。})光的性质。
            表面散射模型是典型的参数化模型,
            因此可以模拟各种外观。
      \item \keyindex{间接光传输}{indirect light transport}{light transport光传输}\sidenote{译者注:这里把transport译作“传输”是为了
                  与下一段propagation译作“传播”区分开,但个人理解似乎就是“传播”的意思。}:因为
            光在一个物体上反射或折射后可能遇到另一个物体,
            所以通常有必要追踪从表面发出的额外光线来捕捉这种效应。
      \item \keyindex{光线传播}{ray propagation}{}:我们需要知道光在空间中沿光线传播时发生了什么。
            如果渲染真空中的场景,则光能量沿光线保持恒定。
            真正的真空虽然在地球上是罕见的,
            但对许多环境而言是合理的近似。
            更多复杂模型可用于追踪穿过雾、烟、大气等的光线。
\end{itemize}

本节将简要讨论以上每个仿真任务。
后续章节会展示pbrt底层仿真组件的高级接口,
了解贯穿主渲染循环的单个光线处理过程。
我们还会介绍基于Turner Whitted的
原始光线追踪算法的表面散射模型实现。

\subsection{相机}\label{sub:相机}

几乎每个人都用过\keyindex{相机}{camera}{},熟悉它的基本功能:
你想用一张图像记下世界(通常是按按钮或点击屏幕),
然后图像就被记录到\keyindex{胶片}{film}{}或电子传感器上。
\keyindex{针孔相机}{pinhole camera}{camera相机}是最简单的拍照设备之一。
它由一端打有小孔的遮光盒组成(\reffig{1.1})。
当针孔未被遮挡时,光射进针孔落到固定在盒子另一端的相纸上。
虽然它很简单,但这种相机至今仍在使用,常用于艺术目的。
要在胶片上获得足够的光以形成图像需要非常长的曝光时间。
\begin{figure}[htbp]
      \centering\input{Pictures/chap01/PinholeCamera.tex}
      \caption{针孔相机}\label{fig:1.1}
\end{figure}

虽然大多数相机都比针孔相机复杂得多,
但针孔相机是仿真的便捷起点。
相机最重要的功能是定义会被记录到胶片上的场景部分。
在\reffig{1.1}中,我们可以看见从针孔到胶片边的连线
是如何构造出延伸到场景中的双锥体的。
不在该锥体内的物体不会在胶片上成像。
因为实际的相机成像形状比锥体更复杂,
所以我们把这个可能在胶片上成像的空间区域称为\keyindex{视见体}{viewing volume}{}。

针孔相机也可以看作是把胶片平面放置在针孔的\emph{前方}但距离不变(\reffig{1.2})。
注意从针孔到胶片的连线正好定义了和之前一样的视见体。
当然,这不是真实相机的实际构建方法,
但对于仿真目的而言它是个方便的抽象。
当胶片(或成像)平面在针孔前时,
针孔也常常改称作\keyindex{眼睛}{eye}{}。
\begin{figure}[htbp]
      \centering\input{Pictures/chap01/Filminfront.tex}
      \caption{当仿真针孔相机时,胶片放在针孔前的平面,针孔改称为\emph{眼睛}。}\label{fig:1.2}
\end{figure}

现在我们说到渲染的关键问题:
相机在图像中的每一点记录下的颜色值是什么?
回想最初的针孔相机,
显然光线只有沿着连接针孔与胶片上一点的向量
传播才能作用于胶片上的相应位置。
在把胶片放在眼睛前的仿真相机中,
我们关注沿着图像点传播到眼睛的
光量\sidenote{译者注:原文the amount of light,
      即“光的数量”。这里译为“光量”,可以笼统理解为光的强弱。}。

因此,相机仿真器的重要任务是取图像上一点
并生成\keyindex{射线}{ray}{},
沿这些射线的\keyindex{入射光}{incident light}{light光}可作用于该图像位置。
因为射线由一个\keyindex{端点}{origin point}{point点}和
一个\keyindex{方向向量}{direction vector}{vector向量}组成,
所以该任务对\reffig{1.2}的针孔相机而言很简单:
它把针孔作为端点,
把从针孔到像平面\sidenote{译者注:原文near plane,按
      个人理解改译为“像平面”。}的向量
作为射线的方向。
对于含有多个透镜的复杂相机模型,
则图像上给定点相应的射线需要更多计算
(\refsec{逼真相机}介绍了这类模型的实现)。


\subsection{光线-物体相交}\label{sub:光线-物体相交}

相机每次生成射线时,
渲染器第一个任务就是确定
如果有的话,哪个物体和该射线最先在哪里\keyindex{相交}{intersect}{}。
该\keyindex{交点}{intersection point}{point点}是沿射线可见的,
我们想要模拟光在该点与物体的相互作用。
为了找到相交处,我们必须把场景中的所有物体拿来和该射线测试,
并选出与该射线首先相交的那个。
给定一射线${\bm r}$,首先将其写成\keyindex{参数形式}{parametric form}{}:
\begin{align*}
      {\bm r}(t)={\bm o}+t{\bm d}\, ,
\end{align*}
其中${\bm o}$是射线端点,${\bm d}$是其方向向量,$t$是定义在$(0,\infty)$的\keyindex{参数}{parameter}{}。
我们可以通过指定参数$t$的值并代入上式求得射线上的一点。

通常很容易寻找射线${\bm r}$和由隐函数$F(x,y,z)=0$定义的\keyindex{曲面}{surface}{}的交点。
首先把射线方程代入\keyindex{隐式方程}{implicit equation}{equation方程},
得到只含有参数$t$的新方程。
然后从中解出$t$并把最小正根代入射线方程得到所需的点。
例如,以\keyindex{原点}{origin}{point点}为球心,$r$为\keyindex{半径}{radius}{}的\keyindex{球}{sphere}{}的隐式方程是
\begin{align*}
      x^2+y^2+z^2-r^2=0\, .
\end{align*}
代入射线方程,得到
\begin{align*}
      (o_x+td_x)^2+(o_y+td_y)^2+(o_z+td_z)^2-r^2=0\, .
\end{align*}
上式除了$t$外的所有值都是已知的,因此是易解的关于$t$的二次方程。
如果没有正根,则射线与球面错开了;
如果有,则最小正根给出了交点。

对于光线追踪器的其余部分。只有交点信息是不够的;
它还需要知道该点表明的特定属性。
首先,必须确定该点的材质表示
并传给光线追踪算法之后的步骤。
其次,还需要交点处额外的几何信息来对该点\keyindex{着色}{shade}{}。
例如,曲面\keyindex{法线}{normal}{}${\bm n}$是必需的。
虽然许多光线追踪器只对${\bm n}$操作,
但像pbrt这类更复杂的渲染系统还需要更多信息,
比如位置的各个\keyindex{偏导数}{partial derivative}{derivative导数}以及
关于曲面局部参数化的曲面法线。

当然,绝大多数场景含有多个物体。
暴力法指依次用每个物体对射线测试,
从所有相交中选出$t$的最小正值来求得最近交点。
该方法虽然正确但对哪怕适度复杂的场景也很慢。
更好的方法是在光线相交处理中
并入一个\keyindex{加速结构}{acceleration structure}{}快速否决整组物体。
快速剔除无关几何体的能力意味着光线追踪常能以$O(I\log{N})$的时间运行
\sidenote{大$O$表示法常用来表示算法运行时间随问题规模增长的变化趋势,
称为\keyindex{渐进时间复杂度}{asymptotic time complexity}{},简称为时间复杂度。},
其中$I$是图像像素数目,$N$是场景中物体的数量
\footnote{虽然光线追踪的对数复杂度是
      常被认为是其主要优点,但该复杂度只在平均意义下是典型正确的。
      在计算几何文献中发表的许多光线追踪算法都保证有对数运行时间,
      但它们只在特定种类场景下才能工作,并且预处理开销和存储要求很高。
      \protect\citet{10.1007/BF02684409}提供了相关参考文献。
      但好在表现真实环境的场景通常不会遇到这种最坏的情形。
      实践中本书介绍的射线相交算法是次线性的,
      但在去掉大量预处理和存储开销时
      总是有可能构造出使光线追踪以$O(IN)$时间运行的最坏情形。}
(然而构建加速结构本身至少需要$O(N)$时间)。

pbrt为各种形状实现的几何接口将在第\refchap{形状}介绍,
加速接口和实现在第\refchap{图元和相交加速}。

\subsection{光分布}\label{sub:光分布}

光线-物体相交阶段给出了要着色的点和该点的局部几何信息。
回想我们的最终目标是找出从该点出发朝相机方向传播的光量。
为此需要知道有多少光\emph{到达}了该点。
这同时涉及到场景中光的\emph{几何}和\emph{辐射}分布。
对于非常简单的光源(例如\keyindex{点光源}{point light}{light source光源}),
光的\keyindex{几何分布}{geometric distribution}{distribution分布}只需
知道光源位置即可。
然而真实世界并不存在点光源,
所以基于物理的光照常基于\keyindex{面光源}{area light source}{light source光源}。
这意味着光源和表面发光的几何体关联。
但本节我们用点光源说明光分布的构成;
光度量和分布的严格讨论是第\refchap{颜色和辐射度学}和第\refchap{光源}的主题。

我们常常想知道光在交点附近的微分面积上积累的功率(\reffig{1.3})。
假设点光源具有功率\sidenote{译者注:称为辐射通量或辐射功率。}
$\varPhi$并向所有方向均匀辐射光
\sidenote{译者注:原文使用正体字母$\Phi$,
      我把标量统一改为了斜体;
      此外原文正确使用了$\pi$的正体表示圆周率,
      但我因为字体环境冲突无法打出正体,
      所以全书被迫用的斜体。}。
这意味着围绕光源的单位球面在单位面积上的功率
为$\displaystyle\frac{\varPhi}{4\mathrm{\pi}}$
(这些度量将在\refsec{辐射度学}解释和形式化)。
\begin{figure}[htbp]
      \centering\input{Pictures/chap01/Basicreflectionsetting.tex}
      \caption{确定由点光源到达一点的单位面积功率的几何结构。
            该点到光源的距离记作$r$.}\label{fig:1.3}
\end{figure}

如果考虑两个这样的球面(\reffig{1.4}),
很明显大球上一点的单位面积功率必然比小球低,
因为相同的总功率分布在了更大的面积上。
特别地,到达半径为$r$的球面上一点的单位面积功率正比于$\displaystyle\frac{1}{r^2}$.
\begin{figure}[htbp]
      \centering\input{Pictures/chap01/Lightspheresequalpower.tex}
      \caption{因为点光源向所有方向均匀辐射光,
            所以以光源为球心的任意球面所积累的总功率相同。}
      \label{fig:1.4}
\end{figure}

此外可以看出,如果一小块曲面$\mathrm{d}A$相对于
从曲面指向光源的向量倾斜了角度$\theta$,
则$\mathrm{d}A$上积累的功率正比于$\cos{\theta}$.
综上所述,单位面积上的微分功率
(\keyindex{辐射照度}{irradiance}{}\sidenote{译者注:
      原文differential irradiance,鉴于其本身就是定义为功率对面积的微分,此处翻译略去了“微分”。})
$\mathrm{d}E$为
\begin{align*}
      \mathrm{d}E=\frac{\varPhi\cos{\theta}}{4\mathrm{\pi}r^2}\,.
\end{align*}

熟悉计算机图形学中基本光照的读者会注意到
该方程包含了两个熟悉的定律:
上述面的光照随倾斜按余弦衰减\sidenote{译者注:
      \keyindex{朗伯余弦定律}{Lambert's cosine law}{}。},
随距离按$\displaystyle\frac{1}{r^2}$衰减。

\subsection{可见性}\label{sub:可见性}

上节所述的光分布忽略了一个重要部分:\keyindex{阴影}{shadow}{}。
对于着色点,只有当该点到光源位置的路径畅通时,
该光源才会照亮该点(\reffig{1.5})。
\begin{figure}[htbp]
      \centering\input{Pictures/chap01/Twolightsoneblocker.tex}
      \caption{只有在视点处能无障碍地看到光源时,
            该光源才能对其表面积累能量。
            左边的光源照亮了点$\bm p$,但右边的不行。}\label{fig:1.5}
\end{figure}

幸运的是,在光线追踪器中很容易确定光对于着色点是否可见。
我们简单构造一条新射线,
其端点是表面上的点,方向指向光源。
这种特殊射线称为\keyindex{阴影射线}{shadow ray}{ray射线}。
如果追踪这些射线穿过环境,
我们就能检查在射线原点与光源之间能否找到相交处,
方法是沿射线方向比较相交处和光源位置的相应参数值$t$.
如果光源与表面之间没有遮挡物,
就应考虑该光源的作用。

\subsection{表面散射}\label{sub:表面散射}
我们现在能计算两种对着色点至关重要的信息了:位置和入射光
\footnote{已熟悉渲染的读者可能会反对本节只考虑直接照明的讨论。但请放心,pbrt支持全局照明。}。
现在需要确定入射光是如何在表面上\keyindex{散射}{scattered}{}的。
特别地,我们关注沿着原来追踪寻找交点的射线散射回的光的能量大小,
因为该光线会通向相机(\reffig{1.6})。
\begin{figure}[htbp]
      \centering\input{Pictures/chap01/Surfacescatteringgeometry.tex}
      \caption{表面散射的几何结构。
            入射光沿${\bm \omega}_\mathrm{i}$方向与表面交于点$\bm p$并沿方向${\bm \omega}_\mathrm{o}$散射回相机。
            朝相机散射的光量由入射光能量与BRDF的积给定。}\label{fig:1.6}
\end{figure}

场景中每个物体都提供了\keyindex{材质}{material}{},
即对表面每点外观属性的描述。
这个描述由\keyindex{双向反射分布函数}{bidirectional reflectance distribution function}{}(BRDF)给定。
该函数告诉我们从\keyindex{入射}{incoming}{}方向${\bm \omega}_\mathrm{i}$
到\keyindex{出射}{outgoing}{}方向${\bm \omega}_\mathrm{o}$会反射多少能量。
我们把$\bm p$处的BRDF写作$f_{\mathrm{r}}({\bm p},{\bm \omega}_\mathrm{o},{\bm \omega}_\mathrm{i})$.
现在计算散射回相机的光量$L$就很直接了:
\begin{lstlisting}
for each light:
      if light is not blocked:
            incident_light = light.L(point)
            amount_reflected =
                  surface.BRDF(hit_point, camera_vector, light_vector)
            L += amount_reflected * incident_light
\end{lstlisting}
这里我们用符号$L$代表光照强弱;
在光度量中它用的单位和前面的$\mathrm{d}E$有一点不同。
$L$即\keyindex{辐射亮度}{radiance}{},是后文经常见到的光度量指标。

很容易从BRDF的表示推广到\keyindex{透射光}{transmitted light}{light光}
(得到BTDF),或者得到到达表面任意一侧的光的一般散射。
描述一般散射的函数称作\keyindex{双向散射分布函数}{bidirectional scattering distribution function}{}(BSDF)。
pbrt支持各种BSDF模型;它们将于第\refchap{反射模型}介绍。
更复杂的还有\keyindex{双向散射表面反射分布函数}{bidirectional scattering surface reflectance distribution function}{}(BSSRDF),
它建模的光在离开表面时的点和进入时不同。
BSSRDF将在\refsub{BSSRDF}、\refsec{BSSRDF}和\refsec{使用扩散方程的次表面散射}介绍。

\subsection{间接光传输}\label{sub:间接光传输}

\citet{10.1145/358876.358882}在关于光线追踪
的原文中强调了其递归性质
是在渲染图像中引入间接\keyindex{镜面反射}{specular reflection}{reflection反射}与\keyindex{透射}{transmission}{}的关键。
例如,如果来自相机的射线命中像镜子那样光滑的物体,
我们可以在交点处以曲面法线为对称轴反射该射线,
递归调用光线追踪程序找到到达镜面上该点的的光,
将其纳入相机原来的射线的考虑范围。
该技术也可用于追踪交于透明物体的透射光线。
很长一段时间大多数早期光线追踪示例都拿
镜子和玻璃球举例(\reffig{1.7}),
因为这种效果很难用其他渲染技术实现。
\begin{figure}[htbp]
      \centering
      \subfloat[Whitted光线追踪]{\includegraphics[width=\linewidth]{chap01/spheres-whitted.png}\label{fig:1.7.1}}\\
      \subfloat[随机渐进光子映射。]{\includegraphics[width=\linewidth]{chap01/spheres-sppm.png}\label{fig:1.7.2}}
      \caption{一个典型的早期光线追踪场景。
            注意镜面和玻璃物体的使用,
            它强调了算法处理这类表面的能力。
            (a)使用Whitted光线追踪渲染,
            (b)使用随机渐进光子映射(SPPM),
            \refsec{随机渐进光子映射}将介绍这一高级光传输算法。
            SPPM能准确模拟光通过球体的聚焦现象。}\label{fig:1.7}
\end{figure}

通常,从物体上一点到达相机的光量
\sidenote{译者注:此处指辐亮度。}
由物体的发光量(如果它自己就是光源)与反射光量之和决定。
它被形式化为\keyindex{光传输方程}{light transport equation}{light transport光传输}
(也称作\keyindex{渲染方程}{rendering equation}{render渲染}),
表示从点$\bm p$沿方向${\bm \omega}_\mathrm{o}$的
出射辐亮度$L_{\mathrm{o}}({\bm p},{\bm \omega}_\mathrm{o})$等于
该点沿该方向的发光亮度加上
点$\bm p$的邻域球面$S^2$所有方向上
经BSDF$f({\bm p},{\bm \omega}_\mathrm{o},{\bm \omega}_\mathrm{i})$和
余弦项调制的入射亮度:
\begin{align}
      L_{\mathrm{o}}({\bm p},{\bm \omega}_\mathrm{o})=L_{\mathrm{e}}({\bm p},{\bm \omega}_\mathrm{o})+\int_{S^2}f({\bm p},{\bm \omega}_\mathrm{o},{\bm \omega}_\mathrm{i})L_{\mathrm{i}}({\bm p},{\bm \omega}_\mathrm{i})|\cos{\theta_{\mathrm{i}}}| \,\mathrm{d}{\bm \omega}_\mathrm{i}\, .
      \label{eq:1.1}
\end{align}
\refsub{BRDF}和\refsec{光传输方程}将展示其更完整的推导。
除了最简单的场景,
解析地求解该方程是几乎不可能的,
所以必须简化假设或使用数值积分技术。

Whitted算法通过忽略绝大多数方向的入射光来简化积分,
只计算到光源方向以及
完美\keyindex{反射}{reflection}{}与\keyindex{折射}{refraction}{}方向
的$L_{\mathrm{i}}({\bm p},{\bm \omega}_\mathrm{i})$.
换句话说,它把积分变为少量方向上的求和。

Whitted的方法可以扩展到实现镜面和玻璃外的更多效果。
例如,追踪许多贴近\keyindex{镜面反射}{mirror reflection}{reflection反射}方向
的递归光线并平均它们的作用,
可以近似得到\keyindex{光泽反射}{glossy reflection}{reflection反射}。
事实上只要命中物体我们就\emph{一直}递归地追踪光线。
例如,随机选取反射方向${\bm \omega}_\mathrm{i}$并
计算BRDF$f_{\mathrm{r}}({\bm p},{\bm \omega}_\mathrm{o},{\bm \omega}_\mathrm{i})$对
这条新建光线赋权。
这个简单有效的办法可以得到非常逼真的图像,
因为它考虑了物体间的光所有的\keyindex{互反射}{interreflection}{reflection反射}。
当然,我们需要知道何时停止递归,
何况完全随机选取方向会让渲染算法收敛到合理结果的速度变慢。
但是这些问题是可以解决的;
第\refchap{蒙特卡罗积分}和第\refchap{光传输III:双向方法}会对这些问题予以介绍。

当用该方法递归追踪光线时,
其实是把光线的一棵\keyindex{树}{tree}{}与图像的每个位置关联(\reffig{1.8}),
来自相机的射线是该树的根。
树中每条光线有关联的\keyindex{权重}{weight}{};
这允许我们对不反射100\%入射光的光滑表面建模。
\begin{figure}[htbp]
      \centering\input{Pictures/chap01/RayTree.tex}
      \caption{递归光线追踪与一整棵关于图像每处位置光线的树关联。}\label{fig:1.8}
\end{figure}

\subsection{光线传播}\label{sub:光线传播}

目前的讨论都假设光线是在\keyindex{真空}{vacuum}{}中传播的。
例如在描述点光源的光分布时,
我们假设光的功率朝着
以光源为球心的球面均匀发散且不随路线衰减
\sidenote{译者注:意思是没有传播介质吸收光的能量,
      和前文反比于半径平方不是一回事。}。
\keyindex{介质}{participating media}{}的出现,
例如烟、雾、尘会破坏该假设。
模拟这些效果很重要:
即使我们不渲染充满烟气的房间,
几乎所有室外场景也都会受到介质的实质影响。
例如,地球大气使得远处物体显得不够饱和(\reffig{1.9})。
\begin{figure}[htbp]
      \centering
      \subfloat[无大气散射]{\includegraphics[width=\linewidth]{chap01/ecosys-nofog.png}\label{fig:1.9.1}}\\
      \subfloat[有大气散射]{\includegraphics[width=\linewidth]{chap01/ecosys-fog.png}\label{fig:1.9.2}}
      \caption{地球大气随距离降低饱和度。
            (a)渲染场景没有模拟该现象,但(b)包含了大气模型。
            大气衰减程度是观察真实场景时重要的深度线索,
            为二次渲染增加了尺度感。}
      \label{fig:1.9}
\end{figure}

介质影响光沿路线传播的方式有两种。
第一种是介质可以通过吸收或沿不同方向散射
来\keyindex{熄灭}{extinguish}{}(或\keyindex{衰减}{attenuate}{})光。
可以通过计算射线端点与交点之间的\keyindex{透射率}{transmittance}{}$T$来
实现这一效果。
透射率表示交点处散射的光有多少成功到达射线端点。

介质也可以沿路线增强光。
在介质发光(例如火焰)或从其他方向把光散射回该射线时可发生该现象(\reffig{1.10})。
可以通过数值计算\keyindex{体积光传输方程}{volume light transport equation}{light transport光传输}来寻求该量,
该方法还能计算光传输方程求得从表面反射回的光量。
介质的描述和体积渲染会留到第\refchap{体积散射}和\refchap{光传输II:体积渲染}。
现在我们就能计算介质效应并将其合并到光线所含的光量中了。
\begin{figure}[htbp]
      \centering
      \includegraphics[width=\linewidth]{chap01/spotfog.png}
      \caption{聚光灯通过雾气照在球上。
            注意因为介质增加了散射,聚光灯的光分布形状和球的阴影清晰可见。}
      \label{fig:1.10}
\end{figure}