\section{点}\label{sec:点}

\keyindex{点}{point}{}是2D或3D空间中的零维位置。
pbrt中的类\refvar{Point2}{}和\refvar{Point3}{}以
明确的方式表示点:使用关于坐标系的$x,y,z$(3D)坐标。
尽管向量也使用同样的表示,
但事实上点表示一个位置而向量表示一个方向,
由此它们的处理方式有很多重要的差异。
文中点记为$\bm p$.

本节中,我们这里仍只涵盖类\refvar{Point3}{}中3D点方法的实现。
\begin{lstlisting}
`\initcode{Point Declarations}{=}\initnext{PointDeclarations}`
template <typename T> class `\initvar{Point2}{}` {
public:
    `\refcode{Point2 Public Methods}{}`
    `\refcode{Point2 Public Data}{}`
};
\end{lstlisting}

\begin{lstlisting}
`\refcode{Point Declarations}{+=}\lastnext{PointDeclarations}`
template <typename T> class `\initvar{Point3}{}` {
public:
    `\refcode{Point3 Public Methods}{}`
    `\refcode{Point3 Public Data}{}`
};
\end{lstlisting}

像向量那样,常用点类型有更短的类型名称会很有用。
\begin{lstlisting}
`\refcode{Point Declarations}{+=}\lastcode{PointDeclarations}`
typedef `\refvar{Point2}{}`<`\refvar{Float}{}`> `\initvar{Point2f}{}`;
typedef `\refvar{Point2}{}`<int>   `\initvar{Point2i}{}`;
typedef `\refvar{Point3}{}`<`\refvar{Float}{}`> `\initvar{Point3f}{}`;
typedef `\refvar{Point3}{}`<int>   `\initvar{Point3i}{}`;

`\initcode{Point2 Public Data}{=}`
T x, y;

`\initcode{Point3 Public Data}{=}`
T x, y, z;
\end{lstlisting}

还像向量那样,\refvar{Point3}{}的构造函数接收参数
设置{\ttfamily x},{\ttfamily y}和{\ttfamily z}坐标值。
\begin{lstlisting}
`\initcode{Point3 Public Methods}{=}\initnext{Point3PublicMethods}`
`\refvar{Point3}{}`() { x = y = z = 0; }
`\refvar{Point3}{}`(T x, T y, T z) : x(x), y(y), z(z) {
    `\refvar{Assert}{}`(!HasNaNs());
}
\end{lstlisting}

通过丢弃$z$坐标将\refvar{Point3}{}转换为\refvar{Point2}{}很有用。
完成这个转换的构造函数有修饰符{\ttfamily explicit}
所以该转换不会在缺乏显式类型转换时发生,以免意外。
\begin{lstlisting}
`\initcode{Point2 Public Methods}{=}`
explicit `\refvar{Point2}{}`(const `\refvar{Point3}{}`<T> &p) : x(p.x), y(p.y) {
    `\refvar{Assert}{}`(!HasNaNs());
}
\end{lstlisting}

将一种元素类型的点(例如\refvar{Point3f}{})转换为另一种点(例如\refvar{Point3i}{})
和将点转换为不同元素类型的向量都很有用。
下列构造函数和转换操作提供了这些转换形式。
它们也都要求有显式类型转换,使得在使用它们时源码更加清楚。
\begin{lstlisting}
`\refcode{Point3 Public Methods}{+=}\lastnext{Point3PublicMethods}`
template <typename U> explicit `\refvar{Point3}{}`(const `\refvar{Point3}{}`<U> &p)
    : x((T)p.x), y((T)p.y), z((T)p.z) { 
        `\refvar{Assert}{}`(!HasNaNs());
}
template <typename U> explicit operator `\refvar{Vector3}{}`<U>() const {
    return `\refvar{Vector3}{}`<U>(x, y, z);
}
\end{lstlisting}

某些\refvar{Point3}{}方法返回或接收\refvar{Vector3}{}。
例如,可以把向量加到点上,使之朝给定方向偏移得到一个新点。
\begin{lstlisting}
`\refcode{Point3 Public Methods}{+=}\lastnext{Point3PublicMethods}`
`\refvar{Point3}{}`<T> operator+(const `\refvar{Vector3}{}`<T> &v) const {
    return `\refvar{Point3}{}`<T>(x + v.x, y + v.y, z + v.z);
}
`\refvar{Point3}{}`<T> &operator+=(const `\refvar{Vector3}{}`<T> &v) {
    x += v.x; y += v.y; z += v.z;
    return *this;
}
\end{lstlisting}

或者,从一个点减去另一个点,得到两者间的向量,如\reffig{2.6}所示。
\begin{figure}[htbp]
    \centering\input{Pictures/chap02/Vecfrompoints.tex}
    \caption{求两点间的向量。向量$\bm v=\bm p'-\bm p$由点$\bm p'$和$\bm p$的逐元素减法得到。}
    \label{fig:2.6}
\end{figure}

\begin{lstlisting}
`\refcode{Point3 Public Methods}{+=}\lastnext{Point3PublicMethods}`
`\refvar{Vector3}{}`<T> operator-(const `\refvar{Point3}{}`<T> &p) const {
    return `\refvar{Vector3}{}`<T>(x - p.x, y - p.y, z - p.z);
}
\end{lstlisting}

从一个点减去向量得到另一个点。
\begin{lstlisting}
`\refcode{Point3 Public Methods}{+=}\lastcode{Point3PublicMethods}`
`\refvar{Point3}{}`<T> operator-(const `\refvar{Vector3}{}`<T> &v) const {
    return `\refvar{Point3}{}`<T>(x - v.x, y - v.y, z - v.z);
}
`\refvar{Point3}{}`<T> &operator-=(const `\refvar{Vector3}{}`<T> &v) {
    x -= v.x; y -= v.y; z -= v.z;
    return *this;
}
\end{lstlisting}

两点间的距离可以由它们相减所得向量的长度算得:
\begin{lstlisting}
`\refcode{Geometry Inline Functions}{+=}\lastnext{GeometryInlineFunctions}`
template <typename T> inline `\refvar{Float}{}`
`\initvar{Distance}{}`(const `\refvar{Point3}{}`<T> &p1, const `\refvar{Point3}{}`<T> &p2) {
    return (p1 - p2).`\refvar{Length}{}`();
}
template <typename T> inline `\refvar{Float}{}`
`\initvar{DistanceSquared}{}`(const `\refvar{Point3}{}`<T> &p1, const `\refvar{Point3}{}`<T> &p2) {
    return (p1 - p2).`\refvar{LengthSquared}{}`();
}
\end{lstlisting}

尽管通常用标量对点赋权或把两点相加没有数学意义,
但为了能计算点的加权和,点类仍允许这类操作,
只要所用权重和为一就有数学意义。
数乘代码以及与点求和的代码和向量的相应代码一样,
所以这里不再赘述。

值得一提的是,两点间的线性\keyindex{插值}{interpolate}{}很有用。
\refvar[Point3::Lerp]{Lerp}{()}在{\ttfamily t==0}时返回{\ttfamily p0},
{\ttfamily t==1}时返回{\ttfamily p1},
{\ttfamily t}取其余值时则在两点间插值。
对于{\ttfamily t<0}或{\ttfamily t>1},
\refvar[Point3::Lerp]{Lerp}{()}进行\keyindex{外推}{extrapolate}{}。
\begin{lstlisting}
`\refcode{Geometry Inline Functions}{+=}\lastnext{GeometryInlineFunctions}`
template <typename T> `\refvar{Point3}{}`<T>
`\initvar[Point3::Lerp]{Lerp}{}`(`\refvar{Float}{}` t, const `\refvar{Point3}{}`<T> &p0, const `\refvar{Point3}{}`<T> &p1) {
    return (1 - t) * p0 + t * p1;
}
\end{lstlisting}

函数\refvar[Point3::Min]{Min}{()}和\refvar[Point3::Max]{Max}{()}返回
给定两点逐元素取最小和最大值后表示的点。
\begin{lstlisting}
`\refcode{Geometry Inline Functions}{+=}\lastnext{GeometryInlineFunctions}`
template <typename T> `\refvar{Point3}{}`<T>
`\initvar[Point3::Min]{Min}{}`(const `\refvar{Point3}{}`<T> &p1, const `\refvar{Point3}{}`<T> &p2) {
    return `\refvar{Point3}{}`<T>(std::min(p1.x, p2.x), std::min(p1.y, p2.y), 
                     std::min(p1.z, p2.z));
}
template <typename T> `\refvar{Point3}{}`<T>
`\initvar[Point3::Max]{Max}{}`(const `\refvar{Point3}{}`<T> &p1, const `\refvar{Point3}{}`<T> &p2) {
    return `\refvar{Point3}{}`<T>(std::max(p1.x, p2.x), std::max(p1.y, p2.y), 
                     std::max(p1.z, p2.z));
}
\end{lstlisting}

\refvar{Floor}{()}、\refvar{Ceil}{()}和\refvar[Point3::Abs]{Abs}{()}对
给定点逐元素应用相应操作。
\begin{lstlisting}
`\refcode{Geometry Inline Functions}{+=}\lastnext{GeometryInlineFunctions}`
template <typename T> `\refvar{Point3}{}`<T> `\initvar{Floor}{}`(const `\refvar{Point3}{}`<T> &p) {
    return `\refvar{Point3}{}`<T>(std::floor(p.x), std::floor(p.y), std::floor(p.z));
}
template <typename T> `\refvar{Point3}{}`<T> `\initvar{Ceil}{}`(const `\refvar{Point3}{}`<T> &p) {
    return `\refvar{Point3}{}`<T>(std::ceil(p.x), std::ceil(p.y), std::ceil(p.z));
}
template <typename T> `\refvar{Point3}{}`<T> `\initvar[Point3::Abs]{Abs}{}`(const `\refvar{Point3}{}`<T> &p) {
    return `\refvar{Point3}{}`<T>(std::abs(p.x), std::abs(p.y), std::abs(p.z));
}
\end{lstlisting}

最后,\refvar[Point3::Permute]{Permute}{()}根据提供的顺序对坐标值排序。
\begin{lstlisting}
`\refcode{Geometry Inline Functions}{+=}\lastnext{GeometryInlineFunctions}`
template <typename T> `\refvar{Point3}{}`<T>
`\initvar[Point3::Permute]{Permute}{}`(const `\refvar{Point3}{}`<T> &p, int x, int y, int z) {
    return `\refvar{Point3}{}`<T>(p[x], p[y], p[z]);
}
\end{lstlisting}