\section{射线}\label{sec:射线}

\keyindex{射线}{ray}{}是由\keyindex{端点}{origin}{}和\keyindex{方向}{direction}{}指定的半无限直线。
pbrt用\refvar{Point3f}{}作为端点、
\refvar{Vector3f}{}作为方向表示\refvar{Ray}{}。
我们只需要端点和方向是浮点的射线,
所以\refvar{Ray}{}没有像点、向量和法线那样被任意类型的模板类参数化。
射线记为$\bm r$;它有端点$\bm o$和方向$\bm d$,如\reffig{2.7}所示。
\begin{figure}[htbp]
    \centering\input{Pictures/chap02/Raypointvector.tex}
    \caption{射线是由端点$\bm o$和方向向量$\bm d$定义的半无限直线。}
    \label{fig:2.7}
\end{figure}

\begin{lstlisting}
`\initcode{Ray Declarations}{=}\initnext{RayDeclarations}`
class `\initvar{Ray}{}` {
public:
    `\refcode{Ray Public Methods}{}`
    `\refcode{Ray Public Data}{}`
};
\end{lstlisting}

因为我们在整个代码中经常引用这些变量,
所以\refvar{Ray}{}的端点和方向成员简洁地命名为{\ttfamily o}和{\ttfamily d}。
注意我们为了方便再次令数据可以公开访问。
\begin{lstlisting}
`\initcode{Ray Public Data}{=}\initnext{RayPublicData}`
`\refvar{Point3f}{}` `\initvar[Ray::o]{o}{}`;
`\refvar{Vector3f}{}` `\initvar[Ray::d]{d}{}`;
\end{lstlisting}

射线的\keyindex{参数形式}{parametric form}{}将其表达为标量值$t$的函数,
给出射线通过的点集:
\begin{align}\label{eq:2.3}
    \bm r(t)=\bm o+t\bm d,\quad 0\le t<\infty\, .
\end{align}

\refvar{Ray}{}还有一个成员变量沿其无限端把射线限制为线段。
该域\refvar{tMax}{}允许我们把射线限制到线段点集$[\bm o,\bm r(t_{\max}))$.

注意到该域声明为{\ttfamily mutable},
意味着即使含有它的\refvar{Ray}{}是{\ttfamily const}它也可以改变——
因此当射线被传入一个接收{\ttfamily const Ray \&}的方法时,
该方法不允许修改其端点或方向但能修改它的范围。
这个约定符合系统中射线最常见的用法,
即作为光线-物体相交测试例程的参数,
用\refvar{tMax}{}记录最近相交处的偏移量。
\begin{lstlisting}
`\refcode{Ray Public Data}{+=}\lastnext{RayPublicData}`
mutable `\refvar{Float}{}` `\initvar{tMax}{}`;
\end{lstlisting}

每条射线有一个关联的时间值。
在有动画物体的场景中,
渲染系统为每条射线构造处于合适时间的场景表示。
\begin{lstlisting}
`\refcode{Ray Public Data}{+=}\lastnext{RayPublicData}`
`\refvar{Float}{}` `\initvar[Ray::time]{time}{}`;
\end{lstlisting}

最后,每条射线记录包含其端点的介质。
\refsec{介质}介绍的类\refvar{Medium}{}封装了
介质(可能在空间上变化)的属性,例如起雾大气、烟,
或者散射液体如牛奶或洗发液。
将这些信息与射线关联让系统的其他部分能够
正确考虑光线从一种介质传播到另一种中的影响。
\begin{lstlisting}
`\refcode{Ray Public Data}{+=}\lastcode{RayPublicData}`
const `\refvar{Medium}{}` *`\initvar[Ray::medium]{medium}{}`;
\end{lstlisting}

构造\refvar{Ray}{}是很简单的。
默认构造函数依赖\refvar{Point3f}{}和\refvar{Vector3f}{}的
构造函数把端点和方向设为$(0,0,0)$.
也可以提供特定点和方向。
如果提供了端点和方向,
则构造函数允许为\refvar{tMax}{}、射线的时间和介质给定一个值。
\begin{lstlisting}
`\initcode{Ray Public Methods}{=}\initnext{RayPublicMethods}`
`\refvar{Ray}{}`() : `\refvar{tMax}{}`(`\refvar{Infinity}{}`), `\refvar[Ray::time]{time}{}`(0.f), `\refvar[Ray::medium]{medium}{}`(nullptr) { }
`\refvar{Ray}{}`(const `\refvar{Point3f}{}` &o, const `\refvar{Vector3f}{}` &d, `\refvar{Float}{}` tMax = `\refvar{Infinity}{}`,
    `\refvar{Float}{}` time = 0.f, const `\refvar{Medium}{}` *medium = nullptr)
    : `\refvar[Ray::o]{o}{}`(o), `\refvar[Ray::d]{d}{}`(d), `\refvar{tMax}{}`(tMax), `\refvar[Ray::time]{time}{}`(time), `\refvar[Ray::medium]{medium}{}`(medium) { }
\end{lstlisting}

因为沿射线的位置可以视作关于单个参数$t$的函数,
所以类\refvar{Ray}{}为射线重载了应用运算符函数。
这样,当我们需要寻找射线上特定位置的点时,
我们可以这样编写代码:

{\ttfamily\indent\indent\refvar{Ray}{} r(\refvar{Point3f}{}(0, 0, 0), \refvar{Vector3f}{}(1, 2, 3));}\newline
{\ttfamily\indent\indent\refvar{Point3f}{} p = r(1.7);}

\begin{lstlisting}
`\refcode{Ray Public Methods}{+=}\lastcode{RayPublicMethods}`
`\refvar{Point3f}{}` operator()(`\refvar{Float}{}` t) const { return `\refvar[Ray::o]{o}{}` + `\refvar[Ray::d]{d}{}` * t; }
\end{lstlisting}

\subsection{射线差分}\label{sub:射线差分}
为了用第\refchap{纹理}定义的纹理函数获得更好的抗锯齿效果,
pbrt可以跟踪一些射线的额外信息。
\refsec{采样与抗锯齿}中,该信息会用于计算
类\refvar{Texture}{}用来估计一小部分场景在图像平面上投影区域的值。
由此,\refvar{Texture}{}能计算出在那片区域的纹理均值,得到高质量的最终图像。

\refvar{RayDifferential}{}是\refvar{Ray}{}的子类,
含有两条辅助射线的额外信息。
这些额外射线表示由胶卷平面上的主光线
在$x$和$y$方向偏移一个样本距离后的相机光线。
通过确定三条射线在待着色物体上的投影区域,
\refvar{Texture}{}可以为合适的抗锯齿估计出要平均的面积。

因为类\refvar{RayDifferential}{}继承自\refvar{Ray}{},
系统的几何接口可以写成接收{\ttfamily const Ray \&}参数,
这样\refvar{Ray}{}或\refvar{RayDifferential}{}都能传入它们。
只有需要考虑抗锯齿和纹理的例程才需要\refvar{RayDifferential}{}参数。
\begin{lstlisting}
`\refcode{Ray Declarations}{+=}\lastcode{RayDeclarations}`
class `\initvar{RayDifferential}{}` : public `\refvar{Ray}{}` {
public:
    `\refcode{RayDifferential Public Methods}{}`
    `\refcode{RayDifferential Public Data}{}`
};
\end{lstlisting}

\refvar{RayDifferential}{}构造函数照搬\refvar{Ray}{}的构造函数。
\begin{lstlisting}
`\initcode{RayDifferential Public Methods}{=}\initnext{RayDifferentialPublicMethods}`
`\refvar{RayDifferential}{}`() { `\refvar{hasDifferentials}{}` = false; }
`\refvar{RayDifferential}{}`(const `\refvar{Point3f}{}` &o, const `\refvar{Vector3f}{}` &d,
        `\refvar{Float}{}` tMax = `\refvar{Infinity}{}`, `\refvar{Float}{}` time = 0.f,
        const `\refvar{Medium}{}` *medium = nullptr)
    : `\refvar{Ray}{}`(o, d, tMax, time, medium) {
    `\refvar{hasDifferentials}{}` = false; 
}
\end{lstlisting}

\begin{lstlisting}
`\initcode{RayDifferential Public Data}{=}`
bool `\initvar{hasDifferentials}{}`;
`\refvar{Point3f}{}` `\initvar{rxOrigin}{}`, `\initvar{ryOrigin}{}`;
`\refvar{Vector3f}{}` `\initvar{rxDirection}{}`, `\initvar{ryDirection}{}`;
\end{lstlisting}

构造函数还能从\refvar{Ray}{}创建\refvar{RayDifferential}{}。
它把\refvar{hasDifferentials}{}设为{\ttfamily false},
因为如果有相邻射线也还是未知的。
\begin{lstlisting}
`\refcode{RayDifferential Public Methods}{+=}\lastnext{RayDifferentialPublicMethods}`
`\refvar{RayDifferential}{}`(const `\refvar{Ray}{}` &ray) : `\refvar{Ray}{}`(ray) {
    `\refvar{hasDifferentials}{}` = false; 
}
\end{lstlisting}

在相机光线间隔一像素的假设下,pbrt中\refvar{Camera}{}的实现为离开相机的射线计算差分。
\refvar{SamplerIntegrator}{}之类的积分器可为每个像素生成多条相机光线,
这时样本间的实际距离会更低。
第\refchap{绪论}中遇到的代码片\refcode{Generate camera ray for current sample}{}
调用了如下定义的方法\refvar{ScaleDifferentials}{()}来
按估计的样本间隔{\ttfamily s}更新差分射线。
\begin{lstlisting}
`\refcode{RayDifferential Public Methods}{+=}\lastcode{RayDifferentialPublicMethods}`
void `\initvar{ScaleDifferentials}{}`(`\refvar{Float}{}` s) {
    `\refvar{rxOrigin}{}` = `\refvar[Ray::o]{o}{}` + (`\refvar{rxOrigin}{}` - `\refvar[Ray::o]{o}{}`) * s;
    `\refvar{ryOrigin}{}` = `\refvar[Ray::o]{o}{}` + (`\refvar{ryOrigin}{}` - `\refvar[Ray::o]{o}{}`) * s;
    `\refvar{rxDirection}{}` = `\refvar[Ray::d]{d}{}` + (`\refvar{rxDirection}{}` - `\refvar[Ray::d]{d}{}`) * s;
    `\refvar{ryDirection}{}` = `\refvar[Ray::d]{d}{}` + (`\refvar{ryDirection}{}` - `\refvar[Ray::d]{d}{}`) * s;
}
\end{lstlisting}