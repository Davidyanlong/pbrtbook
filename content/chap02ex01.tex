\section{译者补充:四元数}\label{sec:译者补充:四元数}
\begin{remark}
    本节内容不是原书内容,而是译者自学补充的,请酌情参考和斧正。
\end{remark}

本节内容主要依据文献\citep{10.5555/90767.90913,enwiki:1013104981}整理而成,
给出四元数相关数学推导,具体介绍
四元数的定义、性质及其在几何变换中的运用。

\subsection{四元数的定义}\label{sub:四元数的定义}
\begin{definition}
    \keyindex{四元数}{quaternion}{}记作
    \begin{align}
        {\bm q}=w+x\mathbf{i}+y\mathbf{j}+z\mathbf{k}\, ,
    \end{align}
    其中$w, x, y, z$是实数,
    $\mathbf{i}, \mathbf{j}, \mathbf{k}$是\keyindex{基四元数}{basic quaternion}{quaternion四元数},
    也称\keyindex{基元}{basis element}{quaternion四元数}。

    基四元数可视为分别指向三个空间轴向的单位向量。
    此时${\bm q}$可视作由一个标量和一个向量构成:
    其中$w$称为\keyindex{实部}{real part}{}或\keyindex{标量部}{scalar part}{quaternion四元数},
    $x\mathbf{i}+y\mathbf{j}+z\mathbf{k}$称为\keyindex{虚部}{imaginary part}{quaternion四元数}、\keyindex{纯部}{pure part}{quaternion四元数}或\keyindex{向量部}{vector part}{quaternion四元数}。
\end{definition}
\begin{definition}
    当$w=0$且$xyz\neq 0$时,${\bm q}$称为\keyindex{向量四元数}{vector quaternion}{quaternion四元数}。
\end{definition}
\begin{definition}
    当$x=y=z=0$时,${\bm q}$称为\keyindex{标量四元数}{scalar quaternion}{quaternion四元数}。
    其中,当$w=x=y=z=0$时,${\bm q}$称为\keyindex{零四元数}{zero quaternion}{quaternion四元数},记作$0$.
\end{definition}
\begin{notation}
    在实际书写时,我们做如下约定
    \protect\sidenote{因为实数域$\mathbb{R}$、向量空间$\mathbb{R}^3$分别
        与$\mathbb{H}$的子集\protect\keyindex{同构}{isomorphic}{},
        所以即便在这种记法下
        实数与标量四元数、三维向量与向量四元数记号一样,
        其逻辑也是自洽的。}:
    \begin{itemize}
        \item $w, x, y, z$之一等于$0$时,略写相应项;
        \item $x, y, z$之一等于$1$时,相应项简写为$\mathbf{i, j}$或$\mathbf{k}$;
        \item 也可写作${\bm q}=w+{\bm u}$,其中$w$为标量,${\bm u}=x\mathbf{i}+y\mathbf{j}+z\mathbf{k}$为向量;
        \item 所有四元数构成的集合记作$\mathbb{H}$.
    \end{itemize}
\end{notation}


\subsection{四元数的运算}\label{sub:四元数的运算}
分别记四元数为
\begin{align}
    {\bm q}   & =w+{\bm u}=w+x\mathbf{i}+y\mathbf{j}+z\mathbf{k}\, ,             \\
    {\bm q}_n & =w_n+{\bm u}_n=w_n+x_n\mathbf{i}+y_n\mathbf{j}+z_n\mathbf{k}\, ,
\end{align}
其中$w, x, y, z, w_n, x_n, y_n, z_n\in\mathbb{R}$,${\bm u}, {\bm u}_n\in\mathbb{R}^3$,$n$为下标。
\begin{definition}
    四元数\keyindex{加法}{addition}{}为
    \begin{align}
        {\bm q}_1+{\bm q}_2 & =(w_1+w_2)+({\bm u}_1+{\bm u}_2)\nonumber                                  \\
                            & =(w_1+w_2)+(x_1+x_2)\mathbf{i}+(y_1+y_2)\mathbf{j}+(z_1+z_2)\mathbf{k}\, .
    \end{align}
\end{definition}
\begin{proposition}
    四元数加法的\keyindex{单位元}{identity element}{}是零四元数。
\end{proposition}
\begin{proposition}
    四元数加法的\keyindex{逆元}{inverse element}{}是
    \begin{align}
        -{\bm q}=-w-x\mathbf{i}-y\mathbf{j}-z\mathbf{k}\, .
    \end{align}
\end{proposition}
\begin{proposition}
    四元数的加法满足\keyindex{交换律}{commutative law}{}
    \begin{align}
        {\bm q}_1+{\bm q}_2={\bm q}_2+{\bm q}_1, \quad \forall {\bm q}_1, {\bm q}_2\in \mathbb{H}\, .
    \end{align}
\end{proposition}
\begin{proposition}
    四元数的加法满足\keyindex{结合律}{associative law}{}
    \begin{align}
        ({\bm q}_1+{\bm q}_2)+{\bm q}_3={\bm q}_1+({\bm q}_2+{\bm q}_3), \quad \forall {\bm q}_1, {\bm q}_2, {\bm q}_3\in \mathbb{H}\, .
    \end{align}
\end{proposition}
\begin{notation}
    约定向量的\keyindex{数乘}{scalar multiplication}{}、\keyindex{内积}{inner product}{}、\keyindex{叉积}{cross product}{}分别记作
    \begin{align}
        \lambda{\bm u}           & =(\lambda x)\mathbf{i}+(\lambda y)\mathbf{j}+(\lambda z)\mathbf{k},\quad \forall \lambda\in\mathbb{R}\, , \\
        {\bm u}_1\cdot{\bm u}_2  & =x_1x_2+y_1y_2+z_1z_2\, ,                                                                                 \\
        {\bm u}_1\times{\bm u}_2 & =(y_1z_2-y_2z_1)\mathbf{i}+(z_1x_2-z_2x_1)\mathbf{j}+(x_1y_2-x_2y_1)\mathbf{k}\, .
    \end{align}
\end{notation}
\begin{definition}
    四元数${\bm q}$与实数$\lambda$的\keyindex{数乘}{scalar multiplication}{}为
    \begin{align}
        \lambda{\bm q}={\bm q}\lambda & =\lambda w+\lambda {\bm u}\nonumber                                              \\
                                      & =\lambda w+(\lambda x)\mathbf{i}+(\lambda y)\mathbf{j}+(\lambda z)\mathbf{k}\, .
    \end{align}
\end{definition}
\begin{proposition}
    四元数的数乘对加法满足\keyindex{分配律}{distributive law}{}
    \begin{align}
        (\lambda_1+\lambda_2){\bm q} & =\lambda_1{\bm q}+\lambda_2{\bm q}, \quad \forall \lambda_1, \lambda_2\in\mathbb{R}, \forall {\bm q}\in\mathbb{H}\, , \\
        \lambda({\bm q}_1+{\bm q}_2) & =\lambda{\bm q}_1+\lambda{\bm q}_2, \quad \forall \lambda\in\mathbb{R}, \forall {\bm q}_1, {\bm q}_2\in\mathbb{H}\, .
    \end{align}
\end{proposition}
\begin{definition}
    基元$\mathbf{i}, \mathbf{j}, \mathbf{k}$的乘法为
    \begin{align}
        1\mathbf{i}  & =\mathbf{i}1=\mathbf{i}, & 1\mathbf{j}  & =\mathbf{j}1=\mathbf{j}, & 1\mathbf{k}  & =\mathbf{k}1=\mathbf{k}\, ,\nonumber \\
        \mathbf{i}^2 & =-1,                     & \mathbf{j}^2 & =-1,                     & \mathbf{k}^2 & =-1\, ,                    \nonumber \\
        \mathbf{ij}  & =\mathbf{k},             & \mathbf{jk}  & =\mathbf{i},             & \mathbf{ki}  & =\mathbf{j}\, ,            \nonumber \\
        \mathbf{ji}  & =-\mathbf{k},            & \mathbf{kj}  & =-\mathbf{i},            & \mathbf{ik}  & =-\mathbf{j}\, .
    \end{align}
\end{definition}
\begin{definition}
    四元数\keyindex{乘法}{multiplication}{}为
    \begin{align}
        {\bm q}_1{\bm q}_2 & =(w_1+{\bm u}_1)(w_2+{\bm u}_2)\nonumber                                                                      \\
                           & =(w_1+x_1\mathbf{i}+y_1\mathbf{j}+z_1\mathbf{k})(w_2+x_2\mathbf{i}+y_2\mathbf{j}+z_2\mathbf{k})\nonumber      \\
                           & =\quad w_1w_2+w_1x_2\mathbf{i}+w_1y_2\mathbf{j}+w_1z_2\mathbf{k}\nonumber                                     \\
                           & \quad +x_1w_2\mathbf{i}+x_1x_2\mathbf{i}^2+x_1y_2\mathbf{i}\mathbf{j}+x_1z_2\mathbf{i}\mathbf{k}\nonumber     \\
                           & \quad +y_1w_2\mathbf{j}+y_1x_2\mathbf{j}\mathbf{i}+y_1y_2\mathbf{j}^2+y_1z_2\mathbf{j}\mathbf{k}\nonumber     \\
                           & \quad +z_1w_2\mathbf{k}+z_1x_2\mathbf{k}\mathbf{i}+z_1y_2\mathbf{k}\mathbf{j}+z_1z_2\mathbf{k}^2\nonumber     \\
                           & =\quad w_1w_2-(x_1x_2+y_1y_2+z_1z_2)\nonumber                                                                 \\
                           & \quad +w_1(x_2\mathbf{i}+y_2\mathbf{j}+z_2\mathbf{k})+w_2(x_1\mathbf{i}+y_1\mathbf{j}+z_1\mathbf{k})\nonumber \\
                           & \quad +(y_1z_2-y_2z_1)\mathbf{i}+(z_1x_2-z_2x_1)\mathbf{j}+(x_1y_2-x_2y_1)\mathbf{k}\nonumber                 \\
                           & =(w_1w_2-{\bm u}_1\cdot{\bm u}_2)+(w_1{\bm u}_2+w_2{\bm u}_1+{\bm u}_1\times{\bm u}_2)\, ,
    \end{align}
    也称为\keyindex{哈密顿积}{Hamilton product}{}。
\end{definition}
\begin{proposition}
    四元数乘法的单位元是$1$.
\end{proposition}
\begin{corollary}
    四元数的乘法不满足交换律。
\end{corollary}
\begin{example}
    取
    \begin{align}
        {\bm q}_1 & =1+2\mathbf{i}+3\mathbf{j}+4\mathbf{k}\, , \\
        {\bm q}_2 & =5+6\mathbf{i}+7\mathbf{j}+8\mathbf{k}\, .
    \end{align}
    于是易得
    \begin{align}
        {\bm q}_1{\bm q}_2 & =-60+12\mathbf{i}+30\mathbf{j}+24\mathbf{k}\, , \\
        {\bm q}_2{\bm q}_1 & =-60+20\mathbf{i}+14\mathbf{j}+32\mathbf{k}\, .
    \end{align}
    显然${\bm q}_1{\bm q}_2\neq{\bm q}_2{\bm q}_1$.
\end{example}
\begin{proposition}
    四元数的乘法满足结合律
    \begin{align}
        ({\bm q}_1{\bm q}_2){\bm q}_3={\bm q}_1({\bm q}_2{\bm q}_3), \quad \forall {\bm q}_1, {\bm q}_2, {\bm q}_3\in \mathbb{H}\, .
    \end{align}
\end{proposition}
\begin{prove}
    根据定义有
    \begin{align}
        ({\bm q}_1{\bm q}_2){\bm q}_3 & =((w_1+{\bm u}_1)(w_2+{\bm u}_2))(w_3+{\bm u}_3)\nonumber                                                                                           \\
                                      & =((w_1w_2-{\bm u}_1\cdot{\bm u}_2)+(w_1{\bm u}_2+w_2{\bm u}_1+{\bm u}_1\times{\bm u}_2))(w_3+{\bm u}_3)\nonumber                                    \\
                                      & =\quad ((w_1w_2-{\bm u}_1\cdot{\bm u}_2)w_3-(w_1{\bm u}_2+w_2{\bm u}_1+{\bm u}_1\times{\bm u}_2)\cdot{\bm u}_3)\nonumber                            \\
                                      & \quad +((w_1w_2-{\bm u}_1\cdot{\bm u}_2){\bm u}_3+w_3(w_1{\bm u}_2+w_2{\bm u}_1+{\bm u}_1\times{\bm u}_2)\nonumber                                  \\
                                      & \quad +(w_1{\bm u}_2+w_2{\bm u}_1+{\bm u}_1\times{\bm u}_2)\times{\bm u}_3)\nonumber                                                                \\
                                      & =\quad w_1w_2w_3-w_3{\bm u}_1\cdot{\bm u}_2-w_1{\bm u}_2\cdot{\bm u}_3-w_2{\bm u}_1\cdot{\bm u}_3-({\bm u}_1\times{\bm u}_2)\cdot{\bm u}_3\nonumber \\
                                      & \quad +w_1w_2{\bm u}_3-({\bm u}_1\cdot{\bm u}_2){\bm u}_3+w_3w_1{\bm u}_2+w_2w_3{\bm u}_1+w_3{\bm u}_1\times{\bm u}_2\nonumber                      \\
                                      & \quad +w_1{\bm u}_2\times{\bm u}_3+w_2{\bm u}_1\times{\bm u}_3+({\bm u}_1\times{\bm u}_2)\times{\bm u}_3\nonumber                                   \\
                                      & =\quad  w_1w_2w_3-({\bm u}_1\times{\bm u}_2)\cdot{\bm u}_3+({\bm u}_1\times{\bm u}_2)\times{\bm u}_3-({\bm u}_1\cdot{\bm u}_2){\bm u}_3\nonumber    \\
                                      & \quad +w_2w_3{\bm u}_1+w_1{\bm u}_2\times{\bm u}_3-w_1{\bm u}_2\cdot{\bm u}_3\nonumber                                                              \\
                                      & \quad +w_3w_1{\bm u}_2+w_2{\bm u}_1\times{\bm u}_3-w_2{\bm u}_1\cdot{\bm u}_3\nonumber                                                              \\
                                      & \quad +w_1w_2{\bm u}_3+w_3{\bm u}_1\times{\bm u}_2-w_3{\bm u}_1\cdot{\bm u}_2\, .
    \end{align}
    同理有
    \begin{align}
        {\bm q}_1({\bm q}_2{\bm q}_3) & =(w_1+{\bm u}_1)((w_2+{\bm u}_2)(w_3+{\bm u}_3))\nonumber                                                                                           \\
                                      & =(w_1+{\bm u}_1)((w_2w_3-{\bm u}_2\cdot{\bm u}_3)+(w_2{\bm u}_3+w_3{\bm u}_2+{\bm u}_2\times{\bm u}_3))\nonumber                                    \\
                                      & =\quad(w_1(w_2w_3-{\bm u}_2\cdot{\bm u}_3)-{\bm u}_1\cdot(w_2{\bm u}_3+w_3{\bm u}_2+{\bm u}_2\times{\bm u}_3))\nonumber                             \\
                                      & \quad +(w_1(w_2{\bm u}_3+w_3{\bm u}_2+{\bm u}_2\times{\bm u}_3)+(w_2w_3-{\bm u}_2\cdot{\bm u}_3){\bm u}_1\nonumber                                  \\
                                      & \quad +{\bm u}_1\times(w_2{\bm u}_3+w_3{\bm u}_2+{\bm u}_2\times{\bm u}_3))\nonumber                                                                \\
                                      & =\quad w_1w_2w_3-w_1{\bm u}_2\cdot{\bm u}_3-w_2{\bm u}_1\cdot{\bm u}_3-w_3{\bm u}_1\cdot{\bm u}_2-{\bm u}_1\cdot({\bm u}_2\times{\bm u}_3)\nonumber \\
                                      & \quad +w_1w_2{\bm u}_3+w_1w_3{\bm u}_2+w_1{\bm u}_2\times{\bm u}_3+w_2w_3{\bm u}_1-({\bm u}_2\cdot{\bm u}_3){\bm u}_1\nonumber                      \\
                                      & \quad +w_2{\bm u}_1\times{\bm u}_3+w_3{\bm u}_1\times{\bm u}_2+{\bm u}_1\times({\bm u}_2\times{\bm u}_3)\nonumber                                   \\
                                      & =\quad w_1w_2w_3-{\bm u}_1\cdot({\bm u}_2\times{\bm u}_3)+{\bm u}_1\times({\bm u}_2\times{\bm u}_3)-({\bm u}_2\cdot{\bm u}_3){\bm u}_1\nonumber     \\
                                      & \quad +w_2w_3{\bm u}_1+w_1{\bm u}_2\times{\bm u}_3-w_1{\bm u}_2\cdot{\bm u}_3\nonumber                                                              \\
                                      & \quad +w_3w_1{\bm u}_2+w_2{\bm u}_1\times{\bm u}_3-w_2{\bm u}_1\cdot{\bm u}_3\nonumber                                                              \\
                                      & \quad +w_1w_2{\bm u}_3+w_3{\bm u}_1\times{\bm u}_2-w_3{\bm u}_1\cdot{\bm u}_2\, .
    \end{align}
    注意到向量运算有\sidenote{它们其实是向量的\protect\keyindex{数量三重积}{scalar triple product}{}的两种等价定义,也称\protect\keyindex{混合积}{mixed product}{}。}
    \begin{align}
        ({\bm u}_1\times{\bm u}_2)\cdot{\bm u}_3\equiv{\bm u}_1\cdot({\bm u}_2\times{\bm u}_3)\, ,
    \end{align}
    以及\sidenote{第一个和第三个等号的变换利用了\protect\keyindex{向量三重积}{vector triple product}{}的性质,
        也称为\protect\keyindex{三重积展开}{triple product expansion}{}或\protect\keyindex{拉格朗日公式}{Lagrange formula}{}。}
    \begin{align}
          & ({\bm u}_1\times{\bm u}_2)\times{\bm u}_3-({\bm u}_1\cdot{\bm u}_2){\bm u}_3\nonumber                             \\
        = & ({\bm u}_1\cdot{\bm u}_3){\bm u}_2-({\bm u}_2\cdot{\bm u}_3){\bm u}_1-({\bm u}_1\cdot{\bm u}_2){\bm u}_3\nonumber \\
        = & ({\bm u}_1\cdot{\bm u}_3){\bm u}_2-({\bm u}_1\cdot{\bm u}_2){\bm u}_3-({\bm u}_2\cdot{\bm u}_3){\bm u}_1\nonumber \\
        = & ({\bm u}_3\times{\bm u}_2)\times{\bm u}_1-({\bm u}_2\cdot{\bm u}_3){\bm u}_1\nonumber                             \\
        = & {\bm u}_1\times(-{\bm u}_3\times{\bm u}_2)-({\bm u}_2\cdot{\bm u}_3){\bm u}_1\nonumber                            \\
        = & {\bm u}_1\times({\bm u}_2\times{\bm u}_3)-({\bm u}_2\cdot{\bm u}_3){\bm u}_1\, .
    \end{align}
    所以$({\bm q}_1{\bm q}_2){\bm q}_3={\bm q}_1({\bm q}_2{\bm q}_3)$成立。
\end{prove}
\begin{proposition}
    四元数的乘法对加法满足分配律
    \begin{align}
        {\bm q}_1({\bm q}_2+{\bm q}_3)={\bm q}_1{\bm q}_2+{\bm q}_1{\bm q}_3, \quad \forall {\bm q}_1, {\bm q}_2, {\bm q}_3\in \mathbb{H}\, , \\
        ({\bm q}_1+{\bm q}_2){\bm q}_3={\bm q}_1{\bm q}_3+{\bm q}_2{\bm q}_3, \quad \forall {\bm q}_1, {\bm q}_2, {\bm q}_3\in \mathbb{H}\, .
    \end{align}
\end{proposition}
\begin{prove}
    根据定义有
    \begin{align}
        {\bm q}_1({\bm q}_2+{\bm q}_3) & =(w_1+{\bm u}_1)((w_2+{\bm u}_2)+(w_3+{\bm u}_3))\nonumber                                            \\
                                       & =(w_1+{\bm u}_1)((w_2+w_3)+({\bm u}_2+{\bm u}_3))\nonumber                                            \\
                                       & =\quad (w_1(w_2+w_3)-{\bm u}_1\cdot({\bm u}_2+{\bm u}_3))\nonumber                                    \\
                                       & \quad +(w_1({\bm u}_2+{\bm u}_3)+(w_2+w_3){\bm u}_1+{\bm u}_1\times({\bm u}_2+{\bm u}_3))\nonumber    \\
                                       & =\quad (w_1w_2-{\bm u}_1\cdot{\bm u}_2)+(w_1{\bm u}_2+w_2{\bm u}_1+{\bm u}_1\times{\bm u}_2)\nonumber \\
                                       & \quad +(w_1w_3-{\bm u}_1\cdot{\bm u}_3)+(w_1{\bm u}_3+w_3{\bm u}_1{\bm u}_1\times{\bm u}_3)\nonumber  \\
                                       & ={\bm q}_1{\bm q}_2+{\bm q}_1{\bm q}_3\, .
    \end{align}
    \begin{align}
        ({\bm q}_1+{\bm q}_2){\bm q}_3 & =((w_1+{\bm u}_1)+(w_2+{\bm u}_2))(w_3+{\bm u}_3)\nonumber                                            \\
                                       & =((w_1+w_2)+({\bm u}_1+{\bm u}_2))(w_3+{\bm u}_3)\nonumber                                            \\
                                       & =\quad ((w_1+w_2)w_3-({\bm u}_1+{\bm u}_2)\cdot{\bm u}_3)\nonumber                                    \\
                                       & \quad +((w_1+w_2){\bm u}_3+w_3({\bm u}_1+{\bm u}_2)+({\bm u}_1+{\bm u}_2)\times{\bm u}_3)\nonumber    \\
                                       & =\quad (w_1w_3-{\bm u}_1\cdot{\bm u}_3)+(w_1{\bm u}_3+w_3{\bm u}_1+{\bm u}_1\times{\bm u}_3)\nonumber \\
                                       & \quad +(w_2w_3-{\bm u}_2\cdot{\bm u}_3)+(w_2{\bm u}_3+w_3{\bm u}_2+{\bm u}_2\times{\bm u}_3)\nonumber \\
                                       & ={\bm q}_1{\bm q}_3+{\bm q}_2{\bm q}_3\, .
    \end{align}
\end{prove}
\begin{definition}
    称四元数${\bm q}=w+{\bm u}$与$\bar{\bm q}=w-{\bm u}$互为\keyindex{共轭}{conjugate}{}。
\end{definition}
\begin{proposition}
    四元数积的共轭等于逆序共轭的积,即
    \begin{align}
        \overline{{\bm q}_1{\bm q}_2}=\bar{\bm q}_2\bar{\bm q}_1, \quad\forall {\bm q}_1, {\bm q}_2\in\mathbb{H}\, .
    \end{align}
\end{proposition}
\begin{prove}
    根据定义有
    \begin{align}
        \overline{{\bm q}_1{\bm q}_2} & =(w_1w_2-{\bm u}_1\cdot{\bm u}_2)-(w_1{\bm u}_2+w_2{\bm u}_1+{\bm u}_1\times{\bm u}_2)\nonumber                   \\
                                      & =(w_2w_1-(-{\bm u}_2)\cdot(-{\bm u}_1))+(w_2(-{\bm u}_1)+w_1(-{\bm u}_2)+(-{\bm u}_2)\times(-{\bm u}_1))\nonumber \\
                                      & =(w_2-{\bm u}_2)(w_1-{\bm u}_1)\nonumber                                                                          \\
                                      & =\bar{\bm q}_2\bar{\bm q}_1\, .
    \end{align}
\end{prove}
\begin{definition}
    四元数的\keyindex{模}{modulus}{}为
    \begin{align}
        \|{\bm q}\|=\sqrt{{\bm q}\bar{\bm q}}=\sqrt{\bar{\bm q}{\bm q}}=\sqrt{w^2+x^2+y^2+z^2}\, .
    \end{align}
\end{definition}
\begin{definition}
    模为$1$的四元数${\bm q}$称为\keyindex{单位四元数}{unit quaternion}{quaternion四元数}。
\end{definition}
\begin{definition}
    称$、\displaystyle\frac{1}{\|{\bm q}\|}{\bm q}$为非零四元数${\bm q}$的\keyindex{规范化四元数}{versor}{quaternion四元数}。
\end{definition}
\begin{proposition}
    当向量四元数${\bm q}$是单位四元数即$w=0$且$x^2+y^2+z^2=1$时,有
    \begin{align}
        {\bm q}^2={\bm q}{\bm q}=-1\, .
    \end{align}
\end{proposition}
\begin{prove}
    依据哈密顿积定义有
    \begin{align}
        {\bm q}^2={\bm q}{\bm q} & =(w^2-{\bm u}\cdot{\bm u})+(2c{\bm u}+{\bm u}\times{\bm u})\nonumber \\
                                 & =-{\bm u}\cdot{\bm u}\nonumber                                       \\
                                 & =-(x^2+y^2+z^2)\nonumber                                             \\
                                 & =-1\, .
    \end{align}
\end{prove}
\begin{proposition}
    四元数取共轭后模不变,即
    \begin{align}
        \|\bar{\bm q}\|=\|{\bm q}\|\, .
    \end{align}
\end{proposition}
\begin{proposition}
    在四元数的数乘中有
    \begin{align}
        \|\lambda{\bm q}\|=|\lambda|\|{\bm q}\|, \quad \forall \lambda\in\mathbb{R}, \forall {\bm q}\in\mathbb{H}\, .
    \end{align}
\end{proposition}
\begin{proposition}
    在四元数的乘法中有
    \begin{align}
        \|{\bm q}_1{\bm q}_2\|=\|{\bm q}_1\|\|{\bm q}_2\|,\quad \forall {\bm q}_1{\bm q}_2\in\mathbb{H}\, .
    \end{align}
\end{proposition}
\begin{prove}
    根据定义可得
    \begin{align}
        \|{\bm q}_1{\bm q}_2\|^2 = & \|(w_1w_2-x_1x_2-y_1y_2-z_1z_2)\nonumber                    \\
                                   & +(w_1x_2+w_2x_1+y_1z_2-y_2z_1)\mathbf{i}\nonumber           \\
                                   & +(w_1y_2+w_2y_1+z_1x_2-z_2x_1)\mathbf{j}\nonumber           \\
                                   & +(w_1z_2+w_2z_1+x_1y_2-x_2y_1)\mathbf{k}\|^2\nonumber       \\
        =                          & (w_1w_2-x_1x_2-y_1y_2-z_1z_2)^2\nonumber                    \\
                                   & +(w_1x_2+w_2x_1+y_1z_2-y_2z_1)^2\nonumber                   \\
                                   & +(w_1y_2+w_2y_1+z_1x_2-z_2x_1)^2\nonumber                   \\
                                   & +(w_1z_2+w_2z_1+x_1y_2-x_2y_1)^2\nonumber                   \\
        =                          & (w_1^2+x_1^2+y_1^2+z_1^2)(w_2^2+x_2^2+y_2^2+z_2^2)\nonumber \\
        =                          & \|{\bm q}_1\|^2\|{\bm q}_2\|^2\, ,
    \end{align}
    所以$\|{\bm q}_1{\bm q}_2\|=\|{\bm q}_1\|\|{\bm q}_2\|$.
\end{prove}

\begin{proposition}
    四元数${\bm q}$与其共轭的积恒有
    \begin{align}
        {\bm q}\bar{\bm q}=\bar{\bm q}{\bm q}=\|{\bm q}\|^2\, .
    \end{align}
\end{proposition}
\begin{prove}
    根据定义有
    \begin{align}
        {\bm q}\bar{\bm q} & =(w+{\bm u})(w-{\bm u})\nonumber                                                      \\
                           & =(w^2-{\bm u}\cdot(-{\bm u}))+(w(-{\bm u})+w{\bm u}+{\bm u}\times(-{\bm u}))\nonumber \\
                           & =w^2+{\bm u}\cdot{\bm u}\nonumber                                                     \\
                           & =\|{\bm q}\|^2\, .
    \end{align}
    同理有$\bar{\bm q}{\bm q}=\|{\bm q}\|^2$.
\end{prove}
\begin{definition}
    非零四元数${\bm q}$的\keyindex{逆}{inverse}{}是
    \begin{align}
        {\bm q}^{-1}=\frac{1}{\|{\bm q}\|^2}\bar{\bm q}\, .
    \end{align}
    它是哈密顿积的逆元。
\end{definition}
\begin{corollary}
    对于非零四元数${\bm q}$有
    \begin{align}
        {\bm q}{\bm q}^{-1}={\bm q}^{-1}{\bm q}=1\, .
    \end{align}
\end{corollary}
\begin{corollary}
    单位四元数的逆与共轭相等,即
    \begin{align}
        \|{\bm q}\|=1 \Rightarrow {\bm q}^{-1}=\bar{\bm q}\, .
    \end{align}
\end{corollary}

\subsection{四元数与旋转变换}\label{sub:四元数与旋转变换}
如\reffig{2.ex1}所示,点$P$绕单位向量$\bm n$给定的轴顺时针旋转角度$\theta$得到点$P'$.
我们现在说明该一般旋转变换和四元数运算的关系。
\begin{figure}[htbp]
    \centering\input{Pictures/chap02/QuaternionAndRotaion.tex}
    \caption{旋转的几何表达。}
    \label{fig:2.ex1}
\end{figure}

作点$P$在$\bm n$所在直线(过点$O$)上的投影$H$,即$PH\perp OH$.
点$P$绕$\bm n$顺时针旋转角度$\theta$得到点$P'$,即$\angle PHP'=\theta$且$HP'\perp OH$.
最后,作点$P$绕$\bm n$顺时针旋转角度$\displaystyle\frac{\pi}{2}$得到的点$V$,即$HV\perp HP$且$HV\perp OH$.
记向量
\begin{align}
    \bm h=\overrightarrow{OH},\quad
    \bm p=\overrightarrow{OP},\quad
    \bm p'=\overrightarrow{OP'},\quad
    \bm u=\overrightarrow{HP},\quad
    \bm u'=\overrightarrow{HP'},\quad
    \bm v=\overrightarrow{HV}\, .
\end{align}
注意到$\bm n$是单位向量,根据上述几何关系,易得
\sidenote{注意$\bm v$的表达式与惯用手有关,
    此处为了与原书统一,使用的是左手坐标系。}
\begin{align}
    \bm h  & =(\bm n\cdot\bm p)\bm n\, ,          \\
    \bm u  & =\bm p-\bm h\, ,                     \\
    \bm v  & =\bm u\times\bm n\, ,                \\
    \bm u' & =\bm u\cos\theta+\bm v\sin\theta\, , \\
    \bm p' & =\bm h+\bm u'\, .
\end{align}
将这些式子从前往后依次代入可得\sidenote{注意这是普通的向量运算。}
\begin{align}
    \bm p' & =(\bm n\cdot\bm p)\bm n+(\bm p-(\bm n\cdot\bm p)\bm n)\cos\theta+(\bm p-(\bm n\cdot\bm p)\bm n)\times\bm n\sin\theta\nonumber \\
           & =\bm p\cos\theta+(1-\cos\theta)(\bm n\cdot\bm p)\bm n+\bm p\times\bm n\sin\theta\, .
\end{align}
另一方面,我们构造四元数$\bm q$:
\begin{align}
    \bm q=c+\bm s=\cos\frac{\theta}{2}+\bm n\sin\frac{\theta}{2}\, ,
\end{align}
其中$\displaystyle c=\cos\frac{\theta}{2}$,$\displaystyle \bm s=\bm n\sin\frac{\theta}{2}$.
于是有四元数积
\begin{align}
    \bar{\bm q}\bm p\bm q & =(c-\bm s)\bm p(c+\bm s)\nonumber                                                                                                                                                                \\
                          & =(\bm s\cdot\bm p+(c\bm p-\bm s\times\bm p))(c+\bm s)\nonumber                                                                                                                                   \\
                          & =(c\bm s\cdot\bm p-(c\bm p-\bm s\times\bm p)\cdot\bm s)+((\bm s\cdot\bm p)\bm s+c(c\bm p-\bm s\times\bm p)+(c\bm p-\bm s\times\bm p)\times\bm s)\nonumber                                        \\
                          & =(\bm s\cdot\bm p)\bm s+c(c\bm p-\bm s\times\bm p)+(c\bm p-\bm s\times\bm p)\times\bm s\nonumber                                                                                                 \\
                          & =(\bm s\cdot\bm p)\bm s+c^2\bm p-c\bm s\times\bm p+c\bm p\times\bm s-\bm s\times\bm p\times\bm s\nonumber                                                                                        \\
                          & =(\bm s\cdot\bm p)\bm s+c^2\bm p+2c\bm p\times\bm s-((\bm s\cdot\bm s)\bm p-(\bm p\cdot\bm s)\bm s)\nonumber                                                                                     \\
                          & =(c^2-\bm s\cdot\bm s)\bm p+2(\bm s\cdot\bm p)\bm s+2c\bm p\times\bm s\nonumber                                                                                                                  \\
                          & =\left(\cos^2\frac{\theta}{2}-\bm n\cdot\bm n\sin^2\frac{\theta}{2}\right)\bm p+2(\bm n\cdot\bm p)\bm n\sin^2\frac{\theta}{2}+2\bm p\times\bm n\cos\frac{\theta}{2}\sin\frac{\theta}{2}\nonumber \\
                          & =\bm p\cos\theta+(1-\cos\theta)(\bm n\cdot\bm p)\bm n+\bm p\times\bm n\sin\theta\nonumber                                                                                                        \\
                          & =\bm p'\, .
\end{align}
其中前三个等号运用了四元数乘法计算规则,
第四个等号起为向量运算,
第六个等号再次运用了向量三重积的性质,
第九个等号运用了三角函数倍角公式。

更一般地,对任意实数$w$有
\begin{align}
    \bar{\bm q}(w+\bm p)\bm q & =w\bar{\bm q}\bm q+\bar{\bm q}\bm p\bm q\nonumber \\
                              & =w\|\bm q\|^2+\bar{\bm q}\bm p\bm q\nonumber      \\
                              & =w+\bar{\bm q}\bm p\bm q\nonumber                 \\
                              & =w+\bm p'\, .
\end{align}
其中第一个等号运用了四元数乘法分配律。

上式给出了旋转变换与四元数的关系是:
\begin{theorem}
    任意点或向量$\bm p$绕任意单位向量$\bm n$顺时针旋转角度$\theta$得到新点或新向量$\bm p'$,
    则$\bm p'$的(左手)齐次坐标恰是四元数积$\bar{\bm q}\bm p\bm q$,其中
    \begin{align}
        \bm q=\cos\frac{\theta}{2}+\bm n\sin\frac{\theta}{2}\, .
    \end{align}
\end{theorem}

\begin{corollary}
    绕单位向量$\bm n$顺时针旋转角度$\theta$与绕$-\bm n$逆时针旋转$\theta$等价。
\end{corollary}

\begin{prove}
    利用四元数积易证:
    \begin{align}
          & (\cos(-\frac{\theta}{2})-(-\bm n)\sin(-\frac{\theta}{2}))\bm p(\cos(-\frac{\theta}{2})+(-\bm n)\sin(-\frac{\theta}{2}))\nonumber \\
        = & (\cos\frac{\theta}{2}-\bm n\sin\frac{\theta}{2})\bm p(\cos\frac{\theta}{2}+\bm n\sin\frac{\theta}{2})\, .
    \end{align}
\end{prove}

\begin{theorem}
    绕单位向量$\bm n$顺时针旋转角度$\theta$的$3\times3$旋转矩阵$\bm R_{\bm n}(\theta)$为
    \begin{align}
        \bm R_{\bm n}(\theta)=\bm I\cos\theta+(1-\cos\theta)\bm n\bm n^\mathrm{T}+\bm A_{\bm n}\sin\theta\, .
    \end{align}
    其中$\bm I$为3阶单位矩阵,$\bm A_{\bm n}$为叉积的等效矩阵
    \begin{align}
        \bm A_{\bm n}=\left[
            \begin{array}{ccc}
                     & n_z  & -n_y \\
                -n_z &      & n_x  \\
                n_y  & -n_x &
            \end{array}
            \right]\, .
    \end{align}
    相应$4\times4$旋转矩阵只需将4阶单位矩阵左上角$3\times3$子阵替换为$\bm R_{\bm n}(\theta)$.
\end{theorem}

\begin{corollary}
    $3\times3$旋转矩阵$\bm R_{\bm n}(\theta)$的分量表达式为
    \begin{align}
        \bm R_{\bm n}(\theta)=\left[\begin{array}{ccc}
                1-2(y^2+z^2) & 2(xy+zw)     & 2(xz-yw)     \\
                2(xy-zw)     & 1-2(x^2+z^2) & 2(yz+xw)     \\
                2(xz+yw)     & 2(yz-xw)     & 1-2(x^2+y^2)
            \end{array}\right]\, .
    \end{align}
    其中$\displaystyle w+x\mathbf{i}+y\mathbf{j}+z\mathbf{k}=\cos\frac{\theta}{2}+\bm n\sin\frac{\theta}{2}$,
    $\bm n$为单位向量。
\end{corollary}

\begin{remark}
    以上都是在左手坐标系下的结论,右手坐标系下则会有所不同。试着推导一下吧!
\end{remark}

\begin{prove}
    考虑$\theta\in[0,2\pi]$的情况\sidenote{思考一下:为什么只需要考虑这个角度区间?超过这个区间又该如何处理?},
    此时有$\displaystyle\frac{\theta}{2}\in[0,\pi]$.因为
    \begin{align}
        w+x\mathbf{i}+y\mathbf{j}+z\mathbf{k}=\cos\frac{\theta}{2}+\bm n\sin\frac{\theta}{2}\, ,
    \end{align}
    所以
    \begin{align}
        \cos\frac{\theta}{2}=w,\qquad\sin\frac{\theta}{2}=\sqrt{1-w^2}\, .
    \end{align}
    于是
    \begin{align}
        \cos\theta & =2\cos^2\frac{\theta}{2}-1=2w^2-1\, ,                         \\
        \sin\theta & =2\sin\frac{\theta}{2}\cos\frac{\theta}{2}=2w\sqrt{1-w^2}\, .
    \end{align}
    并且
    \begin{align}
        n_x         & =\frac{x}{\sin\frac{\theta}{2}}=\frac{x}{\sqrt{1-w^2}}\, ,          \\
        n_y         & =\frac{y}{\sin\frac{\theta}{2}}=\frac{y}{\sqrt{1-w^2}}\, ,          \\
        n_z         & =\frac{z}{\sin\frac{\theta}{2}}=\frac{z}{\sqrt{1-w^2}}\, ,          \\
        x^2+y^2+z^2 & =\sin^2\frac{\theta}{2}\|\bm n\|^2=\sin^2\frac{\theta}{2}=1-w^2\, .
    \end{align}

    将上面的式子代入旋转矩阵表达式得
    \begin{align}
        \bm R_{\bm n}(\theta) & =\bm I\cos\theta+(1-\cos\theta)\bm n\bm n^\mathrm{T}+\bm A_{\bm n}\sin\theta\nonumber                                         \\
                              & =(2w^2-1)\bm I+2(1-w^2)\bm n\bm n^\mathrm{T}+2w\sqrt{1-w^2}\bm A_{\bm n}\nonumber                                             \\
                              & =(2w^2-1)\bm I+2(1-w^2)\left[\begin{array}{ccc}
                n_x^2  & n_xn_y & n_xn_z \\
                n_xn_y & n_y^2  & n_yn_z \\
                n_xn_z & n_yn_z & n_z^2
            \end{array}\right]+2w\sqrt{1-w^2}\left[\begin{array}{ccc}
                     & n_z  & -n_y \\
                -n_z &      & n_x  \\
                n_y  & -n_x &
            \end{array}\right]\nonumber \\
                              & =(1-2x^2-2y^2-2z^2)\bm I+2\left[\begin{array}{ccc}
                x^2 & xy  & xz  \\
                xy  & y^2 & yz  \\
                xz  & yz  & z^2
            \end{array}\right]+2w\left[\begin{array}{ccc}
                   & z  & -y \\
                -z &    & x  \\
                y  & -x &
            \end{array}\right]\nonumber          \\
                              & =\left[\begin{array}{ccc}
                1-2(y^2+z^2) & 2(xy+zw)     & 2(xz-yw)     \\
                2(xy-zw)     & 1-2(x^2+z^2) & 2(yz+xw)     \\
                2(xz+yw)     & 2(yz-xw)     & 1-2(x^2+y^2)
            \end{array}\right]\, .
    \end{align}
\end{prove}

\subsubsection*{由旋转矩阵求解四元数}
已知$3\times3$旋转矩阵为$\bm R$,根据上文定理,
其对应的单位四元数$\bm q=w+x\mathbf{i}+y\mathbf{j}+z\mathbf{k}$满足
\begin{align}
    \bm R=\left[\begin{array}{ccc}
            r_{00} & r_{01} & r_{02} \\
            r_{10} & r_{11} & r_{12} \\
            r_{20} & r_{21} & r_{22}
        \end{array}\right]
    =\left[\begin{array}{ccc}
            1-2(y^2+z^2) & 2(xy+zw)     & 2(xz-yw)     \\
            2(xy-zw)     & 1-2(x^2+z^2) & 2(yz+xw)     \\
            2(xz+yw)     & 2(yz-xw)     & 1-2(x^2+y^2)
        \end{array}\right]\, .
\end{align}
注意到$-\bm q$也将是$\bm R$对应的四元数
\sidenote{思考一下:$\bm q$和$-\bm q$的几何含义有什么关联?};此外
\begin{align}\label{eq:02ex.1}
    \bm R-\bm R^\mathrm{T}=4\left[\begin{array}{ccc}
                & zw  & -yw \\
            -zw &     & xw  \\
            yw  & -xw &
        \end{array}\right],\quad
    \bm R+\bm R^\mathrm{T}=4\left[\begin{array}{ccc}
            \displaystyle\frac{r_{00}}{2} & xy                            & xz                            \\
            xy                            & \displaystyle\frac{r_{11}}{2} & yz                            \\
            xz                            & yz                            & \displaystyle\frac{r_{22}}{2}
        \end{array}\right]\, .
\end{align}
观察上式可以发现,若已知$x,y,z,w$中任意一项,则可以利用$\bm R-\bm R^\mathrm{T}$
或$\bm R+\bm R^\mathrm{T}$快速求得其他项。
因此我们首先利用$\bm R$主对角线上的元素有
\begin{align}
    \bm B[x^2\ y^2\ z^2\ w^2]^\mathrm{T}=\bm C[r_{00}\ r_{11}\ r_{22}\ 1]^\mathrm{T}\, ,
\end{align}
其中
\begin{align}
    \bm B=\left[\begin{array}{rrrr}
               & -2 & -2 &   \\
            -2 &    & -2 &   \\
            -2 & -2 &    &   \\
            1  & 1  & 1  & 1
        \end{array}\right],\
    \bm C=\left[\begin{array}{rrrr}
            1 &   &   & -1 \\
              & 1 &   & -1 \\
              &   & 1 & -1 \\
              &   &   & 1
        \end{array}\right]\, .
\end{align}
这个方程组不难解得
\begin{align}\label{eq:02ex.2}
    \left[\begin{array}{c}
            x^2 \\y^2\\z^2\\w^2
        \end{array}\right]=\bm B^{-1}\bm C\left[\begin{array}{c}
            r_{00} \\r_{11}\\r_{22}\\1
        \end{array}\right]=\frac{1}{4}\left[\begin{array}{rrrr}
            1  & -1 & -1 & 1 \\
            -1 & 1  & -1 & 1 \\
            -1 & -1 & 1  & 1 \\
            1  & 1  & 1  & 1
        \end{array}\right]\left[\begin{array}{c}
            r_{00} \\r_{11}\\r_{22}\\1
        \end{array}\right]\, .
\end{align}
虽然我们可以从$x^2,y^2,z^2,w^2$中任解一项并开方,
然后利用上文的$\bm R-\bm R^\mathrm{T}$或$\bm R+\bm R^\mathrm{T}$求解出其他项,
但是考虑到开方运算一般比除法的开销大,我们希望最高效地利用这次开方运算——
选最大的一项开方\sidenote{思考一下:不全部开方求解的原因除了开销大外还有别的吗?},
使下一步的除数远离零以保持足够的数值精度。
为此我们需要用简单的方式判断哪一项最大。
设$a=w^2-x^2-y^2-z^2$,于是有
\begin{align}
    [x^2\ y^2\ z^2\ a]^\mathrm{T}=\bm D[x^2\ y^2\ z^2\ w^2]^\mathrm{T}\, ,
\end{align}
其中
\begin{align}
    \bm D=\left[\begin{array}{rrrr}
            1  &    &    &   \\
               & 1  &    &   \\
               &    & 1  &   \\
            -1 & -1 & -1 & 1
        \end{array}\right]\, .
\end{align}
于是联立可解得
\begin{align}
    \left[\begin{array}{c}
            r_{00} \\r_{11}\\r_{22}\\1
        \end{array}\right]=\bm C^{-1}\bm B\bm D^{-1}\left[\begin{array}{c}
            x^2 \\ y^2\\ z^2\\ a
        \end{array}\right]=\left[\begin{array}{cccc}
            2 &   &   & 1 \\
              & 2 &   & 1 \\
              &   & 2 & 1 \\
            2 & 2 & 2 & 1
        \end{array}\right]\left[\begin{array}{c}
            x^2 \\ y^2\\ z^2\\ a
        \end{array}\right]\, .
\end{align}
整理可得
\begin{align}
    r_{00}             & =2x^2+a\, , \\
    r_{11}             & =2y^2+a\, , \\
    r_{22}             & =2z^2+a\, , \\
    \mathrm{tr}(\bm R) & =2w^2+a\, .
\end{align}
这说明$x^2,y^2,z^2,w^2$和$r_{00},r_{11},r_{22},\mathrm{tr}(\bm R)$的大小关系一致。

综上所述,由旋转矩阵$\bm R$求解单位四元数$\bm q=w+x\mathbf{i}+y\mathbf{j}+z\mathbf{k}$的方式是:
\begin{enumerate}
    \item 比较$r_{00},r_{11},r_{22},\mathrm{tr}(\bm R)$的大小关系,
          对应于$x^2,y^2,z^2,w^2$的大小关系;
    \item 取最大的那项,按\refeq{02ex.2}对应行计算该项并求算术平方根;
    \item 计算$\bm R-\bm R^\mathrm{T}$或$\bm R+\bm R^\mathrm{T}$,
          分别用所得矩阵的元素(\refeq{02ex.1})除以上一步的
          已知项并乘以系数$\displaystyle\frac{1}{4}$得到剩余未知项,
          此时得到第一个四元数。
    \item (可选)将该四元数各分量取相反数,得到第二个四元数。
\end{enumerate}
