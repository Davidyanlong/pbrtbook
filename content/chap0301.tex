\section{基本形状接口}\label{sec:基本形状接口}

\refvar{Shape}{}的接口定义于
源文件\href{https://github.com/mmp/pbrt-v3/tree/master/src/core/shape.h}{\ttfamily core/shape.h}中,
\href{https://github.com/mmp/pbrt-v3/tree/master/src/core/shape.cpp}{\ttfamily core/shape.cpp}中
可以找到\refvar{Shape}{}公共方法的定义。
基类\refvar{Shape}{}定义了通用形状接口。
它也暴露了一些对所有\refvar{Shape}{}实现有用的公有数据成员。
\begin{lstlisting}
`\initcode{Shape Declarations}{=}`
class `\initvar{Shape}{}` {
public:
`\refcode{Shape Interface}{}`
`\refcode{Shape Public Data}{}`
};
\end{lstlisting}

所有形状都定义在物体的坐标空间中;
例如,所有球体都定义在球心位于原点的坐标系中。
为了在场景别处放置球体,必须提供描述从\keyindex{物体空间}{object space}{}到世界空间映射的变换。
类\refvar{Shape}{}保存了该变换及其逆。

\refvar{Shape}{}也接收一个布尔参数\refvar{reverseOrientation}{}来
表示其曲面法线方向是否与默认的相反。
该能力很有用,因为曲面法线的朝向用于决定形状哪一面是“外面”。
例如,发光形状只在曲面法线所在那侧是发光的。
该参数值通过pbrt输入文件中的{\ttfamily ReverseOrientation}语句控制。

\refvar{Shape}{}还保存了调用\refvar[SwapsHandedness]{Transform::SwapsHandedness}{()}
来进行物体到世界变换的返回值。
每次找到光线相交处就调用的\refvar{SurfaceInteraction}{}构造函数需要该值,
所以\refvar{Shape}{}构造函数一次性计算并保存它。
\begin{lstlisting}
`\initcode{Shape Method Definitions}{=}\initnext{ShapeMethodDefinitions}`
`\refvar{Shape}{}`::`\refvar{Shape}{}`(const `\refvar{Transform}{}` *ObjectToWorld,
        const `\refvar{Transform}{}` *WorldToObject, bool reverseOrientation)
    : `\refvar{ObjectToWorld}{}`(ObjectToWorld), `\refvar{WorldToObject}{}`(WorldToObject),
      `\refvar{reverseOrientation}{}`(reverseOrientation),
      `\refvar{transformSwapsHandedness}{}`(ObjectToWorld->`\refvar{SwapsHandedness}{}`()) {
}
\end{lstlisting}

一个重要细节是形状保存了指向其变换的指针而不是直接保存\refvar{Transform}{}
对象。回想\refsec{变换}\refvar{Transform}{}对象由总共32个浮点数表示,需要128字节内存;
因为场景中许多形状经常被施加同样的变换,
pbrt维持了一个\refvar{Transform}{}池使它们可以复用,
并把指向共享\refvar{Transform}{}的指针传给形状。
这样,\refvar{Shape}{}析构函数不会删除其\refvar{Transform}{}指针,
而是让\refvar{Transform}{}控制代码来管理那些内存。
\begin{lstlisting}
`\initcode{Shape Public Data}{=}`
const `\refvar{Transform}{}` *`\initvar{ObjectToWorld}{}`, *`\initvar{WorldToObject}{}`;
const bool `\initvar{reverseOrientation}{}`;
const bool `\initvar{transformSwapsHandedness}{}`;
\end{lstlisting}

\subsection{边界}\label{sub:边界}
pbrt要渲染的场景中经常包含处理计算量极大的物体。
对于许多操作,有一个3D\keyindex{包围盒}{bounding volume}{}来封住物体常常很有用。
例如,如果一条光线没有穿过某个包围盒,对于该光线pbrt可以避免处理里面所有物体。

轴对齐边界框是方便的包围盒,因为它们只需要六个浮点值保存且适合很多形状。
而且测试光线与轴对齐边界框相交性的成本极低。
因此每个\refvar{Shape}{}实现必须能
用\refvar{Bounds3f}{}表示的轴对齐边界框包围自己。
有两种不同的包围方法。
第一个是\refvar{ObjectBound}{()},返回在形状的物体空间里的边界框。
\begin{lstlisting}
`\initcode{Shape Interface}{=}\initnext{ShapeInterface}`
virtual `\refvar{Bounds3f}{}` `\initvar{ObjectBound}{}`() const = 0;
\end{lstlisting}

第二个方法是\refvar[Shape::WorldBound]{WorldBound}{()},返回世界空间的边界框。
pbrt提供该方法的默认实现将物体空间边界变换到世界空间。
然而可以轻松计算更紧致的世界空间边界的形状应该重载该方法。
这种形状的一个例子是三角形(\reffig{3.1})。
\begin{figure}[htbp]
    \centering\input{Pictures/chap03/Objworldbounds.tex}
    \caption{(a)三角形的世界空间边界框通过将其物体空间边界框
        变换到世界空间再寻找包围结果边界框的边界框算得;可能得到肥大的边界框。
        (b)然而,如果三角形的顶点首先从物体空间变换到世界空间再包围,边界框可能合适得多。}
    \label{fig:3.1}
\end{figure}

\begin{lstlisting}
`\refcode{Shape Method Definitions}{+=}\lastnext{ShapeMethodDefinitions}`
`\refvar{Bounds3f}{}` `\refvar{Shape}{}`::`\initvar[Shape::WorldBound]{WorldBound}{}`() const {
    return (*`\refvar{ObjectToWorld}{}`)(`\refvar{ObjectBound}{}`());
}
\end{lstlisting}

\subsection{光线-边界相交}\label{sub:光线-边界相交}
有了\refvar{Bounds3f}{}实例包围形状的用法后,我们将添加一个\refvar{Bounds3}{}方法,即
\refvar{Bounds3::IntersectP}{()},以检查光线-框相交性,
并且如果有的话还返回相交处的两个参数$t$值。

认识边界框的一种方式是将其看作三块厚板\sidenote{译者注:原文slab。}的交集,
每一块是两平行平面之间的空间区域。
为了求光线与框的相交性,我们让光线依次与框的三块厚板相交。
因为厚板与三个坐标轴对齐,在光线-厚板相交中可以应用大量优化。

基本的光线-边界框相交算法算法工作如下:
我们从一个参数化区间开始,它沿我们想要求交的光线覆盖了位置$t$的范围;
该范围通常是$(0,+\infty)$.
然后我们先后计算光线与每个厚板相交的两个参数化$t$位置。
我们计算每个厚板相交区间与当前相交区间的交集,
如果我们发现结果区间退化则返回失败。
如果检查完全部三个厚板后区间是非退化的,
我们就有了光线位于框内的参数范围。
\reffig{3.2}{}说明了该过程,
\reffig{3.3}展示了光线与厚板相交的基本几何结构。
\begin{figure}[htb]
    \centering\input{Pictures/chap03/RayAABBintersect.tex}
    \caption{光线与轴对齐边界框相交。我们依次计算与每个厚板的交点,逐渐缩窄参数区间。
        这里的2D形式中,沿光线与$x$和$y$相交范围(蓝色线段)给出了框中光线的范围。}
    \label{fig:3.2}
\end{figure}
\begin{figure}[htb]
    \centering\input{Pictures/chap03/Rayslabintersect.tex}
    \caption{光线与轴对齐厚板相交。这里展示的两平面由关于常数$c$的$x=c$描述。
        每个平面的法线为$(1,0,0)$.除非光线平行于平面,否则它会在
        参数位置$t_{\mathrm{near}}$和$t_{\mathrm{far}}$处与厚板相交两次。}
    \label{fig:3.3}
\end{figure}

如果方法\refvar{Bounds3::IntersectP}{()}返回{\ttfamily true},
相交的参数范围在可选参数{\ttfamily hitt0}和{\ttfamily hitt1}内返回。
光线超出范围{\ttfamily (0, \refvar[tMax]{Ray::tMax}{})}的相交部分被忽略。
如果射线端点在框内,{\ttfamily hitt0}返回$0$.
\begin{lstlisting}
`\refcode{Geometry Inline Functions}{+=}\lastnext{GeometryInlineFunctions}`
template <typename T>
inline bool `\refvar{Bounds3}{}`<T>::`\initvar[Bounds3::IntersectP]{IntersectP}{}`(const `\refvar{Ray}{}` &ray, `\refvar{Float}{}` *hitt0,
        `\refvar{Float}{}` *hitt1) const {
    `\refvar{Float}{}` t0 = 0, t1 = ray.`\refvar{tMax}{}`;
    for (int i = 0; i < 3; ++i) {
        `\refcode{Update interval for ith bounding box slab}{}`
    }
    if (hitt0) *hitt0 = t0;
    if (hitt1) *hitt1 = t1;
    return true;
}
\end{lstlisting}

对于每对平面,该例程需要两处光线-平面相交。
例如,两个垂直于$x$轴的平面描述的厚板可以由
通过点$(x_0,0,0)$和$(x_1,0,0)$的平面描述,每个的法线都是$(1,0,0)$.
考虑平面相交的第一个$t$值即$t_0$.
端点为$\bm o$方向为$\bm d$的射线与平面$ax+by+cz+d=0$相交处的参数$t$值
可通过将射线方程代入平面方程求得:
\begin{align*}
    0 & =a(o_x+td_x)+b(o_y+td_y)+c(o_z+td_z)+d      \\
      & =(a,b,c)\cdot\bm o+t(a,b,c)\cdot\bm d+d\, .
\end{align*}
解得$t$为
\begin{align*}
    t=\frac{-d-(a,b,c)\cdot\bm o}{(a,b,c)\cdot\bm d}\, .
\end{align*}

因为平面法线的$y$和$z$分量是零,所以$b$和$c$是零,$a$是1。
平面的系数$d$是$-x_0$.我们可以用这些信息和点积的定义大大简化计算:
\begin{align*}
    t_0=\frac{x_0-o_x}{d_x}\, .
\end{align*}

计算厚板相交处$t$值的代码从计算光线方向相应分量的倒数开始,
这样它可以乘以该因子而不是执行多次除法。
注意尽管它除以该分量,但却不需要检查除数非零。
如果是零,则{\ttfamily invRayDir}会有无限值$-\infty$或$\infty$,
且算法剩余部分仍能正确工作
\footnote{这里假设所用架构支持现代系统通用的IEEE浮点算术。
IEEE浮点算术相关属性是对于所有{\ttfamily v>0},{\ttfamily v/0=}$\infty$,
对所有{\ttfamily w<0},{\ttfamily w/0=-}$\infty$,
其中$\infty$是特殊值即任意正数乘以$\infty$为$\infty$,
任意负数乘以$\infty$为$-\infty$.
关于浮点算术的更多信息见\protect\refsub{浮点算术}。}。
\begin{lstlisting}
`\initcode{Update interval for ith bounding box slab}{=}`
`\refvar{Float}{}` invRayDir = 1 / ray.`\refvar[Ray::d]{d}{}`[i];
`\refvar{Float}{}` tNear = (`\refvar{pMin}{}`[i] - ray.`\refvar[Ray::o]{o}{}`[i]) * invRayDir;
`\refvar{Float}{}` tFar  = (`\refvar{pMax}{}`[i] - ray.`\refvar[Ray::o]{o}{}`[i]) * invRayDir;
`\refcode{Update parametric interval from slab intersection t values}{}`
\end{lstlisting}

两个距离被记下来,这样{\ttfamily tNear}存有最近的相交处而{\ttfamily tFar}的最远。
这给出了参数范围$[${\ttfamily tNear,tFar}$]$,
用于计算和当前范围$[${\ttfamily t0,t1}$]$的交集以计算新范围。
如果新范围为空(即{\ttfamily t0>t1}),则代码会立即返回错误。

这里还有另一个浮点相关的细节:当射线端点在边界框厚板平面上
且射线在厚板平面内时,{\ttfamily tNear}或{\ttfamily tFar}可能会
由$\frac{0}{0}$的表达式算出,得到IEEE浮点“not a number”(NaN)值。
像无限值那样,NaN有明确指定的含义:例如任何含有NaN的逻辑比较结果都是假。
因此应细心编写更新{\ttfamily t0}和{\ttfamily t1}值的代码,
这样如果{\ttfamily tNear}或{\ttfamily tFar}是NaN,则
{\ttfamily t0}或{\ttfamily t1}不会取NaN值而总是保持不变。
\begin{lstlisting}
`\initcode{Update parametric interval from slab intersection t values}{=}`
if (tNear > tFar) std::swap(tNear, tFar);
`\refcode{Update tFar to ensure robust ray-bounds intersection}{}`
t0 = tNear > t0 ? tNear : t0;
t1 = tFar  < t1 ? tFar  : t1;
if (t0 > t1) return false;
\end{lstlisting}

\refvar{Bounds3}{}也提供了特殊化的方法\refvar[Bounds3::IntersectP2]{IntersectP}{()},
它接收射线方向的倒数作为额外参数,
这样每次调用\refvar[Bounds3::IntersectP2]{IntersectP}{()}就不用计算三个倒数了。

这个版本还接收表示每个方向分量是否为负的预计算值,
使得可以去掉原始例程中对算出的{\ttfamily tNear}和{\ttfamily tFar}值的比较
而是直接分别计算近和远的值。
因为原始代码中将这些值从低到高排序的比较取决于算出的值,
它们对于执行的处理器会很低效,
因为在比较之前必须完成计算它们的值。

如果射线段全在边界框内则该例程返回{\ttfamily true},
即使相交处不在射线的范围$(0,${\ttfamily tMax}$)$内也是这样。
\begin{lstlisting}
`\refcode{Geometry Inline Functions}{+=}\lastnext{GeometryInlineFunctions}`
template <typename T>
inline bool `\refvar{Bounds3}{}`<T>::`\initvar[Bounds3::IntersectP2]{IntersectP}{}`(const `\refvar{Ray}{}` &ray, const `\refvar{Vector3f}{}` &invDir,
        const int dirIsNeg[3]) const {
    const `\refvar{Bounds3f}{}` &bounds = *this;
    `\refcode{Check for ray intersection against x and y slabs}{}`
    `\refcode{Check for ray intersection against z slab}{}`
    return (tMin < ray.`\refvar{tMax}{}`) && (tMax > 0);
}
\end{lstlisting}

如果射线方向向量为负\sidenote{译者注:指分量为负。},
则用厚板和较大的两个边界值计算“近”参数化的相交处,较小的计算“远”相交处。
该实现可用该观察直接在每个方向计算近和远参数值。
\begin{lstlisting}
`\initcode{Check for ray intersection against x and y slabs}{=}`
`\refvar{Float}{}` tMin =  (bounds[  dirIsNeg[0]].x - ray.o.x) * invDir.x;
`\refvar{Float}{}` tMax =  (bounds[1-dirIsNeg[0]].x - ray.o.x) * invDir.x;
`\refvar{Float}{}` tyMin = (bounds[  dirIsNeg[1]].y - ray.o.y) * invDir.y;
`\refvar{Float}{}` tyMax = (bounds[1-dirIsNeg[1]].y - ray.o.y) * invDir.y;
`\refcode{Update tMax and tyMax to ensure robust bounds intersection}{}`
if (tMin > tyMax || tyMin > tMax) 
    return false;
if (tyMin > tMin) tMin = tyMin; 
if (tyMax < tMax) tMax = tyMax;
\end{lstlisting}

代码片\refcode{Check for ray intersection against z slab}{}类似,这里就不介绍了
\sidenote{译者注:笔者还是补充上来了。}。
\begin{lstlisting}
`\initcode{Check for ray intersection against z slab}{=}`
`\refvar{Float}{}` tzMin = (bounds[  dirIsNeg[2]].z - ray.o.z) * invDir.z;
`\refvar{Float}{}` tzMax = (bounds[1-dirIsNeg[2]].z - ray.o.z) * invDir.z;
`\refcode{Update tzMax to ensure robust bounds intersection}{}`
if (tMin > tzMax || tzMin > tMax)
    return false;
if (tzMin > tMin)
    tMin = tzMin;
if (tzMax < tMax)
    tMax = tzMax;
\end{lstlisting}

相交测试是遍历\refsec{包围盒层次}介绍的加速结构\refvar{BVHAccel}{}的核心。
因为遍历BVH树时要执行如此多的光线-边界框相交测试,
我们发现比起没有接收预计算方向导数和符号的\refvar{Bounds3::IntersectP}{()},
该优化方法在整个渲染时间中提供了大约15\%的性能提升。

\subsection{相交测试}\label{sub:相交测试}
\refvar{Shape}{}实现必须提供一个(也可能两个)
方法的实现以测试光线与其形状的相交性。
第一个是\refvar{Shape::Intersect}{()},
如果有的话就返回关于单个光线-形状相交处的几何信息,
它在沿光线的参数范围{\ttfamily (0, \refvar{tMax}{})}内对应于首次相交。
\begin{lstlisting}
`\refcode{Shape Interface}{+=}\lastnext{ShapeInterface}`
virtual bool `\initvar[Shape::Intersect]{Intersect}{}`(const `\refvar{Ray}{}` &ray, `\refvar{Float}{}` *tHit,
    `\refvar{SurfaceInteraction}{}` *isect, bool testAlphaTexture = true) const = 0;
\end{lstlisting}

在阅读(和编写)相交例程时要记住一些重点:
\begin{itemize}
    \item \refvar{Ray}{}结构含有成员\refvar[tMax]{Ray::tMax}{}定义光线终点。
          相交例程必须忽略发生在该点后的任何相交。
    \item 如果找到相交处,其沿光线的参数距离应该存于{\ttfamily tHit}指针
          传入相交例程。如果沿光线有多次相交,则应报告最近的那个。
    \item 关于相交处的信息存于结构\refvar{SurfaceInteraction}{},
          它完全刻画了曲面的局部几何属性。该类在整个pbrt中大量使用,
          它用于将光线追踪器的几何部分与着色和照明部分干净地隔离开。
          类\refvar{SurfaceInteraction}{}已在\refsec{交互作用}定义
          \footnote{几乎所有光线追踪器都使用这个通用习惯来返回
              关于与形状相交的几何信息。作为一种优化,许多光线追踪器在找到相交处时
              都只部分初始化相交处信息,并只存储够用的信息,如果确实需要剩余值则稍后再计算。
              在稍后会找到与另一个形状更近的相交处的情况下该方法节省了工作。
              按我们的经验,计算所有信息的额外工作并不多,而对于有着复杂场景数据管理算法
              (例如当占用太多内存时从主内存剔除几何体并写入磁盘)的渲染器而言,
              延迟的方法可能会失败,因为形状不再在内存中了。}。
    \item 传入相交例程的光线在世界空间中,所以如果相交测试需要,
          形状应负责将它们变换到物体空间。返回的相交信息应在世界空间中。
\end{itemize}

一些形状实现支持用纹理切掉它们的一些表面;
参数{\ttfamily testAlphaTexture}表示这些形状是否应该为当前的相交测试执行此操作。
第二个相交测试方法\refvar{Shape::IntersectP}{()},
是一个决定相交是否发生的断言函数,不返回关于相交的任何信息。
类\refvar{Shape}{}提供了方法\refvar[Shape::IntersectP]{IntersectP}{()}的
一个默认实现,即调用方法\refvar{Shape::Intersect}{()}并
只是忽略算得的关于交点的额外信息。
因为这非常浪费,pbrt中几乎所有形状实现都
为\refvar[Shape::IntersectP]{IntersectP}{()}提供了
一个更高效的实现,即确定是否存在相交而不计算其所有细节。
\begin{lstlisting}
`\refcode{Shape Interface}{+=}\lastnext{ShapeInterface}`
virtual bool `\initvar[Shape::IntersectP]{IntersectP}{}`(const `\refvar{Ray}{}` &ray,
        bool testAlphaTexture = true) const {
    `\refvar{Float}{}` tHit = ray.`\refvar{tMax}{}`;
    `\refvar{SurfaceInteraction}{}` isect;
    return `\refvar[Shape::Intersect]{Intersect}{}`(ray, &tHit, &isect, testAlphaTexture);
}
\end{lstlisting}

\subsection{表面积}\label{sub:表面积}
为了合理地将\refvar{Shape}{}用作面光源,需要能够计算物体空间中形状的表面积。
\begin{lstlisting}
`\refcode{Shape Interface}{+=}\lastnext{ShapeInterface}`
virtual `\refvar{Float}{}` `\initvar[Shape::Area]{Area}{}`() const = 0;
\end{lstlisting}

\subsection{面性}\label{sub:面性}
许多渲染系统,尤其是基于扫描线或缓冲区算法的,
支持“单面”\sidenote{译者注:原文one-sided。此外“面性”原文sidedness。}形状概念——
形状可从前面看到但从后面看就消失了。
特别地,若几何体封闭且总是从外面看,
则其背面部分可以去掉而不改变结果图像。
该优化能极大提升这类隐藏表面去除算法的速度。
然而,当光线追踪使用该技术时,提升性能的潜力下降了,
因为它经常需要在确定曲面法线以进行背面测试之前执行光线-物体相交。
而且,若单面物体实际上并不封闭则该功能会导致物理矛盾的场景描述。
例如,当阴影射线从光源追踪到表面上一点时,另一表面可能阻挡光线,
但若阴影射线从另一方向追踪时则不会这样。
由于这些原因,pbrt不支持该功能。