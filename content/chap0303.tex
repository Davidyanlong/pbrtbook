\section{圆柱体}\label{sec:圆柱体}

另一个有用的二次曲面是\keyindex{圆柱体}{cylinder}{};
pbrt提供以$z$轴为中心的圆柱体\refvar{Shape}{}。
实现在文件\href{https://github.com/mmp/pbrt-v3/tree/master/src/shapes/cylinder.h}{\ttfamily shapes/cylinder.h}和
\href{https://github.com/mmp/pbrt-v3/tree/master/src/shapes/cylinder.cpp}{\ttfamily shapes/cylinder.cpp}内。
用户为圆柱体提供最小和最大$z$值,以及半径和最大扫掠值$\varphi$(\reffig{3.6})。
\begin{figure}[htbp]
    \centering\input{Pictures/chap03/Cylindersetting.tex}
    \caption{圆柱体形状基本设置。它半径为$r$并沿$z$轴覆盖一定范围。
        通过指定最大值$\varphi$可扫掠出部分圆柱体。}
    \label{fig:3.6}
\end{figure}
\begin{lstlisting}
`\initcode{Cylinder Declarations}{=}`
class `\initvar{Cylinder}{}` : public `\refvar{Shape}{}` {
public:
    `\refcode{Cylinder Public Methods}{}`
protected:
    `\refcode{Cylinder Private Data}{}`
};
\end{lstlisting}

参数形式下,圆柱体由下列方程描述:
\begin{align*}
    \varphi & =u\varphi_{\max}\, ,               \\
    x       & =r\cos\varphi\, ,                  \\
    y       & =r\sin\varphi\, ,                  \\
    z       & =z_{\min}+v(z_{\max}-z_{\min})\, .
\end{align*}

\reffig{3.7}展示了两个圆柱体的渲染图像。像球体图像那样,
左边圆柱体是完整圆柱体,右边是部分圆柱体,因为它的$\varphi_{\max}$值小于$2\pi$.
\begin{figure}[htbp]
    \centering\includegraphics[width=\linewidth]{chap03/twocylinders.png}
    \caption{两个圆柱体。左边是完整圆柱体,右边是部分圆柱体。}
    \label{fig:3.7}
\end{figure}

\begin{lstlisting}
`\initcode{Cylinder Public Methods}{=}`
`\refvar{Cylinder}{}`(const `\refvar{Transform}{}` *ObjectToWorld, const `\refvar{Transform}{}` *WorldToObject,
         bool reverseOrientation, `\refvar{Float}{}` radius, `\refvar{Float}{}` zMin, `\refvar{Float}{}` zMax,
         `\refvar{Float}{}` phiMax)
    : `\refvar{Shape}{}`(ObjectToWorld, WorldToObject, reverseOrientation),
      `\refvar[Cylinder::radius]{radius}{}`(radius), `\refvar[Cylinder::zMin]{zMin}{}`(std::min(zMin, zMax)),
      `\refvar[Cylinder::zMax]{zMax}{}`(std::max(zMin, zMax)),
      `\refvar[Cylinder::phiMax]{phiMax}{}`(`\refvar{Radians}{}`(`\refvar{Clamp}{}`(phiMax, 0, 360))) { }
\end{lstlisting}
\begin{lstlisting}
`\initcode{Cylinder Private Data}{=}`
const `\refvar{Float}{}` `\initvar[Cylinder::radius]{radius}{}`, `\initvar[Cylinder::zMin]{zMin}{}`, `\initvar[Cylinder::zMax]{zMax}{}`, `\initvar[Cylinder::phiMax]{phiMax}{}`;
\end{lstlisting}

\subsection{边界}\label{sub:边界3}
像球那样,圆柱体边界方法用$z$的范围计算保守边界框但不考虑最大的$\varphi$.
\begin{lstlisting}
`\initcode{Cylinder Method Definitions}{=}\initnext{CylinderMethodDefinitions}`
`\refvar{Bounds3f}{}` `\refvar{Cylinder}{}`::`\initvar[Cylinder::ObjectBound]{\refvar{ObjectBound}{}}{}`() const {
    return `\refvar{Bounds3f}{}`(`\refvar{Point3f}{}`(-`\refvar[Cylinder::radius]{radius}{}`, -`\refvar[Cylinder::radius]{radius}{}`, `\refvar[Cylinder::zMin]{zMin}{}`),
                    `\refvar{Point3f}{}`( `\refvar[Cylinder::radius]{radius}{}`,  `\refvar[Cylinder::radius]{radius}{}`, `\refvar[Cylinder::zMax]{zMax}{}`));
}
\end{lstlisting}

\subsection{相交测试}\label{sub:相交测试3}
光线-圆柱体相交公式可以通过把射线方程代入圆柱体的隐式方程得到,和球体的情况一样。
以$z$轴为中心$r$为半径的无限长圆柱体的隐式方程为
\begin{align*}
    x^2+y^2-r^2=0\, .
\end{align*}
代入射线方程\refeq{2.3},我们有
\begin{align*}
    (o_x+td_x)^2+(o_y+td_y)^2=r^2\, .
\end{align*}
我们将其展开并求二次式方程$at^2+bt+c=0$的系数得
\begin{align*}
    a & =d_x^2+d_y^2\, ,      \\
    b & =2(d_xo_x+d_yo_y)\, , \\
    c & =o_x^2+o_y^2-r^2\, .
\end{align*}
\begin{lstlisting}
`\initcode{Compute quadratic cylinder coefficients}{=}`
`\refcode{Initialize EFloat ray coordinate values}{}`
`\refvar{EFloat}{}` a = dx * dx + dy * dy;
`\refvar{EFloat}{}` b = 2 * (dx * ox + dy * oy);
`\refvar{EFloat}{}` c = ox * ox + oy * oy - `\refvar{EFloat}{}`(`\refvar[Cylinder::radius]{radius}{}`) * `\refvar{EFloat}{}`(`\refvar[Cylinder::radius]{radius}{}`);
\end{lstlisting}

所有二次曲面形状求解二次方程的过程都一样,
所以下面会复用来自\refvar{Sphere}{}
相交方法的一些代码片。
\begin{lstlisting}
`\refcode{Cylinder Method Definitions}{+=}\lastnext{CylinderMethodDefinitions}`
bool `\refvar{Cylinder}{}`::`\initvar[Cylinder::Intersect]{\refvar[Shape::Intersect]{Intersect}{}}{}`(const `\refvar{Ray}{}` &r, `\refvar{EFloat}{}` *tHit,
        `\refvar{SurfaceInteraction}{}` *isect, bool testAlphaTexture) const {
    `\refvar{Float}{}` phi;
    `\refvar{Point3f}{}` pHit;
    `\refcode{Transform Ray to object space}{}`
    `\refcode{Compute quadratic cylinder coefficients}{}`
    `\refcode{Solve quadratic equation for t values}{}`
    `\refcode{Compute cylinder hit point and $\varphi$}{}`
    `\refcode{Test cylinder intersection against clipping parameters}{}`
    `\refcode{Find parametric representation of cylinder hit}{}`
    `\refcode{Compute error bounds for cylinder intersection}{}`
    `\refcode{Initialize SurfaceInteraction from parametric information}{}`
    `\refcode{Update tHit for quadric intersection}{}`
    return true;
}
\end{lstlisting}

像球体那样,这里的实现改进了计算的交点
以降低在计算射线方程中累积舍入误差的影响;见\refsub{定界交点误差}。
于是我们由圆柱体的参数化描述从$x$和$y$反解出$\varphi$;
最后所得结果与球体相同。
\begin{lstlisting}
`\initcode{Compute cylinder hit point and $\varphi$}{=}`
pHit = ray((`\refvar{Float}{}`)tShapeHit);
`\refcode{Refine cylinder intersection point}{}`
phi = std::atan2(pHit.y, pHit.x);
if (phi < 0) phi += 2 * `\refvar{Pi}{}`;
\end{lstlisting}

相交方法下一部分保证命中点在指定$z$范围内且能接受角度$\varphi$.
否则它就拒绝该命中点,并在之前还没尝试的前提下检查$t_1$——这和
\refvar{Sphere::Intersect}{()}
中的条件逻辑很像。
\begin{lstlisting}
`\initcode{Test cylinder intersection against clipping parameters}{=}`
if (pHit.z < `\refvar[Cylinder::zMin]{zMin}{}` || pHit.z > `\refvar[Cylinder::zMax]{zMax}{}` || phi > `\refvar[Cylinder::phiMax]{phiMax}{}`) {
    if (tShapeHit == t1) return false;
    tShapeHit = t1;
    if (t1.`\refvar{UpperBound}{}`() > ray.`\refvar{tMax}{}`) return false;
    `\refcode{Compute cylinder hit point and $\varphi$}{}`
    if (pHit.z < `\refvar[Cylinder::zMin]{zMin}{}` || pHit.z > `\refvar[Cylinder::zMax]{zMax}{}` || phi > `\refvar[Cylinder::phiMax]{phiMax}{}`)
        return false;
}
\end{lstlisting}

再次将$\varphi$缩放到0到1计算$u$值。
直接反解圆柱体参数方程的$z$值求得$v$参数坐标。
\begin{lstlisting}
`\initcode{Find parametric representation of cylinder hit}{=}`
`\refvar{Float}{}` u = phi / `\refvar[Cylinder::phiMax]{phiMax}{}`;
`\refvar{Float}{}` v = (pHit.z - `\refvar[Cylinder::zMin]{zMin}{}`) / (`\refvar[Cylinder::zMax]{zMax}{}` - `\refvar[Cylinder::zMin]{zMin}{}`);
`\refcode{Compute cylinder $\partial$p/$\partial$u and $\partial$p/$\partial$v}{}`
`\refcode{Compute cylinder $\partial$n/$\partial$u and $\partial$n/$\partial$v}{}`
\end{lstlisting}

圆柱体的偏导数很容易推导:
\begin{align*}
    \frac{\partial\bm p}{\partial u} & =(-\varphi_{\max}y,\varphi_{\max}x,0)\, , \\
    \frac{\partial\bm p}{\partial v} & =(0,0,z_{\max}-z_{\min})\, .
\end{align*}
\begin{lstlisting}
`\initcode{Compute cylinder $\partial$p/$\partial$u and $\partial$p/$\partial$v}{=}`
`\refvar{Vector3f}{}` dpdu(-`\refvar[Cylinder::phiMax]{phiMax}{}` * pHit.y, `\refvar[Cylinder::phiMax]{phiMax}{}` * pHit.x, 0);
`\refvar{Vector3f}{}` dpdv(0, 0, `\refvar[Cylinder::zMax]{zMax}{}` - `\refvar[Cylinder::zMin]{zMin}{}`);
\end{lstlisting}

我们又用外恩加滕方程计算圆柱法线的参数化偏导数。
相关偏导数为:
\begin{align*}
    \frac{\partial^2\bm p}{\partial u^2}         & =-\varphi_{\max}^2(x,y,0)\, , \\
    \frac{\partial^2\bm p}{\partial u\partial v} & =(0,0,0)\, ,                  \\
    \frac{\partial^2\bm p}{\partial v^2}         & =(0,0,0)\, .
\end{align*}
\begin{lstlisting}
`\initcode{Compute cylinder $\partial$n/$\partial$u and $\partial$n/$\partial$v}{=}`
`\refvar{Vector3f}{}` d2Pduu = -`\refvar[Cylinder::phiMax]{phiMax}{}` * `\refvar[Cylinder::phiMax]{phiMax}{}` * `\refvar{Vector3f}{}`(pHit.x, pHit.y, 0);
`\refvar{Vector3f}{}` d2Pduv(0, 0, 0), d2Pdvv(0, 0, 0);
`\refcode{Compute coefficients for fundamental forms}{}`
`\refcode{Compute $\partial$n/$\partial$u and $\partial$n/$\partial$v from fundamental form coefficients}{}`
\end{lstlisting}

\subsection{表面积}\label{sub:表面积3}
圆柱体就是卷起来的矩形。如果你展开矩形,
其高是$z_{\max}-z_{\min}$,宽是$r\varphi_{\max}$:
\begin{lstlisting}
`\refcode{Cylinder Method Definitions}{+=}\lastnext{CylinderMethodDefinitions}`
`\refvar{Float}{}` `\refvar{Cylinder}{}`::`\initvar[Cylinder::Area]{\refvar[Shape::Area]{Area}{}}{}`() const {
    return (`\refvar[Cylinder::zMax]{zMax}{}` - `\refvar[Cylinder::zMin]{zMin}{}`) * `\refvar[Cylinder::radius]{radius}{}` * `\refvar[Cylinder::phiMax]{phiMax}{}`;
}
\end{lstlisting}