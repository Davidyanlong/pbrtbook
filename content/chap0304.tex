\section{圆盘}\label{sec:圆盘}

\keyindex{圆盘}{disk}{}是一种有趣的二次曲面,
因为它有特别简单的相交例程避免求解二次方程。
pbrt中,\refvar{Disk}{}是半径为$r$沿$z$轴高度为$h$的圆盘。
它的实现在文件\href{https://github.com/mmp/pbrt-v3/tree/master/src/shapes/disk.h}{\ttfamily shapes/disk.h}和
\href{https://github.com/mmp/pbrt-v3/tree/master/src/shapes/disk.cpp}{\ttfamily shapes/disk.cpp}中。
\begin{lstlisting}
`\initcode{Disk Declarations}{=}`
class `\initvar{Disk}{}` : public `\refvar{Shape}{}` {
public:
    `\refcode{Disk Public Methods}{}`
private:
    `\refcode{Disk Private Data}{}`
};
\end{lstlisting}

为了描述部分圆盘,用户可指定最大$\varphi$值,
超出的圆盘部分被截去(\reffig{3.8})。
圆盘还能通过指定内径$r_{\mathrm{i}}$推广为\keyindex{圆环}{annulus}{}。
参数形式下,它描述为:
\begin{align*}
    \varphi & =u\varphi_{\max}\, ,                     \\
    x       & =((1-v)r+vr_{\mathrm{i}})\cos\varphi\, , \\
    y       & =((1-v)r+vr_{\mathrm{i}})\sin\varphi\, , \\
    z       & =h\, .
\end{align*}
\begin{figure}[htbp]
    \centering\input{Pictures/chap03/Disksetting.tex}
    \caption{圆盘形状基本设置。圆盘半径为$r$,沿$z$轴高度为$h$.
        通过指定最大$\varphi$值和内径$r_{\mathrm{i}}$可扫掠出部分圆盘。}
    \label{fig:3.8}
\end{figure}

\reffig{3.9}是两个圆盘的渲染图像。
\begin{figure}[htbp]
    \centering\includegraphics[width=\linewidth]{chap03/twodisks.png}
    \caption{两个圆盘。左边是完整圆盘,右边是部分圆盘。}
    \label{fig:3.9}
\end{figure}
\begin{lstlisting}
`\initcode{Disk Public Methods}{=}`
`\refvar{Disk}{}`(const `\refvar{Transform}{}` *ObjectToWorld, const `\refvar{Transform}{}` *WorldToObject,
     bool reverseOrientation, `\refvar{Float}{}` height, `\refvar{Float}{}` radius,
     `\refvar{Float}{}` innerRadius, `\refvar{Float}{}` phiMax)
    : `\refvar{Shape}{}`(ObjectToWorld, WorldToObject, reverseOrientation),
      `\refvar[Disk::height]{height}{}`(height), `\refvar[Disk::radius]{radius}{}`(radius), `\refvar[Disk::innerRadius]{innerRadius}{}`(innerRadius),
      `\refvar[Disk::phiMax]{phiMax}{}`(`\refvar{Radians}{}`(`\refvar{Clamp}{}`(phiMax, 0, 360))) { }
\end{lstlisting}
\begin{lstlisting}
`\initcode{Disk Private Data}{=}`
const `\refvar{Float}{}` `\initvar[Disk::height]{height}{}`, `\initvar[Disk::radius]{radius}{}`, `\initvar[Disk::innerRadius]{innerRadius}{}`, `\initvar[Disk::phiMax]{phiMax}{}`;
\end{lstlisting}

\subsection{边界}\label{sub:边界4}
边界方法非常简单;
它计算的边界框的中心沿$z$轴高度为$h$,
$x$和$y$方向边长都为\refvar[Disk::radius]{radius}{}。
\begin{lstlisting}
`\initcode{Disk Method Definitions}{=}\initnext{DiskMethodDefinitions}`
`\refvar{Bounds3f}{}` `\refvar{Disk}{}`::`\initvar[Disk::ObjectBound]{\refvar{ObjectBound}{}}{}`() const {
    return `\refvar{Bounds3f}{}`(`\refvar{Point3f}{}`(-`\refvar[Disk::radius]{radius}{}`, -`\refvar[Disk::radius]{radius}{}`, `\refvar[Disk::height]{height}{}`),
                    `\refvar{Point3f}{}`( `\refvar[Disk::radius]{radius}{}`,  `\refvar[Disk::radius]{radius}{}`, `\refvar[Disk::height]{height}{}`));
}
\end{lstlisting}

\subsection{相交测试}\label{sub:相交测试4}
射线与圆盘相交也很简单。射线与圆盘所在平面$z=h$相交,再检查交点是否在圆盘内。
\begin{lstlisting}
`\refcode{Disk Method Definitions}{+=}\lastnext{DiskMethodDefinitions}`
bool `\refvar{Disk}{}`::`\initvar[Disk::Intersect]{\refvar[Shape::Intersect]{Intersect}{}}{}`(const `\refvar{Ray}{}` &r, `\refvar{Float}{}` *tHit,
        `\refvar{SurfaceInteraction}{}` *isect, bool testAlphaTexture) const {
    `\refcode{Transform Ray to object space}{}`
    `\refcode{Compute plane intersection for disk}{}`
    `\refcode{See if hit point is inside disk radii and $\varphi$max}{}`
    `\refcode{Find parametric representation of disk hit}{}`
    `\refcode{Refine disk intersection point}{}`
    `\refcode{Compute error bounds for disk intersection}{}`
    `\refcode{Initialize SurfaceInteraction from parametric information}{}`
    `\refcode{Update tHit for quadric intersection}{}`
    return true;
}
\end{lstlisting}

第一步是计算射线与圆盘所在平面相交处的参数$t$值。
我们想求得射线位置$z$分量等于圆盘高度时的$t$.因此
\begin{align*}
    h=o_z+td_z\, ,
\end{align*}
即
\begin{align*}
    t=\frac{h-o_z}{d_z}\, .
\end{align*}

相交方法算出$t$值并检查它是否在合法值范围{\ttfamily (0, \refvar{tMax}{})}内。
如果不在,例程就返回{\ttfamily false}。
\begin{lstlisting}
`\initcode{Compute plane intersection for disk}{=}`
`\refcode{Reject disk intersections for rays parallel to the disk's plane}{}`
`\refvar{Float}{}` tShapeHit = (`\refvar[Disk::height]{height}{}` - ray.o.z) / ray.d.z;
if (tShapeHit <= 0 || tShapeHit >= ray.`\refvar{tMax}{}`)
    return false;
\end{lstlisting}

如果射线平行于圆盘平面(即它的$z$分量为0),则报告不相交。
射线平行于圆盘平面且在该平面内的情况有些模棱两可,
但把和圆盘边缘相交定义为“不相交”是最合理的。
必须明确处理该情况以避免下列代码生成NaN浮点值。
\begin{lstlisting}
`\initcode{Reject disk intersections for rays parallel to the disk's plane}{=}`
if (ray.d.z == 0)
    return false;
\end{lstlisting}

现在相交方法可以计算射线与平面相交的点{\ttfamily pHit}了。
知道平面相交处后,如果从命中点到圆盘中心的距离
大于\refvar{Disk::radius}{}或小于\refvar{Disk::innerRadius}{}
则返回{\ttfamily false}。
该过程可以利用中心点{\ttfamily (0,0,\refvar[Disk::height]{height}{})}的$x$和$y$坐标是零,
{\ttfamily pHit}的$z$分量等于\refvar[Disk::height]{height}{}的事实,
通过实际计算到中心的平方距离来优化。
\begin{lstlisting}
`\initcode{See if hit point is inside disk radii and $\varphi$max}{=}`
`\refvar{Point3f}{}` pHit = ray(tShapeHit);
`\refvar{Float}{}` dist2 = pHit.x * pHit.x + pHit.y * pHit.y;
if (dist2 > `\refvar[Disk::radius]{radius}{}` * `\refvar[Disk::radius]{radius}{}` || dist2 < `\refvar[Disk::innerRadius]{innerRadius}{}` * `\refvar[Disk::innerRadius]{innerRadius}{}`)
    return false;
`\refcode{Test disk $\varphi$ value against $\varphi$max}{}`
\end{lstlisting}

如果距离检查通过,则最终测试保证命中点的$\varphi$值
在零到调用者指定的$\varphi_{\max}$之间。
反解圆盘参数化得到和其他二次曲面一样的$\varphi$表达式。
\begin{lstlisting}
`\initcode{Test disk $\varphi$ value against $\varphi$max}{=}`
`\refvar{Float}{}` phi = std::atan2(pHit.y, pHit.x);
if (phi < 0) phi += 2 * `\refvar{Pi}{}`;
if (phi > `\refvar[Disk::phiMax]{phiMax}{}`)
    return false;
\end{lstlisting}

如果我们走到这一步,则与圆盘存在相交。
参数{\ttfamily u}被缩放来反映$\varphi_{\max}$指定的部分圆盘,
{\ttfamily v}通过反解参数方程算得。
命中点的偏导数方程推导过程和之前二次曲面用的一样。
因为圆盘的法线在任何地方都一样,
偏导数$\displaystyle\frac{\partial\bm n}{\partial u}$和$\displaystyle\frac{\partial\bm n}{\partial v}$都是平凡的$(0,0,0)$.
\begin{lstlisting}
`\initcode{Find parametric representation of disk hit}{=}`
`\refvar{Float}{}` u = phi / `\refvar[Disk::phiMax]{phiMax}{}`;
`\refvar{Float}{}` rHit = std::sqrt(dist2);
`\refvar{Float}{}` oneMinusV = ((rHit - `\refvar[Disk::innerRadius]{innerRadius}{}`) /
                   (`\refvar[Disk::radius]{radius}{}` - `\refvar[Disk::innerRadius]{innerRadius}{}`));
`\refvar{Float}{}` v = 1 - oneMinusV;
`\refvar{Vector3f}{}` dpdu(-`\refvar[Disk::phiMax]{phiMax}{}` * pHit.y, `\refvar[Disk::phiMax]{phiMax}{}` * pHit.x, 0);
`\refvar{Vector3f}{}` dpdv = `\refvar{Vector3f}{}`(pHit.x, pHit.y, 0.) * (`\refvar[Disk::innerRadius]{innerRadius}{}` - `\refvar[Disk::radius]{radius}{}`) /
                rHit;
`\refvar{Normal3f}{}` dndu(0, 0, 0), dndv(0, 0, 0);
\end{lstlisting}

\subsection{表面积}\label{sub:表面积4}
圆盘有计算简单的表面积,因为它们只是圆环的一部分:
\begin{align*}
    A=\frac{\varphi_{\max}}{2}(r^2-r_{\mathrm{i}}^2)\, .
\end{align*}
\begin{lstlisting}
`\refcode{Disk Method Definitions}{+=}\lastnext{DiskMethodDefinitions}`
`\refvar{Float}{}` `\refvar{Disk}{}`::`\initvar[Disk::Area]{\refvar[Shape::Area]{Area}{}}{}`() const { 
    return `\refvar[Disk::phiMax]{phiMax}{}` * 0.5 * (`\refvar[Disk::radius]{radius}{}` * `\refvar[Disk::radius]{radius}{}` - `\refvar[Disk::innerRadius]{innerRadius}{}` * `\refvar[Disk::innerRadius]{innerRadius}{}`);
}
\end{lstlisting}