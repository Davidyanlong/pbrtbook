\chapterimage{Pictures/chap04/landscape-above-1300x650.png}
\chapter{图元和相交加速}\label{chap:图元和相交加速}
\setcounter{sidenote}{1}

上一章描述的类只关注表示3D对象的几何性质。
虽然类\refvar{Shape}{}为诸如相交和定界等几何操作提供了方便的抽象,
但它没有包含能完全描述场景中一个物体的足够信息。
例如,有必要给每个形状\keyindex{绑定}{bind}{}材质属性以指定其外观。
为了实现这个目标,本章介绍抽象基类\refvar{Primitive}{}并提供大量实现。

要直接渲染的形状表示为类\refvar{GeometricPrimitive}{}。
该类结合了\refvar{Shape}{}及其外观属性的描述。
这样pbrt的几何与着色部分可以干净地分开,
这些外观属性包含在第\refchap{材质}描述的类\refvar{Material}{}中。

类\refvar{TransformedPrimitive}{}处理场景中\refvar{Shape}{}的两个更一般用途:
动画变换矩阵和物体实例化,对于包含许多在不同位置的同一几何体的场景(例如\reffig{4.1}),
它们能大大减少内存需求。
实现这些特性本质上要求在世界空间\refvar{Shape}{}的表示
和实际场景世界空间之间插入额外的变换矩阵。
因此两者都由一个类处理。
\begin{figure}[htbp]
    \centering\includegraphics[width=\linewidth]{chap04/landscape-above.png}
    \caption{该室外场景大量运用实例化作为场景描述的压缩机制。
        场景中只有2400万个不同的三角形,但因为通过实例化复用物体,
        总的几何复杂度为31亿个三角形(场景由Laubwerk提供)。}
    \label{fig:4.1}
\end{figure}

本章还介绍类\refvar{Aggregate}{},
它表示可以容纳许多\refvar{Primitive}{}的容器。
pbrt用该类作为\keyindex{加速结构}{acceleration structure}{}的基础——
帮助减少测试射线与场景中所有$n$个物体相交本来的复杂度$O(n)$.
大多数射线只与少量图元相交而与剩下的错开很大距离。
如果相交加速算法能一次拒绝整组图元,
比起简单地依次测试每条射线与每个图元,性能会有巨大的提升。
对这些加速结构复用\refvar{Primitive}{}接口的一个好处是
让支持一种\keyindex{加速器}{accelerator}{}包含另一种加速器的混合方法更容易。

本章描述了两个加速器的实现,
第一个\refvar{BVHAccel}{}基于构建场景中围绕物体的边界框的层级,
第二个\refvar{KdTreeAccel}{}基于自适应递归空间细分。
虽然提出了许多其他的加速结构,但如今几乎所有光线追踪器都用这两种。
本章末的“扩展阅读”一节有对其他可能性的大量参考。

\input{content/chap0401.tex}

\input{content/chap0402.tex}

\section{包围盒层次}\label{sec:包围盒层次}

\keyindex{包围盒层次}{bounding volume hierarchy}{}(BVH)是一种
基于图元细分的光线相交加速方法,把图元划分为不相交集合的层次
(相反,空间细分一般把空间划分为不相交集合的层次)。
\reffig{4.2}展示了简单场景的包围盒层次\sidenote{译者注:包围盒是边界框的近义词。}。
图元存于\keyindex{叶子}{leaf}{}中,只要它不与节点的边界相交,
该节点下的子树就可以跳过。
\begin{figure}[htbp]
    \centering\input{Pictures/chap04/Primitivesandhierarchy.tex}
    \caption{简单场景的包围盒层次。(a)一小部分图元,边界框用虚线表示。
        图元基于邻近度聚合;这里,球体和等边三角形在被框住整个场景的边界框
        围住之前都被另一个边界框包围了(都用实线表示)。(b)相应的包围盒层次。
        根节点持有整个场景。这里它有两个孩子,一个保存包围了球体和等边三角形的边界框
        (又把这些图元作为其孩子),另一个保存持有瘦三角形的边界框。}
    \label{fig:4.2}
\end{figure}

图元细分的一个性质是每个图元只在层次中出现一次。
相反,一个图元可能与空间细分的多个空间区域重合,
因此在光线穿过它们时要多次测试相交
\footnote{\protect\keyindex{邮箱}{mailboxing}{}技术可用于
    让使用空间细分的加速器避免这样的多次相交,但它的实现在存在多进程时会很棘手。
    “扩展阅读”一节有关于邮箱的更多信息。}。
该性质还意味着表示图元细分层次所需的内存量是有界的。
对于每个叶子中保存单个图元的二叉BVH,节点总数为$2n-1$,其中$n$是图元数量。
有$n$个叶子节点和$n-1$个内部节点
\sidenote{译者注:这些结论利用了二叉BVH的前提:每个节点要么是叶子节点,要么是有两个孩子的内部节点。}。
如果叶子保存了多个图元,则需要的节点更少。

构建BVH比kd树更高效,kd树分发光线相交测试通常比BVH稍快但构建时间长得多。
另一方面,BVH通常数值更稳定,比起kd树更不容易因为舍入误差错过相交。

BVH加速器{\refvar{BVHAccel}{}}定义在\href{https://github.com/mmp/pbrt-v3/tree/master/src/accelerators/bvh.h}{\ttfamily accelerators/bvh.h}
和\href{https://github.com/mmp/pbrt-v3/tree/master/src/accelerators/bvh.cpp}{\ttfamily accelerators/bvh.cpp}
中。除了要保存的图元以及任何叶子节点中的最大图元数目,
其构造函数还接收一个描述当划分图元以构建树时要用四个算法中哪一个的枚举值。
应该用默认值\refvar{SAH}{},它表示\refsub{表面积启发法}讨论的基于“表面积启发法”的算法。
另一个是\refsub{线性包围盒层次}讨论的\refvar{HLBVH}{},
它能更高效地构造(且更易并行化),但建立的树不如\refvar{SAH}{}高效。
剩下的两种方法使用的计算量甚至更少,但创建的树的质量非常低。
\begin{lstlisting}
`\initcode{BVHAccel Public Types}{=}`
enum class `\initvar{SplitMethod}{}` { `\initvar{SAH}{}`, `\initvar{HLBVH}{}`, `\initvar{Middle}{}`, `\initvar{EqualCounts}{}` };
\end{lstlisting}
\begin{lstlisting}
`\initcode{BVHAccel Method Definitions}{=}\initnext{BVHAccelMethodDefinitions}`
`\initvar{BVHAccel}{}`::`\refvar{BVHAccel}{}`(const std::vector<std::shared_ptr<`\refvar{Primitive}{}`>> &p,
         int maxPrimsInNode, `\refvar{SplitMethod}{}` splitMethod)
     : `\refvar{maxPrimsInNode}{}`(std::min(255, maxPrimsInNode)), `\refvar[BVHAccel::primitives]{primitives}{}`(p),
       `\refvar{splitMethod}{}`(splitMethod) {
    if (primitives.size() == 0)
        return;
    `\refcode{Build BVH from primitives}{}`
}
\end{lstlisting}
\begin{lstlisting}
`\initcode{BVHAccel Private Data}{=}\initnext{BVHAccelPrivateData}`
const int `\initvar{maxPrimsInNode}{}`;
const `\refvar{SplitMethod}{}` `\initvar{splitMethod}{}`;
std::vector<std::shared_ptr<`\refvar{Primitive}{}`>> `\initvar[BVHAccel::primitives]{primitives}{}`;
\end{lstlisting}

\subsection{BVH构建}\label{sub:BVH构建}
这里的实现中构建BVH有三个阶段。
首先,计算关于每个图元的边界信息并保存到将于树构建期间使用的数组中。
接着,用选择的编码于\refvar{SplitMethod}{}的算法构建树。
结果是\keyindex{二叉树}{binary tree}{}每个内部节点
都有指针指向其孩子且每个叶子节点都有指向一个或多个图元的引用。
最后,该树转化为更紧实(且因此更高效)的无指针表示以供渲染时使用
(虽然在构建树期间直接计算无指针表示也可以,但用该方法实现更简单)。
\begin{lstlisting}
`\initcode{Build BVH from primitives}{=}`
`\refcode{Initialize primitiveInfo array for primitives}{}`
`\refcode{Build BVH tree for primitives using primitiveInfo}{}`
`\refcode{Compute representation of depth-first traversal of BVH tree}{}`
\end{lstlisting}

对于每个要存于BVH的图元,我们在结构体\refvar{BVHPrimitiveInfo}{}的一个实例中
存储其边界框的形心、完整边界框以及它在\refvar{primitives}{}数组中的索引。
\begin{lstlisting}
`\initcode{Initialize primitiveInfo array for primitives}{=}`
std::vector<`\refvar{BVHPrimitiveInfo}{}`> primitiveInfo(`\refvar[BVHAccel::primitives]{primitives}{}`.size());
for (size_t i = 0; i < `\refvar[BVHAccel::primitives]{primitives}{}`.size(); ++i)
    primitiveInfo[i] = { i, `\refvar[BVHAccel::primitives]{primitives}{}`[i]->`\refvar[Primitive::WorldBound]{WorldBound}{}`() };
\end{lstlisting}
\begin{lstlisting}
`\initcode{BVHAccel Local Declarations}{=}\initnext{BVHAccelLocalDeclarations}`
struct `\initvar{BVHPrimitiveInfo}{}` {
    `\refvar{BVHPrimitiveInfo}{}`(size_t primitiveNumber, const `\refvar{Bounds3f}{}` &bounds)
        : `\refvar{primitiveNumber}{}`(primitiveNumber), `\refvar[BVHPrimitiveInfo::bounds]{bounds}{}`(bounds),
          `\refvar{centroid}{}`(.5f * bounds.`\refvar{pMin}{}` + .5f * bounds.`\refvar{pMax}{}`) { }
    size_t `\initvar{primitiveNumber}{}`;
    `\refvar{Bounds3f}{}` `\initvar[BVHPrimitiveInfo::bounds]{bounds}{}`;
    `\refvar{Point3f}{}` `\initvar{centroid}{}`;
};
\end{lstlisting}

现在可以开始层次构建了。如果选择HLBVH构建算法,则调用\refvar{HLBVHBuild}{()}
构建树。其他三种构建算法都由\refvar{recursiveBuild}{()}负责。
初始调用这些函数时传递了所有要存于树中的图元。
它们返回一个指向树根的指针,用结构体\refvar{BVHBuildNode}{}表示。
树节点应该用提供的\refvar{MemoryArena}{}分配内存,
创建的总数应存于{\ttfamily *totalNodes}中。

树构建过程的一个重要副作用是通过参数{\ttfamily orderedPrims}返回指向图元的新指针数组;
该数组保存了有序的图元这样叶子节点的图元在数组中占有连续的范围。
在树构建后它与原始的\refvar[BVHAccel::primitives]{primitives}{}数组交换。
\begin{lstlisting}
`\initcode{Build BVH tree for primitives using primitiveInfo}{=}`
`\refvar{MemoryArena}{}` arena(1024 * 1024);
int totalNodes = 0;
std::vector<std::shared_ptr<`\refvar{Primitive}{}`>> orderedPrims;
`\refvar{BVHBuildNode}{}` *root;
if (splitMethod == `\refvar{SplitMethod}{}`::`\refvar{HLBVH}{}`)
    root = `\refvar{HLBVHBuild}{}`(arena, primitiveInfo, &totalNodes, orderedPrims);
else
    root = `\refvar{recursiveBuild}{}`(arena, primitiveInfo, 0, `\refvar[BVHAccel::primitives]{primitives}{}`.size(),
                          &totalNodes, orderedPrims);
`\refvar[BVHAccel::primitives]{primitives}{}`.swap(orderedPrims);
\end{lstlisting}

每个\refvar{BVHBuildNode}{}表示一个BVH节点。
所有节点存储一个\refvar{Bounds3f}{}以表示该节点下所有孩子的边界。
每个内部节点在\refvar[BVHBuildNode::children]{children}{}中存有指向其两个孩子的指针。
内部节点也记录图元沿哪个坐标轴划分分给它们的两个孩子;
该信息用于提高遍历算法的性能。
叶子节点需要记录哪个或哪些图元保存在它们中;
数组\refvar{BVHAccel::primitives}{}中从偏移量\refvar{firstPrimOffset}{}起
直到但不包括{\ttfamily\refvar{firstPrimOffset}{}+\refvar[BVHBuildNode:nPrimitives]{nPrimitives}{}}的元素是叶子中的元素
(因此需要记录图元数组,这样就可以利用该表示,
而不是例如在每个叶子节点中保存一个大小可变的图元索引数组)。
\begin{lstlisting}
`\refcode{BVHAccel Local Declarations}{+=}\lastnext{BVHAccelLocalDeclarations}`
struct `\initvar{BVHBuildNode}{}` {
    `\refcode{BVHBuildNode Public Methods}{}`
    `\refvar{Bounds3f}{}` `\initvar[BVHBuildNode::bounds]{bounds}{}`;
    `\refvar{BVHBuildNode}{}` *`\initvar[BVHBuildNode::children]{children}{}`[2];
    int `\initvar[BVHBuildNode::splitAxis]{splitAxis}{}`, `\initvar{firstPrimOffset}{}`, `\initvar[BVHBuildNode:nPrimitives]{nPrimitives}{}`;
};
\end{lstlisting}

我们将通过其孩子指针是否有值{\ttfamily nullptr}来分别区分叶子和内部节点。
\begin{lstlisting}
`\initcode{BVHBuildNode Public Methods}{=}\initnext{BVHBuildNodePublicMethods}`
void `\initvar[BVHBuildNode::InitLeaf]{InitLeaf}{}`(int first, int n, const `\refvar{Bounds3f}{}` &b) {
    `\refvar{firstPrimOffset}{}` = first;
    `\refvar[BVHBuildNode:nPrimitives]{nPrimitives}{}` = n;
    `\refvar[BVHBuildNode::bounds]{bounds}{}` = b;
    `\refvar[BVHBuildNode::children]{children}{}`[0] = `\refvar[BVHBuildNode::children]{children}{}`[1] = nullptr;
}
\end{lstlisting}

方法\refvar[BVHBuildNode::InitInterior]{InitInterior}{()}要求已创建两个孩子节点,这样它们的指针才能传入。
该要求让计算内部节点的边界更加容易了,因为孩子的边界可以立刻获得。
\begin{lstlisting}
`\refcode{BVHBuildNode Public Methods}{+=}\lastcode{BVHBuildNodePublicMethods}`
void `\initvar[BVHBuildNode::InitInterior]{InitInterior}{}`(int axis, `\refvar{BVHBuildNode}{}` *c0, `\refvar{BVHBuildNode}{}` *c1) {
    `\refvar[BVHBuildNode::children]{children}{}`[0] = c0;
    `\refvar[BVHBuildNode::children]{children}{}`[1] = c1;
    `\refvar[BVHBuildNode::bounds]{bounds}{}` = `\refvar[Union2]{Union}{}`(c0->`\refvar[BVHBuildNode::bounds]{bounds}{}`, c1->`\refvar[BVHBuildNode::bounds]{bounds}{}`);
    `\refvar[BVHBuildNode::splitAxis]{splitAxis}{}` = axis;
    `\refvar[BVHBuildNode:nPrimitives]{nPrimitives}{}` = 0;
}
\end{lstlisting}

除了用于分配节点和\refvar{BVHPrimitiveInfo}{}结构体数组的\refvar{MemoryArena}{}外,\linebreak
\refvar{recursiveBuild}{()}接收范围参数{\ttfamily[start,end)}。
它负责为从{\ttfamily primitiveInfo [start]}直到并包括{\ttfamily primitiveInfo[end-1]}的
范围表示的图元子集返回一个BVH。
如果该范围只含有单个图元,则递归触底并创建一个叶子节点。
否则,该方法用划分算法之一来划分数组该范围内的元素并相应地重新排列它们,
这样范围{\ttfamily[start,mid)}和{\ttfamily[mid,end)}表示分开的子集。
如果划分成功,则这两个图元集合又传入将会为当前节点的两个孩子返回节点指针的递归调用。

{\ttfamily totalNodes}跟踪已创建的BVH节点总数;
利用该数目使得之后可以分配数目恰好正确的更紧实的\refvar{LinearBVHNode}{}。
最终,数组{\ttfamily orderedPrims}用于保存图元引用就像图元存于树的叶子节点一样。
该数组初始化为空;当创建一个叶子节点时,\refvar{recursiveBuild}{()}把
与之重合的图元添加到数列末尾,让叶子节点可以只存储对该数组的偏移量以及
表示与之重合的图元集的图元数量。
回想当完成树构建时,用这里创建的有序图元数组代替\refvar{BVHAccel::primitives}{}。
\begin{lstlisting}
`\refcode{BVHAccel Method Definitions}{+=}\lastnext{BVHAccelMethodDefinitions}`
`\refvar{BVHBuildNode}{}` *`\refvar{BVHAccel}{}`::`\initvar{recursiveBuild}{}`(`\refvar{MemoryArena}{}` &arena,
        std::vector<`\refvar{BVHPrimitiveInfo}{}`> &primitiveInfo, int start,
        int end, int *totalNodes,
        std::vector<std::shared_ptr<`\refvar{Primitive}{}`>> &orderedPrims) {
    `\refvar{BVHBuildNode}{}` *node = arena.`\refvar[MemoryArena:Alloc2]{Alloc}{}`<`\refvar{BVHBuildNode}{}`>();
    (*totalNodes)++;
    `\refcode{Compute bounds of all primitives in BVH node}{}`
    int nPrimitives = end - start;
    if (nPrimitives == 1) {
        `\refcode{Create leaf BVHBuildNode}{}`
    } else {
        `\refcode{Compute bound of primitive centroids, choose split dimension dim}{}`
        `\refcode{Partition primitives into two sets and build children}{}`
    }
    return node;
}
\end{lstlisting}
\begin{lstlisting}
`\initcode{Compute bounds of all primitives in BVH node}{=}`
`\refvar{Bounds3f}{}` bounds;
for (int i = start; i < end; ++i)
    bounds = `\refvar[Union2]{Union}{}`(bounds, primitiveInfo[i].`\refvar[BVHPrimitiveInfo::bounds]{bounds}{}`);
\end{lstlisting}

在叶子节点处,与该叶子重合的图元被添到{\ttfamily orderedPrims}数组末尾并初始化一个叶子节点对象。
\begin{lstlisting}
`\initcode{Create leaf BVHBuildNode}{=}`
int firstPrimOffset = orderedPrims.size();
for (int i = start; i < end; ++i) {
    int primNum = primitiveInfo[i].`\refvar{primitiveNumber}{}`;
    orderedPrims.push_back(`\refvar[BVHAccel::primitives]{primitives}{}`[primNum]);
}
node->`\refvar[BVHBuildNode::InitLeaf]{InitLeaf}{}`(firstPrimOffset, nPrimitives, bounds);
return node;
\end{lstlisting}

对于内部节点,一组图元必须在两个子树之间划分。
给定$n$个图元,有$2^{n-1}-1$种\sidenote{译者注:原文误写为$2(n-1)-2$,已修正。}
可能的方法将它们划分到两个非空组。
实际中构建BVH时,一般考虑沿一个坐标轴划分,这意味着大约有$3n$个候选划分
(沿每个轴方向,每个图元可能放到第一分区或第二分区)。

这里我们就选择三个坐标轴的一个用来划分图元。
当为当前图元集合投影边界框形心时,我们选择有最大范围的轴
(另一种是尝试所有三个轴并选择给出最好结果的那个,但实际中本方法更好)。
该方法在许多场景下给出良好划分;\reffig{4.3}说明了该策略。
\begin{figure}[htbp]
    \centering\input{Pictures/chap04/BVHchoosesplitaxis.tex}
    \caption{选择沿哪个轴划分图元。\protect\refvar{BVHAccel}{}基于
        图元边界框形心在哪个轴有最大范围来选择划分图元所沿的轴。
        这里在二维中,它们沿$y$轴的范围最大(轴上的实心点),所以图元会在$y$上划分。}
    \label{fig:4.3}
\end{figure}

这里划分的一般目标是选择图元的一个划分使得
得到的两个图元集合的边界框没有太多重合——
如果有大量重合则在遍历树时它需要更频繁地遍历两个子树,
比起本应可以更高效剪除一些图元它需要更多计算量。
待会儿在讨论表面积启发法时会更严谨地表述该求取高效图元划分的思想。
\begin{lstlisting}
`\initcode{Compute bound of primitive centroids, choose split dimension dim}{=}`
`\refvar{Bounds3f}{}` centroidBounds;
for (int i = start; i < end; ++i)
    centroidBounds = `\refvar[Union2]{Union}{}`(centroidBounds, primitiveInfo[i].`\refvar{centroid}{}`);
int dim = centroidBounds.`\refvar{MaximumExtent}{}`();
\end{lstlisting}

如果所有形心点都在同一位置(即形心边界为零体积),
则递归停止并用该图元创建一个叶子结点;
这里没有划分方法能对那种(非常)情况有效。
否则用选择的方法划分图元并传入两个对\refvar{recursiveBuild}{()}的递归调用。
\begin{lstlisting}
`\initcode{Partition primitives into two sets and build children}{=}`
int mid = (start + end) / 2;
if (centroidBounds.`\refvar{pMax}{}`[dim] == centroidBounds.`\refvar{pMin}{}`[dim]) {
    `\refcode{Create leaf BVHBuildNode}{}`
} else {
    `\refcode{Partition primitives based on splitMethod}{}`
    node->`\refvar[BVHBuildNode::InitInterior]{InitInterior}{}`(dim,
                       `\refvar{recursiveBuild}{}`(arena, primitiveInfo, start, mid,
                                      totalNodes, orderedPrims),
                       `\refvar{recursiveBuild}{}`(arena, primitiveInfo, mid, end,
                                      totalNodes, orderedPrims));
}
\end{lstlisting}

代码片\refcode{Partition primitives based on splitMethod}{}只是用
\refvar[splitMethod]{BVHAccel::splitMethod}{}
的值决定该用哪个图元划分方案。接下来的几页将介绍这三个方案。
\begin{lstlisting}
`\initcode{Partition primitives based on splitMethod}{=}`
switch (`\refvar{splitMethod}{}`) {
case `\refvar{SplitMethod}{}`::`\refvar{Middle}{}`: {
    `\refcode{Partition primitives through node's midpoint}{}`
}
case `\refvar{SplitMethod}{}`::`\refvar{EqualCounts}{}`: {
    `\refcode{Partition primitives into equally sized subsets}{}`
    break;
}
case `\refvar{SplitMethod}{}`::`\refvar{SAH}{}`:
default: {
    `\refcode{Partition primitives using approximate SAH}{}`
    break;
}
}
\end{lstlisting}

\refvar{Middle}{}是一个简单的\refvar{SplitMethod}{},
它首先计算图元形心沿划分轴的中点。
该方法在代码片\refcode{Partition primitives through node's midpoint}{}中实现。
图元按其形心在中点之上还是之下分为两个集合。
该划分用C++标准库函数{\ttfamily std::partition()}很容易完成,
它接收数组中的一系列元素和比较函数,并对数组元素排序使得
对于判定函数而言所有返回{\ttfamily true}的元素都出现在返回{\ttfamily false}的范围之前
\footnote{在调用{\ttfamily std::partition()}时,
注意数组{\ttfamily primitiveInfo}索引的特殊表达式即{\ttfamily \&primitiveInfo[end-1]+1}。
这样写代码有些晦涩的理由。在C和C++程序语言中,
计算数组末尾后下一个元素的指针是合法的,
这样遍历数组元素能持续到当前指针等于末端点。
为此,我们这里想就写成表达式{\ttfamily \&primitiveInfo[end]}。
然而{\ttfamily primitiveInfo}分配为C++的{\ttfamily vector};
一些{\ttfamily vector}实现在传给其{\ttfamily []}操作符
的偏移量是在数组末端之后时会报运行时错误。
因为我们不会尝试引用数组末端后下一个元素的值而只是想计算其地址,
所以该操作事实上是安全的。
因此我们最终用这里的表达式计算同一地址,并且也满足任何{\ttfamily vector}错误检查。}。
{\ttfamily std::partition()}返回指向第一个对于判定有{\ttfamily false}值的元素的指针,
它转化为对数组{\ttfamily primitiveInfo}的偏移量,这样我们就可以将其传入递归调用。
\reffig{4.4}说明了该方法,包括其有效和无效的情况。

如果图元都有巨大的重叠边界框,则该划分方法可能无法把图元分为两组。
这种情况下,执行往下进入{\ttfamily \refvar{SplitMethod}{}::\refvar{EqualCounts}{}}方法再试一次。
\begin{lstlisting}
`\initcode{Partition primitives through node's midpoint}{=}`
`\refvar{Float}{}` pmid = (centroidBounds.`\refvar{pMin}{}`[dim] + centroidBounds.`\refvar{pMax}{}`[dim]) / 2;
`\refvar{BVHPrimitiveInfo}{}` *midPtr =
    std::partition(&primitiveInfo[start], &primitiveInfo[end-1]+1,
        [dim, pmid](const `\refvar{BVHPrimitiveInfo}{}` &pi) {
            return pi.`\refvar{centroid}{}`[dim] < pmid;
        });
mid = midPtr - &primitiveInfo[0];
if (mid != start && mid != end)
    break;
\end{lstlisting}

当\refvar{splitMethod}{}是{\ttfamily\refvar{SplitMethod}{}::\refvar{EqualCounts}{}}时,
则运行代码片\refcode{Partition primitives into equally sized subsets}{}。
它把图元划分为两个数量相等的子集使得$n$个中前一半的$\displaystyle\frac{n}{2}$个
沿所选轴的形心坐标最小,另一半的则有最大形心坐标值。
尽管该方法有时能起效,但也有\reffig{4.4}(b)效果不好的情况。
\begin{figure}[htb]
    \centering\input{Pictures/chap04/Midpointgoodbadsplit.tex}
    \caption{一轴上基于形心中点划分图元。(a)对于一些图元分布,
        例如这里所示,沿所选轴(粗蓝线)基于形心中点的划分效果很好。
        (b)对于像这个的分布,中点是次优选项;所得两个边界框大量重叠。
        (c)若来自(b)的同一组图元换为用这里展示的线分开,所得边界框
        更小且根本不会重叠,使渲染时性能更好。}
    \label{fig:4.4}
\end{figure}

该方案也易于调用标准库的{\ttfamily std::nth\_element()}实现。
它接收起点、中点和终点指针以及一个比较函数。
它对数组排序使得中点指针处元素的位置是,如果数组完全排序,
则所有中点之前的元素都比中点元素小且所有后面的元素都比它大。
对于$n$个元素该排序可以在$O(n)$时间内完成,
比完全排序数组的$O(n\log{n})$更高效。
\begin{lstlisting}
`\initcode{Partition primitives into equally sized subsets}{}`
mid = (start + end) / 2;
std::nth_element(&primitiveInfo[start], &primitiveInfo[mid], 
                 &primitiveInfo[end-1]+1,
    [dim](const BVHPrimitiveInfo &a, const BVHPrimitiveInfo &b) { 
        return a.`\refvar{centroid}{}`[dim] < b.`\refvar{centroid}{}`[dim];
    });
\end{lstlisting}

\subsection{表面积启发法}\label{sub:表面积启发法}
上述两个图元划分方法对一些图元分布效果不错,
但实际中它们经常选择性能较差的划分,导致光线要访问树的更多节点,
因此带来渲染时不必要的低效光线-图元相交计算。
当下光线追踪大部分最好的构建加速结构算法都基于“\keyindex{表面积启发法}{surface area heuristic}{}”(SAH),
它提供了全面的开销模型来回答问题,例如
“大量图元划分中哪个会为光线-图元相交测试带来更好的BVH?”,
或者“在空间划分方案里大量划分空间的可选位置中哪个会带来更好的加速结构?”

SAH模型估计执行光线相交测试的开销,包括穿行树的节点花的时间和
为特定的图元划分进行光线-图元相交测试花的时间。
然后构建加速结构的算法可以遵循最小化总开销的目标。
通常用贪婪算法独立地为正在构建的每个层次节点最小化开销。

SAH开销模型背后的思想很简单:构建自适应加速结构(图元划分或空间划分)的任意点处,
我们可以为当前区域和几何体创建一个叶子结点。
这种情况下,任何穿过该区域的光线都要对所有重合的图元测试,
且带来的开销为
\begin{align*}
    \sum\limits_{i=1}^{N}{t_{\mathrm{isect}}(i)}\, ,
\end{align*}
其中$N$是图元数量,$t_{\mathrm{isect}}(i)$是对第$i$个图元计算光线-物体相交的时间。

另一选项是划分该空间。这种情况下,光线会带来开销
\begin{align}\label{eq:4.1}
    c(A,B)=t_{\mathrm{trav}}+p_A\sum\limits_{i=1}^{N_A}{t_\mathrm{isect}(a_i)}+p_B\sum\limits_{i=1}^{N_B}{t_{\mathrm{isect}}(b_i)}\, ,
\end{align}
其中$t_{\mathrm{trav}}$是穿行内部节点并确定光线穿过哪个子树所花的时间,分别地,
$p_A$和$p_B$是光线穿过每个子节点(假设二分划分)的概率,
$a_i$和$b_i$是两个子节点中图元的索引,
$N_A$和$N_B$是与两个子节点区域重合的图元数量。
怎样划分图元的选项会影响两个概率值以及划分出的两边的图元集合。

pbrt中,我们将作出简化假设即所有图元的$t_{\mathrm{trav}}(i)$都相同;
该假设可能和实际差不了多少,且它引入的任何误差看起来并不太影响加速器的性能。
另一种可能是向\refvar{Primitive}{}添加一个方法返回估计的相交测试所需的CPU周期数。

概率$p_A$和$p_B$可用来自几何概型的思想计算。
可以证明当凸体$A$包含于另一凸体$B$中,
穿过$B$的均匀分布的随机光线也穿过$A$的条件概率是它们表面积$s_A$与$s_B$的比:
\begin{align*}
    \displaystyle p(A|B)=\frac{s_A}{s_B}\, .
\end{align*}

因为我们对光线穿过节点的开销感兴趣,我们可以直接利用该结果。
因此,如果我们考虑细化一空间区域$A$使得有两个新子区域边界为$B$和$C$(\reffig{4.5}),
则穿过$A$的光线也会穿过两个子区域之一的概率很容易计算。
\begin{figure}[htbp]
    \centering\input{Pictures/chap04/Surfaceareasplit.tex}
    \caption{如果边界层次一个表面积为$s_A$的节点分为两个表面积为$s_B$和$s_C$的孩子,
        则穿过$A$的光线也穿过$B$和$C$的概率分别为$\displaystyle\frac{s_B}{s_A}$和$\displaystyle\frac{s_C}{s_A}$.}
    \label{fig:4.5}
\end{figure}

当\refvar{splitMethod}{}值为{\ttfamily\refvar{SplitMethod}{}::\refvar{SAH}{}}时,SAH用于构建BVH;
通过考虑大量候选划分来寻找沿所选轴给出最小SAH估计开销的图元划分
(这是默认\refvar{SplitMethod}{},且它为渲染创建最高效的树)。
然而,一旦它细分少量图元,则实现切换为划分成等量的子集。
这时应用SAH所增加的计算量是不划算的。
\begin{lstlisting}
`\initcode{Partition primitives using approximate SAH}{=}`
if (nPrimitives <= 4) {
    `\refcode{Partition primitives into equally sized subsets}{}`
} else {
    `\refcode{Allocate BucketInfo for SAH partition buckets}{}`
    `\refcode{Initialize BucketInfo for SAH partition buckets}{}`
    `\refcode{Compute costs for splitting after each bucket}{}`
    `\refcode{Find bucket to split at that minimizes SAH metric}{}`
    `\refcode{Either create leaf or split primitives at selected SAH bucket}{}`
}
\end{lstlisting}

比起穷举沿该轴所有$2^n$个可能的划分
\sidenote{译者注:原文写作$2n$,笔者认为是笔误,已修正。},
这里的实现是将沿该轴的范围分为少量相等的较大范围,并为每个计算SAH以选择最好的。
然后它只考虑在该范围边界内的划分。
该方法比考虑所有划分更高效且通常仍产出几乎一样高效的划分。
\reffig{4.6}说明了该思想。
\begin{figure}[htbp]
    \centering\input{Pictures/chap04/BVHsplitbucketing.tex}
    \caption{用表面积启发法为BVH选择划分平面。图元边界形心的投影范围被投影在选定的划分轴上。
        每个图元都基于其边界形心被放在沿轴的一个桶中。
        然后实现会估计沿每个桶的边界(粗蓝线)的平面划分图元的开销;
        谁的表面积启发法开销最小就选谁。}
    \label{fig:4.6}
\end{figure}
\begin{lstlisting}
`\initcode{Allocate BucketInfo for SAH partition buckets}{=}`
constexpr int nBuckets = 12;
struct `\initvar{BucketInfo}{}` {
    int `\initvar[BucketInfo::count]{count}{}` = 0;
    `\refvar{Bounds3f}{}` `\initvar[BucketInfo::bounds]{bounds}{}`;
};
`\refvar{BucketInfo}{}` buckets[nBuckets];
\end{lstlisting}

对于该范围内的每个图元,我们确定包含其形心的桶\protect\sidenote{译者注:原文bucket。}并更新桶的边界以包含图元的边界。
\begin{lstlisting}
`\initcode{Initialize BucketInfo for SAH partition buckets}{=}`
for (int i = start; i < end; ++i) {
    int b = nBuckets * 
        centroidBounds.`\refvar{Offset}{}`(primitiveInfo[i].`\refvar{centroid}{}`)[dim];
    if (b == nBuckets) b = nBuckets - 1;
    buckets[b].`\refvar[BucketInfo::count]{count}{}`++;
    buckets[b].`\refvar[BucketInfo::bounds]{bounds}{}` = `\refvar[Union2]{Union}{}`(buckets[b].`\refvar[BucketInfo::bounds]{bounds}{}`, primitiveInfo[i].`\refvar[BVHPrimitiveInfo::bounds]{bounds}{}`);
}
\end{lstlisting}

对于每个桶,我们现在都有图元数量以及全部相应边界框的边界。
我们想用SAH估计在每个桶边界处作划分的开销。
下面的代码片遍历所有桶并初始化数组{\ttfamily cost[i]}来
保存估计的在第{\ttfamily i}个桶后划分的SAH开销
(不考虑在最后一个桶之后划分,因为根据定义它并不划分图元)。

我们任意设置估计的相交开销为1,然后设置估计的遍历开销
为$\displaystyle\frac{1}{8}$(两者之一总是可以设为1,
因为是估计的遍历和相交开销的相对量级而不是绝对量级决定其影响)。
尽管遍历节点即光线-边界框相交的绝对计算量仅稍低于光线与形状相交所需的计算量,
但pbrt中光线-图元相交测试经过了两次虚函数调用,增加了大量开销,
所以这里我们估计其开销大于光线-框相交的八倍。

基于对桶前向和后向扫描而增量式地计算、存储和计数边界的线性时间实现是可能的,
但这里的计算关于桶的数量有$O(n^2)$复杂度。
更加高度优化解决该低效问题的渲染器是值得的,
但这里对于小的$n$,性能影响通常是可接受的。
\begin{lstlisting}
`\initcode{Compute costs for splitting after each bucket}{=}`
`\refvar{Float}{}` cost[nBuckets - 1];
for (int i = 0; i < nBuckets - 1; ++i) {
    `\refvar{Bounds3f}{}` b0, b1;
    int count0 = 0, count1 = 0;
    for (int j = 0; j <= i; ++j) {
        b0 = `\refvar[Union2]{Union}{}`(b0, buckets[j].`\refvar[BucketInfo::bounds]{bounds}{}`);
        count0 += buckets[j].`\refvar[BucketInfo::count]{count}{}`;
    }
    for (int j = i+1; j < nBuckets; ++j) {
        b1 = `\refvar[Union2]{Union}{}`(b1, buckets[j].`\refvar[BucketInfo::bounds]{bounds}{}`);
        count1 += buckets[j].`\refvar[BucketInfo::count]{count}{}`;
    }
    cost[i] = .125f + (count0 * b0.`\refvar{SurfaceArea}{}`() +
                       count1 * b1.`\refvar{SurfaceArea}{}`()) / bounds.`\refvar{SurfaceArea}{}`();
}
\end{lstlisting}

有了所有开销,对数组{\ttfamily cost}的线性扫描找到最小开销的划分。
\begin{lstlisting}
`\initcode{Find bucket to split at that minimizes SAH metric}{=}`
`\refvar{Float}{}` minCost = cost[0];
int minCostSplitBucket = 0;
for (int i = 1; i < nBuckets - 1; ++i) {
    if (cost[i] < minCost) {
        minCost = cost[i];
        minCostSplitBucket = i;
    }
}
\end{lstlisting}

如果为划分选的桶边界有比用存在的图元构建节点更低的估计开销,
或者出现一个节点的图元超过了允许的最大数量,
则用函数{\ttfamily std::partition()}来完成
在数组{\ttfamily primitiveInfo}中记录节点的工作。
回想之前它的用法即该函数确保数组的所有对于给定判定函数返回{\ttfamily true}的元素
都出现在返回{\ttfamily false}的之前,
并且它返回指向第一个判定函数返回{\ttfamily false}的元素的指针。
因为我们之前任意设置估计的相交开销为1,
所以只是创建叶子结点的估计开销等于图元的数量{\ttfamily nPrimitives}。
\begin{lstlisting}
`\initcode{Either create leaf or split primitives at selected SAH bucket}{=}`
`\refvar{Float}{}` leafCost = nPrimitives;
if (nPrimitives > maxPrimsInNode || minCost < leafCost) {
    `\refvar{BVHPrimitiveInfo}{}` *pmid = std::partition(&primitiveInfo[start],
        &primitiveInfo[end-1]+1, 
        [=](const `\refvar{BVHPrimitiveInfo}{}` &pi) {
            int b = nBuckets * centroidBounds.`\refvar{Offset}{}`(pi.centroid)[dim];
            if (b == nBuckets) b = nBuckets - 1;
            return b <= minCostSplitBucket;
        });
    mid = pmid - &primitiveInfo[0];
} else {
    `\refcode{Create leaf BVHBuildNode}{}`
}
\end{lstlisting}

\subsection{线性包围盒层次}\label{sub:线性包围盒层次}
尽管用表面积启发法构建包围盒层次给出了很好的结果,
但该方法有两个缺点:第一,接收了许多传入的场景图元来在树的所有层次上计算SAH开销。
第二,自顶向下的BVH构建难以很好地并行化:
最明显的并行化方法——执行独立子树的并行化构建——
在直到该树顶部几层构建好前都受困于有限的独立任务,
这反过来又抑制了并行的可扩展性
(第二个问题在GPU上尤为突出,如果大规模并行化不可用则性能会很差)。

开发\keyindex{线性包围盒层次}{linear bounding volume hierarchy}{}(LBVH)来解决这些问题。
通过LBVH,以传递轻量级小次数的图元来构建树;
树的构建时间与图元数量呈线性关系。
而且算法快速地把图元划分为可以独立处理的\keyindex{群集}{cluster}{}。
该过程很容易并行化且很适合GPU实现。

LBVH背后的关键思想是把BVH构建变为排序问题。
因为没有排序多维数据的单一顺序函数,所以LBVH是基于\keyindex{莫顿码}{Morton code}{}的,
它将$n$维中的邻近点映射为1D直线上的邻近点,使之有明显的排序函数。
在图元排序后,空间上相邻的图元群集在排序数组的连续段内。

莫顿码基于简单的变换:给定$n$维整数坐标值,
其莫顿码表示由交错二进制坐标数位求得。
例如,考虑一个2D坐标$(x,y)$,其中$x$和$y$的数位表示为$x_i$和$y_i$.
相应的莫顿码值为
\begin{align*}
    \cdots y_3x_3y_2x_2y_1x_1y_0x_0\, .
\end{align*}

\reffig{4.7}展示了2D点按莫顿顺序的图示——
注意沿递归的“z”形路径访问它们
(因此莫顿路径有时也称为“z序”)。
我们可以看到2D中坐标相近的点通常在莫顿曲线上也是相近的
\footnote{许多GPU用莫顿布局在内存中存储纹理贴图。
    这样做的一个优点是当在四个纹素值之间执行双线性插值时,
    比起纹理按扫描线顺序排布,这些值更有极大可能在内存中相邻。
    反过来,这有利于纹理缓存性能。}。
\begin{figure}[htbp]
    \centering\input{Pictures/chap04/MortonBasic.tex}
    \caption{沿莫顿曲线访问点的顺序。沿$x$和$y$轴的坐标值用二进制表示。
        如果我们按其莫顿索引的顺序连接整数坐标点,我们可以看见莫顿曲线沿分层级的“z”形路径访问这些点。}
    \label{fig:4.7}
\end{figure}

莫顿编码值也编码了关于点所表示的位置的有用信息。
考虑2D中4位坐标值的情况:$x$和$y$坐标是$[0,15]$内的整数且
莫顿码有8位:$y_3x_3y_2x_2y_1x_1y_0x_0$.
编码会得出许多整数性质;一些例子包括:
\begin{itemize}
    \item 对于莫顿编码8位值中高位$y_3$置位的,我们就知道
          其本身的$y$坐标高位置位且因此$y\ge8$(\reffig{4.8}(a))。
    \item 下一位值$x_3$在中间划分$x$轴(\reffig{4.8}(b))。
          例如如果$y_3$置位而$x_3$没有,则相应的点一定位于\reffig{4.8}(c)的阴影区。
          具有许多高位匹配的点通常位于由匹配位值决定的幂2边长的对齐空间区域。
    \item $y_2$值将$y$轴划分为四个区域(\reffig{4.8}(d))。
\end{itemize}
\begin{figure}[htb]
    \centering\input{Pictures/chap04/MortonBinary.tex}\\
    \input{Pictures/chap04/MortonSplitHoriz.tex}
    \input{Pictures/chap04/MortonSplitVert.tex}
    \input{Pictures/chap04/MortonShadedRegion.tex}\\
    \input{Pictures/chap04/MortonSplitHoriz3.tex}
    \caption{莫顿编码的实现。莫顿值中不同位的值表示相应坐标所在的空间区域。
        (a)在2D中,一个点坐标的莫顿编码值高位定义了沿$y$轴中点的划分平面。
        如果该高位置位,则点在平面以上。(b)同样,莫顿值的第二高位在中间划分$x$轴。
        (c)如果高$y$位是1且高$x$位是0,则该点一定在阴影区域内。
        (d)第二高$y$位把$y$轴划分为四个区域。}
    \label{fig:4.8}
\end{figure}

另一个解释这些基于数位的性质的方法是按莫顿编码值来。
例如,\reffig{4.8}(a)对应在范围$[8,15]$内的索引,
\reffig{4.8}(c)对应$[8,11]$.因此,给定一组排序了的莫顿索引,
我们可以通过二分搜索找到对应于像\reffig{4.8}(c)区域那样的点的范围,
以在数组中找到每个终点。

LBVH是通过用每个空间区域中点处的划分平面
(即等价于之前定义的路径\refvar[Middle]{SplitMethod::Middle}{})划分图元而构建的BVH。
因为它基于上述的莫顿编码性质,所以划分非常高效。

仅仅以不同方式复现\refvar{Middle}{}并不有趣,所以这里的实现中我们将
构建\keyindex{分层线性包围盒层次}{hierarchical linear bounding volume hierarchy}{linear bounding volume hierarchy线性包围盒层次}(HLBVH)。
通过该方法首先用基于莫顿曲线的聚类为低层级构建树(以下称“小树”\sidenote{译者注:原文treelet。}),
然后用表面积启发法创建高层级树。
方法\refvar{HLBVHBuild}{()}实现了该方法并返回所得树的根节点。
\begin{lstlisting}
`\refcode{BVHAccel Method Definitions}{+=}\lastnext{BVHAccelMethodDefinitions}`
`\refvar{BVHBuildNode}{}` *`\refvar{BVHAccel}{}`::`\initvar{HLBVHBuild}{}`(`\refvar{MemoryArena}{}` &arena, 
        const std::vector<`\refvar{BVHPrimitiveInfo}{}`> &primitiveInfo,
        int *totalNodes,
        std::vector<std::shared_ptr<`\refvar{Primitive}{}`>> &orderedPrims) const {
    `\refcode{Compute bounding box of all primitive centroids}{}`
    `\refcode{Compute Morton indices of primitives}{}`
    `\refcode{Radix sort primitive Morton indices}{}`
    `\refcode{Create LBVH treelets at bottom of BVH}{}`
    `\refcode{Create and return SAH BVH from LBVH treelets}{}`
}
\end{lstlisting}

只用图元边界框形心构建BVH来对其排序——
它不考虑每个图元的实际空间范围。
该简化对HLBVH提供的性能很关键,
但它也意味着对于一些占有很大尺寸范围的场景而言,
构建的树不会像基于SAH的树那样考虑这些特例。

因为莫顿编码在整数坐标上操作,我们首先需要定界所有图元的形心,
这样我们就可以量化相对于整体边界的形心位置。
\begin{lstlisting}
`\initcode{Compute bounding box of all primitive centroids}{=}`
`\refvar{Bounds3f}{}` bounds;
for (const `\refvar{BVHPrimitiveInfo}{}` &pi : primitiveInfo)
    bounds = `\refvar[Union2]{Union}{}`(bounds, pi.`\refvar{centroid}{}`);
\end{lstlisting}

有了整体边界,我们现在可以为每个图元计算莫顿码了。
这是非常轻量的计算,但假定可能有数百万图元,则它是值得并行化的。
注意下面把512的循环块尺寸传入\refvar{ParallelFor}{()};
这让工作线程每次要处理512组图元而不是默认的一组。
因为每个图元计算莫顿码要执行的工作量相对较小,
这粒度更好地把分布式任务的开销分摊给了工作线程。
\begin{lstlisting}
`\initcode{Compute Morton indices of primitives}{=}`
std::vector<`\refvar{MortonPrimitive}{}`> mortonPrims(primitiveInfo.size());
`\refvar{ParallelFor}{}`(
    [&](int i) {
        `\refcode{Initialize mortonPrims[i] for ith primitive}{}`
    }, primitiveInfo.size(), 512);
\end{lstlisting}

为每个图元都创建一个实例\refvar{MortonPrimitive}{};
它保存了图元在数组\newline{\ttfamily primitiveInfo}中的图元索引及其莫顿码。
\begin{lstlisting}
`\refcode{BVHAccel Local Declarations}{+=}\lastnext{BVHAccelLocalDeclarations}`
struct `\initvar{MortonPrimitive}{}` {
    int `\initvar{primitiveIndex}{}`;
    uint32_t `\initvar{mortonCode}{}`;
};
\end{lstlisting}

我们为每个$x$、$y$和$z$坐标用10位数,这样莫顿码一共30位。
该粒度允许值与单个32位变量相容。
边界框内的浮点形心偏移量在$[0,1]$内,
所以我们用$2^{10}$缩放它们得到10位表示的整数坐标
(对于恰等于1的边界情况,可能得到出界的量化值1024;
这种情况在即将介绍的函数\refvar{LeftShift3}{()}中处理)。
\begin{lstlisting}
`\initcode{Initialize mortonPrims[i] for ith primitive}{=}`
constexpr int mortonBits = 10;
constexpr int mortonScale = 1 << mortonBits;
mortonPrims[i].`\refvar{primitiveIndex}{}` = primitiveInfo[i].`\refvar{primitiveNumber}{}`;
`\refvar{Vector3f}{}` centroidOffset = bounds.`\refvar{Offset}{}`(primitiveInfo[i].`\refvar{centroid}{}`);
mortonPrims[i].`\refvar{mortonCode}{}` = `\refvar{EncodeMorton3}{}`(centroidOffset * mortonScale);
\end{lstlisting}

为计算3D莫顿码,首先我们定义一个辅助函数:\refvar{LeftShift3}{()}接收
一个32位值并返回把第$i$位移到第$3i$位的结果,剩下位为零。
\reffig{4.9}说明了该操作。
\begin{figure}[htbp]
    \centering\includegraphics[width=0.75\linewidth]{chap04/LeftShift3.eps}
    \caption{移位以计算3D莫顿码。函数\refvar{LeftShift3}{()}接收32位整数值,
        且对于低10位,把第$i$位移到第$3i$位的位置——换句话说,向左移$2i$位。剩下全部位设为零。}
    \label{fig:4.9}
\end{figure}

实现该操作最明显的方法,即单独移动每一位,并不是最高效的
(它需要总共9次位移以及逻辑或来计算最终值)。
取而代之的是,我们可以把每位的移动分解为多个幂2尺寸的移位且一并把数位值移到其最终位置。
然后,所有需要移动给定幂2数位的位可以一并移动。
函数\refvar{LeftShift3}{()}实现了该计算,\reffig{4.10}展示了它如何工作的
\sidenote{译者注:原图第4行“9”的位置有误,此处已修正。}。
\begin{figure}[htbp]
    \centering\includegraphics[width=0.75\linewidth]{Pictures/chap04/Mortonpow2decomposition.eps}
    \caption{莫顿移位的幂2分解。通过一系列幂2尺寸的移动来为每个3D坐标计算莫顿码执行移位。
        首先,位8和9向左移16位,这把位8放在了其最终位置。接着位4到7移动8位。
        移动4和2位后(适当掩模使每位最终都移动适当数位),所有数位都在正确的位置上。
        该计算由函数\refvar{LeftShift3}{()}实现。}
    \label{fig:4.10}
\end{figure}

\begin{lstlisting}
`\initcode{BVHAccel Utility Functions}{=}\initnext{BVHAccelUtilityFunctions}`
inline uint32_t `\initvar{LeftShift3}{}`(uint32_t x) {
    if (x == (1 << 10)) --x;
    x = (x | (x << 16)) & 0b00000011000000000000000011111111;
    x = (x | (x <<  8)) & 0b00000011000000001111000000001111;
    x = (x | (x <<  4)) & 0b00000011000011000011000011000011;
    x = (x | (x <<  2)) & 0b00001001001001001001001001001001;
    return x;
}
\end{lstlisting}

\refvar{EncodeMorton3}{()}接收每个分量都是$0$到$2^{10}$之间浮点值的3D坐标值。
它把这些值转换为整数然后通过展开三个10位量化值使第$i$位在位置$3i$上来计算莫顿码,
然后$y$位再移一位,$z$位再移两位,全部结果求或(\reffig{4.11})。
\begin{figure}[htbp]
    \centering\includegraphics[width=0.75\linewidth]{chap04/Mortonxyzinterleave.eps}
    \caption{最终交错坐标值。有了\refvar{LeftShift3}{()}为$x$、$y$和$z$计算的交错值,
        最终莫顿编码值通过分别移动$y$和$z$一位和两位然后全部结果求或算得。}
    \label{fig:4.11}
\end{figure}

\begin{lstlisting}
`\refcode{BVHAccel Utility Functions}{+=}\lastnext{BVHAccelUtilityFunctions}`
inline uint32_t `\initvar{EncodeMorton3}{}`(const `\refvar{Vector3f}{}` &v) {
    return (`\refvar{LeftShift3}{}`(v.z) << 2) | (`\refvar{LeftShift3}{}`(v.y) << 1) |
            `\refvar{LeftShift3}{}`(v.x);
}
\end{lstlisting}

一旦计算了莫顿索引,我们将用\keyindex{基数排序}{radix sort}{}按莫顿索引值对
\newline{\ttfamily mortonPrims}排序。
我们已经发现对于BVH的构建,我们的基数排序实现明显快于使用我们系统标准库的{\ttfamily std::sort()}
(它是\keyindex{快速排序}{quicksort}{}和\keyindex{插入排序}{insertion sort}{}的混合)。
\begin{lstlisting}
`\initcode{Radix sort primitive Morton indices}{=}`
`\refvar{RadixSort}{}`(&mortonPrims);
\end{lstlisting}

回想基数排序和大多数排序算法的不同在于它不是基于比较一对值
而是基于依赖一些键值的桶子项。基数排序可用于排序整数值,
它从最右边数位到最左边每次排序一个数码。
它尤其值得每次对二进制值排序多个数码;
这样做减少了传递数据的次数。
这里的实现中,{\ttfamily bitsPerPass}设置每次传递处理的位数;
取值6后,我们有5次传递来排序30位。
\begin{lstlisting}
`\refcode{BVHAccel Utility Functions}{+=}\lastcode{BVHAccelUtilityFunctions}`
static void `\initvar{RadixSort}{}`(std::vector<`\refvar{MortonPrimitive}{}`> *v) {
    std::vector<`\refvar{MortonPrimitive}{}`> tempVector(v->size());
    constexpr int bitsPerPass = 6;
    constexpr int nBits = 30;
    constexpr int nPasses = nBits / bitsPerPass;
    for (int pass = 0; pass < nPasses; ++pass) {
        `\refcode{Perform one pass of radix sort, sorting bitsPerPass bits}{}`
    }
    `\refcode{Copy final result from tempVector, if needed}{}`
}
\end{lstlisting}

当前传递会排序{\ttfamily bitsPerPass}位,从{\ttfamily lowBit}开始。
\begin{lstlisting}
`\initcode{Perform one pass of radix sort, sorting bitsPerPass bits}{=}`
int lowBit = pass * bitsPerPass;
`\refcode{Set in and out vector pointers for radix sort pass}{}`
`\refcode{Count number of zero bits in array for current radix sort bit}{}`
`\refcode{Compute starting index in output array for each bucket}{}`
`\refcode{Store sorted values in output array}{}`
\end{lstlisting}

{\ttfamily in}和{\ttfamily out}引用分别对应于要排序的向量以及保存排序值的向量。
每次通过循环的传递都在输入向量{\ttfamily *v}和临时向量之间交替。
\begin{lstlisting}
`\initcode{Set in and out vector pointers for radix sort pass}{=}`
std::vector<`\refvar{MortonPrimitive}{}`> &in  = (pass & 1) ? tempVector : *v;
std::vector<`\refvar{MortonPrimitive}{}`> &out = (pass & 1) ? *v : tempVector;
\end{lstlisting}

如果我们每次传递中排序$n$位,则每个值可能落入的桶有$2^n$个。
我们先计数每个桶中会落入多少个值;这能让我们确定在输出数组中的何处保存排序值。
为了给当前值计算桶索引,该实现对索引移位使得在索引{\ttfamily lowBit}上的位
位于0号位,再掩模低处{\ttfamily bitsPerPass}个位。
\begin{lstlisting}
`\initcode{Count number of zero bits in array for current radix sort bit}{=}`
constexpr int nBuckets = 1 << bitsPerPass;
int bucketCount[nBuckets] = { 0 };
constexpr int bitMask = (1 << bitsPerPass) - 1;
for (const `\refvar{MortonPrimitive}{}` &mp : in) {
    int bucket = (mp.`\refvar{mortonCode}{}` >> lowBit) & bitMask;
    ++bucketCount[bucket];
}
\end{lstlisting}

有了每个桶中落入值的数量,我们就可以计算在输出数组中
每个桶的值开始处的偏移量;这就是之前的桶中落入值数量的和。
\begin{lstlisting}
`\initcode{Compute starting index in output array for each bucket}{=}`
int outIndex[nBuckets];
outIndex[0] = 0;
for (int i = 1; i < nBuckets; ++i)
    outIndex[i] = outIndex[i - 1] + bucketCount[i - 1];
\end{lstlisting}

现在我们知道每个桶在何处开始排序值了,
可以接收另一次图元传递来重算每个图元所在的桶并
将其\refvar{MortonPrimitive}{}保存在输出数组中。
这为当前这组数位完成了排序传递。
\begin{lstlisting}
`\initcode{Store sorted values in output array}{=}`
for (const `\refvar{MortonPrimitive}{}` &mp : in) {
    int bucket = (mp.`\refvar{mortonCode}{}` >> lowBit) & bitMask;
    out[outIndex[bucket]++] = mp;
}
\end{lstlisting}

当完成排序时,如果执行了奇数次基数排序传递,则最终排序值需要
从临时向量复制到原来传入\refvar{RadixSort}{()}的输出向量。
\begin{lstlisting}
`\initcode{Copy final result from tempVector, if needed}{}`
if (nPasses & 1)
    std::swap(*v, tempVector);
\end{lstlisting}

有了图元排序数组,我们将用附近的形心求得图元群集,
再在每个群集内创建图元上的LBVH。
这一步很适合并行化,因为通常有很多群集且每个群集都可以独立处理。
\begin{lstlisting}
`\initcode{Create LBVH treelets at bottom of BVH}{=}`
`\refcode{Find intervals of primitives for each treelet}{}`
`\refcode{Create LBVHs for treelets in parallel}{}`
\end{lstlisting}

每个图元群集都表示为一个\refvar{LBVHTreelet}{}。
它对群集中第一个图元在{\ttfamily mortonPrims}数组中的索引
以及后续图元的数量进行编码(见\reffig{4.12})。
\begin{lstlisting}
`\refcode{BVHAccel Local Declarations}{+=}\lastnext{BVHAccelLocalDeclarations}`
struct `\initvar{LBVHTreelet}{}` {
   int `\initvar{startIndex}{}`, `\initvar[LBVHTreelet:nPrimitives]{nPrimitives}{}`;
   `\refvar{BVHBuildNode}{}` *`\initvar{buildNodes}{}`;
};
\end{lstlisting}

\begin{figure}[htbp]
    \centering\input{Pictures/chap04/LBVHtreeletclusters.tex}
    \caption{LBVH小树的图元群集。图元形心聚类到覆盖其边界的均匀网格内。
        为每个格子中位于排序后的莫顿索引值连续段内的图元群集创建LBVH。}
    \label{fig:4.12}
\end{figure}

回想\reffig{4.8}中,莫顿码高位值匹配的点集位于原始盒中一个幂2对齐和幂2边长的子集内。
因为我们已经用莫顿编码值对数组{\ttfamily mortonPrims}排过序了,
所以高位值匹配的图元已经位于一段连续数组中。

这里我们将求30位莫顿码中有相同的高12位对应的图元集。
通过线性浏览数组{\ttfamily mortonPrims}来求得群集并
找到这12位中有任何一位发生变化的地方。
这对应了在$2^{12}=4096$个规范网格中对图元聚类,
每维有$2^4=16$个格子。
实践中,许多网格是空的,
尽管这里我们仍然希望求得许多独立的群集。
\begin{lstlisting}
`\initcode{Find intervals of primitives for each treelet}{=}`
std::vector<`\refvar{LBVHTreelet}{}`> treeletsToBuild;
for (int start = 0, end = 1; end <= (int)mortonPrims.size(); ++end) {
    uint32_t mask = 0b00111111111111000000000000000000;
    if (end == (int)mortonPrims.size() ||
        ((mortonPrims[start].`\refvar{mortonCode}{}` & mask) !=
         (mortonPrims[end].`\refvar{mortonCode}{}` & mask))) {
        `\refcode{Add entry to treeletsToBuild for this treelet}{}`
        start = end;
    }
}
\end{lstlisting}

当为一个小树求得一个图元群集后,就立即为它分配\refvar{BVHBuildNode}{}
(回想一个BVH中的节点数量不超过叶子节点数量的两倍,而后者又不超过图元数量)。
在现在的串行执行阶段中预分配这些内存比在LBVH的并行构建中分配更简单。

这里的一个重要细节是传入\refvar[MemoryArena:Alloc2]{MemoryArena::Alloc}{()}的{\ttfamily false}值;
它表示不要执行正在分配的底层对象的构造函数。
令人惊讶的是,运行\refvar{BVHBuildNode}{}构造函数会引入很大开销
并明显降低构建HLBVH时的整体性能。
因为在后面的代码中\refvar{BVHBuildNode}{}的所有成员都会初始化,
所以无论如何这里都没必要用构造函数执行初始化。

\begin{lstlisting}
`\initcode{Add entry to treeletsToBuild for this treelet}{=}`
int nPrimitives = end - start;
int maxBVHNodes = 2 * nPrimitives - 1;
`\refvar{BVHBuildNode}{}` *nodes = arena.`\refvar[MemoryArena:Alloc2]{Alloc}{}`<`\refvar{BVHBuildNode}{}`>(maxBVHNodes, false);
treeletsToBuild.push_back({start, nPrimitives, nodes});
\end{lstlisting}

一旦每个小树的图元都定好了,我们就可以并行地为它们创建LBVH了。
当构建完成后,每个\refvar{LBVHTreelet}{}的指针\refvar{buildNodes}{}会指向相应LBVH的根。

工作线程在构建LBVH时有两个地方必须相互配合。
第一,需要计算所有LBVH中的节点总数并通过传入\refvar{HLBVHBuild}{()}的
指针{\ttfamily totalNode}返回。
第二,当为LBVH创建叶子节点时,
需要数组{\ttfamily orderedPrims}的连续一段来
记录叶子节点中图元的索引。
我们的实现为两者都使用了原子变量——
{\ttfamily atomicTotal}跟踪节点数目,
{\ttfamily orderedPrimsOffset}跟踪{\ttfamily orderedPrims}下一有效项的索引。
\begin{lstlisting}
`\initcode{Create LBVHs for treelets in parallel}{=}`
std::atomic<int> atomicTotal(0), orderedPrimsOffset(0);
orderedPrims.resize(primitives.size());
`\refvar{ParallelFor}{}`(
    [&](int i) {
        `\refcode{Generate ith LBVH treelet}{}`
    }, treeletsToBuild.size());
*totalNodes = atomicTotal;
\end{lstlisting}

构建小树的工作由\refvar{emitLBVH}{()}执行,
它取用形心位于某空间区域内的图元并不断用划分平面
沿着三轴之一上的区域中心将当前空间区域划分为两半。

注意到\refvar{emitLBVH}{()}没有用
指向原子变量{\ttfamily atomicTotal}的指针来计数创建的节点,
而是更新一个非原子局部变量。
然后当每棵小树建成时这里的代码片只对每棵小树更新一次{\ttfamily atomicTotal}。
该方法比让工作线程在其执行过程中频繁修改{\ttfamily atomicTotal}具有明显更好的性能
(见附录\refsub{内存连续模型与性能}关于多核内存连续模型开销的讨论)。
\begin{lstlisting}
`\initcode{Generate ith LBVH treelet}{=}`
int nodesCreated = 0;
const int firstBitIndex = 29 - 12;
`\refvar{LBVHTreelet}{}` &tr = treeletsToBuild[i];
tr.`\refvar{buildNodes}{}` = 
    `\refvar{emitLBVH}{}`(tr.`\refvar{buildNodes}{}`, primitiveInfo, &mortonPrims[tr.`\refvar{startIndex}{}`],
             tr.`\refvar[LBVHTreelet:nPrimitives]{nPrimitives}{}`, &nodesCreated, orderedPrims,
             &orderedPrimsOffset, firstBitIndex);
atomicTotal += nodesCreated;
\end{lstlisting}

幸好有莫顿编码,当前空间区域不需要在\refvar{emitLBVH}{()}中显式表示:
传入的有序{\ttfamily mortonPrims}有一些匹配的高位,反过来对应了一个空间框。
对于莫顿码中剩下的每一位,该函数尝试沿着对应于数位{\ttfamily bitIndex}的平面来划分图元
(回想\reffig{4.8}(d)),然后再递归地调用自己。
尝试用来划分的下一位索引被作为该函数的最后一个参数传入:它最初为$29-12$,
因为29是从零计起的第30位索引,而我们之前用了高12位莫顿码值来聚类图元;
所以,我们知道那些数位都一定匹配该群集。
\begin{lstlisting}
`\refcode{BVHAccel Method Definitions}{+=}\lastnext{BVHAccelMethodDefinitions}`
`\refvar{BVHBuildNode}{}` *`\refvar{BVHAccel}{}`::`\initvar{emitLBVH}{}`(`\refvar{BVHBuildNode}{}` *&buildNodes,
        const std::vector<`\refvar{BVHPrimitiveInfo}{}`> &primitiveInfo,
        `\refvar{MortonPrimitive}{}` *mortonPrims, int nPrimitives, int *totalNodes,
        std::vector<std::shared_ptr<`\refvar{Primitive}{}`>> &orderedPrims,
        std::atomic<int> *orderedPrimsOffset, int bitIndex) const {
    if (bitIndex == -1 || nPrimitives < maxPrimsInNode) {
        `\refcode{Create and return leaf node of LBVH treelet}{}`
    } else {
        int mask = 1 << bitIndex;
        `\refcode{Advance to next subtree level if there's no LBVH split for this bit}{}`
        `\refcode{Find LBVH split point for this dimension}{}`
        `\refcode{Create and return interior LBVH node}{}`
    }
}
\end{lstlisting}

在\refvar{emitLBVH}{()}用最后低位划分完图元后,
不可能再有更多划分了,就创建一个叶子节点。
或者它也可以在只有很少的图元时就停止并创建叶子节点。

回想{\ttfamily orderedPrimsOffset}是数组{\ttfamily orderedPrims}中
下一有效元素的偏移量。
这里,对{\ttfamily fetch\_add()}的调用原子地将{\ttfamily nPrimitives}的值加到\newline
{\ttfamily orderedPrimsOffset}上并返回相加前它的旧值。
因为这些计算是原子的,多个LBVH构建线程可以
同时分割数组{\ttfamily orderedPrims}中的空间且没有数据竞争。
有了数组中的空间,叶子的构建和之前\refcode{Create leaf BVHBuildNode}{}中实现的方法一样。
\begin{lstlisting}
`\initcode{Create and return leaf node of LBVH treelet}{=}`
(*totalNodes)++;
`\refvar{BVHBuildNode}{}` *node = buildNodes++;
`\refvar{Bounds3f}{}` bounds;
int firstPrimOffset = orderedPrimsOffset->fetch_add(nPrimitives);
for (int i = 0; i < nPrimitives; ++i) {
    int primitiveIndex = mortonPrims[i].`\refvar{primitiveIndex}{}`;
    orderedPrims[firstPrimOffset + i] = primitives[primitiveIndex];
    bounds = `\refvar[Union2]{Union}{}`(bounds, primitiveInfo[primitiveIndex].`\refvar[BVHPrimitiveInfo::bounds]{bounds}{}`);
}
node->`\refvar[BVHBuildNode::InitLeaf]{InitLeaf}{}`(firstPrimOffset, nPrimitives, bounds);
return node;
\end{lstlisting}

可能有所有图元都位于划分平面同一侧的情况;因为图元按其莫顿索引排序,
可以通过看该范围内第一个和最后一个图元对于该平面是否有相同的数位来高效检测该情况。
这种情况下,\refvar{emitLBVH}{()}行进到下一位而不会无用地创建一个节点。
\begin{lstlisting}
`\initcode{Advance to next subtree level if there's no LBVH split for this bit}{=}`
if ((mortonPrims[0].`\refvar{mortonCode}{}` & mask) ==
    (mortonPrims[nPrimitives - 1].`\refvar{mortonCode}{}` & mask))
    return `\refvar{emitLBVH}{}`(buildNodes, primitiveInfo, mortonPrims, nPrimitives,
                    totalNodes, orderedPrims, orderedPrimsOffset,
                    bitIndex - 1);
\end{lstlisting}

如果划分平面两侧都有图元,则二分搜索会高效地找到
当前图元集中第{\ttfamily bitIndex}位从0变为1的划分点。
\begin{lstlisting}
`\initcode{Find LBVH split point for this dimension}{=}`
int searchStart = 0, searchEnd = nPrimitives - 1;
while (searchStart + 1 != searchEnd) {
    int mid = (searchStart + searchEnd) / 2;
    if ((mortonPrims[searchStart].`\refvar{mortonCode}{}` & mask) ==
        (mortonPrims[mid].`\refvar{mortonCode}{}` & mask))
        searchStart = mid;
    else
        searchEnd = mid;
}
int splitOffset = searchEnd;
\end{lstlisting}

有了划分偏移量,该方法现在可以声明一个节点用作内部节点
并为划出的两个图元集递归地构建LBVH了。
注意到莫顿编码还带来了一个方便:
不需要为了划分而复制或记录数组{\ttfamily mortonPrims}中的项:
因为它们都按莫顿码值排序了且从高到低处理数位,
两部分图元已经在划分平面的正确一侧。
\begin{lstlisting}
`\initcode{Create and return interior LBVH node}{=}`
(*totalNodes)++;
`\refvar{BVHBuildNode}{}` *node = buildNodes++;
`\refvar{BVHBuildNode}{}` *lbvh[2] = {
    `\refvar{emitLBVH}{}`(buildNodes, primitiveInfo, mortonPrims, splitOffset,
             totalNodes, orderedPrims, orderedPrimsOffset, bitIndex - 1),
    `\refvar{emitLBVH}{}`(buildNodes, primitiveInfo, &mortonPrims[splitOffset],
             nPrimitives - splitOffset, totalNodes, orderedPrims,
             orderedPrimsOffset, bitIndex - 1)
};
int axis = bitIndex % 3;
node->`\refvar[BVHBuildNode::InitInterior]{InitInterior}{}`(axis, lbvh[0], lbvh[1]);
return node;
\end{lstlisting}

一旦创建好所有LBVH小树,
\refvar{buildUpperSAH}{()}就为所有小树创建一个BVH。
因为它们通常只有几十个或几百个(无论如何不超过4096),
该步骤花的时间很少。
\begin{lstlisting}
`\initcode{Create and return SAH BVH from LBVH treelets}{=}`
std::vector<`\refvar{BVHBuildNode}{}` *> finishedTreelets;
for (`\refvar{LBVHTreelet}{}` &treelet : treeletsToBuild)
    finishedTreelets.push_back(treelet.`\refvar{buildNodes}{}`);
return `\refvar{buildUpperSAH}{}`(arena, finishedTreelets, 0,
                     finishedTreelets.size(), totalNodes);
\end{lstlisting}

这里不介绍该方法的实现,
因为它和完全基于SAH的BVH构建遵循相同的方法,
只不过是在小树根节点上而不是在场景图元上进行。
\begin{lstlisting}
`\initcode{BVHAccel Private Methods}{=}`
`\refvar{BVHBuildNode}{}` *`\initvar{buildUpperSAH}{}`(`\refvar{MemoryArena}{}` &arena,
    std::vector<`\refvar{BVHBuildNode}{}` *> &treeletRoots, int start, int end,
    int *totalNodes) const;
\end{lstlisting}

\subsection{为遍历而压实的BVH}\label{sub:为遍历而压实的BVH}
一旦建好BVH树,最后一步就是将其转换为紧凑的\sidenote{译者注:原文compact。}表达——
这样做能提升缓存、内存以及整个系统的性能。
最后的BVH存于内存中的一个线性数组内。
原始树的节点按\keyindex{深度优先}{depth-first}{}顺序排布,
即意味着在内存中每个内部节点的第一个孩子会立刻排在该节点之后。
这种情况下,只需要显式保存每个内部节点第二个孩子的偏移量。
见\reffig{4.13}关于树的拓扑与内存中节点顺序之间关系的图示。
\begin{figure}[htbp]
    \centering\input{Pictures/chap04/BVHlinearization.tex}
    \caption{BVH在内存中的线性排布。BVH的节点(左)按深度优先顺序(右)存储于内存中。
        因此,对于该树的任意中间节点(例如该例中的A和B),
        第一个孩子可在内存中父节点之后立刻找到。
        第二个孩子则通过偏移指针找到,这里用带箭头的线表示。
        树的叶子结点(D、E和C)没有孩子。}
    \label{fig:4.13}
\end{figure}

结构体\refvar{LinearBVHNode}{}保存遍历BVH所需的信息。
除了每个节点的边界框,它还为每个叶子节点保存偏移量和该节点内的图元数量。
对于内部节点,它保存了第二个孩子的偏移量以及构建层次时
是沿哪个坐标轴划分图元的;
这些信息用于下面的遍历例程以沿着光线按从前往后的顺序访问节点。
\begin{lstlisting}
`\refcode{BVHAccel Local Declarations}{+=}\lastcode{BVHAccelLocalDeclarations}`
struct `\initvar{LinearBVHNode}{}` {
    `\refvar{Bounds3f}{}` `\initvar[LinearBVHNode::bounds]{bounds}{}`;
    union {
        int `\initvar{primitivesOffset}{}`;    // leaf
        int `\initvar{secondChildOffset}{}`;   // interior
    };
    uint16_t `\initvar[LinearBVHNode::nPrimitives]{nPrimitives}{}`;  // 0 -> interior node
    uint8_t `\initvar[LinearBVHNode::axis]{axis}{}`;          // interior node: xyz
    uint8_t `\initvar[LinearBVHNode::pad]{pad}{}`[1];        // ensure 32 byte total size
};
\end{lstlisting}

该结构体被填充了以保证是32字节大小。
这样做保证了如果分配节点时第一个节点是对齐\keyindex{缓存行}{cache line}{}的,
则后续节点不会跨越缓存行
(只要缓存行大小至少为32字节,即现代CPU架构的情况)。

建好的树被方法\refvar{flattenBVHTree}{()}变换为\refvar{LinearBVHNode}{}表示,
它执行深度优先遍历并在内存中按线性顺序存储节点。
\begin{lstlisting}
`\initcode{Compute representation of depth-first traversal of BVH tree}{=}`
nodes = AllocAligned<`\refvar{LinearBVHNode}{}`>(totalNodes);
int offset = 0;
`\refvar{flattenBVHTree}{}`(root, &offset);
\end{lstlisting}

指向\refvar{LinearBVHNode}{}数组的指针保存为\refvar{BVHAccel}{}的一个成员变量,
所以它可以在\refvar{BVHAccel}{}的析构函数中释放。
\begin{lstlisting}
`\refcode{BVHAccel Private Data}{+=}\lastcode{BVHAccelPrivateData}`
`\refvar{LinearBVHNode}{}` *`\initvar[BVHAccel::nodes]{nodes}{}` = nullptr;
\end{lstlisting}

将树展平为线性表示很简单;
参数{\ttfamily *offset}跟踪当前在数组\refvar{BVHAccel::nodes}{}
中的偏移量。注意在递归调用处理其孩子之前
(如果该节点是内部节点)要把当前节点添加到该数组中。
\begin{lstlisting}
`\refcode{BVHAccel Method Definitions}{+=}\lastnext{BVHAccelMethodDefinitions}`
int `\refvar{BVHAccel}{}`::`\initvar{flattenBVHTree}{}`(`\refvar{BVHBuildNode}{}` *node, int *offset) {
    `\refvar{LinearBVHNode}{}` *linearNode = &`\refvar[BVHAccel::nodes]{nodes}{}`[*offset];
    linearNode->`\refvar[LinearBVHNode::bounds]{bounds}{}` = node->`\refvar[BVHBuildNode::bounds]{bounds}{}`;
    int myOffset = (*offset)++;
    if (node->`\refvar[BVHBuildNode:nPrimitives]{nPrimitives}{}` > 0) {
        linearNode->`\refvar{primitivesOffset}{}` = node->`\refvar{firstPrimOffset}{}`;
        linearNode->`\refvar[LinearBVHNode::nPrimitives]{nPrimitives}{}` = node->`\refvar[BVHBuildNode:nPrimitives]{nPrimitives}{}`;
    } else {
        `\refcode{Create interior flattened BVH node}{}`
    }
    return myOffset;
}
\end{lstlisting}

在内部节点时,递归调用会展平两棵子树。
第一棵如愿在数组中当前节点之后立刻结束,
而其递归调用\refvar{flattenBVHTree}{()}返回的第二棵偏移量
则保存在该节点的成员\refvar{secondChildOffset}{}中。
\begin{lstlisting}
`\initcode{Create interior flattened BVH node}{=}`
linearNode->`\refvar[LinearBVHNode::axis]{axis}{}` = node->`\refvar[BVHBuildNode::splitAxis]{splitAxis}{}`;
linearNode->`\refvar[LinearBVHNode::nPrimitives]{nPrimitives}{}` = 0;
`\refvar{flattenBVHTree}{}`(node->`\refvar[BVHBuildNode::children]{children}{}`[0], offset);
linearNode->`\refvar{secondChildOffset}{}` =
    `\refvar{flattenBVHTree}{}`(node->`\refvar[BVHBuildNode::children]{children}{}`[1], offset);
\end{lstlisting}

\subsection{遍历}\label{sub:遍历}
BVH的遍历代码非常简单——没有递归调用,只有少量数据用来维护当前遍历的状态。
方法\refvar[BVHAccel::Intersect]{Intersect}{()}从
预先计算一些与将要反复用到的光线相关的值开始。
\begin{lstlisting}
`\refcode{BVHAccel Method Definitions}{+=}\lastcode{BVHAccelMethodDefinitions}`
bool `\refvar{BVHAccel}{}`::`\initvar[BVHAccel::Intersect]{Intersect}{}`(const `\refvar{Ray}{}` &ray,
        `\refvar{SurfaceInteraction}{}` *isect) const {
    bool hit = false;
    `\refvar{Vector3f}{}` invDir(1 / ray.d.x, 1 / ray.d.y, 1 / ray.d.z);
    int dirIsNeg[3] = { invDir.x < 0, invDir.y < 0, invDir.z < 0 };
    `\refcode{Follow ray through BVH nodes to find primitive intersections}{}`
    return hit;
}
\end{lstlisting}

每当\refvar[BVHAccel::Intersect]{Intersect}{()}中的{\ttfamily while}循环开始一次迭代时,
{\ttfamily currentNodeIndex}保有将要访问的节点在数组\refvar[BVHAccel::nodes]{nodes}{}中的偏移量。
它起始于0值,表示树根。仍需要访问的节点存于数组{\ttfamily nodesToVisit[]}中,即充当一个栈;
{\ttfamily toVisitOffset}存有栈中下一个可弹出元素的偏移量。
\begin{lstlisting}
`\initcode{Follow ray through BVH nodes to find primitive intersections}{=}`
int toVisitOffset = 0, currentNodeIndex = 0;
int nodesToVisit[64];
while (true) {
    const `\refvar{LinearBVHNode}{}` *node = &`\refvar[BVHAccel::nodes]{nodes}{}`[currentNodeIndex];
    `\refcode{Check ray against BVH node}{}`
}
\end{lstlisting}

对于每个节点,我们都检查光线是否与节点的边界框相交(或从其内部发射)。
如果是我们就访问该节点,如果它是叶子节点就对其图元做相交测试,
如果是内部节点就处理它的孩子。
如果发现没有相交,就从{\ttfamily nodesToVisit[]}取出下一个将要访问的节点的偏移量
(或者如果栈空了就完成遍历了)。
\begin{lstlisting}
`\initcode{Check ray against BVH node}{=}`
if (node->`\refvar[LinearBVHNode::bounds]{bounds}{}`.`\refvar[Bounds3::IntersectP2]{IntersectP}{}`(ray, invDir, dirIsNeg)) {
    if (node->`\refvar[LinearBVHNode::nPrimitives]{nPrimitives}{}` > 0) {
        `\refcode{Intersect ray with primitives in leaf BVH node}{}`
    } else {
        `\refcode{Put far BVH node on nodesToVisit stack, advance to near node}{}`
    }
} else {
    if (toVisitOffset == 0) break;
    currentNodeIndex = nodesToVisit[--toVisitOffset];
}
\end{lstlisting}

如果当前节点是叶子,则该光线必须和它里面的图元做相交测试。
然后从栈{\ttfamily nodesToVisit}中找到下一个要访问的节点;
即使当前节点求得了交点,也必须访问剩下的节点,
以防万一它们中有一个给出更近的交点。
然而,如果求得一个交点,则该光线的\refvar{tMax}{}值将会更新为该相交距离;
这样可以更高效地丢弃剩下的节点中任何比该距离更远的部分。
\begin{lstlisting}
`\initcode{Intersect ray with primitives in leaf BVH node}{=}`
for (int i = 0; i < node->`\refvar[LinearBVHNode::nPrimitives]{nPrimitives}{}`; ++i)
    if (`\refvar[BVHAccel::primitives]{primitives}{}`[node->`\refvar{primitivesOffset}{}` + i]->`\refvar[Primitive::Intersect]{Intersect}{}`(ray, isect))
        hit = true;
if (toVisitOffset == 0) break;
currentNodeIndex = nodesToVisit[--toVisitOffset];
\end{lstlisting}

对于光线命中的内部节点,需要访问它的两个孩子。
如上所述,万一与光线相交的图元在第一个里面,
则访问光线穿过的第一个孩子再访问第二个是可取的,
这样光线的\refvar{tMax}{}值就能更新,
进而缩减光线的范围以及与之相交的边界框节点数目。

一个执行从前往后遍历而不会因光线与两个子节点相交
以及比较距离而带来开销的高效方法是使用光线方向向量
在该节点划分图元时所沿的坐标轴上的符号:
如果符号为负,我们应该先访问第二个孩子再访问第一个孩子,
因为进入第二棵子树的图元在划分点的上面一侧
(正号方向则反过来)。
这样做很简单:先要访问的节点的偏移量被复制到{\ttfamily currentNodeIndex},
另一个节点的偏移量被加到栈{\ttfamily nodesToVisit}中
(回想因为内存中节点按深度优先排列,第一个孩子正好在当前节点之后)。
\begin{lstlisting}
`\initcode{Put far BVH node on nodesToVisit stack, advance to near node}{=}`
if (dirIsNeg[node->`\refvar[LinearBVHNode::axis]{axis}{}`]) {
   nodesToVisit[toVisitOffset++] = currentNodeIndex + 1;
   currentNodeIndex = node->`\refvar{secondChildOffset}{}`;
} else {
   nodesToVisit[toVisitOffset++] = node->`\refvar{secondChildOffset}{}`;
   currentNodeIndex = currentNodeIndex + 1;
}
\end{lstlisting}

方法{\initvar{BVHAccel::IntersectP}{()}}本质上和常规相交方法一样,有两处区别是,它调用了\refvar{Primitive}{}的
方法\refvar[Primitive::IntersectP]{IntersectP}{()}而不是\refvar[Primitive::Intersect]{Intersect}{()},
以及当找到任何交点时就立刻停止遍历。


\section{kd树加速器}\label{sec:kd树加速器}

\keyindex{二叉空间划分}{binary space partitioning}{}(BSP)树用平面自适应地细分空间。
一个BSP树从包含整个场景的边界框开始。
如果框内图元的数量大于某个阈值,则用平面将该框分为两半。
然后图元和与之重合的任意一半关联,
同时位于两半里的图元就都与它们关联
(相比之下,BVH中划分后每个图元只能分配到两个组中的一个)。

划分过程递归进行,直到结果树中的每个叶子区域都包含足够少的图元或者达到最大深度。
因为划分平面可以放置于整个框内的任意位置,
且3D空间的不同部分可以精确到不同程度,
所以BSP易于处理分布不均的几何体。

BSP树的两个变种是\keyindex{kd树}{kd-tree}{tree树}和\keyindex{八叉树}{octree}{tree树}。
kd树\sidenote{译者注:“kd”是k个维度的缩写。}简单地限制划分平面垂直于一个坐标轴;
这让树的遍历和构建都更高效,而在如何划分空间上牺牲一些灵活性。
八叉树每一步用三个垂直于轴的平面同时将该框分为八个区域
(通常在每个方向沿范围中心划分)。

本节中,我们将在类\refvar{KdTreeAccel}{}中为光线相交加速实现一个kd树。
该类源码可在文件\href{https://github.com/mmp/pbrt-v3/tree/master/src/accelerators/kdtreeaccel.h}{\ttfamily accelerators/kdtreeaccel.h}
和\href{https://github.com/mmp/pbrt-v3/tree/master/src/accelerators/kdtreeaccel.cpp}{\ttfamily accelerators/kdtreeaccel.cpp}
中找到。

\begin{lstlisting}
`\initcode{KdTreeAccel Declarations}{=}\initnext{KdTreeAccelDeclarations}`
class `\initvar{KdTreeAccel}{}` : public `\refvar{Aggregate}{}` {
public:
    `\refcode{KdTreeAccel Public Methods}{}`
private:
    `\refcode{KdTreeAccel Private Methods}{}`
    `\refcode{KdTreeAccel Private Data}{}`
};
\end{lstlisting}

除了要保存的图元外,\refvar{KdTreeAccel}{}构造函数
还接收一些参数用于在构建树时指导要作出的决定;
这些参数存于成员变量中(\refvar{isectCost}{}、\refvar{traversalCost}{}、
\refvar{maxPrims}{}、{\ttfamily maxDepth}和\refvar{emptyBonus}{})留待后用。
见\reffig{4.14}中构建树的图示。
\begin{figure}[htbp]
    \centering\input{Pictures/chap04/kdtreesplits.tex}
    \caption{通过沿坐标轴之一递归地划分场景几何边界框来构建kd树。这里,第一次划分沿$x$轴;
        它摆放后使三角形刚好单独在右边区域而其余图元则在左边。
        然后再用轴对齐的划分平面细化若干次左边的区域。
        细化标准的细节——每一步用哪个轴划分空间、沿轴上哪个位置放置平面
        以及何时结束细分——在实践中均会极大影响树的性能。}
    \label{fig:4.14}
\end{figure}

\begin{lstlisting}
`\initcode{KdTreeAccel Method Definitions}{=}\initnext{KdTreeAccelMethodDefinitions}`
`\refvar{KdTreeAccel}{}`::`\refvar{KdTreeAccel}{}`(
        const std::vector<std::shared_ptr<`\refvar{Primitive}{}`>> &p,
        int isectCost, int traversalCost, `\refvar{Float}{}` emptyBonus,
        int maxPrims, int maxDepth)
    : `\refvar{isectCost}{}`(isectCost), `\refvar{traversalCost}{}`(traversalCost),
      `\refvar{maxPrims}{}`(maxPrims), `\refvar{emptyBonus}{}`(emptyBonus), `\refvar[KdTreeAccel::primitives]{primitives}{}`(p) {
    `\refcode{Build kd-tree for accelerator}{}`
}
\end{lstlisting}

\begin{lstlisting}
`\initcode{KdTreeAccel Private Data}{=}\initnext{KdTreeAccelPrivateData}`
const int `\initvar{isectCost}{}`, `\initvar{traversalCost}{}`, `\initvar{maxPrims}{}`;
const `\refvar{Float}{}` `\initvar{emptyBonus}{}`;
std::vector<std::shared_ptr<`\refvar{Primitive}{}`>> `\initvar[KdTreeAccel::primitives]{primitives}{}`;
\end{lstlisting}

\subsection{树状表示}\label{sub:树状表示}
kd树是二叉树,每个内部节点总是有两个孩子且树的叶子存有与之重合的图元。
每个内部节点必须提供三块信息的访问渠道:
\begin{itemize}
    \item 划分轴:该节点划分了$x,y$和$z$中的哪一个轴;
    \item 划分位置:划分平面沿该轴的位置;
    \item 孩子:关于如何到达其下两个子节点的信息。
\end{itemize}
每个叶子节点只需要记录哪个图元与之重合。

为了保证所有内部节点和许多叶子节点只用8字节内存
(假设\refvar{Float}{}占4字节)而麻烦一下是值得的,
因为这样做保证了八个节点契合一个64字节的缓存行。
因为树中经常有许多节点且每条光线通常都要访问许多节点,
最小化节点表示的大小能极大提高缓存性能。
我们最初的实现使用了16字节节点表示;
当我们把大小减少到8字节时我们得到了几乎20\%的提速。

叶子和内部节点都用下面的结构体\refvar{KdAccelNode}{}表示。
每个{\ttfamily union}成员后的注释都说明了特定域是用于内部节点、叶子节点还是两者都是。
\begin{lstlisting}
`\initcode{KdTreeAccel Local Declarations}{=}\initnext{KdTreeAccelLocalDeclarations}`
struct `\initvar{KdAccelNode}{}` {
    `\refcode{KdAccelNode Methods}{}`
    union {
        `\refvar{Float}{}` `\initvar[KdAccelNode::split]{split}{}`;                  // Interior
        int `\initvar{onePrimitive}{}`;             // Leaf
        int `\initvar{primitiveIndicesOffset}{}`;   // Leaf
    };
    union {
        int `\initvar[KdAccelNode::flags]{flags}{}`;         // Both
        int `\initvar{nPrims}{}`;        // Leaf
        int `\initvar{aboveChild}{}`;    // Interior
    };
};
\end{lstlisting}

变量\refvar{KdAccelNode::flags}{}的低两位用于区分用$x,y$和$z$划分的内部节点
(这些数位分别取值0,1和2)以及叶子节点(这些数位取值3)。
在8字节中保存叶子节点相对简单:\refvar{KdAccelNode::flags}{}的低2位
用于表示这是一个叶子,\refvar[nPrims]{KdAccelNode::nPrims}{}的高30位
可用于记录有多少个图元与之重合。
然后,如果只有一个图元与\refvar{KdAccelNode}{}叶子重合,
则指向数组\refvar{KdTreeAccel::primitives}{}
的整数索引会指出该\refvar{Primitive}{}。如果重合的图元多于一个,
则它们的索引保存于\refvar[primitiveIndices]{KdTreeAccel::primitiveIndices}{}的一段中。
该叶子第一个索引的偏移量存于\refvar[primitiveIndicesOffset]{KdAccelNode::primitiveIndicesOffset}{}且后面直接跟着剩下的索引。
\begin{lstlisting}
`\refcode{KdTreeAccel Private Data}{+=}\lastnext{KdTreeAccelPrivateData}`
std::vector<int> `\initvar{primitiveIndices}{}`;
\end{lstlisting}

叶子节点很容易初始化,不过我们要注意细节,
因为\refvar[KdAccelNode::flags]{flags}{}和\refvar{nPrims}{}共享同一存储;
我们需要注意在初始化其中一个时不要搞乱了另一个。
此外,在保存图元数量前必须向左移两位,
这样\refvar{KdAccelNode::flags}{}的低两位可以都设为1以表示这是一个叶子节点。
\begin{lstlisting}
`\refcode{KdTreeAccel Method Definitions}{+=}\lastnext{KdTreeAccelMethodDefinitions}`
void `\refvar{KdAccelNode}{}`::`\initvar[KdAccelNode::InitLeaf]{InitLeaf}{}`(int *primNums, int np,
        std::vector<int> *primitiveIndices) {
    `\refvar[KdAccelNode::flags]{flags}{}` = 3;
    `\refvar{nPrims}{}` |= (np << 2);
    `\refcode{Store primitive ids for leaf node}{}`
}
\end{lstlisting}

对于有零或一个重合图元的叶子节点,
因为有\refvar[onePrimitive]{KdAccelNode::onePrimitive}{}
域了,所以不再需要额外分配内存。
对于有多个重合图元的情况,则在数组{\ttfamily primitiveIndices}中分配存储。
\begin{lstlisting}
`\initcode{Store primitive ids for leaf node}{=}`
if (np == 0)
    `\refvar{onePrimitive}{}` = 0;
else if (np == 1)
    `\refvar{onePrimitive}{}` = primNums[0];
else {
    `\refvar{primitiveIndicesOffset}{}` = primitiveIndices->size();
    for (int i = 0; i < np; ++i)
        primitiveIndices->push_back(primNums[i]);
}
\end{lstlisting}

让内部节点减少到8字节也相当简单。
一个\refvar{Float}{}(当\refvar{Float}{}定义为{\ttfamily float}时其大小为32位)
保存了节点沿所选划分轴分割空间的位置,并且如之前所述,
\refvar{KdAccelNode::flags}{}低两位用于记录该节点是沿哪个轴划分的。
剩下的就是存储足够的信息使我们遍历树时能找到该节点的两个孩子。

我们排布节点的方式是只存储一个孩子指针,而不是存储两个指针或偏移量:
所有节点都分配到单个连续内存块,
内部节点的对应划分平面下方空间的孩子在数组中的保存位置总是紧跟其父亲
(通过在内存中保持至少一个孩子挨着其父亲,这样的排布也提高了缓存性能)。
另一个对应于划分平面上方的孩子,则在数组其他某处出现;
单个整数偏移量\refvar[aboveChild]{KdAccelNode::aboveChild}{}保存了它在节点数组中的位置。
该表示和\refsub{为遍历而压实的BVH}中BVH节点用的类似。

有了所有这些约定,初始化内部节点的代码就很简单了。
就像方法\refvar[KdAccelNode::InitLeaf]{InitLeaf}{()}
那样,在设置\refvar{aboveChild}{}前为\refvar[KdAccelNode::flags]{flags}{}赋值、
计算移位的\refvar{aboveChild}{}逻辑或值很重要,
这样才不会搞乱保存在\refvar[KdAccelNode::flags]{flags}{}中的数位。
\begin{lstlisting}
`\initcode{KdAccelNode Methods}{=}\initnext{KdAccelNodeMethods}`
void `\initvar[KdAccelNode::InitInterior]{InitInterior}{}`(int axis, int ac, `\refvar{Float}{}` s) {
    `\refvar[KdAccelNode::split]{split}{}` = s;
    `\refvar[KdAccelNode::flags]{flags}{}` = axis;
    `\refvar{aboveChild}{}` |= (ac << 2);
}
\end{lstlisting}

最后,我们将提供一些方法从节点中提取各种值,
这样调用者就不需要了解其内存表示的细节了。
\begin{lstlisting}
`\refcode{KdAccelNode Methods}{+=}\lastcode{KdAccelNodeMethods}`
`\refvar{Float}{}` `\initvar{SplitPos}{()}` const { return `\refvar[KdAccelNode::split]{split}{}`; }
int `\initvar[KdAccelNode::nPrimitives]{nPrimitives}{()}` const { return `\refvar{nPrims}{}` >> 2; }
int `\initvar[KdAccelNode::SplitAxis]{SplitAxis}{()}` const { return `\refvar[KdAccelNode::flags]{flags}{}` & 3; }
bool `\initvar{IsLeaf}{()}` const { return (`\refvar[KdAccelNode::flags]{flags}{}` & 3) == 3; }
int `\initvar{AboveChild}{()}` const { return `\refvar{aboveChild}{}` >> 2; }
\end{lstlisting}

\subsection{树的构建}\label{sub:树的构建}
kd树是用递归自顶向下算法构建的。
每一步中,我们有一个轴对齐空间区域和与该区域重合的图元集。
要么该区域分为两个子区域且转化为内部节点,
要么用重合的图元创建一个叶子节点,结束递归。

正如讨论\refvar{KdAccelNode}{}时所提到的,
所有树节点都保存于连续数组中。\newline
\refvar[nextFreeNode]{KdTreeAccel::nextFreeNode}{}记录了该数组中下一个有效节点,
\refvar[nAllocedNodes]{KdTreeAccel::\newline nAllocedNodes}{}记录了已经分配的总数。
通过一开始设置两者为0且不分配任何节点,这里的实现保证了当初始化树的第一个节点时能立即完成分配。

如果没有为构造函数给定,则还有必要确定树的最大深度。
尽管树的构建过程通常会自然地在合理的深度结束,
但限制最大深度很重要,这样极端情况下树所用的内存数量才不会无限增长。
我们已经发现值$8+1.3\log_2N$为大量场景给出了合理的最大深度。

\begin{lstlisting}
`\initcode{Build kd-tree for accelerator}{=}`
`\refvar{nextFreeNode}{}` = `\refvar{nAllocedNodes}{}` = 0;
if (maxDepth <= 0)
    maxDepth = std::round(8 + 1.3f * `\refvar{Log2Int}{}`(`\refvar[KdTreeAccel::primitives]{primitives}{}`.size()));
`\refcode{Compute bounds for kd-tree construction}{}`
`\refcode{Allocate working memory for kd-tree construction}{}`
`\refcode{Initialize primNums for kd-tree construction}{}`
`\refcode{Start recursive construction of kd-tree}{}`
\end{lstlisting}

\begin{lstlisting}
`\refcode{KdTreeAccel Private Data}{+=}\lastnext{KdTreeAccelPrivateData}`
`\refvar{KdAccelNode}{}` *`\initvar[KdTreeAccel::nodes]{nodes}{}`;
int `\initvar{nAllocedNodes}{}`, `\initvar{nextFreeNode}{}`;
\end{lstlisting}

因为构建例程会一路重复使用图元边界框,
所以在开始构建树前它们被保存在{\ttfamily vector}中,
这样就不需重复调用可能更慢的方法\refvar{Primitive::WorldBound}{()}。
\begin{lstlisting}
`\initcode{Compute bounds for kd-tree construction}{=}`
std::vector<`\refvar{Bounds3f}{}`> primBounds;
for (const std::shared_ptr<`\refvar{Primitive}{}`> &prim : `\refvar[KdTreeAccel::primitives]{primitives}{}`) {
    `\refvar{Bounds3f}{}` b = prim->`\refvar[Primitive::WorldBound]{WorldBound}{}`();
    `\refvar[KdTreeAccel::bounds]{bounds}{}` = `\refvar[Union2]{Union}{}`(`\refvar[KdTreeAccel::bounds]{bounds}{}`, b);
    primBounds.push_back(b);
}
\end{lstlisting}

\begin{lstlisting}
`\refcode{KdTreeAccel Private Data}{+=}\lastcode{KdTreeAccelPrivateData}`
`\refvar{Bounds3f}{}` `\initvar[KdTreeAccel::bounds]{bounds}{}`;
\end{lstlisting}

树构建例程的参数之一是图元索引数组,表示哪个图元与当前节点重合。
因为(当递归开始时)所有图元都和根节点重合,
所以我们从初始化值为零到{\ttfamily primitives.size()-1}的数组开始。
\begin{lstlisting}
`\initcode{Initialize primNums for kd-tree construction}{=}`
std::unique_ptr<int[]> primNums(new int[`\refvar[KdTreeAccel::primitives]{primitives}{}`.size()]);
for (size_t i = 0; i < `\refvar[KdTreeAccel::primitives]{primitives}{}`.size(); ++i)
    primNums[i] = i;
\end{lstlisting}

每个树节点都会调用\refvar{KdTreeAccel::buildTree}{()}。
它负责决定该节点应该是内部节点还是叶子并适当更新数据结构。
最后三个参数{\ttfamily edges}、{\ttfamily prims0}、{\ttfamily prims1}是
指向分配于代码片\refcode{Allocate working memory for kd-tree construction}{}的数据的指针,
稍后几页会对此作定义和介绍。
\begin{lstlisting}
`\initcode{Start recursive construction of kd-tree}{=}`
`\refvar[KdTreeAccel::buildTree]{buildTree}{}`(0, `\refvar[KdTreeAccel::bounds]{bounds}{}`, primBounds, primNums.get(), `\refvar[KdTreeAccel::primitives]{primitives}{}`.size(), 
          maxDepth, edges, prims0.get(), prims1.get());
\end{lstlisting}

\refvar{KdTreeAccel::buildTree}{()}的主要参数是供创建的节点使用的相对于
\refvar{KdAccelNode}{}数组的偏移量{\ttfamily nodeNum}、
给出该节点覆盖的空间区域边界框的{\ttfamily nodeBounds},
以及与之重合的图元索引{\ttfamily primNums}。
其余参数稍后在快用到它们时阐述。
\begin{lstlisting}
`\refcode{KdTreeAccel Method Definitions}{+=}\lastnext{KdTreeAccelMethodDefinitions}`
void `\refvar{KdTreeAccel}{}::\initvar[KdTreeAccel::buildTree]{buildTree}{}`(int nodeNum, const `\refvar{Bounds3f}{}` &nodeBounds,
        const std::vector<`\refvar{Bounds3f}{}`> &allPrimBounds, int *primNums,
        int nPrimitives, int depth,
        const std::unique_ptr<`\refvar{BoundEdge}{}`[]> edges[3], 
        int *prims0, int *prims1, int badRefines) {
    `\refcode{Get next free node from nodes array}{}`
    `\refcode{Initialize leaf node if termination criteria met}{}`
    `\refcode{Initialize interior node and continue recursion}{}`
}
\end{lstlisting}

如果所有分配的节点都已经用完了,则重新分配两倍数量的节点内存并复制旧值。
第一次调用\refvar{KdTreeAccel::buildTree}{()}时,
\refvar[nAllocedNodes]{KdTreeAccel::nAllocedNodes}{}
为0并分配树节点的一个初始块。
\begin{lstlisting}
`\initcode{Get next free node from nodes array}{=}`
if (`\refvar{nextFreeNode}{}` == `\refvar{nAllocedNodes}{}`) {
    int nNewAllocNodes = std::max(2 * `\refvar{nAllocedNodes}{}`, 512);
    `\refvar{KdAccelNode}{}` *n = `\refvar{AllocAligned}{}`<`\refvar{KdAccelNode}{}`>(nNewAllocNodes);
    if (`\refvar{nAllocedNodes}{}` > 0) {
        memcpy(n, `\refvar[KdTreeAccel::nodes]{nodes}{}`, `\refvar{nAllocedNodes}{}` * sizeof(`\refvar{KdAccelNode}{}`));
        `\refvar{FreeAligned}{}`(`\refvar[KdTreeAccel::nodes]{nodes}{}`);
    }
    `\refvar[KdTreeAccel::nodes]{nodes}{}` = n;
    `\refvar{nAllocedNodes}{}` = nNewAllocNodes;
}
++`\refvar{nextFreeNode}{}`;
\end{lstlisting}

当区域内有足够少量的图元或达到最大深度时就创建叶子节点(停止递归)。
参数{\ttfamily depth}一开始为树的最大深度,且每一层递减。
\begin{lstlisting}
`\initcode{Initialize leaf node if termination criteria met}{=}`
if (nPrimitives <= `\refvar{maxPrims}{}` || depth == 0) {
    `\refvar[KdTreeAccel::nodes]{nodes}{}`[nodeNum].`\refvar[KdAccelNode::InitLeaf]{InitLeaf}{}`(primNums, nPrimitives, &`\refvar{primitiveIndices}{}`);
    return;
}
\end{lstlisting}

若这是个内部节点,则需要选择一个划分平面,按该平面划分图元并递归。
\begin{lstlisting}
`\initcode{Initialize interior node and continue recursion}{=}`
`\refcode{Choose split axis position for interior node}{}`
`\refcode{Create leaf if no good splits were found}{}`
`\refcode{Classify primitives with respect to split}{}`
`\refcode{Recursively initialize children nodes}{}`
\end{lstlisting}

我们的实现选择用\refsub{表面积启发法}介绍的SAH来划分。
SAH适用于kd树和BVH;为节点中一系列候选划分平面计算估计的开销,
并选择给出最少开销的划分。

在这里的实现中,相交开销$t_{\text{isect}}$和遍历开销$t_{\text{trav}}$可由用户设置;
它们的默认值分别是80和1.
重要的是,这两个值的比例决定了树构建算法的表现
\footnote{该方法的许多其他实现似乎给这些开销使用了接近得多的值,
    有时甚至接近相等值(例如,见\citet{hurley2002fast})。
    在pbrt中这里所用的值为大量测试场景给出了最好的性能。
    我们怀疑这一矛盾是因为pbrt中光线-图元相交测试需要两次虚函数调用
    以及一次光线从世界到物体空间的变换这一事实,
    此外还有执行实际相交测试的开销。
    只支持三角图元的高度优化的光线追踪器不会有此类任何额外开销。
    见\refsub{只有三角形}关于这一平衡设计的更多讨论。}。
比起BVH所用的值,这些值之间更大的比例反映的事实是
访问kd树的节点比访问BVH节点的开销更少。

针对用于BVH树的SAH的一点修改是,对于kd树值得稍微偏好选择
使其中一个孩子没有与之重合的图元的划分,
因为光线穿过这些区域可以立即进行到下一个kd树节点而无需任何光线-图元相交测试。
因此,未划分和划分后区域的改进开销分别为
\begin{align*}
    t_{\text{isect}}N \quad \text{和} \quad t_{\text{trav}}+(1-b_{\mathrm{e}})(p_BN_Bt_{\text{isect}}+p_AN_At_{\text{isect}})\, ,
\end{align*}
其中$b_{\mathrm{e}}$是为零的“补贴”\sidenote{译者注:原文bonus。}值,
除非两个区域之一完全为空时取值0到1.

有了为开销模型计算概率的方法,唯一要解决的问题是
怎么生成候选划分位置以及怎么为每个候选者高效计算开销。
可以证明该模型最小开销能于在某一图元边界框的一个面上划分时取得——
不需要考虑在中间位置的划分(为了帮助你自己理解,
考虑一下开销函数在面的边界之间时的特性)。
这里,我们将考虑该区域内三个坐标轴之一或以上的所有边界框面。

利用精心构造的算法可以把检查所有这些候选者的开销维持在相对低的水平。
为了计算这些开销,我们将扫掠边界框在每个轴上的投影并追踪开销最低的那些(\reffig{4.15})。
每个边界框在每个轴上有两处边界,每处都用结构体\refvar{BoundEdge}{}的实例表示。
该结构体记录了边界沿轴的位置,它表示边界框的开始或结束
(沿轴从低到高),以及哪个图元与之关联。
\begin{figure}[htbp]
    \centering\input{Pictures/chap04/kdtreeprojectedbboxes.tex}
    \caption{给定我们要考虑的可能划分所沿的轴,图元的边界框被投影到该轴上,
    这带来了一个高效算法以追踪特定划分平面两侧会各有多少图元。
    例如这里,在$a_1$处划分会让$A$完全留在划分平面下方,$B$横跨之,而$C$完全在其上方。
    轴上每一个点$a_0,a_1,b_0,b_1,c_0$和$c_1$都由结构体\refvar{BoundEdge}{}的一个实例表示。}
    \label{fig:4.15}
\end{figure}
\begin{lstlisting}
`\refcode{KdTreeAccel Local Declarations}{+=}\lastnext{KdTreeAccelLocalDeclarations}`
enum class `\initvar{EdgeType}{}` { `\initvar[EdgeType::Start]{Start}{}`, `\initvar[EdgeType::End]{End}{}` };
\end{lstlisting}
\begin{lstlisting}
`\refcode{KdTreeAccel Local Declarations}{+=}\lastcode{KdTreeAccelLocalDeclarations}`
struct `\initvar{BoundEdge}{}` {
    `\refcode{BoundEdge Public Methods}{}`
    `\refvar{Float}{}` `\initvar[BoundEdge::t]{t}{}`;
    int `\initvar[BoundEdge::primNum]{primNum}{}`;
    `\refvar{EdgeType}{}` `\initvar[BoundEdge::type]{type}{}`;
};
\end{lstlisting}
\begin{lstlisting}
`\initcode{BoundEdge Public Methods}{=}`
`\refvar{BoundEdge}{}`(`\refvar{Float}{}` t, int primNum, bool starting)
    : `\refvar[BoundEdge::t]{t}{}`(t), `\refvar[BoundEdge::primNum]{primNum}{}`(primNum) {
    `\refvar[BoundEdge::type]{type}{}` = starting ? `\refvar{EdgeType::Start}{}` : `\refvar{EdgeType::End}{}`; 
}
\end{lstlisting}

对于任意树节点至多需要为{\ttfamily 2*\refvar[KdTreeAccel::primitives]{primitives}{}.size()}个\refvar{BoundEdge}{}计算开销,
所以一次分配全部三轴上所有边界的内存然后再为每个创建的节点复用。
\begin{lstlisting}
`\initcode{Allocate working memory for kd-tree construction}{=}\initnext{Allocateworkingmemoryforkdtreeconstruction}`
std::unique_ptr<`\refvar{BoundEdge}{}`[]> edges[3];
for (int i = 0; i < 3; ++i)
    edges[i].reset(new `\refvar{BoundEdge}{}`[2 * `\refvar[KdTreeAccel::primitives]{primitives}{}`.size()]);
\end{lstlisting}

在为创建的叶子确定估计的开销后,\refvar{KdTreeAccel::buildTree}{()}选择
一个轴尝试沿其划分并为每个候选划分计算开销函数。
{\ttfamily bestAxis}和{\ttfamily bestOffset}记录了该轴
以及目前给出最低开销{\ttfamily bestCost}的边界框边界索引。
{\ttfamily invTotalSA}初始化为节点表面积的倒数;
当计算光线穿过每个候选孩子节点的概率时会用到它的值。
\begin{lstlisting}
`\initcode{Choose split axis position for interior node}{=}`
int bestAxis = -1, bestOffset = -1;
`\refvar{Float}{}` bestCost = `\refvar{Infinity}{}`;
`\refvar{Float}{}` oldCost = `\refvar{isectCost}{}` * `\refvar{Float}{}`(nPrimitives);
`\refvar{Float}{}` totalSA = nodeBounds.`\refvar{SurfaceArea}{}`();
`\refvar{Float}{}` invTotalSA = 1 / totalSA;
`\refvar{Vector3f}{}` d = nodeBounds.`\refvar{pMax}{}` - nodeBounds.`\refvar{pMin}{}`;
`\refcode{Choose which axis to split along}{}`
int retries = 0;
retrySplit:
`\refcode{Initialize edges for axis}{}`
`\refcode{Compute cost of all splits for axis to find best}{}`
\end{lstlisting}

该方法首先尝试沿具有最大空间范围的轴寻找一个划分;
如果成功,该选项则有助于给出形状上趋于方形的空间区域。
这在直觉上是合理的方法。
如果沿该轴没有成功找到好的划分,则回退并依次尝试其他的。
\begin{lstlisting}
`\initcode{Choose which axis to split along}{=}`
int axis = nodeBounds.`\refvar{MaximumExtent}{}`();
\end{lstlisting}

首先用重合图元的边界框初始化该轴的数组{\ttfamily edges}。
然后该数组沿该轴从低到高存储,这样它就能从头到尾扫掠框的边界。
\begin{lstlisting}
`\initcode{Initialize edges for axis}{=}`
for (int i = 0; i < nPrimitives; ++i) {
    int pn = primNums[i];
    const `\refvar{Bounds3f}{}` &bounds = allPrimBounds[pn];
    edges[axis][2 * i] =     `\refvar{BoundEdge}{}`(bounds.`\refvar{pMin}{}`[axis], pn, true);
    edges[axis][2 * i + 1] = `\refvar{BoundEdge}{}`(bounds.`\refvar{pMax}{}`[axis], pn, false);
}
`\refcode{Sort edges for axis}{}`
\end{lstlisting}

C++标准库例程{\ttfamily sort()}要求被排序的结构要定义顺序;
这用值\refvar{BoundEdge::t}{}
来完成。然而,一个细微之处是如果值\refvar{BoundEdge::t}{}相同,
则需要通过比较节点类型来打破平局;
这是必要的,因为{\ttfamily sort()}所取决的事实是
{\ttfamily a < b}和{\ttfamily b < a}都为{\ttfamily false}的唯一时刻是{\ttfamily a == b}。
\begin{lstlisting}
`\initcode{Sort edges for axis}{=}`
std::sort(&edges[axis][0], &edges[axis][2*nPrimitives],
    [](const `\refvar{BoundEdge}{}` &e0, const `\refvar{BoundEdge}{}` &e1) -> bool {
        if (e0.`\refvar[BoundEdge::t]{t}{}` == e1.`\refvar[BoundEdge::t]{t}{}`)
            return (int)e0.`\refvar[BoundEdge::type]{type}{}` < (int)e1.`\refvar[BoundEdge::type]{type}{}`;
        else return e0.`\refvar[BoundEdge::t]{t}{}` < e1.`\refvar[BoundEdge::t]{t}{}`; 
    });
\end{lstlisting}

有了排好序的边界数组,我们想为它们中的每一处划分快速计算开销函数。
光线穿过每个孩子节点的概率很容易用其表面积计算,
划分处每侧的图元数量由变量{\ttfamily nBelow}和{\ttfamily nAbove}跟踪。
我们想保持它们的值是最新的,这样如果我们在某次循环中选择在{\ttfamily edgeT}处划分,
{\ttfamily nBelow}会给出最终在划分平面之下的图元数量而{\ttfamily nAbove}则给出之上的数量
\footnote{当多个边界框面投影到轴上同一点时,这一特性在这些点处可能不成立。
然而此处的实现只会高估数量,而且更重要的是,
它会于在这些点的每一个上进行的多次循环中的某一次取到正确的值,
所以无论如何最终算法的功能是正确的。}。

在第一个边界处,依据定义所有图元必须在该边界之上,
所以{\ttfamily nAbove}初始化为{\ttfamily nPrimitives}且{\ttfamily nBelow}设为0.
当循环考虑在边界框范围尾部的划分时,
{\ttfamily nAbove}需要递减,因为该框以前一定在划分平面之上,
如果在该点完成划分则它不会再在平面之上。
同样,计算划分开销后,如果划分候选项在边界框范围的起点处,
则所有后续划分中该框都会在下侧。
在循环体开头和结尾的测试为这两种情况更新了图元数量。
\begin{lstlisting}
`\initcode{Compute cost of all splits for axis to find best}{=}`
int nBelow = 0, nAbove = nPrimitives;
for (int i = 0; i < 2 * nPrimitives; ++i) {
    if (edges[axis][i].`\refvar[BoundEdge::type]{type}{}` == `\refvar{EdgeType::End}{}`) --nAbove;
    `\refvar{Float}{}` edgeT = edges[axis][i].`\refvar[BoundEdge::t]{t}{}`;
    if (edgeT > nodeBounds.`\refvar{pMin}{}`[axis] &&
        edgeT < nodeBounds.`\refvar{pMax}{}`[axis]) {
        `\refcode{Compute cost for split at ith edge}{}`
    }
    if (edges[axis][i].`\refvar[BoundEdge::type]{type}{}` == `\refvar{EdgeType::Start}{}`) ++nBelow;
}
\end{lstlisting}

{\ttfamily belowSA}和{\ttfamily aboveSA}持有两个候选孩子边框的表面积;
通过把六个面的面积加在一起很容易将其算出来。
\begin{lstlisting}
`\initcode{Compute child surface areas for split at edgeT}{=}`
int otherAxis0 = (axis + 1) % 3, otherAxis1 = (axis + 2) % 3;
`\refvar{Float}{}` belowSA = 2 * (d[otherAxis0] * d[otherAxis1] +
                     (edgeT - nodeBounds.`\refvar{pMin}{}`[axis]) * 
                     (d[otherAxis0] + d[otherAxis1]));
`\refvar{Float}{}` aboveSA = 2 * (d[otherAxis0] * d[otherAxis1] +
                     (nodeBounds.`\refvar{pMax}{}`[axis] - edgeT) * 
                     (d[otherAxis0] + d[otherAxis1]));
\end{lstlisting}

有了所有这些信息,就可以计算特定划分的开销了。
\begin{lstlisting}
`\initcode{Compute cost for split at ith edge}{=}`
`\refcode{Compute child surface areas for split at edgeT}{}`
`\refvar{Float}{}` pBelow = belowSA * invTotalSA; 
`\refvar{Float}{}` pAbove = aboveSA * invTotalSA;
`\refvar{Float}{}` eb = (nAbove == 0 || nBelow == 0) ? emptyBonus : 0;
`\refvar{Float}{}` cost = `\refvar{traversalCost}{}` + 
             `\refvar{isectCost}{}` * (1 - eb) * (pBelow * nBelow + pAbove * nAbove);
`\refcode{Update best split if this is lowest cost so far}{}`
\end{lstlisting}

如果为该候选划分计算的开销是目前最好的,则记录该划分的细节。
\begin{lstlisting}
`\initcode{Update best split if this is lowest cost so far}{=}`
if (cost < bestCost)  {
    bestCost = cost;
    bestAxis = axis;
    bestOffset = i;
}
\end{lstlisting}

可能在之前的测试中没有找到可行的划分(\reffig{4.16}展示了一种可能发生的情况)。
这种情况下,沿当前轴不存在可以把该节点划分开的单个候选位置。
这时,依次尝试另外两轴的划分。
(当{\ttfamily retries}等于2时)如果它们也没有找到划分,
则没有有用的方式细化该节点,因为两个孩子都仍会有同样多的重合图元。
当这种条件发生时,所能做的就是放弃并构建一个叶子节点。
\begin{figure}[htbp]
    \centering\input{Pictures/chap04/Overlappingbboxes.tex}
    \caption{如果多个边界框(虚线)如上所示与一个kd树节点(实线)重合,
    则不可能有划分位置能让其两侧的图元比总和更少。}
    \label{fig:4.16}
\end{figure}

也可能最佳划分的开销仍高于根本不划分该节点的开销。
如果它差得多且图元也不太多,就立即创建叶子节点。
否则,{\ttfamily badRefines}保持追踪目前在树的当前节点以上已经做了多少次不良划分。
允许稍微差些的细化是值得的,因为要考虑的图元子集更小后,之后的划分可能会找到更好的结果。
\begin{lstlisting}
`\initcode{Create leaf if no good splits were found}{=}`
if (bestAxis == -1 && retries < 2) {
    ++retries;
    axis = (axis + 1) % 3;
    goto retrySplit;
}
if (bestCost > oldCost) ++badRefines;
if ((bestCost > 4 * oldCost && nPrimitives < 16) || 
    bestAxis == -1 || badRefines == 3) {
    nodes[nodeNum].`\refvar[KdAccelNode::InitLeaf]{InitLeaf}{}`(primNums, nPrimitives, &`\refvar{primitiveIndices}{}`);
    return; 
}
\end{lstlisting}

选好划分位置后,按照之前代码中跟踪{\ttfamily nBelow}和{\ttfamily nAbove}的同样方式,
边界框的边界可用于把图元分为在划分处上方、下方或同在两侧。
注意下面的循环中跳过了数组中的项{\ttfamily bestOffset};
这是必要的,这样边界框被用作划分处的图元不会被错误地分类为同时位于划分处的两侧。
\begin{lstlisting}
`\initcode{Classify primitives with respect to split}{=}`
int n0 = 0, n1 = 0;
for (int i = 0; i < bestOffset; ++i)
    if (edges[bestAxis][i].`\refvar[BoundEdge::type]{type}{}` == `\refvar{EdgeType::Start}{}`)
        prims0[n0++] = edges[bestAxis][i].`\refvar[BoundEdge::primNum]{primNum}{}`;
for (int i = bestOffset + 1; i < 2 * nPrimitives; ++i)
    if (edges[bestAxis][i].`\refvar[BoundEdge::type]{type}{}` == `\refvar{EdgeType::End}{}`)
        prims1[n1++] = edges[bestAxis][i].`\refvar[BoundEdge::primNum]{primNum}{}`;
\end{lstlisting}

回想在kd树节点数组中该节点的“下方”孩子的节点序数是当前节点序数加一。
在递归从树的这一侧返回后,偏移量\refvar{nextFreeNode}{}被用于“上方”孩子。
这里唯一的重要细节是内存{\ttfamily prims0}被直接传入给两个孩子复用,
而{\ttfamily prims1}指针则首先向前推进
\sidenote{译者注:这段内容较难,笔者的理解是:由于\refvar{KdTreeAccel::buildTree}{()}在构建
过程中会原位修改传入的{\ttfamily prims0}和{\ttfamily prims1},所以需要保护现场。
左子树构建完成后,{\ttfamily prims0}的内容就没有用处了,可在右子树构建时被覆盖;
但构建左子树时不能变动还未构建的右子树所需的{\ttfamily prims1},
所以需要挪到新存储位置{\ttfamily prims1 + nPrimitives}。}。
这是必要的,因为当前对\refvar{KdTreeAccel::buildTree}{()}的调用取决于
下文中它首次递归调用\refvar{KdTreeAccel::buildTree}{()}时贮藏的{\ttfamily prims1}值,
毕竟它必须作为参数传给第二次调用。
然而,在首次递归调用立即使用之后就没有相应的必要贮藏{\ttfamily edges}值或贮藏{\ttfamily prims0}了。
\begin{lstlisting}
`\initcode{Recursively initialize children nodes}{=}`
`\refvar{Float}{}` tSplit = edges[bestAxis][bestOffset].`\refvar[BoundEdge::t]{t}{}`;
`\refvar{Bounds3f}{}` bounds0 = nodeBounds, bounds1 = nodeBounds;
bounds0.`\refvar{pMax}{}`[bestAxis] = bounds1.`\refvar{pMin}{}`[bestAxis] = tSplit;
`\refvar[KdTreeAccel::buildTree]{buildTree}{}`(nodeNum + 1, bounds0, allPrimBounds, prims0, n0,
          depth - 1, edges, prims0, prims1 + nPrimitives, badRefines);
int aboveChild = `\refvar{nextFreeNode}{}`;
nodes[nodeNum].`\refvar[KdAccelNode::InitInterior]{InitInterior}{}`(bestAxis, aboveChild, tSplit);
`\refvar[KdTreeAccel::buildTree]{buildTree}{}`(aboveChild, bounds1, allPrimBounds, prims1, n1, 
          depth - 1, edges, prims0, prims1 + nPrimitives, badRefines);
\end{lstlisting}

因此,整数数组{\ttfamily prims1}比数组{\ttfamily prims0}需要
多得多的空间存储最坏情况下重合图元可能的数目,
后者只需要一次处理单个层级的图元。
\begin{lstlisting}
`\refcode{Allocate working memory for kd-tree construction}{+=}\lastcode{Allocateworkingmemoryforkdtreeconstruction}`
std::unique_ptr<int[]> prims0(new int[`\refvar[KdTreeAccel::primitives]{primitives}{}`.size()]);
std::unique_ptr<int[]> prims1(new int[(maxDepth+1) * `\refvar[KdTreeAccel::primitives]{primitives}{}`.size()]);
\end{lstlisting}

\subsection{遍历}\label{sub:遍历2}
\reffig{4.17}展示了光线遍历树的基本过程
\sidenote{译者注:原文中该图题注将左右孩子写反了,已修正。}。
让光线与树的整体边框相交给出了初始的{\ttfamily tMin}和{\ttfamily tMax}值,即图中标出的点。
像本章的\refvar{BVHAccel}{}那样,如果光线错开了整体图元边框,
则该方法可立即返回{\ttfamily false}。
否则,它从根开始下沉到树中。
在每个内部节点处,它确定光线首先进入两个孩子中的哪个并按顺序处理两个孩子。
当光线退出树或者找到最近相交处时遍历结束。
\begin{figure}[htbp]
    \centering\input{Pictures/chap04/kdraytraversal.tex}
    \caption{光线遍历穿过kd树。(a)光线与树的边框相交,
    给出了要考虑的初始参数范围$[t_{\min},t_{\max}]$.
    (b)因为该范围非空,所以这里需要考虑根节点的两个孩子。
    光线首先进入左侧孩子,标记为“near”,有参数范围$[t_{\min},t_{\text{split}}]$.
    如果近处节点是含有图元的叶子节点,则执行光线-图元相交测试;
    否则处理其孩子节点。(c)如果该节点内没有找到命中处,或者找到的命中处
    超出了$[t_{\min},t_{\text{split}}]$,则处理右边的远处节点。
    (d)继续该过程——按深度优先处理树节点,从前往后遍历——直到
    求得最近相交处或者光线退出该树。}
    \label{fig:4.17}
\end{figure}
\begin{lstlisting}
`\refcode{KdTreeAccel Method Definitions}{+=}\lastcode{KdTreeAccelMethodDefinitions}`
bool `\refvar{KdTreeAccel}{}`::`\initvar[KdTreeAccel::Intersect]{Intersect}{}`(const `\refvar{Ray}{}` &ray,
        `\refvar{SurfaceInteraction}{}` *isect) const {
    `\refcode{Compute initial parametric range of ray inside kd-tree extent}{}`
    `\refcode{Prepare to traverse kd-tree for ray}{}`
    `\refcode{Traverse kd-tree nodes in order for ray}{}`
}
\end{lstlisting}

算法从寻找光线与树重合的整体参数范围$[t_{\min},t_{\max}]$开始,
如果没有重合则立即退出。
\begin{lstlisting}
`\initcode{Compute initial parametric range of ray inside kd-tree extent}{=}`
`\refvar{Float}{}` tMin, tMax;
if (!bounds.`\refvar[Bounds3::IntersectP]{IntersectP}{}`(ray, &tMin, &tMax)) 
    return false;
\end{lstlisting}

结构体\refvar{KdToDo}{}数组用于记录该光线目前要处理的节点;
它排了序使得数组中最后一个活跃项是应该考虑的下一个节点。
该数组中需要的最大项数是kd树的最大深度;
下文所用的数组大小在实践中应该是绰绰有余的。
\begin{lstlisting}
`\initcode{Prepare to traverse kd-tree for ray}{=}`
`\refvar{Vector3f}{}` invDir(1 / ray.`\refvar[Ray::d]{d}{}`.x, 1 / ray.`\refvar[Ray::d]{d}{}`.y, 1 / ray.`\refvar[Ray::d]{d}{}`.z);
constexpr int maxTodo = 64;
`\refvar{KdToDo}{}` todo[maxTodo];
int todoPos = 0;
\end{lstlisting}
\begin{lstlisting}
`\refcode{KdTreeAccel Declarations}{+=}\lastcode{KdTreeAccelDeclarations}`
struct `\initvar{KdToDo}{}` {
    const `\refvar{KdAccelNode}{}` *`\initvar[KdToDo::node]{node}{}`;
    `\refvar{Float}{}` `\initvar[KdToDo::tMin]{tMin}{}`, `\initvar[KdToDo::tMax]{tMax}{}`;
};
\end{lstlisting}

遍历继续穿过节点,循环中每次处理单个叶子或内部节点。
值\refvar[KdToDo::tMin]{tMin}{}和\refvar[KdToDo::tMax]{tMax}{}总是持有
光线与当前节点重合的参数范围。
\begin{lstlisting}
`\initcode{Traverse kd-tree nodes in order for ray}{=}`
bool hit = false;
const `\refvar{KdAccelNode}{}` *node = &`\refvar[KdTreeAccel::nodes]{nodes}{}`[0];
while (node != nullptr) {
    `\refcode{Bail out if we found a hit closer than the current node}{}`
    if (!node->`\refvar{IsLeaf}{}`()) {
        `\refcode{Process kd-tree interior node}{}`
    } else {
        `\refcode{Check for intersections inside leaf node}{}`
        `\refcode{Grab next node to process from todo list}{}`
    }
}
return hit;
\end{lstlisting}

对于和多个节点重合的图元可能以前就找到相交处了。
首次检测到时如果相交处在当前节点之外,
则有必要继续遍历树直到我们遇到一个节点的{\ttfamily tMin}超过相交处。
只有这时才能确定和其他图元不会再有更近的相交处了。
\begin{lstlisting}
`\initcode{Bail out if we found a hit closer than the current node}{=}`
if (ray.`\refvar{tMax}{}` < tMin) break;
\end{lstlisting}

对于内部节点要做的第一件事是让光线与节点的划分平面相交;
有了交点后,我们可以确定需要处理一个还是两个孩子节点以及光线穿过它们的顺序。
\begin{lstlisting}
`\initcode{Process kd-tree interior node}{=}`
`\refcode{Compute parametric distance along ray to split plane}{}`
`\refcode{Get node children pointers for ray}{}`
`\refcode{Advance to next child node, possibly enqueue other child}{}`
\end{lstlisting}

按照在光线-边界框测试中和计算光线与轴对齐平面相交一样的方式计算到划分平面的参数距离。
循环中我们每次用预先计算的值{\ttfamily invDir}保存除数。
\begin{lstlisting}
`\initcode{Compute parametric distance along ray to split plane}{=}`
int axis = node->`\refvar[KdAccelNode::SplitAxis]{SplitAxis}{}`();
`\refvar{Float}{}` tPlane = (node->`\refvar{SplitPos}{}`() - ray.`\refvar[Ray::o]{o}{}`[axis]) * invDir[axis];
\end{lstlisting}

现在需要确定光线遇到孩子节点的顺序使得是沿光线按从前往后的顺序遍历树的。
\reffig{4.18}展示了该计算的几何结构。
射线端点关于划分平面的位置足够区分两种情况,
现在忽略光线实际上没有穿过两节点之一的情况。
射线端点位于划分平面上的罕见情况需仔细处理,
需要改用它的方向来区分两种情况。
\begin{figure}[htbp]
    \centering\input{Pictures/chap04/Raybelowabove.tex}
    \caption{射线端点关于划分平面的位置可用于确定该首先处理该节点的哪个孩子。
    如果射线如$\bm r_1$的端点在划分平面的“下方”一侧,
    则我们在处理上方孩子前应该先处理下方孩子,反之亦然。}
    \label{fig:4.18}
\end{figure}
\begin{lstlisting}
`\initcode{Get node children pointers for ray}{=}`
const `\refvar{KdAccelNode}{}` *firstChild, *secondChild;
int belowFirst = (ray.`\refvar[Ray::o]{o}{}`[axis] <  node->`\refvar{SplitPos}{}`()) ||
                 (ray.`\refvar[Ray::o]{o}{}`[axis] == node->`\refvar{SplitPos}{}`() && ray.`\refvar[Ray::d]{d}{}`[axis] <= 0);
if (belowFirst) {
    firstChild = node + 1;
    secondChild = &`\refvar[KdTreeAccel::nodes]{nodes}{}`[node->`\refvar{AboveChild}{}`()];
} else {
    firstChild = &`\refvar[KdTreeAccel::nodes]{nodes}{}`[node->`\refvar{AboveChild}{}`()];
    secondChild = node + 1;
}
\end{lstlisting}

该节点的两个孩子可能没有必要都处理。
\reffig{4.19}展示了一些光线只穿过一个孩子的配置。
光线绝不会同时错过两个孩子,因为否则当前内部节点就不该被访问到。
\begin{figure}[htbp]
    \centering\input{Pictures/chap04/kdskipanode.tex}
    \caption{节点的两个孩子不需要都处理的两种情况,因为光线没有与之重合。
    (a)上方光线与划分平面的相交超出了光线的$t_{\max}$位置,
    因此没有进入更远的孩子。下方光线背对划分平面,这由负的$t_{\text{split}}$值表示。
    (b)光线在其$t_{\min}$值之前与平面相交,意味着不需要处理近处孩子。}
    \label{fig:4.19}
\end{figure}

下面的代码中第一个{\ttfamily if}测试与\reffig{4.19}(a)对应:
如果可以证明由于光线背对平面或因$t_{\text{split}}>t_{\max}$没与节点重合,
即光线没有与远处节点重合,则只有近处节点需要处理。
\reffig{4.19}(b)展示了第二个{\ttfamily if}测试中类似的情况:
如果光线没与之重合,则可能不需要处理近处节点。
否则,{\ttfamily else}语句负责两个孩子都需要处理的情况;
接着会处理近处节点,而远处节点列入列表{\ttfamily todo}。
\begin{lstlisting}
`\initcode{Advance to next child node, possibly enqueue other child}{=}`
if (tPlane > tMax || tPlane <= 0)
    node = firstChild;
else if (tPlane < tMin)
    node = secondChild;
else {
    `\refcode{Enqueue secondChild in todo list}{}`
    node = firstChild;
    tMax = tPlane;
}
\end{lstlisting}
\begin{lstlisting}
`\initcode{Enqueue secondChild in todo list}{=}`
todo[todoPos].`\refvar[KdToDo::node]{node}{}` = secondChild;
todo[todoPos].`\refvar[KdToDo::tMin]{tMin}{}` = tPlane;
todo[todoPos].`\refvar[KdToDo::tMax]{tMax}{}` = tMax;
++todoPos;
\end{lstlisting}

如果当前节点是叶子,则对叶子里的图元执行相交测试。
\begin{lstlisting}
`\initcode{Check for intersections inside leaf node}{=}`
int nPrimitives = node->`\refvar[KdAccelNode::nPrimitives]{nPrimitives}{}`();
if (nPrimitives == 1) {
    const std::shared_ptr<`\refvar{Primitive}{}`> &p = `\refvar[KdTreeAccel::primitives]{primitives}{}`[node->`\refvar{onePrimitive}{}`];
    `\refcode{Check one primitive inside leaf node}{}`
} else {
    for (int i = 0; i < nPrimitives; ++i) {
        int index = `\refvar{primitiveIndices}{}`[node->`\refvar{primitiveIndicesOffset}{}` + i];
        const std::shared_ptr<`\refvar{Primitive}{}`> &p = `\refvar[KdTreeAccel::primitives]{primitives}{}`[index];
        `\refcode{Check one primitive inside leaf node}{}`
    }
}
\end{lstlisting}

处理单个图元就是把相交请求传给图元的事。
\begin{lstlisting}
`\initcode{Check one primitive inside leaf node}{=}`
if (p->`\refvar[Primitive::Intersect]{Intersect}{}`(ray, isect)) 
    hit = true;
\end{lstlisting}

在叶子节点做完相交测试后,从数组{\ttfamily todo}加载下一个要处理的节点。
如果没有剩下更多节点了,则光线穿过该树没有命中任何东西。
\begin{lstlisting}
`\initcode{Grab next node to process from todo list}{=}`
if (todoPos > 0) {
    --todoPos;
    node = todo[todoPos].`\refvar[KdToDo::node]{node}{}`;
    tMin = todo[todoPos].`\refvar[KdToDo::tMin]{tMin}{}`;
    tMax = todo[todoPos].`\refvar[KdToDo::tMax]{tMax}{}`;
}
else
    break;
\end{lstlisting}

像\refvar{BVHAccel}{}那样,此处没有展示\refvar{KdTreeAccel}{}对于
阴影射线的特殊化相交方法。它和方法\refvar[KdTreeAccel::Intersect]{Intersect}{()}类似
但调用的是方法\refvar{Primitive::IntersectP}{()}且一旦
找到任何相交处就返回{\ttfamily true}而不担心是否找到了最近的那个。
\begin{lstlisting}
`\initcode{KdTreeAccel Public Methods}{=}`
bool `\initvar[KdTreeAccel::IntersectP]{IntersectP}{}`(const `\refvar{Ray}{}` &ray) const;
\end{lstlisting}

\section{扩展阅读}\label{sec:扩展阅读04}
引入光线追踪算法之后,涌现了大量尝试寻找高效方法对其加速的研究,
主要是通过开发改进的光线追踪加速结构。
《\citetitle{10.5555/94788}》\citep{10.5555/94788}中Arvo和Kirk的章节
总结了1989年最新进展并为区分不同光线相交加速方法提供了优秀的分类方案。

\citet{Kirk88theray}引入了\keyindex{元层次}{meta-hierarchies}{}的统一原则。
它们证明了通过让实现的加速数据结构与场景图元遵照相同的接口,
很容易混合与匹配不同的相交加速框架。
pbrt遵循这一模型,因为\refvar{Aggregate}{}继承自基类\refvar{Primitive}{}。

\subsection{网格}\label{sub:网格}
\citet{4056861}引入了均匀网格,
即把场景边界分解为等长网格的空间细分方法。
\citet{10.2312:egtp.19871000}
以及\citet{Cleary1988}描述了更高效的网格遍历方法。
\citet{10.1145/37401.37417}描述了
对该方法的大量改进并证明了网格对于渲染极其复杂场景的用处。
\citet{Jevans1989:23}引入了层次化网格,
即含有许多图元的网格自我细化为小格。
\citet{cazals1995filtering}以及
\citet{576857}为层次化网格开发了更复杂的技术。

\citet{4061545}为网格的并行创建开发了高效算法。
他们的有趣发现之一是随着所用处理核数量的增长,
网格创建性能很快被有效内存带宽所限制。

选择最优网格分辨率对于从网格中获得优异性能很重要。
\citet{4342587}有该话题的优秀论文,
为完全自动化选择分辨率以及在使用层次化网格时决定何时细化为子网格提供了坚实基础。
他们用大量简化假设推导出理论结果,然后证明了这些结果渲染真实世界场景的适用性。
他们的论文也包括对该领域前人工作很好的筛选引用。

\citet{lagae2008compact}基于
哈希法\sidenote{译者注:即hashing,也称散列法。}为均匀网格
描述了一种新颖的表示,它具有的优良性质是不仅每个图元
拥有对网格的单个索引,而且每个网格也只有单个图元索引。
他们证明了该表示有很低的内存使用量且仍然非常高效。

\citet{4634613}证明在透视空间中构建网格,
即投影中心是相机或光源时,能让追踪相机或光源发出的光线高效得多。
尽管该方法需要多种加速结构,但从为不同种类光线专门设计的多种结构中获得的性能提升可以很高。
他们的方法也因在某种意义上是栅格化和光线追踪的中间地带而令人瞩目。

\subsection{包围盒层次}\label{sub:包围盒层次}
\citet{10.1145/360349.360354}首先建议为标准可见曲面确定算法使用包围盒来剔除物体集。
在此基础上,\citet{10.1145/800250.807479}首先为快速光线追踪
的场景表示开发了层次化数据结构,尽管他们的方法依赖于用户去定义层次。
\citet{10.1145/15922.15916}基于用厚板集定界物体实现了最早之一的实用物体细分方法。
\citet{4057175}描述了自动计算包围盒层次的首个算法。
尽管他们的算法是基于依据盒的表面积来估计光线与包围盒相交的概率,
但它比现代SAH BVH算法低效得多。

本章的\refvar{BVHAccel}{}实现基于\citet{4342588}以及\citet{4342598}描述的构建算法。
边界框测试则是\citet{10.1145/1198555.1198748}引入的。
\citet{10.1080/2151237X.2007.10129248}开发了甚至更高效的边界框测试,
当同一光线对许多边界框做相交测试时它进行额外的预计算以换取更高的性能;
我们把实现他们的方法留作习题。

pbrt中用的BVH遍历算法由多位研究者同时开发出来;
见\citet{bouloshaines2006}的批注了解更多细节和背景。
树遍历的另一选项是\citet{10.1145/15922.15916};
他们维护一个按光线距离排序的节点堆。
在单片存储\sidenote{译者注:原文on-chip memory。}数量相对有限的GPU上,
为每条光线维护一个将要访问的节点的栈可能会有极其高的内存开销。
\citet{10.1145/1071866.1071869}引入了
“无栈”\sidenote{译者注:原文stackless。}kd树遍历算法,
它周期性地从树根开始回溯和搜索以找到下一个要访问的节点而不是显式保存所有要访问的节点。
\citet{10.5555/1921479.1921496}对该方法做了大量改进,
减少了从树根重新遍历的频率并将该方法应用于BVH。

许多研究者已经为构建BVH后提升其质量开发了许多技术。
\citet{10.2312:EGWR:EGSR07:073-084}和\citet{4634624}提出了
对BVH做局部调整的算法,\citet{10.1145/2159616.2159649}在
一个动画的多个坐标系上复用BVH,通过更新包围运动物体的部分来保持其质量。
也见\citet{BittnerFast2013}、\citet{10.1145/2492045.2492055}
以及\citet{BITTNER2015135}了解该领域的最新工作。

当前大多数构建BVH方法都基于自顶向下的树构建,
首先创建树节点然后将图元划分到孩子中并继续递归。
\citet{4634626}证明了另一个方法,
他说明自底向上的构建即首先创建叶子然后聚为父亲节点是可行选项。
\citet{10.1145/2492045.2492054}开发了该方法高效得多的实现并
证明了其对并行实现的适应性。

BVH的一个缺点是即便少量与边界框重合的相对较大图元也会极大降低BVH的效率:
仅因为下沉到叶子的几何体的重合边界框\sidenote{译者注:此句翻译不确定。},
许多树节点就会重合。\citet{4342593}提出
“分割剪裁”\sidenote{译者注:原文split clipping。}的办法;
提出了树中每个图元只出现一次的约束,
且巨大输入图元的边界框被细分为更紧致的子框集再用于树的构建。
\citet{4634636}观察到有问题的图元是
那些相对于其表面积在其边界框内有大量空白空间的,
所以它们细分了最异常的三角形并报告有巨大性能提升。
\citet{10.1145/1572769.1572771}开发了
在BVH构建期间划分图元的方法,使得当发现SAH开销下降时可以只划分图元。
也见\citet{10.1145/1572769.1572772}理论上
优化BVH划分算法及其与之前方法的关系,
以及\citet{10.1145/2492045.2492055}改进
决定何时分割三角形的准则。
\citet{10.5555/2980009.2980014}开发了一种方法为长细几何体如毛发等构建BVH;
因为这类几何体相对于其边界框体积来说非常细,
在大多数加速结构上它一般都有很差的性能。

BVH的内存要求可以非常大。在我们的实现中,每个节点为32字节。
场景中每个图元至多需要2个BVH树节点,每个图元的总开销可以高达64字节。
\citet{10.5555/2383894.2383909}建议为BVH节点使用更紧实的表示,牺牲一些效率。
首先,他们量化了每个节点中保存的边界框,用8或16字节来编码其相对于节点父亲边界框的位置。
然后,他们使用了{\itshape 隐式索引}\sidenote{译者注:原文implicit indexing。},
其中节点$i$的孩子在节点数组中的位置为$2i$和$2i+1$(假设分支系数为$2\times$)。
他们证明节约了大量内存,而性能影响适中。
\citet{10.2312:PE:VMV:VMV10:227-234}开发了另一种在空间上高效的BVH表示。
也见\citet{10.5555/1839214.1839242}开发的BVH节点与三角网格的紧实表示。

\citet{10.1111/j.1467-8659.2006.00970.x}为
缓存高效的BVH和kd树布局提出了算法并展示了来自它们的性能提升。
也见\citet{10.5555/1121584}的书籍了解该话题的广泛讨论。

\citet{10.1111/j.1467-8659.2009.01377.x}引入了线性BVH。
\citet{10.5555/1921479.1921493}在树的上层用SAH开发了HLBVH推广型。
他们还注意到莫顿编码值的高位可用于高效寻找图元群集——两种思想都用于我们HLBVH的开发。
\citet{10.1145/2018323.2018333}对HLBVH引入了更多改进,
它们多数都针对GPU实现。

不像HLBVH路线,这里\refvar{BVHAccel}{}中的BVH构建实现没有并行化。
详见\citet{5669303}了解始终利用SAH进行高性能并行BVH构建的方法。

\subsection{kd树}\label{sub:kd树}
\citet{6429331}为光线相交计算引入使用了八叉树。
\citet{kaplan1985use}\sidenote{译者注:未能找到该文献信息。}首先提出了为光线追踪使用kd树。
\citeauthor{kaplan1985use}的树构建算法总是从中间划分节点;
\citet{MacDonald1990}引入了SAH方法,
用相对表面积估计光线-节点遍历概率。
\citet{Naylor1993:27}也写过关于构建优良kd树的一般问题。
\citet{HavranImproving2002}回顾了许多这些问题并介绍了有用的改进。
\citet{hurley2002fast}提出为完全为空的树节点
添加补贴\sidenote{译者注:原文bonus。}因子到SAH中,就像我们的实现中做的那样。
见\citet{Havran2000:PhD}的博士论文了解对于高性能kd树构建和遍历算法的出色综述。


\citet{10.1007/978-3-642-71071-1_4}首次为kd树开发了高效的光线遍历算法。
\citet{ArvoRay1988}也研究了该问题并在《{\itshape\citefield{ArvoRay1988}{journaltitle}}》中一篇笔记里中作了讨论。
\citet{SUNG1992271}为BSP树加速器描述了一种光线遍历算法的实现;
我们的\refvar{KdTreeAccel}{}遍历代码一定程度上是基于它们的。

pbrt中kd树构建算法的渐进复杂度是$O(n\log^2n)$.
\citet{4061547}证明了用一些额外繁琐的实现可以在$O(n\log n)$时间内构建kd-树;
它们为典型场景报告了$2$到$3\times$的构建时间加速。

光线追踪最好的kd树是用“完美划分”\sidenote{译者注:原文perfect splits。}构建的,
每一步中剪裁正插入到树中的图元以适应当前节点的边界。
这避免了一个问题,例如一个物体的边界框可能与节点的边界框相交因而被存储于其中,
然而该物体自己并没有与该节点的边界框相交。
该方法由\citet{HavranImproving2002}提出并
由\citet{hurley2002fast}以及\citet{4061547}进一步讨论。
也见\citet{4634623}。
即便用完美划分,大型图元仍可能存于许多个kd树叶子中;
\citet{10.1111/cgf.12241}建议在内部节点中存储一些图元以解决该问题。

kd树构建往往比BVH构建慢得多(尤其是如果使用了“完美划分”),
所以并行构建算法特别有意义。该领域的最新工作包括
\citet{10.1111/j.1467-8659.2007.01062.x}和
\citet{10.5555/1921479.1921492},
他们提出了对多处理器有良好扩展性的高效并行kd树构建算法。

\subsection{表面积启发法}\label{sub:表面积启发法2}
自\citet{MacDonald1990}把SAH引入到
光线追踪以来许多研究者已经钻研了对SAH的改进。
\citet{10.2312:egs.20091046}派生出的一个版本是把
光线在空间中均匀分布的假设替换为光线的起点均匀分布于场景的边界框中。
\citet{4634614}引入了新的SAH,
它导致事实上光线一般并不均匀分布但是它们许多都从单个点
或一组相邻点(分别为相机和光源)发出。\citet{4634625}展示了当使用
“邮箱”\sidenote{译者注:原文mailboxing。}优化时应该怎样修改SAH,
而\citet{VINKLER2012283}用关于图元可见性的假设来调整它们的SAH开销。
\citet{10.1111/j.1467-8659.2011.01861.x}派生出
“光线终止表面积启发法”\sidenote{译者注:原文ray termination surface area heuristic。}(RTSAH),
它们用其来为阴影射线调整BVH遍历顺序以更快找到与遮挡物的相交处。
也见\citet{10.2312:sre.20151164}调整SAH以在kd树
遍历时对正被遮挡的阴影射线负责。

计算SAH会开销很大,尤其是当考虑许多不同的划分或图元分割时。
该问题的一个办法是只在候选点子集处计算它——例如,
沿着pbrt中\refvar{BVHAccel}{}里用的桶方法的直线来。
\citet{hurley2002fast}为构建kd树推荐该方法,
而\citet{4061550}详细讨论了它。
\citet{10.1111/j.1467-8659.2007.01062.x}引入了
将三角形全范围而不仅仅是其形心归入统计\sidenote{译者注:原文binning。}的改进。

\citet{4061549}注意到例如若你只需在一点计算SAH,
则你不需要对图元排序而只需对它们做线性扫描以计算图元数量和该点的边界框。
他们证明了用基于其在许多独立点上的值得到的分段二次式来逼近SAH并
用它选择良好划分会得到高效的树。
\citet{4061550}用了类似的近似。

尽管SAH能得到非常高效的kd树和BVH,但明确的是它并不完美:
许多研究者已经注意到遇到有更高SAH估计开销的kd树或BVH比
具有最低估计开销的树给出更好性能的情况并不罕见。
\citet{10.1145/2492045.2492056}调查了他们的一些结果并
提出两个额外启发法帮助解决之;一个考虑了事实上大多数光线始于曲面——
光线起点实际上并不在场景中随机分布,另一个考虑了当多条光线一起穿过层级时
的SIMD\sidenote{译者注:single instruction multiple data,单指令流多数据流。}分散度。
尽管这些新层次在解释为什么给定的树提供了这样的性能上很有效,
但至今也不知道怎样将其与树构建算法搭配。

\subsection{加速结构的其他话题}\label{sub:加速结构的其他话题}
\citet{10.1145/357332.357335}讨论了
为包围盒使用不同形状的权衡方法并建议把物体投影到屏幕上
再使用$z$-缓存区渲染来为相机光线寻找相交处加速。

许多研究者已经研究了划分平面不必是轴对齐时一般BSP树的适用性,就像kd树的那样。
\citet{4342591}用预选的候选划分平面集来构建树,
然而因为比kd树更慢的构建阶段和更慢的遍历,他们的结果在实际中只接近kd树的性能。
\citet{4634637}展示了比现代kd树更快渲染场景的BSP实现但会花非常长的构建时间。

有很多让一组光线一起而不是每次一个遍历加速结构的技术。
该方法(“包追踪”\sidenote{译者注:原文packet tracing。})是高性能光线追踪的重要组成;
\refsub{包追踪}会更深入地讨论它。

动画图元给光线追踪器带来两个挑战:
第一,如果物体在移动则尝试在多帧动画上复用加速结构的渲染器必须更新加速结构。
\citet{10.2312:egst.20071056}展示了这种情况下怎样渐进更新BVH,
\citet{10.1111/j.1467-8659.2009.01497.x}建议创建相邻图元群集
然后构建这些群集的BVH(因此减轻BVH构建算法的负担)。
第二个问题是对于快速移动的图元,在它们在帧时间上整个运动的边界框可能非常大,
导致有许多不必要的光线-图元相交测试。
关于该问题的著名工作包括\citet{504}加上时间把光线追踪(以及加速的八叉树)推广到四维。
最近Gr\"{u}nschlo\ss{}等\parencite*{10.1145/2018323.2018334}
\sidenote{译者注:参考文献列表中该作者名字中的德文字母“\ss{}”错误显示为“SS”,
目前无法解决,下同,请读者见谅。同时欢迎提供解决办法!}
为运动图元开发了针对BVH的改进。
也见\citet{10.2312:egst.20071056}关于光线追踪动画场景的综述论文。

\citet{10.1145/37401.37409}提出了加速结构的新型方法,
他们引入了5D数据结构来同时基于3D空间和2D光线方向进行细分。
\citet{10.1111/j.1467-8659.2008.01269.x}为
使用三角网格描述的场景提出了另一个有趣的方法:
他们计算一个受约束的四面体网格划分\sidenote{译者注:原文tetrahedralization。},
其中模型的所有三角形面都在四面体网格划分中表示。
然后光线逐步穿过四面体直到它们与来自场景描述的三角形相交。
该方法仍慢于最新kd树和BVH数倍,却是思考该问题的一个新的有趣方式。

在kd树和BVH之间有个有趣的中间地带,树节点为每个子节点持有划分平面而不是只有单个划分平面。
例如,这一改进使得能在像kd树那样的加速结构中作物体划分,
把每个图元放入仅一个子树并允许子树重合,
但仍保留了kd树高效遍历的许多好处。
\citet{10.1007/978-3-642-72617-0_17}\sidenote{译者注:未找到原文引用文献的信息,
译文改用同年同名同第一作者但发表位置不同的文献。}首次
将该改进引入到kd树中用于排序空间数据,命名为“空间kd树”
\sidenote{译者注:原文spatial kd-tree。}(skd-tree)。
skd树最近已经被许多研究者应用到光线追踪,包括
\citet{10.1145/585740.585761}、\citet{10.1145/1283900.1283912}、
\citet{10.5555/2383894.2383912}、
\citet{4061548}以及\citet{2151237X.2006.10129224}。

当使用像网格或kd树的空间划分方法时,图元可能与结构的多个节点重合,
光线可能在其穿过该结构时多次与同一个图元做相交测试。
\citet{Arnaldi1987}以及
\citet{10.2312:egtp.19871000}开发了
“邮箱”技术来解决这个问题:每条光线都给定唯一的整数标识符,
每个图元都记录与之测试的最后一条光线的id。
如果该id匹配,则相交测试是不必要的,可以跳过它。

尽管很高效,但该方法不适用于多线程光线追踪器。
为了解决该问题,\citet{Benthin_2006}建议排序
每条光线的小哈希表以记录最近相交的图元。
\citet{shevtsov2007ray}维护了
最后$n$个相交图元id的小数组并在执行相交测试前线性搜索它。
尽管两种方法仍可能多次检出一些图元,但它们通常剔除了大多数冗余测试。

\input{content/chap0406.tex}
