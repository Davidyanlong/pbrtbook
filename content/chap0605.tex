\section{扩展阅读}\label{sec:扩展阅读06}

在他的开创性画板系统中,\citet{10.1145/1461551.1461591}是
第一个为计算机图形学使用投影矩阵的人。
\citet{10.1201/9781315365459}\sidenote{译者注:该书已有第四版\citep{10.1201/b22086}。}给出了
写得非常好的正交和透视投影矩阵的推导。
关于投影的其他优秀参考有\citet{10.5555/63448}的《\citetitle{10.5555/63448}》,
以及\citet{EBERLY2007}关于游戏引擎设计的书籍\sidenote{译者注:原文引用第一版,此处改为引用第二版。}。

\citet{4056910}使用独特的投影方法为Omnimax\textsuperscript{\textregistered}影院
\sidenote{译者注:即现在的IMAX影院。}生成图像。
本章的\refvar{EnvironmentCamera}{}与\citet{KENTON1992288}描述的相机模型相同。

\citet{10.1145/800224.806818,10.1145/357299.357300,10.1145/800059.801169}做了
计算机图形学中景深和运动模糊的早期工作。
Cook和合作者们基于薄透镜模型为这些效应开发了更精确的模型;
它即\refsub{薄透镜模型与景深}中用于景深计算的方法\citep{10.1145/800031.808590,10.1145/7529.8927}。
见\citet{10.2312:EGWR:EGSR07:121-126}关于
具有非针孔光圈的相机可采用的辐射度量类别的广泛分析。

\citet{10.1145/218380.218463}展示了怎样用光线追踪
模拟复杂相机透镜以建模真实相机的成像效应;
\refsec{逼真相机}中的\refvar{RealisticCamera}{}就基于他们的方法。
\citet{10.1111/j.1467-8659.2011.01851.x}改进了该模拟的大量细节,
合并了依赖波长的效应并一起实现衍射和眩光。
\refsub{出射光瞳}中我们模拟出射光瞳的方法和他们的一样。
见\citet{0321188780}和\citet{9780071476874}的书籍了解
对光学和透镜系统的精彩介绍。

\citet{10.1111/j.1467-8659.2012.03132.x}用多项式来
建模透镜对穿过它们的光线的影响;
它们可以从单个透镜的多项式近似中构造模拟整个透镜系统的多项式。
该方法节约了追踪穿过透镜的光线的计算成本,
不过对于复杂场景其开销相比于剩余的渲染计算量而言一般可以忽略。
\citet{10.1111/cgf.12301}提升了该方法的准确性并展示了怎样将该方法与双向路径追踪结合。

\citet{5280315}介绍了几乎完整模拟数字相机的实现,
包括模数转换\sidenote{译者注:原文analog-to-digital conversion。}和
该过程固有的像素度量值噪声。