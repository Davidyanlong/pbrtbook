\section{扩展阅读}\label{sec:扩展阅读07}
\subsection{采样理论与混叠}\label{sub:采样理论与混叠}
关于信号处理、采样、重建以及傅里叶变换的最好书籍之一
是\citeauthor{91250711}的《\citetitle{91250711}》\parencite*{91250711}。
\citeauthor{GLASSNER1995}的《\citetitle{GLASSNER1995}》\parencite*{GLASSNER1995}有
一系列章节是关于均匀和非均匀采样与重建的理论及其在计算机图形学中的应用。
详见\citet{993400}了解包括采样理论在内的采样数据插值技术历史的大量调研。
\citet{843002}也调研了包括最近不再纯粹关注带限函数在内的采样和重建理论近期进展。
详见\citet{4815542}了解该领域更新的研究。

\citet{10.1145/359863.359869}首先发现混叠是计算机生成图像中伪影的主要来源。
\citet{10.1145/7529.8927}以及\citet{10.1145/325334.325182}介绍了用
非均匀采样将混叠转化为噪声;他们的工作以\citet{10.1126/science.6867716}
研究猴子眼中光感受器\sidenote{译者注:原文photoreceptor。}分布的实验为基础。
\citeauthor{10.1145/325334.325182}还首先将像素滤波方程引入图形学并
用样本间的最小距离发明了泊松样本模式。
\citet{10.1145/325334.325179}基于统计测试开发了自适应采样技术
使算出的图像在给定的误差容忍度内。
\citeauthor{10.1145/37401.37410}为光线追踪广泛研究了采样模式。
他\cite*{10.1145/37401.37410}和\cite*{10.1145/122718.122736}年关于
该主题的SIGGRAPH论文有许多关键见解。

\citet{HECKBERT1990246}写了一篇论文解释了当给像素用浮点坐标时
可能存在的缺陷并发展出这里用的惯例。

\citet{10.1145/237170.237265}研究了实践中分层采样模式比随机模式能好多少,
被采样的函数越平滑,它们就越有效。对于变化很快的函数(例如复杂几何结构覆盖的像素区域),
复杂的分层模式并不比非分层的随机模式表现得更优。
因此,对于在高维函数图像中有复杂变化的场景,
花哨的采样方案相对于简单分层模式的优势变少了。

\citet{CHIU1994370}基于打乱一个典型扰动模式的$x$和$y$坐标,
提出了\keyindex{多重扰动的}{multijittered}{jitter扰动}2D采样技术,
结合了分层和拉丁超立方方法的特性。
最近,\citet{Kensler2013Pixar}表明用他们的方法给两个维度使用一样的重排
得到的结果比用独立重排好得多。他表明该方法比Sobol模式给出了更低偏差,
同时还保持了因为用了扰动样本而将混叠转化为噪声的可感知优点。

\citet{10.1111/j.1467-8659.2007.01100.x}调研了生成泊松圆盘
样本模式的最新成果并比较了各种算法生成的点集质量。
关于该领域的最近研究,尤其参见\citet{10.1080/2151237X.2006.10129217}
\sidenote{译者注:原文标记该文献为2005年,笔者查到为2006年,已修改。}、
\citet{10.1145/1179352.1141915}、\citet{10.1145/1399504.1360619}、
\citet{10.1145/1866158.1866189}、\citet{10.1145/1964921.1964944}、
以及\citet{10.1111/j.1467-8659.2012.03059.x}的论文。
然而我们注意到\citet{10.1145/122718.122736}的一项观察的重要性,
即$n$维泊松圆盘分布对于图形学中的一般积分问题并不是最理想的;
尽管图像平面上前两维的投影具有泊松圆盘的属性很有用,但重要的是
剩余维度会比泊松圆盘单独保证的质量分布得更广。
最近,\citet{10.1111/cgf.12725}提出了一种在投影到更低维子集上时
可保持其特征样本分离性的$n$维泊松圆盘样本构造方法。

pbrt不包含执行自适应采样的采样器——在图像变化大的部分取用更多样本。
尽管自适应采样已是活跃的研究领域,