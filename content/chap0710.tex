\section{扩展阅读}\label{sec:扩展阅读07}
\subsection{采样理论与混叠}\label{sub:采样理论与混叠}
关于信号处理、采样、重建以及傅里叶变换的最好书籍之一
是\citeauthor{91250711}的《\citetitle{91250711}》\parencite*{91250711}。
\citeauthor{GLASSNER1995}的《\citetitle{GLASSNER1995}》\parencite*{GLASSNER1995}有
一系列章节是关于均匀和非均匀采样与重建的理论及其在计算机图形学中的应用。
详见\citet{993400}了解包括采样理论在内的采样数据插值技术历史的大量调研。
\citet{843002}也调研了包括最近不再纯粹关注带限函数在内的采样和重建理论近期进展。
详见\citet{4815542}了解该领域更新的研究。

\citet{10.1145/359863.359869}首先发现混叠是计算机生成图像中伪影的主要来源。
\citet{10.1145/7529.8927}以及\citet{10.1145/325334.325182}介绍了用
非均匀采样将混叠转化为噪声;他们的工作以\citet{10.1126/science.6867716}
研究猴子眼中光感受器\sidenote{译者注:原文photoreceptor。}分布的实验为基础。
\citeauthor{10.1145/325334.325182}还首先将像素滤波方程引入图形学并
用样本间的最小距离发明了泊松样本模式。
\citet{10.1145/325334.325179}基于统计测试开发了自适应采样技术
使算出的图像在给定的误差容忍度内。
\citeauthor{10.1145/37401.37410}为光线追踪广泛研究了采样模式。
他\cite*{10.1145/37401.37410}和\cite*{10.1145/122718.122736}年关于
该主题的SIGGRAPH论文有许多关键见解。

\citet{HECKBERT1990246}写了一篇论文解释了当给像素用浮点坐标时
可能存在的缺陷并发展出这里用的惯例。

\citet{10.1145/237170.237265}研究了实践中分层采样模式比随机模式能好多少,
被采样的函数越平滑,它们就越有效。对于变化很快的函数(例如复杂几何结构覆盖的像素区域),
复杂的分层模式并不比非分层的随机模式表现得更优。
因此,对于在高维函数图像中有复杂变化的场景,
花哨的采样方案相对于简单分层模式的优势变少了。

\citet{CHIU1994370}基于打乱一个典型扰动模式的$x$和$y$坐标,
提出了\keyindex{多重扰动的}{multijittered}{jitter扰动}2D采样技术,
结合了分层和拉丁超立方方法的特性。
最近,\citet{Kensler2013Pixar}表明用他们的方法给两个维度使用一样的重排
得到的结果比用独立重排好得多。他表明该方法比Sobol模式给出了更低偏差,
同时还保持了因为用了扰动样本而将混叠转化为噪声的可感知优点。

\citet{10.1111/j.1467-8659.2007.01100.x}调研了生成泊松圆盘
样本模式的最新成果并比较了各种算法生成的点集质量。
关于该领域的最近研究,尤其参见\citet{10.1080/2151237X.2006.10129217}
\sidenote{译者注:原文标记该文献为2005年,笔者查到为2006年,已修改。}、
\citet{10.1145/1179352.1141915}、\citet{10.1145/1399504.1360619}、
\citet{10.1145/1866158.1866189}、\citet{10.1145/1964921.1964944}、
以及\citet{10.1111/j.1467-8659.2012.03059.x}的论文。
然而我们注意到\citet{10.1145/122718.122736}的一项观察的重要性,
即$n$维泊松圆盘分布对于图形学中的一般积分问题并不是最理想的;
尽管图像平面上前两维的投影具有泊松圆盘的属性很有用,但重要的是
剩余维度会比泊松圆盘单独保证的质量分布得更广。
最近,\citet{10.1111/cgf.12725}提出了一种在投影到更低维子集上时
可保持其特征样本分离性的$n$维泊松圆盘样本构造方法。

pbrt不包含执行自适应采样的采样器——在图像变化大的部分取用更多样本。
尽管自适应采样已经是活跃的研究领域,
但我们关于这些算法的经验是虽然大部分能在一些情况下工作得很好,
不过很少有能跨大范围场景保持稳定的。
自\citet{10.1145/325334.325179}、\citet{10.1145/15922.15902}以及
\citet{PURGATHOFER1987157}关于自适应采样的最初工作以来,
近年已发展出大量复杂有效的自适应采样方法。
著名工作包括\citet{10.1145/1360612.1360632},
他们不是只在图像位置上,而是在图像位置、时间和透镜位置构成的5D域上
自适应采样,并引入一种新颖的多维滤波方法;
\citet{10.1145/166117.166154},他们开发了专注于
渲染运动模糊的自适应采样和重建技术;
\citet{10.1145/1618452.1618486},他们基于小波为图像重建
开发了自适应采样算法。最近,\citet{10.1145/2487228.2487239}计算
5D成像(图像、时间与镜头散焦)的协方差并应用自适应采样和高质量重建,
\citet{10.1145/2641762}已将局部回归理论应用到该问题。

\citet{10.1145/122718.122735}用自适应采样定位了一个隐蔽的问题:
简而言之,如果一组样本既用于决定是否应该取用更多样本
又被加入图像,最终结果是\keyindex{有偏的}{biased}{}且
极限不会收敛到正确结果。\citet{10.1145/37401.37410}观察到
标准图像重建技术在出现自适应采样时会失败:滤波器一部分范围内
的一团密集样本的贡献可能对最终值有很大错误影响,这纯粹是
该区域所取样本的数量造成的。他描述了一个能解决该问题的多阶段矩形滤波器。

\keyindex{压缩感知}{compressed sensing}{}是近来的采样方法,
其所需采样率取决于信号的稀疏性,而不是其频率内容。
\citet{5432169}应用压缩感知来渲染,使其能以很低的采样率生成高质量图像。

\subsection{低偏差采样}\label{sub:低偏差采样}
\citet{10.2312:egtp.19911013}首先引入使用偏差来
评估计算机图形学中样本模式的质量。
\citet{Mitchell92raytracing}、\citet{Dobkin1993:9}
以及\citet{10.1145/234535.234536}建立起该工作。
Dobkin等的论文中一个重要观察是本章和其他工作所用的将偏差应用到
像素采样模式的矩形偏差\sidenote{译者注:原文box discrepancy。}度量
并不特别适合于度量采样模式在穿过像素的随机朝向边缘处的精度,
而应替代使用基于随机边缘的偏差度量。
该观察解释了为什么一些理论上很好的低偏差模式
在用于图像采样时并不如预期那样表现得很好。

\citet{Mitchell92raytracing}首篇关于偏差的论文引入了
为采样使用确定的低偏差序列的思想,为了有更低偏差而去掉了所有随机性。
这样的\keyindex{拟随机}{quasi-random}{}序列是第\refchap{蒙特卡罗积分}将要介绍的
\keyindex{拟蒙特卡罗法}{quasi-Monte Carlo method}{Monte Carlo method蒙特卡罗法}的基础。
\citet{10.1137/1.9781611970081}撰写了关于拟随机采样
和生成低偏差模式算法的开创性书籍。
关于最近的处理,详见\citet{dick_pillichshammer_2010}的出色书籍。

\citet{FAURE199247}介绍了一个确定性方法来计算置乱倒根的重排。本章函数
\refvar{ComputeRadicalInversePermutations}{()}使用了
更易实现且实际中工作得几乎一样好的随机重排。
\refsec{Halton采样器}和\refsec{Sobol采样器}中用于
在给定像素中计算样本索引的算法是Gr\"{u}nschlo\ss{}, Raab 和 Keller
\parencite*{10.1007/978-3-642-27440-4_21}介绍的。

\citet{10.1007/978-3-7091-7484-5_11,10.1145/258734.258769,Keller03strictlydeterministic}
与合作者为图形学中的各种应用研究了拟随机采样模式。
\refvar{ZeroTwoSequenceSampler}{}中用的$(0,2)$序列采样技术
是基于\citet{10.1111/1467-8659.00706}的论文。
$(0,2)$序列是称为$(t,s)$序列和$(t,m,s)$网络的一般类型低偏差序列的特例。
\citet{10.1137/1.9781611970081}与\citet{dick_pillichshammer_2010}作了进一步讨论。
一些\citet{10.1111/1467-8659.00706}的技术
基于了\citet{10.1007/978-3-642-56046-0_17}开发的算法。
\citet{Keller03strictlydeterministic}
\sidenote{译者注:原文此处还引用了一篇2006年的文献,但缺失了引文信息。}
认为因为低偏差模式倾向于比其他收敛得更快,所以
它们是生成高质量影像最高效的采样方法。

\refsec{最大化最小距离采样器}的\refvar{MaxMinDistSampler}{}基于了Gr\"{u}nschlo\ss{}与
合作者\parencite*{10.1007/978-3-642-04107-5_25,10.1007/978-3-540-74496-2_23}发现的
生成矩阵。\citet{SOBOL196786}介绍了\refsec{Sobol采样器}中用的生成矩阵簇;
\citet{Wächter_2008}的博士论文讨论了基2生成矩阵运算的高性能实现。
我们实现所用的Sobol生成矩阵是\citet{10.1137/070709359}导出的改进版。

\subsection{滤波与重建}\label{sub:滤波与重建}
\citet{10.1145/7529.8927}首先将高斯滤波器引入图形学。
\citet{10.1145/54852.378514}通过人类观察者的实验
研究了一系列滤波器以寻找最高效者;本章的\refvar{MitchellFilter}{}
就是它们选出的最优者。\citet{10.1145/800224.806784}研究的图像滤波方法
考虑了来自CRT\sidenote{译者注:即\keyindex{阴极射线管}{cathode-ray tube}{}。}像素
的高斯衰减重建特性的影响,而最近\citet{10.1889/1.1832941}介绍了
微软在LCD上显示文字的ClearType技术。
\citet{10.5555/2532129.2532161}最近应用重建技术
尝试最小化重建图像与原始连续图像间的误差,甚至出现间断时也如此。

已经有很多针对图像重采样应用的重建滤波器研究了。
虽然该应用和图像生成中的重建非均匀样本并不一样,但其许多经验是适用的。
\citet{10.5555/90767.90805}报告了在许多滤波器中,
加Lanczos窗的sinc滤波器给出了图像重采样的最佳结果。
\citet{10.1007/10704282_23}通过把一系列变换应用到图像上
为图像重采样测试了各种滤波器,这样如果执行了完美重采样则最终图像会和原始的一样。
他们还发现Lanczos窗(和其他少数一样)表现得不错,
而无窗时截断sinc给出了最差结果。该领域的其他工作
包括\citet{597800}与\citet{556504}的论文。

即使采样率固定,聪明的重建算法对提升图像质量也很有用。
例如见\citet{10.1145/1572769.1572787}用图像梯度
寻找跨越多个像素的边缘来为抗锯齿估计像素覆盖范围,
以及\citet{Guertin2014MotionBlur}为运动模糊开发的滤波方法。

\citet{55149}首先建议使用\keyindex{中值滤波器}{median filter}{filter滤波器},
即用一组样本的中位数来求每个像素的值,是种去噪技术。
最近,\citet{10.1145/2010324.1964950}、\citet{10.1145/2185520.2185547}、
\citet{10.1111/cgf.12029}、\citet{10.1145/2366145.2366214}、
\citet{10.1111/cgf.12219}、\citet{10.1145/2532708}、
\citet{10.1111/cgf.12415}以及\citet{10.1111/cgf.12587}
开发了滤波技术以减少用蒙特卡罗算法渲染的图像中的噪声。
\citet{10.1145/2766977}把\keyindex{机器学习}{machine learning}{}应用到
寻求高效去噪滤波器的问题并展示了令人印象深刻的结果。

\citet{jensen1995optimizing}观察到基于其表示的照明类型
来区分对像素值的贡献会更高效;可以对低频间接照明做不同于高频直接照明的滤波,
以减少最终图像中的噪声。他们基于该观察开发了高效的滤波技术。
