\section{译者补充:信号处理}\label{sec:译者补充:信号处理}
\begin{remark}
    本节内容不是原书内容,而是译者根据\citet{DigitalSignalProcessing}、
    \citet{enwiki:1115652231}补充的,请酌情参考和斧正。
\end{remark}

\subsection{单位冲激函数}\label{sub:单位冲激函数}
\begin{definition}
    数学中,\keyindex{狄拉克$\delta$分布}{Dirac delta distribution}{}是定义在实数域上的广义分布或函数。
    它在除零以外的点上都取零,且在整个实数域上的积分等于一。通常记作$\delta(\cdot)$.
\end{definition}

狄拉克$\delta$分布也称\keyindex{狄拉克$\delta$函数}{Dirac delta function}{},
简称\keyindex{$\delta$分布}{delta distribution}{}或
\keyindex{$\delta$函数}{delta function}{},
它最早由英国理论物理学家保罗·狄拉克(Paul Adrien Maurice Dirac)提出,
在物理和工程界有广泛应用,也称作\keyindex{单位冲激函数}{unit impulse function}{}。

单位冲激函数不是严格意义上的函数,但形式上遵守微积分运算法则。
可以将其视作在非零处取零,在零处取无穷大,即
\begin{align}
    \delta(t)\approx\left\{
    \begin{array}{ll}
        +\infty, & \text{当}t=0,     \\
        0,       & \text{当}t\neq 0.
    \end{array}
    \right.
\end{align}
且满足如下积分约束的函数:
\begin{align}
    \int_{-\infty}^{\infty}\delta(t)\mathrm{d}t=1\, .
\end{align}

依据单位冲激函数的定义,可推导出以下性质:
\begin{theorem}
    单位冲激函数具有缩放性质:对任意实数$\alpha\neq0$,有
    \begin{align}
        \delta(\alpha t)=\frac{\delta(t)}{|\alpha|}\, .
    \end{align}
\end{theorem}
\begin{corollary}
    单位冲激函数具有对称性,即
    \begin{align}
        \delta(t)=\delta(-t)\, .
    \end{align}
\end{corollary}
\begin{theorem}
    单位冲激函数具有时延性质,也称平移性质或筛选性质,
    即对于可积函数$f$,它可以采样出$t=\tau$处的值:
    \begin{align}
        \int_{-\infty}^{\infty}\delta(t-\tau)f(t)\mathrm{d}t=f(\tau)\, .
    \end{align}
\end{theorem}
\subsection{关于傅里叶变换的推导}\label{sub:关于傅里叶变换的推导}
\subsubsection*{矩形函数}
对于矩形函数
\begin{align}
    f(t)=\left\{\begin{array}{ll}
        1, & \displaystyle\text{若}|t|<\frac{1}{2}, \\
        0, & \text{其他}.
    \end{array}\right.
\end{align}
其频率表示为
\begin{align}
    F(\omega) & =\int_{-\infty}^{\infty}f(t)\mathrm{e}^{-\mathrm{i}2\pi\omega t}\mathrm{d}t
    =\int_{-\frac{1}{2}}^{\frac{1}{2}}\mathrm{e}^{-\mathrm{i}2\pi\omega t}\mathrm{d}t
    =-\frac{\mathrm{e}^{-\mathrm{i}2\pi\omega t}}{\mathrm{i}2\pi\omega}\bigg|_{t=-\frac{1}{2}}^{\frac{1}{2}}\nonumber \\
              & =-\frac{\mathrm{e}^{-\mathrm{i}\pi\omega}-\mathrm{e}^{\mathrm{i}\pi\omega}}{\mathrm{i}2\pi\omega}
    =\frac{\mathrm{i}2\sin(\pi\omega)}{\mathrm{i}2\pi\omega}
    =\frac{\sin(\pi\omega)}{\pi\omega}\, .
\end{align}

\subsubsection*{高斯函数}
对于高斯函数
\begin{align}
    f(t)=\mathrm{e}^{-\pi t^2}\, ,
\end{align}
其频率表示为
\begin{align}
    F(\omega) & =\int_{-\infty}^{\infty}f(t)\mathrm{e}^{-\mathrm{i}2\pi\omega t}\mathrm{d}t
    =\int_{-\infty}^{\infty}\mathrm{e}^{-\pi t^2}\mathrm{e}^{-\mathrm{i}2\pi\omega t}\mathrm{d}t
    =\int_{-\infty}^{\infty}\mathrm{e}^{-\pi((t+\mathrm{i}\omega)^2+\omega^2)}\mathrm{d}t\nonumber                  \\
              & =\mathrm{e}^{-\pi\omega^2}\int_{-\infty}^{\infty}\mathrm{e}^{-\pi(t+\mathrm{i}\omega)^2}\mathrm{d}t
    =\mathrm{e}^{-\pi\omega^2}\int_{-\infty}^{\infty}\mathrm{e}^{-\pi t^2}\mathrm{d}t
    =\mathrm{e}^{-\pi\omega^2}\, .
\end{align}
\subsubsection*{单位冲激函数}
对于单位冲激函数
\begin{align}
    f(t)=\delta(t)\, ,
\end{align}
其频率表示为
\begin{align}
    F(\omega)=\int_{-\infty}^{\infty}f(t)\mathrm{e}^{-\mathrm{i}2\pi\omega t}\mathrm{d}t
    =\int_{-\infty}^{\infty}\delta(t)\mathrm{e}^{-\mathrm{i}2\pi\omega t}\mathrm{d}t
    =\mathrm{e}^{-\mathrm{i}2\pi\omega\cdot0}
    =1\, .
\end{align}
\subsection*{单位常函数}
定义指数衰减函数为
\begin{align}
    f_a(t)=\mathrm{e}^{-a|t|},\quad (a>0)\, .
\end{align}
则单位常函数可视作指数衰减函数的极限,即
\begin{align}
    f(t)=\lim\limits_{a\rightarrow0^+}f_a(t)=1\, .
\end{align}
于是常函数的频率表示满足
\begin{align}
    F(\omega) & =\int_{-\infty}^{\infty}f(t)\mathrm{e}^{-\mathrm{i}2\pi\omega t}\mathrm{d}t
    =\int_{-\infty}^{\infty}\lim\limits_{a\rightarrow0^+}\mathrm{e}^{-a|t|}\mathrm{e}^{-\mathrm{i}2\pi\omega t}\mathrm{d}t
    =\lim\limits_{a\rightarrow0^+}\int_{-\infty}^{\infty}\mathrm{e}^{-a|t|-\mathrm{i}2\pi\omega t}\mathrm{d}t\nonumber                                                                                  \\
              & =\lim\limits_{a\rightarrow0^+}\left(\int_{-\infty}^0\mathrm{e}^{(a-\mathrm{i}2\pi\omega)t}\mathrm{d}t+\int_0^{\infty}\mathrm{e}^{-(a+\mathrm{i}2\pi\omega)t}\mathrm{d}t\right)\nonumber \\
              & =\lim\limits_{a\rightarrow0^+}\left(\frac{\mathrm{e}^{(a-\mathrm{i}2\pi\omega)t}}{a-\mathrm{i}2\pi\omega}\bigg|_{t=-\infty}^0
    +\frac{\mathrm{e}^{-(a+\mathrm{i}2\pi\omega)t}}{-(a+\mathrm{i}2\pi\omega)}\bigg|_{t=0}^{\infty}\right)\nonumber                                                                                     \\
              & =\lim\limits_{a\rightarrow0^+}\left(\frac{1}{a-\mathrm{i}2\pi\omega}+\frac{1}{a+\mathrm{i}2\pi\omega}\right)=\lim\limits_{a\rightarrow0^+}\frac{2a}{a^2+4\pi^2\omega^2}\nonumber        \\
              & =\left\{\begin{array}{ll}
        0,      & \text{若}\omega\neq0, \\
        \infty, & \text{若}\omega=0.
    \end{array}\right.
\end{align}
注意到该频率表示取极限的部分在实数域上积分与$a$无关且为
\begin{align}
    \int_{-\infty}^{\infty}\frac{2a}{a^2+4\pi^2\omega^2}\mathrm{d}\omega
    =\frac{1}{\pi}\int_{-\infty}^{\infty}\frac{1}{1+\left(\frac{2\pi\omega}{a}\right)^2}\mathrm{d}\frac{2\pi\omega}{a}
    =\frac{1}{\pi}\arctan\frac{2\pi\omega}{a}\bigg|_{\omega=-\infty}^{\infty}=1\, .
\end{align}
于是该频率表示实际上就是单位冲激函数,即
\begin{align}
    F(\omega)=\delta(\omega)\, .
\end{align}

\subsubsection*{傅里叶变换的性质}
\begin{theorem}
    傅里叶变换具有频移与时移性质,即对于傅里叶变换对$f(t)\leftrightarrow F(\omega)$,
    给定任意常数$\tau$和$\omega_0$,则有相应的变换对
    \begin{align}
        f(t)\mathrm{e}^{\mathrm{i}2\pi\omega_0 t} & \leftrightarrow F(\omega-\omega_0)\, ,                              \\
        f(t-\tau)                                 & \leftrightarrow F(\omega)\mathrm{e}^{-\mathrm{i}2\pi\omega\tau}\, .
    \end{align}
\end{theorem}
\begin{prove}
    对于时域表示$f(t)\mathrm{e}^{\mathrm{i}2\pi\omega_0 t}$,其傅里叶变换为
    \begin{align}
        \int_{-\infty}^{\infty}f(t)\mathrm{e}^{\mathrm{i}2\pi\omega_0 t}\mathrm{e}^{-\mathrm{i}2\pi\omega t}\mathrm{d}t
        =\int_{-\infty}^{\infty}f(t)\mathrm{e}^{-\mathrm{i}2\pi(\omega-\omega_0) t}\mathrm{d}t=F(\omega-\omega_0)\, .
    \end{align}
    对于频率表示$F(\omega)\mathrm{e}^{-\mathrm{i}2\pi\omega\tau}$,其傅里叶逆变换为
    \begin{align}
        \int_{-\infty}^{\infty}F(\omega)\mathrm{e}^{-\mathrm{i}2\pi\omega\tau}\mathrm{e}^{\mathrm{i}2\pi\omega t}\mathrm{d}\omega
        =\int_{-\infty}^{\infty}F(\omega)\mathrm{e}^{\mathrm{i}2\pi\omega(t-\tau)}\mathrm{d}\omega
        =f(t-\tau)\, .
    \end{align}
\end{prove}
\begin{theorem}
    傅里叶变换和逆变换互为逆运算,即
    \begin{align}
        \mathcal{F}^{-1}\{\mathcal{F}\{f(t)\}\}      & =f(t)\, ,      \\
        \mathcal{F}\{\mathcal{F}^{-1}\{F(\omega)\}\} & =F(\omega)\, .
    \end{align}
\end{theorem}

\begin{prove}
    利用时延性质和单位冲激函数的傅里叶变换对可得
    \begin{align}
        \mathcal{F}^{-1}\{\mathcal{F}\{f(t)\}\}= & \int_{-\infty}^{\infty}\left(\int_{-\infty}^{\infty}f(\tau)\mathrm{e}^{-\mathrm{i}2\pi\omega\tau}\mathrm{d}\tau\right)\mathrm{e}^{\mathrm{i}2\pi\omega t}\mathrm{d}\omega\nonumber \\
        =                                        & \int_{-\infty}^{\infty}f(\tau)\left(\int_{-\infty}^{\infty}\mathrm{e}^{\mathrm{i}2\pi\omega(t-\tau)}\mathrm{d}\omega\right)\mathrm{d}\tau\nonumber                                 \\
        =                                        & \int_{-\infty}^{\infty}f(\tau)\delta(t-\tau)\mathrm{d}\tau\nonumber                                                                                                                \\
        =                                        & f(t)\, .
    \end{align}
    第二个式子同理运用频移性质和单位常函数的傅里叶变换对可证。
\end{prove}

\subsubsection*{余弦函数}
对于余弦函数
\begin{align}
    f(t)=\cos t\, ,
\end{align}
其频率表示为
\begin{align}
    F(\omega) & =\int_{-\infty}^{\infty}f(t)\mathrm{e}^{-\mathrm{i}2\pi\omega t}\mathrm{d}t\nonumber                                                                                                \\
              & =\int_{-\infty}^{\infty}\mathrm{e}^{-\mathrm{i}2\pi\omega t}\cos t\mathrm{d}t\nonumber                                                                                              \\
              & =\int_{-\infty}^{\infty}\frac{1}{2}(\mathrm{e}^{\mathrm{i}t}+\mathrm{e}^{-\mathrm{i}t})\mathrm{e}^{-\mathrm{i}2\pi\omega t}\mathrm{d}t\nonumber                                     \\
              & =\frac{1}{2}\int_{-\infty}^{\infty}(\mathrm{e}^{\mathrm{i}2\pi\frac{1}{2\pi}t}+\mathrm{e}^{\mathrm{i}2\pi\frac{-1}{2\pi}t})\mathrm{e}^{-\mathrm{i}2\pi\omega t}\mathrm{d}t\nonumber \\
              & =\frac{1}{2}(\delta(\omega-\frac{1}{2\pi})+\delta(\omega+\frac{1}{2\pi}))\nonumber                                                                                                  \\
              & =\pi(\delta(1-2\pi\omega)+\delta(1+2\pi\omega))\, .
\end{align}

\subsubsection*{shah函数}
\begin{theorem}
    周期为$T$的函数$f(t)$可被展开为唯一的\keyindex{傅里叶级数}{Fourier series}{},其指数形式为
    \begin{align}
        f(t)=\sum\limits_{n=-\infty}^{\infty}a_n\mathrm{e}^{\mathrm{i}2\pi\frac{n}{T}t}\, ,
    \end{align}
    其中系数
    \begin{align}
        a_n=\frac{1}{T}\int\limits_T f(t)\mathrm{e}^{-\mathrm{i}2\pi\frac{n}{T}t}\mathrm{d}t\, .
    \end{align}
\end{theorem}

对于周期为$T$的shah函数
\begin{align}
    f(t)=T\sum\limits_{k=-\infty}^{\infty}\delta(t-kT)\, ,
\end{align}
其傅里叶展开中的系数为
\begin{align}
    a_n=\frac{1}{T}\int_{-\frac{T}{2}}^{\frac{T}{2}}f(t)\mathrm{e}^{-\mathrm{i}2\pi\frac{n}{T}t}\mathrm{d}t
    =\frac{1}{T}\int_{-\frac{T}{2}}^{\frac{T}{2}}T\delta(t)\mathrm{e}^{-\mathrm{i}2\pi\frac{n}{T}t}\mathrm{d}t
    =1\, .
\end{align}
于是shah函数可展开为
\begin{align}
    f(t)=\sum\limits_{n=-\infty}^{\infty}\mathrm{e}^{\mathrm{i}2\pi\frac{n}{T}t}\, .
\end{align}
因此其频域表示为
\begin{align}
    F(\omega) & =\int_{-\infty}^{\infty}f(t)\mathrm{e}^{-\mathrm{i}2\pi\omega t}\mathrm{d}t\nonumber                                                                    \\
              & =\int_{-\infty}^{\infty}\sum\limits_{n=-\infty}^{\infty}\mathrm{e}^{\mathrm{i}2\pi\frac{n}{T}t}\mathrm{e}^{-\mathrm{i}2\pi\omega t}\mathrm{d}t\nonumber \\
              & =\sum\limits_{n=-\infty}^{\infty}\int_{-\infty}^{\infty}\mathrm{e}^{-\mathrm{i}2\pi(\omega-\frac{n}{T})t}\mathrm{d}t\nonumber                           \\
              & =\sum\limits_{n=-\infty}^{\infty}\delta(\omega-\frac{n}{T})\, .
\end{align}