\section{材质接口与实现}\label{sec:材质接口与实现}

\begin{lstlisting}
`\initcode{Material Declarations}{=}`
class `\initvar{Material}{}` {
public:
    `\refcode{Material Interface}{}`
};
\end{lstlisting}

\begin{lstlisting}
`\initcode{Material Interface}{=}`
virtual void `\initvar[Material::ComputeScatteringFunctions]{ComputeScatteringFunctions}{}`(`\refvar{SurfaceInteraction}{}` *si,
    `\refvar{MemoryArena}{}` &arena, `\refvar{TransportMode}{}` mode,
    bool allowMultipleLobes) const = 0;
\end{lstlisting}

\begin{lstlisting}
`\refcode{SurfaceInteraction Method Definitions}{+=}\lastnext{SurfaceInteractionMethodDefinitions}`
void `\refvar{SurfaceInteraction}{}`::`\initvar[SurfaceInteraction::ComputeScatteringFunctions]{\refvar{ComputeScatteringFunctions}{}}{}`(
    const `\refvar{RayDifferential}{}` &ray, `\refvar{MemoryArena}{}` &arena,
    bool allowMultipleLobes, `\refvar{TransportMode}{}` mode) {
    `\refvar{ComputeDifferentials}{}`(ray);
    primitive->`\refvar[Primitive::ComputeScatteringFunctions]{ComputeScatteringFunctions}{}`(this, arena, mode,
    allowMultipleLobes);
}
\end{lstlisting}

\subsection{哑光材料}\label{sub:哑光材料}
\begin{lstlisting}
`\initcode{MatteMaterial Declarations}{=}`
class `\initvar{MatteMaterial}{}` : public `\refvar{Material}{}` {
public:
    `\refcode{MatteMaterial Public Methods}{}`
private:
    `\refcode{MatteMaterial Private Data}{}`
};
\end{lstlisting}

\subsection{混合材料}\label{sub:混合材料}
\subsection{附加材料}\label{sub:附加材料}
\begin{figure}[htbp]
    \centering
    \includegraphics[width=\linewidth]{Pictures/chap09/dragon-metal.png}
    \caption{用\refvar{MetalMaterial}{}渲染的龙,它基于实测的金属散射数据
        (感谢Christian Schüller提供模型)。}
    \label{fig:9.4}
\end{figure}