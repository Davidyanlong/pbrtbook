%%%%%%%%%%%%%%%%%%%%%%%%%%%%%%%%%%%%%%%%%
% The Legrand Orange Book
% LaTeX Template
% Version 2.4 (26/09/2018)
%
% This template was downloaded from:
% http://www.LaTeXTemplates.com
%
% Original author:
% Mathias Legrand (legrand.mathias@gmail.com) with modifications by:
% Vel (vel@latextemplates.com)
%
% License:
% CC BY-NC-SA 3.0 (http://creativecommons.org/licenses/by-nc-sa/3.0/)
%
% Compiling this template:
% This template uses biber for its bibliography and makeindex for its index.
% When you first open the template, compile it from the command line with the 
% commands below to make sure your LaTeX distribution is configured correctly:
%
% 1) pdflatex main
% 2) makeindex main.idx -s StyleInd.ist
% 3) biber main
% 4) pdflatex main x 2
%
% After this, when you wish to update the bibliography/index use the appropriate
% command above and make sure to compile with pdflatex several times 
% afterwards to propagate your changes to the document.
%
% This template also uses a number of packages which may need to be
% updated to the newest versions for the template to compile. It is strongly
% recommended you update your LaTeX distribution if you have any
% compilation errors.
%
% Important note:
% Chapter heading images should have a 2:1 width:height ratio,
% e.g. 920px width and 460px height.
%
%%%%%%%%%%%%%%%%%%%%%%%%%%%%%%%%%%%%%%%%%

%----------------------------------------------------------------------------------------
%	PACKAGES AND OTHER DOCUMENT CONFIGURATIONS
%----------------------------------------------------------------------------------------

\documentclass[11pt,twoside]{book} % Default font size and left-justified equations

%%%%%%%%%%%%%%%%%%%%%%%%%%%%%%%%%%%%%%%%%
% The Legrand Orange Book
% Structural Definitions File
% Version 2.1 (26/09/2018)
%
% Original author:
% Mathias Legrand (legrand.mathias@gmail.com) with modifications by:
% Vel (vel@latextemplates.com)
% 
% This file was downloaded from:
% http://www.LaTeXTemplates.com
%
% License:
% CC BY-NC-SA 3.0 (http://creativecommons.org/licenses/by-nc-sa/3.0/)
%
%%%%%%%%%%%%%%%%%%%%%%%%%%%%%%%%%%%%%%%%%

%----------------------------------------------------------------------------------------
%	VARIOUS REQUIRED PACKAGES AND CONFIGURATIONS
%----------------------------------------------------------------------------------------

\usepackage{graphicx} % Required for including pictures
\graphicspath{{Pictures/}} % Specifies the directory where pictures are stored
\usepackage{subfig}
\usepackage{tikz} % Required for drawing custom shapes
\usepackage{pstricks}
\usepackage[english]{babel} % English language/hyphenation

\usepackage{enumitem} % Customize lists
\setlist{nolistsep} % Reduce spacing between bullet points and numbered lists

\usepackage{booktabs} % Required for nicer horizontal rules in tables
\usepackage{bm}
\usepackage{xcolor} % Required for specifying colors by name
\definecolor{ocre}{RGB}{243,102,25} % Define the orange color used for highlighting throughout the book

%----------------------------------------------------------------------------------------
%	MARGINS
%----------------------------------------------------------------------------------------

\usepackage{geometry} % Required for adjusting page dimensions and margins
\geometry{
	paper=a4paper, % Paper size, change to letterpaper for US letter size
	top=3cm, % Top margin
	bottom=3cm, % Bottom margin
	inner=3.2cm, % Left margin
	outer=4.8cm, % Right margin
	headheight=14pt, % Header height
	footskip=1.4cm, % Space from the bottom margin to the baseline of the footer
	headsep=10pt, % Space from the top margin to the baseline of the header
	%showframe, % Uncomment to show how the type block is set on the page
}

%----------------------------------------------------------------------------------------
%	FONTS
%----------------------------------------------------------------------------------------
\usepackage[UTF8]{ctex}
\setCJKmainfont[Path=fonts/,BoldFont={SourceHanSerifSC-Bold.otf},ItalicFont={simkai.ttf}]{SourceHanSerifSC-Regular.otf}
\setCJKsansfont[Path=fonts/,BoldFont={SourceHanSansSC-Bold.otf}]{SourceHanSansSC-Regular.otf}
\usepackage{avant} % Use the Avantgarde font for headings
%\usepackage{times} % Use the Times font for headings
\usepackage{mathptmx} % Use the Adobe Times Roman as the default text font together with math symbols from the Sym­bol, Chancery and Com­puter Modern fonts

\usepackage{microtype} % Slightly tweak font spacing for aesthetics
\usepackage[utf8]{inputenc} % Required for including letters with accents
\usepackage[T1]{fontenc} % Use 8-bit encoding that has 256 glyphs

%---------------------------------------------------------------------------------------
% use sidenotes but fix its bug to change font size
% see: https://tex.stackexchange.com/questions/532245/how-to-modify-fonts-in-sidenotes
%---------------------------------------------------------------------------------------
\usepackage{sidenotes}
\usepackage{xparse}
\let\oldmarginpar\marginpar
\RenewDocumentCommand{\marginpar}{om}{%
	\IfNoValueTF{#1}
	{\oldmarginpar{\mymparsetup #2}}
	{\oldmarginpar[\mymparsetup #1]{\mymparsetup #2}}}
\newcommand{\mymparsetup}{\scriptsize\itshape}

%---------------------------------------------------------------------------------------
% use \eg ... see: https://stackoverflow.com/a/39363004/12128185
%---------------------------------------------------------------------------------------
\usepackage{xspace}
\makeatletter
\DeclareRobustCommand\onedot{\futurelet\@let@token\@onedot}
\def\@onedot{\ifx\@let@token.\else.\null\fi\xspace}
\def\eg{\emph{e.g}\onedot} \def\Eg{\emph{E.g}\onedot}
\def\ie{\emph{i.e}\onedot} \def\Ie{\emph{I.e}\onedot}
\def\cf{\emph{c.f}\onedot} \def\Cf{\emph{C.f}\onedot}
\def\etc{\emph{etc}\onedot} \def\vs{\emph{vs}\onedot}
\def\wrt{w.r.t\onedot} \def\dof{d.o.f\onedot}
\def\etal{\emph{et al}\onedot}
\makeatother

%----------------------------------------------------------------------------------------
%	BIBLIOGRAPHY AND INDEX
%----------------------------------------------------------------------------------------

\usepackage[style=authoryear,citestyle=authoryear-comp,maxbibnames=99,maxcitenames=3,mincitenames=1,uniquename=false,sorting=nyt,sortcites=true,autopunct=true,babel=hyphen,hyperref=true,abbreviate=false,backref=true,backend=biber,natbib=true]{biblatex}
\addbibresource{bibliography.bib} % BibTeX bibliography file
\defbibheading{bibempty}{}
\renewcommand*{\finalnamedelim}{{\ifcitation{和}{, and }}}
\renewcommand*{\andothersdelim}{\ifcitation{}{, }}
\DefineBibliographyStrings{english}{andothers={\ifcitation{等}{ et al.}}}
\usepackage{calc} % For simpler calculation - used for spacing the index letter headings correctly
\usepackage{makeidx} % Required to make an index
\makeindex % Tells LaTeX to create the files required for indexing
\usepackage{xifthen} % provides \isempty test
\newcommand\keyindex[3]{\ifthenelse{\isempty{#1}}{}{\ifthenelse{\isempty{#2}}{\ifthenelse{\isempty{#3}}{{\sffamily#1}}{{\sffamily#1}\index{#3!#1}}}{\ifthenelse{\isempty{#3}}{{\sffamily#1}(#2)\index{#2#1}}{{\sffamily#1}(#2)\index{#3!#2#1}}}}}

%----------------------------------------------------------------------------------------
%	MAIN TABLE OF CONTENTS
%----------------------------------------------------------------------------------------

\usepackage{titletoc} % Required for manipulating the table of contents

\contentsmargin{0cm} % Removes the default margin

% Part text styling (this is mostly taken care of in the PART HEADINGS section of this file)
\titlecontents{part}
[0cm] % Left indentation
{\addvspace{20pt}\bfseries} % Spacing and font options for parts
{}
{}
{}

% Chapter text styling
\titlecontents{chapter}
[1.25cm] % Left indentation
{\addvspace{12pt}\large\sffamily\bfseries} % Spacing and font options for chapters
{\color{ocre!60}\contentslabel[\Large\thecontentslabel]{1.25cm}\color{ocre}} % Formatting of numbered sections of this type
{\color{ocre}} % Formatting of numberless sections of this type
{\color{ocre!60}\normalsize\;\titlerule*[.5pc]{.}\;\thecontentspage} % Formatting of the filler to the right of the heading and the page number

% Section text styling
\titlecontents{section}
[1.25cm] % Left indentation
{\addvspace{3pt}\sffamily\bfseries} % Spacing and font options for sections
{\contentslabel[\thecontentslabel]{1.25cm}} % Formatting of numbered sections of this type
{} % Formatting of numberless sections of this type
{\hfill\color{black}\thecontentspage} % Formatting of the filler to the right of the heading and the page number

% Subsection text styling
\titlecontents{subsection}
[1.25cm] % Left indentation
{\addvspace{1pt}\sffamily\small} % Spacing and font options for subsections
{\contentslabel[\thecontentslabel]{1.25cm}} % Formatting of numbered sections of this type
{} % Formatting of numberless sections of this type
{\ \titlerule*[.5pc]{.}\;\thecontentspage} % Formatting of the filler to the right of the heading and the page number

% Figure text styling
\titlecontents{figure}
[1.25cm] % Left indentation
{\addvspace{1pt}\sffamily\small} % Spacing and font options for figures
{\thecontentslabel\hspace*{1em}} % Formatting of numbered sections of this type
{} % Formatting of numberless sections of this type
{\ \titlerule*[.5pc]{.}\;\thecontentspage} % Formatting of the filler to the right of the heading and the page number

% Table text styling
\titlecontents{table}
[1.25cm] % Left indentation
{\addvspace{1pt}\sffamily\small} % Spacing and font options for tables
{\thecontentslabel\hspace*{1em}} % Formatting of numbered sections of this type
{} % Formatting of numberless sections of this type
{\ \titlerule*[.5pc]{.}\;\thecontentspage} % Formatting of the filler to the right of the heading and the page number

%----------------------------------------------------------------------------------------
%	MINI TABLE OF CONTENTS IN PART HEADS
%----------------------------------------------------------------------------------------

% Chapter text styling
\titlecontents{lchapter}
[0em] % Left indentation
{\addvspace{15pt}\large\sffamily\bfseries} % Spacing and font options for chapters
{\color{ocre}\contentslabel[\Large\thecontentslabel]{1.25cm}\color{ocre}} % Chapter number
{}
{\color{ocre}\normalsize\sffamily\bfseries\;\titlerule*[.5pc]{.}\;\thecontentspage} % Page number

% Section text styling
\titlecontents{lsection}
[0em] % Left indentation
{\sffamily\small} % Spacing and font options for sections
{\contentslabel[\thecontentslabel]{1.25cm}} % Section number
{}
{}

% Subsection text styling (note these aren't shown by default, display them by searchings this file for tocdepth and reading the commented text)
\titlecontents{lsubsection}
[.5em] % Left indentation
{\sffamily\footnotesize} % Spacing and font options for subsections
{\contentslabel[\thecontentslabel]{1.25cm}}
{}
{}

%----------------------------------------------------------------------------------------
%	HEADERS AND FOOTERS
%----------------------------------------------------------------------------------------

\usepackage{fancyhdr} % Required for header and footer configuration

\pagestyle{fancy} % Enable the custom headers and footers

\renewcommand{\chaptermark}[1]{\markboth{\sffamily\normalsize\bfseries 第\thechapter 章\ #1}{}} % Styling for the current chapter in the header
\renewcommand{\sectionmark}[1]{\markright{\sffamily\normalsize\thesection\hspace{5pt}#1}{}} % Styling for the current section in the header

\fancyhf{} % Clear default headers and footers
\fancyhead[LE,RO]{\sffamily\normalsize\thepage} % Styling for the page number in the header
\fancyhead[LO]{\rightmark} % Print the nearest section name on the left side of odd pages
\fancyhead[RE]{\leftmark} % Print the current chapter name on the right side of even pages
%\fancyfoot[C]{\thepage} % Uncomment to include a footer

\renewcommand{\headrulewidth}{0.5pt} % Thickness of the rule under the header

\fancypagestyle{plain}{% Style for when a plain pagestyle is specified
	\fancyhead{}\renewcommand{\headrulewidth}{0pt}%
}

% Removes the header from odd empty pages at the end of chapters
\makeatletter
\renewcommand{\cleardoublepage}{
	\clearpage\ifodd\c@page\else
		\hbox{}
		\vspace*{\fill}
		\thispagestyle{empty}
		\newpage
	\fi}

%----------------------------------------------------------------------------------------
%	THEOREM STYLES
%----------------------------------------------------------------------------------------

\usepackage{amsmath,amsfonts,amssymb,amsthm} % For math equations, theorems, symbols, etc
\usepackage{mathrsfs}
\usepackage{yhmath}
\usepackage{boondox-calo}
\newcommand{\intoo}[2]{\mathopen{]}#1\,;#2\mathclose{[}}
\newcommand{\ud}{\mathop{\mathrm{{}d}}\mathopen{}}
\newcommand{\intff}[2]{\mathopen{[}#1\,;#2\mathclose{]}}
\renewcommand{\qedsymbol}{$\blacksquare$}
\newtheorem{notation}{记号}[section]
\allowdisplaybreaks[4] % 公式跨页断行最大强度

% Boxed/framed environments
\newtheoremstyle{ocrenumbox}% Theorem style name
{0pt}% Space above
{0pt}% Space below
{\normalfont}% Body font
{}% Indent amount
{\small\bf\sffamily\color{ocre}}% Theorem head font
{\;}% Punctuation after theorem head
{0.25em}% Space after theorem head
{\small\sffamily\color{ocre}\thmname{#1}\nobreakspace\thmnumber{\@ifnotempty{#1}{}\@upn{#2}}% Theorem text (e.g. Theorem 2.1)
	\thmnote{\nobreakspace\the\thm@notefont\sffamily\bfseries\color{black}---\nobreakspace#3.}} % Optional theorem note

\newtheoremstyle{blacknumex}% Theorem style name
{5pt}% Space above
{5pt}% Space below
{\normalfont}% Body font
{} % Indent amount
{\small\bf\sffamily}% Theorem head font
{\;}% Punctuation after theorem head
{0.25em}% Space after theorem head
{\small\sffamily{\tiny\ensuremath{\blacksquare}}\nobreakspace\thmname{#1}\nobreakspace\thmnumber{\@ifnotempty{#1}{}\@upn{#2}}% Theorem text (e.g. Theorem 2.1)
	\thmnote{\nobreakspace\the\thm@notefont\sffamily\bfseries---\nobreakspace#3.}}% Optional theorem note

\newtheoremstyle{blacknumbox} % Theorem style name
{0pt}% Space above
{0pt}% Space below
{\normalfont}% Body font
{}% Indent amount
{\small\bf\sffamily}% Theorem head font
{\;}% Punctuation after theorem head
{0.25em}% Space after theorem head
{\small\sffamily\thmname{#1}\nobreakspace\thmnumber{\@ifnotempty{#1}{}\@upn{#2}}% Theorem text (e.g. Theorem 2.1)
	\thmnote{\nobreakspace\the\thm@notefont\sffamily\bfseries---\nobreakspace#3.}}% Optional theorem note

% Non-boxed/non-framed environments
\newtheoremstyle{ocrenum}% Theorem style name
{5pt}% Space above
{5pt}% Space below
{\normalfont}% Body font
{}% Indent amount
{\small\bf\sffamily\color{ocre}}% Theorem head font
{\;}% Punctuation after theorem head
{0.25em}% Space after theorem head
{\small\sffamily\color{ocre}\thmname{#1}\nobreakspace\thmnumber{\@ifnotempty{#1}{}\@upn{#2}}% Theorem text (e.g. Theorem 2.1)
	\thmnote{\nobreakspace\the\thm@notefont\sffamily\bfseries\color{black}---\nobreakspace#3.}} % Optional theorem note
\makeatother

% Defines the theorem text style for each type of theorem to one of the three styles above
\newcounter{dummy}
\numberwithin{dummy}{section}
\theoremstyle{ocrenumbox}
\newtheorem{theoremeT}[dummy]{定理}
\newtheorem{problem}{Problem}[chapter]
\newtheorem{exerciseT}{Exercise}[chapter]
\theoremstyle{blacknumex}
\newtheorem{exampleT}{例}[section]
\newtheorem{proveT}{证明}[section]
\newenvironment{prove}{\begin{proveT}}{\hfill{\tiny\ensuremath{\qedhere\blacksquare}}\end{proveT}}
\theoremstyle{blacknumbox}
\newtheorem{vocabulary}{Vocabulary}[chapter]
\newtheorem{definitionT}{定义}[section]
\newtheorem{corollaryT}[dummy]{推论}
\theoremstyle{ocrenum}
\newtheorem{proposition}[dummy]{定理}

%----------------------------------------------------------------------------------------
%	DEFINITION OF COLORED BOXES
%----------------------------------------------------------------------------------------

\RequirePackage[framemethod=default]{mdframed} % Required for creating the theorem, definition, exercise and corollary boxes

% Theorem box
\newmdenv[skipabove=7pt,
	skipbelow=7pt,
	backgroundcolor=black!5,
	linecolor=ocre,
	innerleftmargin=5pt,
	innerrightmargin=5pt,
	innertopmargin=5pt,
	leftmargin=0cm,
	rightmargin=0cm,
	innerbottommargin=5pt]{tBox}

% Exercise box	  
\newmdenv[skipabove=7pt,
	skipbelow=7pt,
	rightline=false,
	leftline=true,
	topline=false,
	bottomline=false,
	backgroundcolor=ocre!10,
	linecolor=ocre,
	innerleftmargin=5pt,
	innerrightmargin=5pt,
	innertopmargin=5pt,
	innerbottommargin=5pt,
	leftmargin=0cm,
	rightmargin=0cm,
	linewidth=4pt]{eBox}

% Definition box
\newmdenv[skipabove=7pt,
	skipbelow=7pt,
	rightline=false,
	leftline=true,
	topline=false,
	bottomline=false,
	linecolor=ocre,
	innerleftmargin=5pt,
	innerrightmargin=5pt,
	innertopmargin=0pt,
	leftmargin=0cm,
	rightmargin=0cm,
	linewidth=4pt,
	innerbottommargin=0pt]{dBox}

% Corollary box
\newmdenv[skipabove=7pt,
	skipbelow=7pt,
	rightline=false,
	leftline=true,
	topline=false,
	bottomline=false,
	linecolor=gray,
	backgroundcolor=black!5,
	innerleftmargin=5pt,
	innerrightmargin=5pt,
	innertopmargin=5pt,
	leftmargin=0cm,
	rightmargin=0cm,
	linewidth=4pt,
	innerbottommargin=5pt]{cBox}

% Creates an environment for each type of theorem and assigns it a theorem text style from the "Theorem Styles" section above and a colored box from above
\newenvironment{theorem}{\begin{tBox}\begin{theoremeT}}{\end{theoremeT}\end{tBox}}
\newenvironment{exercise}{\begin{eBox}\begin{exerciseT}}{\hfill{\color{ocre}\tiny\ensuremath{\blacksquare}}\end{exerciseT}\end{eBox}}
\newenvironment{definition}{\begin{dBox}\begin{definitionT}}{\end{definitionT}\end{dBox}}
\newenvironment{example}{\begin{exampleT}}{\hfill{\tiny\ensuremath{\blacksquare}}\end{exampleT}}
\newenvironment{corollary}{\begin{cBox}\begin{corollaryT}}{\end{corollaryT}\end{cBox}}

%----------------------------------------------------------------------------------------
%	REMARK ENVIRONMENT
%----------------------------------------------------------------------------------------

\newenvironment{remark}{\par\vspace{10pt}\small % Vertical white space above the remark and smaller font size
	\begin{list}{}{
			\leftmargin=35pt % Indentation on the left
			\rightmargin=25pt}\item\ignorespaces % Indentation on the right
		      \makebox[-2.5pt]{\begin{tikzpicture}[overlay]
				      \node[draw=ocre!60,line width=1pt,circle,fill=ocre!25,font=\sffamily\bfseries,inner sep=2pt,outer sep=0pt] at (-15pt,0pt){\textcolor{ocre}{R}};\end{tikzpicture}} % Orange R in a circle
		      \advance\baselineskip -1pt}{\end{list}\vskip5pt} % Tighter line spacing and white space after remark

%----------------------------------------------------------------------------------------
%	SECTION NUMBERING IN THE MARGIN
%----------------------------------------------------------------------------------------

\makeatletter
\renewcommand{\@seccntformat}[1]{\llap{\textcolor{ocre}{\csname the#1\endcsname}\hspace{1em}}}
\renewcommand{\section}{\@startsection{section}{1}{\z@}
	{-4ex \@plus -1ex \@minus -.4ex}
	{1ex \@plus.2ex }
	{\normalfont\large\sffamily\bfseries}}
\renewcommand{\subsection}{\@startsection {subsection}{2}{\z@}
	{-3ex \@plus -0.1ex \@minus -.4ex}
	{0.5ex \@plus.2ex }
	{\normalfont\sffamily\bfseries}}
\renewcommand{\subsubsection}{\@startsection {subsubsection}{3}{\z@}
	{-2ex \@plus -0.1ex \@minus -.2ex}
	{.2ex \@plus.2ex }
	{\normalfont\small\sffamily\bfseries}}
\renewcommand\paragraph{\@startsection{paragraph}{4}{\z@}
	{-2ex \@plus-.2ex \@minus .2ex}
	{.1ex}
	{\normalfont\small\sffamily\bfseries}}

%----------------------------------------------------------------------------------------
%	PART HEADINGS
%----------------------------------------------------------------------------------------

% Numbered part in the table of contents
\newcommand{\@mypartnumtocformat}[2]{%
	\setlength\fboxsep{0pt}%
	\noindent\colorbox{ocre!20}{\strut\parbox[c][.7cm]{\ecart}{\color{ocre!70}\Large\sffamily\bfseries\centering#1}}\hskip\esp\colorbox{ocre!40}{\strut\parbox[c][.7cm]{\linewidth-\ecart-\esp}{\Large\sffamily\centering#2}}%
}

% Unnumbered part in the table of contents
\newcommand{\@myparttocformat}[1]{%
	\setlength\fboxsep{0pt}%
	\noindent\colorbox{ocre!40}{\strut\parbox[c][.7cm]{\linewidth}{\Large\sffamily\centering#1}}%
}

\newlength\esp
\setlength\esp{4pt}
\newlength\ecart
\setlength\ecart{1.2cm-\esp}
\newcommand{\thepartimage}{}%
\newcommand{\partimage}[1]{\renewcommand{\thepartimage}{#1}}%
\def\@part[#1]#2{%
	\ifnum \c@secnumdepth >-2\relax%
		\refstepcounter{part}%
		\addcontentsline{toc}{part}{\texorpdfstring{\protect\@mypartnumtocformat{\thepart}{#1}}{\partname~\thepart\ ---\ #1}}
	\else%
		\addcontentsline{toc}{part}{\texorpdfstring{\protect\@myparttocformat{#1}}{#1}}%
	\fi%
	\startcontents%
	\markboth{}{}%
	{\thispagestyle{empty}%
		\begin{tikzpicture}[remember picture,overlay]%
			\node at (current page.north west){\begin{tikzpicture}[remember picture,overlay]%	
					\fill[ocre!20](0cm,0cm) rectangle (\paperwidth,-\paperheight);
					\node[anchor=north] at (4cm,-3.25cm){\color{ocre!40}\fontsize{220}{100}\sffamily\bfseries\thepart};
					\node[anchor=south east] at (\paperwidth-1cm,-\paperheight+1cm){\parbox[t][][t]{8.5cm}{
							\printcontents{l}{0}{\setcounter{tocdepth}{1}}% The depth to which the Part mini table of contents displays headings; 0 for chapters only, 1 for chapters and sections and 2 for chapters, sections and subsections
						}};
					\node[anchor=north east] at (\paperwidth-1.5cm,-3.25cm){\parbox[t][][t]{15cm}{\strut\raggedleft\color{white}\fontsize{30}{30}\sffamily\bfseries#2}};
				\end{tikzpicture}};
		\end{tikzpicture}}%
	\@endpart}
\def\@spart#1{%
	\startcontents%
	\phantomsection
	{\thispagestyle{empty}%
		\begin{tikzpicture}[remember picture,overlay]%
			\node at (current page.north west){\begin{tikzpicture}[remember picture,overlay]%	
					\fill[ocre!20](0cm,0cm) rectangle (\paperwidth,-\paperheight);
					\node[anchor=north east] at (\paperwidth-1.5cm,-3.25cm){\parbox[t][][t]{15cm}{\strut\raggedleft\color{white}\fontsize{30}{30}\sffamily\bfseries#1}};
				\end{tikzpicture}};
		\end{tikzpicture}}
	\addcontentsline{toc}{part}{\texorpdfstring{%
			\setlength\fboxsep{0pt}%
			\noindent\protect\colorbox{ocre!40}{\strut\protect\parbox[c][.7cm]{\linewidth}{\Large\sffamily\protect\centering #1\quad\mbox{}}}}{#1}}%
	\@endpart}
\def\@endpart{\vfil\newpage
	\if@twoside
		\if@openright
			\null
			\thispagestyle{empty}%
			\newpage
		\fi
	\fi
	\if@tempswa
		\twocolumn
	\fi}

%----------------------------------------------------------------------------------------
%	CHAPTER HEADINGS
%----------------------------------------------------------------------------------------

% A switch to conditionally include a picture, implemented by Christian Hupfer
\newif\ifusechapterimage
\usechapterimagetrue
\newcommand{\thechapterimage}{}%
\newcommand{\chapterimage}[1]{\ifusechapterimage\renewcommand{\thechapterimage}{#1}\fi}%
\newcommand{\autodot}{.}
\def\@makechapterhead#1{%
	{\parindent \z@ \raggedright \normalfont
			\ifnum \c@secnumdepth >\m@ne
				\if@mainmatter
					\begin{tikzpicture}[remember picture,overlay]
						\node at (current page.north west)
						{\begin{tikzpicture}[remember picture,overlay]
								\node[anchor=north west,inner sep=0pt] at (0,0) {\ifusechapterimage\includegraphics[width=\paperwidth]{\thechapterimage}\fi};
								\draw[anchor=west] (\Gm@lmargin,-9cm) node [line width=2pt,rounded corners=15pt,draw=ocre,fill=white,fill opacity=0.5,inner sep=15pt]{\strut\makebox[22cm]{}};
								\draw[anchor=west] (\Gm@lmargin+1.3cm,-9cm) node {\huge\sffamily\bfseries\color{black}\thechapter\autodot~#1\strut};
							\end{tikzpicture}};
					\end{tikzpicture}
				\else
					\begin{tikzpicture}[remember picture,overlay]
						\node at (current page.north west)
						{\begin{tikzpicture}[remember picture,overlay]
								\node[anchor=north west,inner sep=0pt] at (0,0) {\ifusechapterimage\includegraphics[width=\paperwidth]{\thechapterimage}\fi};
								\draw[anchor=west] (\Gm@lmargin,-9cm) node [line width=2pt,rounded corners=15pt,draw=ocre,fill=white,fill opacity=0.5,inner sep=15pt]{\strut\makebox[22cm]{}};
								\draw[anchor=west] (\Gm@lmargin+1.3cm,-9cm) node {\huge\sffamily\bfseries\color{black}#1\strut};
							\end{tikzpicture}};
					\end{tikzpicture}
				\fi\fi\par\vspace*{270\p@}}}

%-------------------------------------------

\def\@makeschapterhead#1{%
	\begin{tikzpicture}[remember picture,overlay]
		\node at (current page.north west)
		{\begin{tikzpicture}[remember picture,overlay]
				\node[anchor=north west,inner sep=0pt] at (0,0) {\ifusechapterimage\includegraphics[width=\paperwidth]{\thechapterimage}\fi};
				\draw[anchor=west] (\Gm@lmargin,-9cm) node [line width=2pt,rounded corners=15pt,draw=ocre,fill=white,fill opacity=0.5,inner sep=15pt]{\strut\makebox[22cm]{}};
				\draw[anchor=west] (\Gm@lmargin+1.3cm,-9cm) node {\huge\sffamily\bfseries\color{black}#1\strut};
			\end{tikzpicture}};
	\end{tikzpicture}
	\par\vspace*{270\p@}}
\makeatother

%----------------------------------------------------------------------------------------
%	LINKS
%----------------------------------------------------------------------------------------

\usepackage{hyperref}
% \hypersetup{hidelinks,backref=true,pagebackref=true,hyperindex=true,colorlinks=false,breaklinks=true,urlcolor=ocre,bookmarks=true,bookmarksopen=false}
\hypersetup{hidelinks,colorlinks,linkcolor=.,citecolor=blue,breaklinks=true,urlcolor=blue,bookmarksopen=false}

\usepackage{bookmark}
\bookmarksetup{
	open,
	numbered,
	addtohook={%
			\ifnum\bookmarkget{level}=0 % chapter
				\bookmarksetup{bold}%
			\fi
			\ifnum\bookmarkget{level}=-1 % part
				\bookmarksetup{color=ocre,bold}%
			\fi
		}
}


%----------------------------------------------------------------------------------------
% code show
%----------------------------------------------------------------------------------------
\usepackage[ruled,linesnumbered]{algorithm2e}
\usepackage{listings}
\lstset{% 
	language={C++}, %language为,还有{[Visual]C++}{[ISO]C++}
	alsolanguage=[ANSI]C, %可以添加很多个alsolanguage,如alsolanguage=matlab,alsolanguage=VHDL等
	tabsize=4, %
	basicstyle=\ttfamily\footnotesize, % 设置代码的大小
	keywordstyle=\color[RGB]{0,84,166}\bfseries, %代码关键字
	stringstyle=\ttfamily\color[RGB]{33,166,86}, % 代码字符串的特殊格式
	commentstyle=\color[RGB]{115,48,11}\scriptsize\rmfamily, %注释
	rulecolor=\color[RGB]{243,102,25},%代码边框
	frame=leftline, %代码框
	framerule=2pt,
	showstringspaces=false,%不显示代码字符串中间的空格标记
	keepspaces=true,
	breakindent=10pt,
	numbers=left,%左侧显示行号 往左靠,还可以为right,或none,即不加行号
	stepnumber=1,%若设置为2,则显示行号为1,3,5,即stepnumber为公差,默认stepnumber=1
	numberstyle={\color[RGB]{33,166,86}\scriptsize} ,%设置行号的大小,大小有tiny,scriptsize,footnotesize,small,normalsize,large等
	numbersep=8pt, %设置行号与代码的距离,默认是5pt
	showspaces=false, %
	flexiblecolumns=true, %
	breaklines=true, %对过长的代码自动换行
	breakautoindent=true,
	aboveskip=1em, %代码块边框
	tabsize=2,
	showstringspaces=false, %不显示字符串中的空格
	backgroundcolor=\color{black!5}, %代码背景色,或\color[rgb]{0.91,0.91,0.91}
	escapeinside=``, %在``里显示中文 %% added by http://bbs.ctex.org/viewthread.php?tid=53451
	fontadjust,
	captionpos=t,
	framextopmargin=2pt,framexbottommargin=2pt,abovecaptionskip=-3pt,belowcaptionskip=3pt,
	xleftmargin=0em,xrightmargin=0em, % 设定listing左右的空白
	texcl=true, % 设定中文冲突,断行,listing数字的样式
	extendedchars=false,% 设定中文冲突
	columns=flexible, % 列模式
	mathescape=true % 设定数学环境输入
}
\newcommand\codecolor[0]{\color[RGB]{142,12,242}} %设置格式

% https://tex.stackexchange.com/questions/17057/hypertarget-seems-to-aim-a-line-too-low
\makeatletter
\newcommand{\htarget}[2]{\Hy@raisedlink{\hypertarget{#1}{}}#2}
\makeatother

\newcommand\initcode[2]{\htarget{code:#1}{\codecolor\itshape{<<{#1}>>#2}}} %代码段名称定义
\newcommand\refcode[2]{\hyperlink{code:#1}{\codecolor\itshape{<<{#1}>>#2}}} %代码段引用跳转
\newcommand\initcnt[1]{\newcounter{#1}\newcounter{#1last}\newcounter{#1next}\setcounter{#1}{0}\setcounter{#1last}{-1}\setcounter{#1next}{1}} %设置三个计数器,记住当前以及前后的编号
\newcommand\addcnt[1]{\stepcounter{#1last}\stepcounter{#1next}\stepcounter{#1}} %三个各自计数器加1
\newcommand\nextcode[1]{\hyperlink{code:#1:\arabic{#1next}}{\htarget{code:#1:\arabic{#1}}{\codecolor $\downarrow$}}} %引用下一段代码
\newcommand\lastcode[1]{\hyperlink{code:#1:\arabic{#1last}}{\htarget{code:#1:\arabic{#1}}{\codecolor $\uparrow$}}} %引用上一段代码
\newcommand\initnext[1]{\initcnt{#1}\nextcode{#1}\addcnt{#1}} %初始化且引用下一段代码
\newcommand\lastnext[1]{\lastcode{#1}\nextcode{#1}\addcnt{#1}} %引用前后代码

% 代码变量定义、引用跳转
\newcommand\refvar[3][]{\ifthenelse{\isempty{#1}}{\hyperlink{codevar:#2}{\ttfamily #2#3}}{\hyperlink{codevar:#1}{\ttfamily #2#3}}}
\newcommand\initvar[3][]{\ifthenelse{\isempty{#1}}{\htarget{codevar:#2}{\ttfamily #2#3}}{\htarget{codevar:#1}{\ttfamily #2#3}}}

\newcommand\reffig[1]{图\ref{fig:#1}}
\newcommand\reftab[1]{表\ref{tab:#1}}
\newcommand\refeq[1]{式\ref{eq:#1}}
\newcommand\refsec[1]{\ref{sec:#1}节}
\newcommand\refchap[1]{\ref{chap:#1}章}
\newcommand\refsub[1]{\ref{sub:#1}节}

\usepackage{pifont}
\newcommand\circleone{\ding{172}}
\newcommand\circletwo{\ding{173}}
\newcommand\circlethree{\ding{174}}

\newcommand\compcolor[1]{\text{\itshape\sffamily\bfseries #1}}
\newcommand{\Equiv}{\ \tikz[baseline=-0.5ex]{\foreach \y in {0.45,0.15,-0.15,-0.45} \draw[yshift=\y ex] (0,0)--(1.5ex,0);}\ }
\usepackage{siunitx}
\usepackage{multirow} % Insert the commands.tex file which contains the majority of the structure behind the template

%\hypersetup{pdftitle={Title},pdfauthor={Author}} % Uncomment and fill out to include PDF metadata for the author and title of the book

%----------------------------------------------------------------------------------------

\begin{document}

%----------------------------------------------------------------------------------------
%	TITLE PAGE
%----------------------------------------------------------------------------------------

\begingroup
\thispagestyle{empty} % Suppress headers and footers on the title page
\begin{tikzpicture}[remember picture,overlay]
    \node[inner sep=0pt] (background) at (current page.center) {\includegraphics[height=\paperheight]{view-3.png}};
    \draw (current page.center) node [fill=yellow!80!green!40,fill opacity=0.1,text opacity=1,inner sep=1cm]
    {\centering\bfseries\sffamily\parbox[c][][t]{\paperwidth}{\color{white}\centering
    {\Large Physically Based Rendering: From Theory To Implementation}\\[25pt]
    {\fontsize{30pt}{15pt}\textrm{从理论到实现}}\\[15pt]
    {\fontsize{60pt}{15pt}\textrm{基于物理的渲染}}\\[25pt]
    {\fontsize{21pt}{15pt}第三版}\\[20pt]
    {\Large\begin{tabular}{rcl}原著 && Matt Pharr\\ && Wenzel Jakob\\ && Greg Humphreys\\ 翻译 && Kanition\end{tabular}}}};
\end{tikzpicture}
\vfill
\endgroup

%----------------------------------------------------------------------------------------
%	COPYRIGHT PAGE
%----------------------------------------------------------------------------------------

\newpage
~\vfill
\thispagestyle{empty}

\noindent \textbf{\LARGE 从理论到实现}\vspace{8pt}\\
\noindent \textbf{\Huge 基于物理的渲染}\vspace{8pt}\\
\noindent \textbf{\large 第三版}\vspace{8pt}\\
\noindent \textbf{\large 原著 \quad Matt Pharr, Wenzel Jakob \& Greg Humphreys}\vspace{5pt}\\
\noindent \textbf{\large 翻译 \quad Kanition}\vspace{16pt}\\

\noindent {\bfseries 英文原版}

\noindent Copyright \copyright\ 2004-2022 Matt Pharr, Wenzel Jakob, and Greg Humphreys

\noindent 官方网址:\url{https://www.pbr-book.org}

\noindent 许可证:CC BY-NC-SA 4.0\\

\noindent {\bfseries 本中译版}

\noindent Copyright \copyright\ 2021-2022 Kanition

\noindent 更新网址:\url{https://github.com/kanition/pbrtbook}

\noindent 许可证:CC BY-NC-SA 4.0

    {\small(详见:\url{https://creativecommons.org/licenses/by-nc-sa/4.0})}

    {\ttfamily\small\input{ver_info.txt}}

{\itshape
本中译版(以下简称“本书”)系译者(笔名 Kanition)自学英文经典书籍
《Physically Based Rendering: From Theory To Implementation》第三版时自行翻译而成。
使用本书及其源码须遵循相关许可证协议。

本书在翻译时遵照原书编排,译文尽力保留了原文词句,但因笔者水平有限,
而原文长句极多,故可能会存在病句甚至误翻,请读者见谅并指正。

此外,笔者根据自己的学习情况对内容作了补充,
例如自行编写补充章节、在边栏进行注释解说、修正一些笔误等。
除补充章节外,行文中凡是笔者自行变动或增添过的地方都有“译者注”的标记。

原书在线版本以网页形式呈现,可以方便地展开、折叠示例代码。
本书虽受到PDF格式限制,但依旧精心保留了代码链接跳转功能,方便读者查阅。
若读者在实践中还有更多需求,建议参考原书所附代码库。

本书由{\scshape \LaTeX} 编写而成,源码已经发布在上述网址,欢迎访问获取最新版。

{\color{red}\sffamily{欢迎提出宝贵意见和建议。如果你发现本书存在错误,请一定要告诉我们!
讨论区:{\normalfont\url{https://github.com/kanition/pbrtbook/discussions}}}}
}

\newpage
\setcounter{page}{1}
\pagestyle{fancy}
{\Huge\bfseries 前言}\vspace{30pt}\\

渲染是计算机图形学的基础组成部分。
最抽象地说,渲染是把三维场景描述转换为图像的过程。
动画算法、几何建模、材质贴图和计算机图形学其他领域
都须经某些渲染过程来可视化其结果。
从电影到游戏等,渲染无处不在,它为创作、娱乐和可视化开辟了新的领域。

早期的渲染研究重点解决基本问题,例如从给定视点确定哪些物体是可见的。
随着这些问题找到高效解法以及图形学其他领域的持续发展使得场景描述更加丰富逼真,
现代渲染已涵盖了广泛内容,包括物理学、天体物理学、天文学、生物学、心理学、感知研究、理论和应用数学。
渲染的跨学科性是它如此引人的原因之一。

本书以文档化代码的形式提供了构建一个完整的渲染系统所需的一批现代渲染算法。
包括封面在内\sidenote{译者注:本书封面作了更换,但和原书封面是同一组渲染结果。},
本书几乎所有图像都由该软件渲染得到。
且生成这些图像的全部算法均有描述。
该pbrt系统按{\itshape 文学编程}的程序设计方法编写,
即把对系统的描述和实现源码结合在一起。
我们认为,用文学编程法介绍计算机图形学和计算机科学是非常合适的。
算法的一些微妙细节在实现之前往往很难弄清楚,
因此读实际代码更有利于充分理解它们。
我们相信,深入理解哪怕少量的算法也比跑马观花更能打牢进一步研究计算机图形学的基础。

除了阐明实践中如何实现算法外,交代其在完整简单软件系统中的上下文
同样有助于解决中型渲染系统的设计和实现问题。
渲染系统的基本抽象和接口设计对实现的优雅性和可扩展性有实质影响,
但本书不会讨论这类设计取舍。

pbrt和本书内容仅关注{\itshape 逼真渲染},
它可定义为这样的图像生成任务:和相机拍摄的照片难以区分,
或者人类看后被激发的响应与看到实际场景时一致。
我们有许多理由关注逼真感。
逼真图像对电影特效工业至关重要,
因为计算机生成的图像经常必须和真实世界的镜头无缝结合。
娱乐应用中所有图像都是合成的,
逼真感是让观察者忘记所见场景并不实际存在的有效手段。
最后,逼真感为衡量渲染系统输出质量提供了定义合理的指标。\\

\noindent{\LARGE\bfseries 读者}

本书主要面向三类读者。
第一类是学习计算机图形学课程的研究生或高年级本科生。
本书假定读者拥有大学入门级计算机图形学知识,
只会回顾一些关键概念,例如基本向量几何和变换。
对于没有编写过上万行源码程序的学生,
文学编程风格更能降低学习难度。
为了让读者领会为何要这样构建系统,
我们特别注意解释关键接口和抽象背后的设计考量。

第二类读者是计算机图形学研究人员。
本书为研究人员全面介绍了该领域,
pbrt源码提供了可用的构建基础(至少可使用一部分源码)。
对于其他领域的读者,
我们认为对透彻理解渲染也有助于了解相关背景。

最后一类读者是工业界软件开发者。
尽管这些读者可能很熟悉本书许多内容,
但阅读文学风格的算法解释也许能获取新的角度。
pbrt涵盖了大量高级或艰深算法的实现和技术,
例如细分曲面、蒙特卡罗采样算法、双向路径追踪、Metropolis采样和次表面散射;
经验丰富的渲染从业者应该会很感兴趣。
我们希望能激发这些读者去钻研一个完整而典型的渲染系统的兴趣。\\

\noindent{\LARGE\bfseries 概述和目标}

pbrt基于{\itshape 光线追踪}算法。
光线追踪是一项优雅的技术,起源于镜片制造。
19世纪Carl Friedrich Gau{\ss}就用透镜手动追踪光线。
计算机上的光线追踪算法跟随无穷小的光线穿过场景直到与曲面相交。
它给出了从特定位置和方向寻找第一个可见物体的简单方法,
这是许多渲染算法的基础。

pbrt的设计和实现贯彻了三个目标:{\itshape 完整性}、{\itshape 解说性}和基于{\itshape 物理性}。

完整性指系统不应缺少高质量商业渲染系统的关键功能。
这意味着要彻底解决重要的实际问题,
例如抗锯齿、稳定性、数值精度以及高效渲染复杂场景的能力。
在设计系统时一开始就应考虑到这些,
因为它们会对系统所有组件产生微妙影响,
且在实现后期阶段很难再改装到系统中。

第二个目标意味着我们着眼于可读性和清晰度,
精心选用算法、数据结构和渲染技术。
因为比起其他渲染系统,我们的实现要接受更多读者的检验,
所以我们尽力选择已知的最优算法并将其实现。
这个目标也要求系统要小到一个人能完全理解的程度。
我们用可扩展的架构实现了pbrt,
即系统核心采用精心设计的抽象基类,
且这些基类尽量实现足够多特定功能。
这样读者不用理解所有特定细节就能明白系统的基本结构。
这更易于钻研感兴趣的部分并跳过其他内容,
且不影响对系统整体配合的把握。

完整性和解说性目标之间是存在矛盾的。
涵盖所有可能有用的技术不仅让本书过于冗长,
而且对于大多数读者而言也太复杂。
针对万一pbrt缺少某项有用功能的情况,
我们尽量使架构便于增添功能而不用改变系统整体设计。

基于物理的渲染的基础是物理定律及其数学表达式。
pbrt的设计对所计算的量和实现的算法使用正确的物理单位。
这样配置后,pbrt能计算出{\itshape 物理正确}的图像;
它们像在真实世界场景中那样准确反映光照。
这样的好处是它为程序正确性提供了具体标准:
对于预期结果可用解析解计算的简单场景,
如果pbrt没有算出相同结果,我们就能知道实现一定有bug。
类似地,如果pbrt中基于物理光照的不同算法对同一场景给出了不同结果,
或者pbrt所得结果和另一个基于物理的渲染器不一致,
则它们中必有一个出错了。
最后,我们认为基于物理的渲染方法是有价值的,因为它是严格的。
当不清楚特定计算该如何执行时,物理学会给出确保一致的答案。

效率的优先度低于以上三个目标。
既然渲染系统生成一张图像通常要花费数分钟或小时,
效率显然是很重要的。
然而我们主要关注{\itshape 算法}层面的效率而非底层代码优化。
尽管系统中计算量集中的部分已尽力做了优化,
但有时明显而微小的优化会让位于清晰的代码组织。

在介绍pbrt和讨论其实现时,
我们希望传授多年来渲染研究和开发的经验教训。
编写好一个渲染器比串接一堆快速算法更需要付出;
让系统既灵活又稳定是项困难的任务。
随着增添越来越多的几何体或光源,
或者其他复杂维度上升,
系统的性能将逐渐下降。
严谨处理数值稳定性、
算法不浪费浮点精度也至关重要。

开发出解决所有这些问题的系统大有益处——
编写新的渲染器或向已有渲染器添加新功能并用它创作出以往无法生成的图片是多么地快乐。
我们编写本书最基本的目标就是给广大读者这样的机会。
我们鼓励读者在阅读本书时使用该系统渲染pbrt发行的示例场景。
每章末的习题会要求修改系统以加深对内部工作原理的理解,
或者完成添加新功能等更复杂的工程。

本书官网为\href{www.pbrt.org}{\ttfamily pbrt.org},
可从该站获取pbrt最新版源码。
我们也会发布勘误、修复bug、新增渲染场景和补充材料。
遇到网站尚未列出的pbrt中的任何bug或行文错误
均发送到邮箱\href{mailto:bugs@pbrt.org}{\url{bugs@pbrt.org}}。
我们非常重视您的反馈\sidenote{译者注:我也欢迎您的反馈!详见扉页更新网址。}!\\

\noindent{\LARGE\bfseries 第一版和第二版的区别}

{\itshape 详见英文原版}\\

\noindent{\LARGE\bfseries 第二版和第三版的区别}

{\itshape 详见英文原版}\\

\noindent{\LARGE\bfseries 致谢}

{\itshape 详见英文原版}\\

\noindent{\LARGE\bfseries 出版}

{\itshape 详见英文原版}\\

\noindent{\LARGE\bfseries 场景和模型}

{\itshape 详见英文原版}\\

\noindent{\LARGE\bfseries 关于封面}

{\itshape 详见英文原版}\\

\noindent{\LARGE\bfseries 扩展阅读}

\citet{10.1093/comjnl/27.2.97}的论文《\emph{Literate Programming}》
描述了文学编程背后的主要思想以及他的网络编程环境。
开创性的\TeX 排版系统是用网络写成的并出版了一系列书籍\citep{10.5555/536126,10.5555/536123}。
最近,\citet{10.1145/164984}在《\citetitle{10.1145/164984}》中
以文学格式出版了图表算法集。
这些程序读起来很有趣,各个算法也展示得很好。
网站\url{www.literateprogramming.com}指向了许多关于文学编程的论文、程序以及大量系统;
自Knuth最初提出该思想以来,文学编程已经进行了许多改进。


我们所知的其他出版成书的文学程序只有对lcc编译器的实现——
由\citet{10.5555/555424}编写并出版的《\citetitle{10.5555/555424}》,
以及\citet{10.5555/1036653}关于MP3音频格式的书《\citetitle{10.5555/1036653}》。


\newpage
\input{content/prefaceonline.tex}
\newpage
{\Huge\bfseries 译者序}\vspace{30pt}\\

“为什么要翻译这本书?”
这是自我翻译本书以来被问过的最多的问题。
同学问过、父母问过、连面试官也问过。
我回答过各种各样的理由,
甚至连自己也不知道哪个理由才是最重要最关键的。

我曾经有过比较低谷的学业经历,
这主要是心中的完美主义作祟。
当主业不能使自己获得一丝满足时,
我不由得把目光投向了其他领域——渲染便是其中之一。
之后很快我便在GitHub上偶遇了pbrt-v4项目。
虽然这并不是我第一次与渲染相遇了——

若干年前读初中的时候,家里终于有了第一台电脑。
那个时候老家电脑并不算普及。爸爸为了学会怎么用
还得去书店买本《电脑入门基础教程》之类的书回来查阅。
我在学校微机课倒是学会了开关机和使用开始菜单,所以对这些书已没多大兴趣。
不过我在这些教程书区中发现了其他宝贝——与电脑相关的视觉设计类书籍,
就是那些封面图片非常有“质感”的与渲染相关的书,厚度通常也很离谱。
同龄人如果家里新买了电脑,几乎都会兴奋地琢磨怎么安装上时下最火的游戏好好玩一把,
而我却被这些书“带偏”了——虽然完全看不懂其中的内容,甚至不知道
里面偶尔出现的单词“run”是“运行”的意思而不是“奔跑”,
但书中花花绿绿的模型和最终展现的成品渲染图深深地吸引了我。
在电脑里生成一幅现实生活中不存在但分毫毕现的图像,
这样的技术对于一个刚刚学会怎么计算有理数的学生而言是非常震撼的。
我让爸爸买回其中一本,回去对照着折腾起怎么安装3ds Max\textsuperscript{\textregistered}
和V-Ray\textsuperscript{\textregistered},
然后磕磕绊绊地按书中步骤设置材质并花三个多小时渲染出一幅室内装修图。
现在想来,当年那台连独显都没有的电脑承受了太多不该它承受的计算开销。
我也只渲染过那么一次——模型、材质都是附带光盘里的,
照着做了一遍后我也学不会什么,只是觉得好玩。
也许那时候对渲染的兴趣种子就这么埋进土里了。

所以当我与它再会时,曾经的回忆便苏醒了。
那时神经网络的黑箱特性让我厌倦,
而渲染技术明晰的数学原理如同命中十环那般精准地满足了我的口味,
让我有一种从迷茫中解脱的释然感。
事实上,我一直有完成一部“作品”的愿望。
在我心里“作品”这个词是有很高门槛的。
我发表了主业相关的论文,但我并不喜欢它——那远远算不上“作品”。
作为骨灰级动漫爱好者,我甚至觉得自己倾注情感剪辑
的MAD(对动画原片进行重新剪辑配乐做成的视频)更算得上“作品”。
何况动漫粉丝的身份让我对渲染技术的滤镜又加深了一层。
在确认pbrt是一套可以自学搞定的完整教程后,
我认定这本内容详实的著作就是我心中追求的“作品”的模样。
我下定决心以翻译的方式学习它,也许要很久,但不怕学不懂。
毕竟检验是否学会的最好办法就是看能否教会别人。
毫不避讳地说,我就是希望从这个过程中获取自我认同,满足完美主义心理。

这本书翻译起来确实不轻松,曾经我只能抽课余时间写,现在只能抽业余时间写。
越写就越觉得原作者能把如此丰富的内容无偿公开是何其慷慨,
所以我效仿着不对获取译本设任何门槛——而且许可证也不允许。
知识本应自由而充分地交流。我赞赏开源精神,并通过这种方式为其贡献力量。
比起从中获取什么经济利益,我更看中是否有人因此而顺利入门渲染技术,成为又一个充满潜力的领域新人。
当然坦诚地说,我没有从翻译本书中获得个人利益是不可能的——
我收获了一些零散的朋友圈点赞满足了自己的虚荣心,
还把翻译经历写进简历里帮助自己熬过了求职关卡。
这也算是对自己一点小小的犒劳吧。

最后是一点关于匿名发表的解释。
“为什么不用真名而是用笔名Kanition署名译作?”
“对你找工作有什么用吗?”
“不怕别人冒充你去攫取个人名利吗?”
“花了这么多心血难道不希望自己的名字在业内传开吗?”
我当然想!可是我起初并不知道自己能翻译到什么水平。
要知道当初我读第一章面对那么多陌生概念是非常痛苦的。
与其搞砸后被人拿去嘲笑黑历史,不如留下一丝神秘感,
像江湖上不见其人的侠者一般只留下一个名号。
此外,这个笔名对我而言有特殊的意义。
它改编自某家我十分钟爱的动画公司名。
在我心中,这家公司就是耕耘“作品”的代表,
正是它制作的动画让我撑过翻译本书之前那段煎熬的岁月。
然而一场人祸夺取了许多鲜活的生命,那些作品成为了永恒……
无论是渲染还是动画,它们都在构建一个更美丽纯粹的世界,
Kanition这个名字正是代表着为此努力的人。
所以即便这本译作将来获得好评,我也不会改回真名署名,
这是我为数不多的纪念方式了。

\vspace{15pt}
{\hfill {\itshape 译者 Kanition}\qquad}

\vspace{15pt}
\noindent{\LARGE\bfseries 致谢}

感谢\href{https://zixuan-zhang.com}{Zixuan Zhang}参与翻译\refsec{相机模型}部分段落。

%----------------------------------------------------------------------------------------
%	TABLE OF CONTENTS
%----------------------------------------------------------------------------------------
\renewcommand{\contentsname}{目录}
\renewcommand{\figurename}{图}
\renewcommand{\tablename}{表}

%\usechapterimagefalse % If you don't want to include a chapter image, use this to toggle images off - it can be enabled later with \usechapterimagetrue

\chapterimage{Pictures/measure-one180-cut1260.png} % Table of contents heading image

\pagestyle{empty} % Disable headers and footers for the following pages

\tableofcontents % Print the table of contents itself

\cleardoublepage % Forces the first chapter to start on an odd page so it's on the right side of the book

\pagestyle{fancy} % Enable headers and footers again

%----------------------------------------------------------------------------------------
%	PART
%----------------------------------------------------------------------------------------

\part{绪论}
\input{content/chap01.tex}

\part{主要几何功能}
\input{content/chap02.tex}

\input{content/chap03.tex}

\chapterimage{Pictures/chap04/landscape-above-1300x650.png}
\chapter{图元和相交加速}\label{chap:图元和相交加速}
\setcounter{sidenote}{1}

上一章描述的类只关注表示3D对象的几何性质。
虽然类\refvar{Shape}{}为诸如相交和定界等几何操作提供了方便的抽象,
但它没有包含能完全描述场景中一个物体的足够信息。
例如,有必要给每个形状\keyindex{绑定}{bind}{}材质属性以指定其外观。
为了实现这个目标,本章介绍抽象基类\refvar{Primitive}{}并提供大量实现。

要直接渲染的形状表示为类\refvar{GeometricPrimitive}{}。
该类结合了\refvar{Shape}{}及其外观属性的描述。
这样pbrt的几何与着色部分可以干净地分开,
这些外观属性包含在第\refchap{材质}描述的类\refvar{Material}{}中。

类\refvar{TransformedPrimitive}{}处理场景中\refvar{Shape}{}的两个更一般用途:
动画变换矩阵和物体实例化,对于包含许多在不同位置的同一几何体的场景(例如\reffig{4.1}),
它们能大大减少内存需求。
实现这些特性本质上要求在世界空间\refvar{Shape}{}的表示
和实际场景世界空间之间插入额外的变换矩阵。
因此两者都由一个类处理。
\begin{figure}[htbp]
    \centering\includegraphics[width=\linewidth]{chap04/landscape-above.png}
    \caption{该室外场景大量运用实例化作为场景描述的压缩机制。
        场景中只有2400万个不同的三角形,但因为通过实例化复用物体,
        总的几何复杂度为31亿个三角形(场景由Laubwerk提供)。}
    \label{fig:4.1}
\end{figure}

本章还介绍类\refvar{Aggregate}{},
它表示可以容纳许多\refvar{Primitive}{}的容器。
pbrt用该类作为\keyindex{加速结构}{acceleration structure}{}的基础——
帮助减少测试射线与场景中所有$n$个物体相交本来的复杂度$O(n)$.
大多数射线只与少量图元相交而与剩下的错开很大距离。
如果相交加速算法能一次拒绝整组图元,
比起简单地依次测试每条射线与每个图元,性能会有巨大的提升。
对这些加速结构复用\refvar{Primitive}{}接口的一个好处是
让支持一种\keyindex{加速器}{accelerator}{}包含另一种加速器的混合方法更容易。

本章描述了两个加速器的实现,
第一个\refvar{BVHAccel}{}基于构建场景中围绕物体的边界框的层级,
第二个\refvar{KdTreeAccel}{}基于自适应递归空间细分。
虽然提出了许多其他的加速结构,但如今几乎所有光线追踪器都用这两种。
本章末的“扩展阅读”一节有对其他可能性的大量参考。

\input{content/chap0401.tex}

\input{content/chap0402.tex}

\section{包围盒层次}\label{sec:包围盒层次}

\keyindex{包围盒层次}{bounding volume hierarchy}{}(BVH)是一种
基于图元细分的光线相交加速方法,把图元划分为不相交集合的层次
(相反,空间细分一般把空间划分为不相交集合的层次)。
\reffig{4.2}展示了简单场景的包围盒层次\sidenote{译者注:包围盒是边界框的近义词。}。
图元存于\keyindex{叶子}{leaf}{}中,只要它不与节点的边界相交,
该节点下的子树就可以跳过。
\begin{figure}[htbp]
    \centering\input{Pictures/chap04/Primitivesandhierarchy.tex}
    \caption{简单场景的包围盒层次。(a)一小部分图元,边界框用虚线表示。
        图元基于邻近度聚合;这里,球体和等边三角形在被框住整个场景的边界框
        围住之前都被另一个边界框包围了(都用实线表示)。(b)相应的包围盒层次。
        根节点持有整个场景。这里它有两个孩子,一个保存包围了球体和等边三角形的边界框
        (又把这些图元作为其孩子),另一个保存持有瘦三角形的边界框。}
    \label{fig:4.2}
\end{figure}

图元细分的一个性质是每个图元只在层次中出现一次。
相反,一个图元可能与空间细分的多个空间区域重合,
因此在光线穿过它们时要多次测试相交
\footnote{\protect\keyindex{邮箱}{mailboxing}{}技术可用于
    让使用空间细分的加速器避免这样的多次相交,但它的实现在存在多进程时会很棘手。
    “扩展阅读”一节有关于邮箱的更多信息。}。
该性质还意味着表示图元细分层次所需的内存量是有界的。
对于每个叶子中保存单个图元的二叉BVH,节点总数为$2n-1$,其中$n$是图元数量。
有$n$个叶子节点和$n-1$个内部节点
\sidenote{译者注:这些结论利用了二叉BVH的前提:每个节点要么是叶子节点,要么是有两个孩子的内部节点。}。
如果叶子保存了多个图元,则需要的节点更少。

构建BVH比kd树更高效,kd树分发光线相交测试通常比BVH稍快但构建时间长得多。
另一方面,BVH通常数值更稳定,比起kd树更不容易因为舍入误差错过相交。

BVH加速器{\refvar{BVHAccel}{}}定义在\href{https://github.com/mmp/pbrt-v3/tree/master/src/accelerators/bvh.h}{\ttfamily accelerators/bvh.h}
和\href{https://github.com/mmp/pbrt-v3/tree/master/src/accelerators/bvh.cpp}{\ttfamily accelerators/bvh.cpp}
中。除了要保存的图元以及任何叶子节点中的最大图元数目,
其构造函数还接收一个描述当划分图元以构建树时要用四个算法中哪一个的枚举值。
应该用默认值\refvar{SAH}{},它表示\refsub{表面积启发法}讨论的基于“表面积启发法”的算法。
另一个是\refsub{线性包围盒层次}讨论的\refvar{HLBVH}{},
它能更高效地构造(且更易并行化),但建立的树不如\refvar{SAH}{}高效。
剩下的两种方法使用的计算量甚至更少,但创建的树的质量非常低。
\begin{lstlisting}
`\initcode{BVHAccel Public Types}{=}`
enum class `\initvar{SplitMethod}{}` { `\initvar{SAH}{}`, `\initvar{HLBVH}{}`, `\initvar{Middle}{}`, `\initvar{EqualCounts}{}` };
\end{lstlisting}
\begin{lstlisting}
`\initcode{BVHAccel Method Definitions}{=}\initnext{BVHAccelMethodDefinitions}`
`\initvar{BVHAccel}{}`::`\refvar{BVHAccel}{}`(const std::vector<std::shared_ptr<`\refvar{Primitive}{}`>> &p,
         int maxPrimsInNode, `\refvar{SplitMethod}{}` splitMethod)
     : `\refvar{maxPrimsInNode}{}`(std::min(255, maxPrimsInNode)), `\refvar[BVHAccel::primitives]{primitives}{}`(p),
       `\refvar{splitMethod}{}`(splitMethod) {
    if (primitives.size() == 0)
        return;
    `\refcode{Build BVH from primitives}{}`
}
\end{lstlisting}
\begin{lstlisting}
`\initcode{BVHAccel Private Data}{=}\initnext{BVHAccelPrivateData}`
const int `\initvar{maxPrimsInNode}{}`;
const `\refvar{SplitMethod}{}` `\initvar{splitMethod}{}`;
std::vector<std::shared_ptr<`\refvar{Primitive}{}`>> `\initvar[BVHAccel::primitives]{primitives}{}`;
\end{lstlisting}

\subsection{BVH构建}\label{sub:BVH构建}
这里的实现中构建BVH有三个阶段。
首先,计算关于每个图元的边界信息并保存到将于树构建期间使用的数组中。
接着,用选择的编码于\refvar{SplitMethod}{}的算法构建树。
结果是\keyindex{二叉树}{binary tree}{}每个内部节点
都有指针指向其孩子且每个叶子节点都有指向一个或多个图元的引用。
最后,该树转化为更紧实(且因此更高效)的无指针表示以供渲染时使用
(虽然在构建树期间直接计算无指针表示也可以,但用该方法实现更简单)。
\begin{lstlisting}
`\initcode{Build BVH from primitives}{=}`
`\refcode{Initialize primitiveInfo array for primitives}{}`
`\refcode{Build BVH tree for primitives using primitiveInfo}{}`
`\refcode{Compute representation of depth-first traversal of BVH tree}{}`
\end{lstlisting}

对于每个要存于BVH的图元,我们在结构体\refvar{BVHPrimitiveInfo}{}的一个实例中
存储其边界框的形心、完整边界框以及它在\refvar{primitives}{}数组中的索引。
\begin{lstlisting}
`\initcode{Initialize primitiveInfo array for primitives}{=}`
std::vector<`\refvar{BVHPrimitiveInfo}{}`> primitiveInfo(`\refvar[BVHAccel::primitives]{primitives}{}`.size());
for (size_t i = 0; i < `\refvar[BVHAccel::primitives]{primitives}{}`.size(); ++i)
    primitiveInfo[i] = { i, `\refvar[BVHAccel::primitives]{primitives}{}`[i]->`\refvar[Primitive::WorldBound]{WorldBound}{}`() };
\end{lstlisting}
\begin{lstlisting}
`\initcode{BVHAccel Local Declarations}{=}\initnext{BVHAccelLocalDeclarations}`
struct `\initvar{BVHPrimitiveInfo}{}` {
    `\refvar{BVHPrimitiveInfo}{}`(size_t primitiveNumber, const `\refvar{Bounds3f}{}` &bounds)
        : `\refvar{primitiveNumber}{}`(primitiveNumber), `\refvar[BVHPrimitiveInfo::bounds]{bounds}{}`(bounds),
          `\refvar{centroid}{}`(.5f * bounds.`\refvar{pMin}{}` + .5f * bounds.`\refvar{pMax}{}`) { }
    size_t `\initvar{primitiveNumber}{}`;
    `\refvar{Bounds3f}{}` `\initvar[BVHPrimitiveInfo::bounds]{bounds}{}`;
    `\refvar{Point3f}{}` `\initvar{centroid}{}`;
};
\end{lstlisting}

现在可以开始层次构建了。如果选择HLBVH构建算法,则调用\refvar{HLBVHBuild}{()}
构建树。其他三种构建算法都由\refvar{recursiveBuild}{()}负责。
初始调用这些函数时传递了所有要存于树中的图元。
它们返回一个指向树根的指针,用结构体\refvar{BVHBuildNode}{}表示。
树节点应该用提供的\refvar{MemoryArena}{}分配内存,
创建的总数应存于{\ttfamily *totalNodes}中。

树构建过程的一个重要副作用是通过参数{\ttfamily orderedPrims}返回指向图元的新指针数组;
该数组保存了有序的图元这样叶子节点的图元在数组中占有连续的范围。
在树构建后它与原始的\refvar[BVHAccel::primitives]{primitives}{}数组交换。
\begin{lstlisting}
`\initcode{Build BVH tree for primitives using primitiveInfo}{=}`
`\refvar{MemoryArena}{}` arena(1024 * 1024);
int totalNodes = 0;
std::vector<std::shared_ptr<`\refvar{Primitive}{}`>> orderedPrims;
`\refvar{BVHBuildNode}{}` *root;
if (splitMethod == `\refvar{SplitMethod}{}`::`\refvar{HLBVH}{}`)
    root = `\refvar{HLBVHBuild}{}`(arena, primitiveInfo, &totalNodes, orderedPrims);
else
    root = `\refvar{recursiveBuild}{}`(arena, primitiveInfo, 0, `\refvar[BVHAccel::primitives]{primitives}{}`.size(),
                          &totalNodes, orderedPrims);
`\refvar[BVHAccel::primitives]{primitives}{}`.swap(orderedPrims);
\end{lstlisting}

每个\refvar{BVHBuildNode}{}表示一个BVH节点。
所有节点存储一个\refvar{Bounds3f}{}以表示该节点下所有孩子的边界。
每个内部节点在\refvar[BVHBuildNode::children]{children}{}中存有指向其两个孩子的指针。
内部节点也记录图元沿哪个坐标轴划分分给它们的两个孩子;
该信息用于提高遍历算法的性能。
叶子节点需要记录哪个或哪些图元保存在它们中;
数组\refvar{BVHAccel::primitives}{}中从偏移量\refvar{firstPrimOffset}{}起
直到但不包括{\ttfamily\refvar{firstPrimOffset}{}+\refvar[BVHBuildNode:nPrimitives]{nPrimitives}{}}的元素是叶子中的元素
(因此需要记录图元数组,这样就可以利用该表示,
而不是例如在每个叶子节点中保存一个大小可变的图元索引数组)。
\begin{lstlisting}
`\refcode{BVHAccel Local Declarations}{+=}\lastnext{BVHAccelLocalDeclarations}`
struct `\initvar{BVHBuildNode}{}` {
    `\refcode{BVHBuildNode Public Methods}{}`
    `\refvar{Bounds3f}{}` `\initvar[BVHBuildNode::bounds]{bounds}{}`;
    `\refvar{BVHBuildNode}{}` *`\initvar[BVHBuildNode::children]{children}{}`[2];
    int `\initvar[BVHBuildNode::splitAxis]{splitAxis}{}`, `\initvar{firstPrimOffset}{}`, `\initvar[BVHBuildNode:nPrimitives]{nPrimitives}{}`;
};
\end{lstlisting}

我们将通过其孩子指针是否有值{\ttfamily nullptr}来分别区分叶子和内部节点。
\begin{lstlisting}
`\initcode{BVHBuildNode Public Methods}{=}\initnext{BVHBuildNodePublicMethods}`
void `\initvar[BVHBuildNode::InitLeaf]{InitLeaf}{}`(int first, int n, const `\refvar{Bounds3f}{}` &b) {
    `\refvar{firstPrimOffset}{}` = first;
    `\refvar[BVHBuildNode:nPrimitives]{nPrimitives}{}` = n;
    `\refvar[BVHBuildNode::bounds]{bounds}{}` = b;
    `\refvar[BVHBuildNode::children]{children}{}`[0] = `\refvar[BVHBuildNode::children]{children}{}`[1] = nullptr;
}
\end{lstlisting}

方法\refvar[BVHBuildNode::InitInterior]{InitInterior}{()}要求已创建两个孩子节点,这样它们的指针才能传入。
该要求让计算内部节点的边界更加容易了,因为孩子的边界可以立刻获得。
\begin{lstlisting}
`\refcode{BVHBuildNode Public Methods}{+=}\lastcode{BVHBuildNodePublicMethods}`
void `\initvar[BVHBuildNode::InitInterior]{InitInterior}{}`(int axis, `\refvar{BVHBuildNode}{}` *c0, `\refvar{BVHBuildNode}{}` *c1) {
    `\refvar[BVHBuildNode::children]{children}{}`[0] = c0;
    `\refvar[BVHBuildNode::children]{children}{}`[1] = c1;
    `\refvar[BVHBuildNode::bounds]{bounds}{}` = `\refvar[Union2]{Union}{}`(c0->`\refvar[BVHBuildNode::bounds]{bounds}{}`, c1->`\refvar[BVHBuildNode::bounds]{bounds}{}`);
    `\refvar[BVHBuildNode::splitAxis]{splitAxis}{}` = axis;
    `\refvar[BVHBuildNode:nPrimitives]{nPrimitives}{}` = 0;
}
\end{lstlisting}

除了用于分配节点和\refvar{BVHPrimitiveInfo}{}结构体数组的\refvar{MemoryArena}{}外,\linebreak
\refvar{recursiveBuild}{()}接收范围参数{\ttfamily[start,end)}。
它负责为从{\ttfamily primitiveInfo [start]}直到并包括{\ttfamily primitiveInfo[end-1]}的
范围表示的图元子集返回一个BVH。
如果该范围只含有单个图元,则递归触底并创建一个叶子节点。
否则,该方法用划分算法之一来划分数组该范围内的元素并相应地重新排列它们,
这样范围{\ttfamily[start,mid)}和{\ttfamily[mid,end)}表示分开的子集。
如果划分成功,则这两个图元集合又传入将会为当前节点的两个孩子返回节点指针的递归调用。

{\ttfamily totalNodes}跟踪已创建的BVH节点总数;
利用该数目使得之后可以分配数目恰好正确的更紧实的\refvar{LinearBVHNode}{}。
最终,数组{\ttfamily orderedPrims}用于保存图元引用就像图元存于树的叶子节点一样。
该数组初始化为空;当创建一个叶子节点时,\refvar{recursiveBuild}{()}把
与之重合的图元添加到数列末尾,让叶子节点可以只存储对该数组的偏移量以及
表示与之重合的图元集的图元数量。
回想当完成树构建时,用这里创建的有序图元数组代替\refvar{BVHAccel::primitives}{}。
\begin{lstlisting}
`\refcode{BVHAccel Method Definitions}{+=}\lastnext{BVHAccelMethodDefinitions}`
`\refvar{BVHBuildNode}{}` *`\refvar{BVHAccel}{}`::`\initvar{recursiveBuild}{}`(`\refvar{MemoryArena}{}` &arena,
        std::vector<`\refvar{BVHPrimitiveInfo}{}`> &primitiveInfo, int start,
        int end, int *totalNodes,
        std::vector<std::shared_ptr<`\refvar{Primitive}{}`>> &orderedPrims) {
    `\refvar{BVHBuildNode}{}` *node = arena.`\refvar[MemoryArena:Alloc2]{Alloc}{}`<`\refvar{BVHBuildNode}{}`>();
    (*totalNodes)++;
    `\refcode{Compute bounds of all primitives in BVH node}{}`
    int nPrimitives = end - start;
    if (nPrimitives == 1) {
        `\refcode{Create leaf BVHBuildNode}{}`
    } else {
        `\refcode{Compute bound of primitive centroids, choose split dimension dim}{}`
        `\refcode{Partition primitives into two sets and build children}{}`
    }
    return node;
}
\end{lstlisting}
\begin{lstlisting}
`\initcode{Compute bounds of all primitives in BVH node}{=}`
`\refvar{Bounds3f}{}` bounds;
for (int i = start; i < end; ++i)
    bounds = `\refvar[Union2]{Union}{}`(bounds, primitiveInfo[i].`\refvar[BVHPrimitiveInfo::bounds]{bounds}{}`);
\end{lstlisting}

在叶子节点处,与该叶子重合的图元被添到{\ttfamily orderedPrims}数组末尾并初始化一个叶子节点对象。
\begin{lstlisting}
`\initcode{Create leaf BVHBuildNode}{=}`
int firstPrimOffset = orderedPrims.size();
for (int i = start; i < end; ++i) {
    int primNum = primitiveInfo[i].`\refvar{primitiveNumber}{}`;
    orderedPrims.push_back(`\refvar[BVHAccel::primitives]{primitives}{}`[primNum]);
}
node->`\refvar[BVHBuildNode::InitLeaf]{InitLeaf}{}`(firstPrimOffset, nPrimitives, bounds);
return node;
\end{lstlisting}

对于内部节点,一组图元必须在两个子树之间划分。
给定$n$个图元,有$2^{n-1}-1$种\sidenote{译者注:原文误写为$2(n-1)-2$,已修正。}
可能的方法将它们划分到两个非空组。
实际中构建BVH时,一般考虑沿一个坐标轴划分,这意味着大约有$3n$个候选划分
(沿每个轴方向,每个图元可能放到第一分区或第二分区)。

这里我们就选择三个坐标轴的一个用来划分图元。
当为当前图元集合投影边界框形心时,我们选择有最大范围的轴
(另一种是尝试所有三个轴并选择给出最好结果的那个,但实际中本方法更好)。
该方法在许多场景下给出良好划分;\reffig{4.3}说明了该策略。
\begin{figure}[htbp]
    \centering\input{Pictures/chap04/BVHchoosesplitaxis.tex}
    \caption{选择沿哪个轴划分图元。\protect\refvar{BVHAccel}{}基于
        图元边界框形心在哪个轴有最大范围来选择划分图元所沿的轴。
        这里在二维中,它们沿$y$轴的范围最大(轴上的实心点),所以图元会在$y$上划分。}
    \label{fig:4.3}
\end{figure}

这里划分的一般目标是选择图元的一个划分使得
得到的两个图元集合的边界框没有太多重合——
如果有大量重合则在遍历树时它需要更频繁地遍历两个子树,
比起本应可以更高效剪除一些图元它需要更多计算量。
待会儿在讨论表面积启发法时会更严谨地表述该求取高效图元划分的思想。
\begin{lstlisting}
`\initcode{Compute bound of primitive centroids, choose split dimension dim}{=}`
`\refvar{Bounds3f}{}` centroidBounds;
for (int i = start; i < end; ++i)
    centroidBounds = `\refvar[Union2]{Union}{}`(centroidBounds, primitiveInfo[i].`\refvar{centroid}{}`);
int dim = centroidBounds.`\refvar{MaximumExtent}{}`();
\end{lstlisting}

如果所有形心点都在同一位置(即形心边界为零体积),
则递归停止并用该图元创建一个叶子结点;
这里没有划分方法能对那种(非常)情况有效。
否则用选择的方法划分图元并传入两个对\refvar{recursiveBuild}{()}的递归调用。
\begin{lstlisting}
`\initcode{Partition primitives into two sets and build children}{=}`
int mid = (start + end) / 2;
if (centroidBounds.`\refvar{pMax}{}`[dim] == centroidBounds.`\refvar{pMin}{}`[dim]) {
    `\refcode{Create leaf BVHBuildNode}{}`
} else {
    `\refcode{Partition primitives based on splitMethod}{}`
    node->`\refvar[BVHBuildNode::InitInterior]{InitInterior}{}`(dim,
                       `\refvar{recursiveBuild}{}`(arena, primitiveInfo, start, mid,
                                      totalNodes, orderedPrims),
                       `\refvar{recursiveBuild}{}`(arena, primitiveInfo, mid, end,
                                      totalNodes, orderedPrims));
}
\end{lstlisting}

代码片\refcode{Partition primitives based on splitMethod}{}只是用
\refvar[splitMethod]{BVHAccel::splitMethod}{}
的值决定该用哪个图元划分方案。接下来的几页将介绍这三个方案。
\begin{lstlisting}
`\initcode{Partition primitives based on splitMethod}{=}`
switch (`\refvar{splitMethod}{}`) {
case `\refvar{SplitMethod}{}`::`\refvar{Middle}{}`: {
    `\refcode{Partition primitives through node's midpoint}{}`
}
case `\refvar{SplitMethod}{}`::`\refvar{EqualCounts}{}`: {
    `\refcode{Partition primitives into equally sized subsets}{}`
    break;
}
case `\refvar{SplitMethod}{}`::`\refvar{SAH}{}`:
default: {
    `\refcode{Partition primitives using approximate SAH}{}`
    break;
}
}
\end{lstlisting}

\refvar{Middle}{}是一个简单的\refvar{SplitMethod}{},
它首先计算图元形心沿划分轴的中点。
该方法在代码片\refcode{Partition primitives through node's midpoint}{}中实现。
图元按其形心在中点之上还是之下分为两个集合。
该划分用C++标准库函数{\ttfamily std::partition()}很容易完成,
它接收数组中的一系列元素和比较函数,并对数组元素排序使得
对于判定函数而言所有返回{\ttfamily true}的元素都出现在返回{\ttfamily false}的范围之前
\footnote{在调用{\ttfamily std::partition()}时,
注意数组{\ttfamily primitiveInfo}索引的特殊表达式即{\ttfamily \&primitiveInfo[end-1]+1}。
这样写代码有些晦涩的理由。在C和C++程序语言中,
计算数组末尾后下一个元素的指针是合法的,
这样遍历数组元素能持续到当前指针等于末端点。
为此,我们这里想就写成表达式{\ttfamily \&primitiveInfo[end]}。
然而{\ttfamily primitiveInfo}分配为C++的{\ttfamily vector};
一些{\ttfamily vector}实现在传给其{\ttfamily []}操作符
的偏移量是在数组末端之后时会报运行时错误。
因为我们不会尝试引用数组末端后下一个元素的值而只是想计算其地址,
所以该操作事实上是安全的。
因此我们最终用这里的表达式计算同一地址,并且也满足任何{\ttfamily vector}错误检查。}。
{\ttfamily std::partition()}返回指向第一个对于判定有{\ttfamily false}值的元素的指针,
它转化为对数组{\ttfamily primitiveInfo}的偏移量,这样我们就可以将其传入递归调用。
\reffig{4.4}说明了该方法,包括其有效和无效的情况。

如果图元都有巨大的重叠边界框,则该划分方法可能无法把图元分为两组。
这种情况下,执行往下进入{\ttfamily \refvar{SplitMethod}{}::\refvar{EqualCounts}{}}方法再试一次。
\begin{lstlisting}
`\initcode{Partition primitives through node's midpoint}{=}`
`\refvar{Float}{}` pmid = (centroidBounds.`\refvar{pMin}{}`[dim] + centroidBounds.`\refvar{pMax}{}`[dim]) / 2;
`\refvar{BVHPrimitiveInfo}{}` *midPtr =
    std::partition(&primitiveInfo[start], &primitiveInfo[end-1]+1,
        [dim, pmid](const `\refvar{BVHPrimitiveInfo}{}` &pi) {
            return pi.`\refvar{centroid}{}`[dim] < pmid;
        });
mid = midPtr - &primitiveInfo[0];
if (mid != start && mid != end)
    break;
\end{lstlisting}

当\refvar{splitMethod}{}是{\ttfamily\refvar{SplitMethod}{}::\refvar{EqualCounts}{}}时,
则运行代码片\refcode{Partition primitives into equally sized subsets}{}。
它把图元划分为两个数量相等的子集使得$n$个中前一半的$\displaystyle\frac{n}{2}$个
沿所选轴的形心坐标最小,另一半的则有最大形心坐标值。
尽管该方法有时能起效,但也有\reffig{4.4}(b)效果不好的情况。
\begin{figure}[htb]
    \centering\input{Pictures/chap04/Midpointgoodbadsplit.tex}
    \caption{一轴上基于形心中点划分图元。(a)对于一些图元分布,
        例如这里所示,沿所选轴(粗蓝线)基于形心中点的划分效果很好。
        (b)对于像这个的分布,中点是次优选项;所得两个边界框大量重叠。
        (c)若来自(b)的同一组图元换为用这里展示的线分开,所得边界框
        更小且根本不会重叠,使渲染时性能更好。}
    \label{fig:4.4}
\end{figure}

该方案也易于调用标准库的{\ttfamily std::nth\_element()}实现。
它接收起点、中点和终点指针以及一个比较函数。
它对数组排序使得中点指针处元素的位置是,如果数组完全排序,
则所有中点之前的元素都比中点元素小且所有后面的元素都比它大。
对于$n$个元素该排序可以在$O(n)$时间内完成,
比完全排序数组的$O(n\log{n})$更高效。
\begin{lstlisting}
`\initcode{Partition primitives into equally sized subsets}{}`
mid = (start + end) / 2;
std::nth_element(&primitiveInfo[start], &primitiveInfo[mid], 
                 &primitiveInfo[end-1]+1,
    [dim](const BVHPrimitiveInfo &a, const BVHPrimitiveInfo &b) { 
        return a.`\refvar{centroid}{}`[dim] < b.`\refvar{centroid}{}`[dim];
    });
\end{lstlisting}

\subsection{表面积启发法}\label{sub:表面积启发法}
上述两个图元划分方法对一些图元分布效果不错,
但实际中它们经常选择性能较差的划分,导致光线要访问树的更多节点,
因此带来渲染时不必要的低效光线-图元相交计算。
当下光线追踪大部分最好的构建加速结构算法都基于“\keyindex{表面积启发法}{surface area heuristic}{}”(SAH),
它提供了全面的开销模型来回答问题,例如
“大量图元划分中哪个会为光线-图元相交测试带来更好的BVH?”,
或者“在空间划分方案里大量划分空间的可选位置中哪个会带来更好的加速结构?”

SAH模型估计执行光线相交测试的开销,包括穿行树的节点花的时间和
为特定的图元划分进行光线-图元相交测试花的时间。
然后构建加速结构的算法可以遵循最小化总开销的目标。
通常用贪婪算法独立地为正在构建的每个层次节点最小化开销。

SAH开销模型背后的思想很简单:构建自适应加速结构(图元划分或空间划分)的任意点处,
我们可以为当前区域和几何体创建一个叶子结点。
这种情况下,任何穿过该区域的光线都要对所有重合的图元测试,
且带来的开销为
\begin{align*}
    \sum\limits_{i=1}^{N}{t_{\mathrm{isect}}(i)}\, ,
\end{align*}
其中$N$是图元数量,$t_{\mathrm{isect}}(i)$是对第$i$个图元计算光线-物体相交的时间。

另一选项是划分该空间。这种情况下,光线会带来开销
\begin{align}\label{eq:4.1}
    c(A,B)=t_{\mathrm{trav}}+p_A\sum\limits_{i=1}^{N_A}{t_\mathrm{isect}(a_i)}+p_B\sum\limits_{i=1}^{N_B}{t_{\mathrm{isect}}(b_i)}\, ,
\end{align}
其中$t_{\mathrm{trav}}$是穿行内部节点并确定光线穿过哪个子树所花的时间,分别地,
$p_A$和$p_B$是光线穿过每个子节点(假设二分划分)的概率,
$a_i$和$b_i$是两个子节点中图元的索引,
$N_A$和$N_B$是与两个子节点区域重合的图元数量。
怎样划分图元的选项会影响两个概率值以及划分出的两边的图元集合。

pbrt中,我们将作出简化假设即所有图元的$t_{\mathrm{trav}}(i)$都相同;
该假设可能和实际差不了多少,且它引入的任何误差看起来并不太影响加速器的性能。
另一种可能是向\refvar{Primitive}{}添加一个方法返回估计的相交测试所需的CPU周期数。

概率$p_A$和$p_B$可用来自几何概型的思想计算。
可以证明当凸体$A$包含于另一凸体$B$中,
穿过$B$的均匀分布的随机光线也穿过$A$的条件概率是它们表面积$s_A$与$s_B$的比:
\begin{align*}
    \displaystyle p(A|B)=\frac{s_A}{s_B}\, .
\end{align*}

因为我们对光线穿过节点的开销感兴趣,我们可以直接利用该结果。
因此,如果我们考虑细化一空间区域$A$使得有两个新子区域边界为$B$和$C$(\reffig{4.5}),
则穿过$A$的光线也会穿过两个子区域之一的概率很容易计算。
\begin{figure}[htbp]
    \centering\input{Pictures/chap04/Surfaceareasplit.tex}
    \caption{如果边界层次一个表面积为$s_A$的节点分为两个表面积为$s_B$和$s_C$的孩子,
        则穿过$A$的光线也穿过$B$和$C$的概率分别为$\displaystyle\frac{s_B}{s_A}$和$\displaystyle\frac{s_C}{s_A}$.}
    \label{fig:4.5}
\end{figure}

当\refvar{splitMethod}{}值为{\ttfamily\refvar{SplitMethod}{}::\refvar{SAH}{}}时,SAH用于构建BVH;
通过考虑大量候选划分来寻找沿所选轴给出最小SAH估计开销的图元划分
(这是默认\refvar{SplitMethod}{},且它为渲染创建最高效的树)。
然而,一旦它细分少量图元,则实现切换为划分成等量的子集。
这时应用SAH所增加的计算量是不划算的。
\begin{lstlisting}
`\initcode{Partition primitives using approximate SAH}{=}`
if (nPrimitives <= 4) {
    `\refcode{Partition primitives into equally sized subsets}{}`
} else {
    `\refcode{Allocate BucketInfo for SAH partition buckets}{}`
    `\refcode{Initialize BucketInfo for SAH partition buckets}{}`
    `\refcode{Compute costs for splitting after each bucket}{}`
    `\refcode{Find bucket to split at that minimizes SAH metric}{}`
    `\refcode{Either create leaf or split primitives at selected SAH bucket}{}`
}
\end{lstlisting}

比起穷举沿该轴所有$2^n$个可能的划分
\sidenote{译者注:原文写作$2n$,笔者认为是笔误,已修正。},
这里的实现是将沿该轴的范围分为少量相等的较大范围,并为每个计算SAH以选择最好的。
然后它只考虑在该范围边界内的划分。
该方法比考虑所有划分更高效且通常仍产出几乎一样高效的划分。
\reffig{4.6}说明了该思想。
\begin{figure}[htbp]
    \centering\input{Pictures/chap04/BVHsplitbucketing.tex}
    \caption{用表面积启发法为BVH选择划分平面。图元边界形心的投影范围被投影在选定的划分轴上。
        每个图元都基于其边界形心被放在沿轴的一个桶中。
        然后实现会估计沿每个桶的边界(粗蓝线)的平面划分图元的开销;
        谁的表面积启发法开销最小就选谁。}
    \label{fig:4.6}
\end{figure}
\begin{lstlisting}
`\initcode{Allocate BucketInfo for SAH partition buckets}{=}`
constexpr int nBuckets = 12;
struct `\initvar{BucketInfo}{}` {
    int `\initvar[BucketInfo::count]{count}{}` = 0;
    `\refvar{Bounds3f}{}` `\initvar[BucketInfo::bounds]{bounds}{}`;
};
`\refvar{BucketInfo}{}` buckets[nBuckets];
\end{lstlisting}

对于该范围内的每个图元,我们确定包含其形心的桶\protect\sidenote{译者注:原文bucket。}并更新桶的边界以包含图元的边界。
\begin{lstlisting}
`\initcode{Initialize BucketInfo for SAH partition buckets}{=}`
for (int i = start; i < end; ++i) {
    int b = nBuckets * 
        centroidBounds.`\refvar{Offset}{}`(primitiveInfo[i].`\refvar{centroid}{}`)[dim];
    if (b == nBuckets) b = nBuckets - 1;
    buckets[b].`\refvar[BucketInfo::count]{count}{}`++;
    buckets[b].`\refvar[BucketInfo::bounds]{bounds}{}` = `\refvar[Union2]{Union}{}`(buckets[b].`\refvar[BucketInfo::bounds]{bounds}{}`, primitiveInfo[i].`\refvar[BVHPrimitiveInfo::bounds]{bounds}{}`);
}
\end{lstlisting}

对于每个桶,我们现在都有图元数量以及全部相应边界框的边界。
我们想用SAH估计在每个桶边界处作划分的开销。
下面的代码片遍历所有桶并初始化数组{\ttfamily cost[i]}来
保存估计的在第{\ttfamily i}个桶后划分的SAH开销
(不考虑在最后一个桶之后划分,因为根据定义它并不划分图元)。

我们任意设置估计的相交开销为1,然后设置估计的遍历开销
为$\displaystyle\frac{1}{8}$(两者之一总是可以设为1,
因为是估计的遍历和相交开销的相对量级而不是绝对量级决定其影响)。
尽管遍历节点即光线-边界框相交的绝对计算量仅稍低于光线与形状相交所需的计算量,
但pbrt中光线-图元相交测试经过了两次虚函数调用,增加了大量开销,
所以这里我们估计其开销大于光线-框相交的八倍。

基于对桶前向和后向扫描而增量式地计算、存储和计数边界的线性时间实现是可能的,
但这里的计算关于桶的数量有$O(n^2)$复杂度。
更加高度优化解决该低效问题的渲染器是值得的,
但这里对于小的$n$,性能影响通常是可接受的。
\begin{lstlisting}
`\initcode{Compute costs for splitting after each bucket}{=}`
`\refvar{Float}{}` cost[nBuckets - 1];
for (int i = 0; i < nBuckets - 1; ++i) {
    `\refvar{Bounds3f}{}` b0, b1;
    int count0 = 0, count1 = 0;
    for (int j = 0; j <= i; ++j) {
        b0 = `\refvar[Union2]{Union}{}`(b0, buckets[j].`\refvar[BucketInfo::bounds]{bounds}{}`);
        count0 += buckets[j].`\refvar[BucketInfo::count]{count}{}`;
    }
    for (int j = i+1; j < nBuckets; ++j) {
        b1 = `\refvar[Union2]{Union}{}`(b1, buckets[j].`\refvar[BucketInfo::bounds]{bounds}{}`);
        count1 += buckets[j].`\refvar[BucketInfo::count]{count}{}`;
    }
    cost[i] = .125f + (count0 * b0.`\refvar{SurfaceArea}{}`() +
                       count1 * b1.`\refvar{SurfaceArea}{}`()) / bounds.`\refvar{SurfaceArea}{}`();
}
\end{lstlisting}

有了所有开销,对数组{\ttfamily cost}的线性扫描找到最小开销的划分。
\begin{lstlisting}
`\initcode{Find bucket to split at that minimizes SAH metric}{=}`
`\refvar{Float}{}` minCost = cost[0];
int minCostSplitBucket = 0;
for (int i = 1; i < nBuckets - 1; ++i) {
    if (cost[i] < minCost) {
        minCost = cost[i];
        minCostSplitBucket = i;
    }
}
\end{lstlisting}

如果为划分选的桶边界有比用存在的图元构建节点更低的估计开销,
或者出现一个节点的图元超过了允许的最大数量,
则用函数{\ttfamily std::partition()}来完成
在数组{\ttfamily primitiveInfo}中记录节点的工作。
回想之前它的用法即该函数确保数组的所有对于给定判定函数返回{\ttfamily true}的元素
都出现在返回{\ttfamily false}的之前,
并且它返回指向第一个判定函数返回{\ttfamily false}的元素的指针。
因为我们之前任意设置估计的相交开销为1,
所以只是创建叶子结点的估计开销等于图元的数量{\ttfamily nPrimitives}。
\begin{lstlisting}
`\initcode{Either create leaf or split primitives at selected SAH bucket}{=}`
`\refvar{Float}{}` leafCost = nPrimitives;
if (nPrimitives > maxPrimsInNode || minCost < leafCost) {
    `\refvar{BVHPrimitiveInfo}{}` *pmid = std::partition(&primitiveInfo[start],
        &primitiveInfo[end-1]+1, 
        [=](const `\refvar{BVHPrimitiveInfo}{}` &pi) {
            int b = nBuckets * centroidBounds.`\refvar{Offset}{}`(pi.centroid)[dim];
            if (b == nBuckets) b = nBuckets - 1;
            return b <= minCostSplitBucket;
        });
    mid = pmid - &primitiveInfo[0];
} else {
    `\refcode{Create leaf BVHBuildNode}{}`
}
\end{lstlisting}

\subsection{线性包围盒层次}\label{sub:线性包围盒层次}
尽管用表面积启发法构建包围盒层次给出了很好的结果,
但该方法有两个缺点:第一,接收了许多传入的场景图元来在树的所有层次上计算SAH开销。
第二,自顶向下的BVH构建难以很好地并行化:
最明显的并行化方法——执行独立子树的并行化构建——
在直到该树顶部几层构建好前都受困于有限的独立任务,
这反过来又抑制了并行的可扩展性
(第二个问题在GPU上尤为突出,如果大规模并行化不可用则性能会很差)。

开发\keyindex{线性包围盒层次}{linear bounding volume hierarchy}{}(LBVH)来解决这些问题。
通过LBVH,以传递轻量级小次数的图元来构建树;
树的构建时间与图元数量呈线性关系。
而且算法快速地把图元划分为可以独立处理的\keyindex{群集}{cluster}{}。
该过程很容易并行化且很适合GPU实现。

LBVH背后的关键思想是把BVH构建变为排序问题。
因为没有排序多维数据的单一顺序函数,所以LBVH是基于\keyindex{莫顿码}{Morton code}{}的,
它将$n$维中的邻近点映射为1D直线上的邻近点,使之有明显的排序函数。
在图元排序后,空间上相邻的图元群集在排序数组的连续段内。

莫顿码基于简单的变换:给定$n$维整数坐标值,
其莫顿码表示由交错二进制坐标数位求得。
例如,考虑一个2D坐标$(x,y)$,其中$x$和$y$的数位表示为$x_i$和$y_i$.
相应的莫顿码值为
\begin{align*}
    \cdots y_3x_3y_2x_2y_1x_1y_0x_0\, .
\end{align*}

\reffig{4.7}展示了2D点按莫顿顺序的图示——
注意沿递归的“z”形路径访问它们
(因此莫顿路径有时也称为“z序”)。
我们可以看到2D中坐标相近的点通常在莫顿曲线上也是相近的
\footnote{许多GPU用莫顿布局在内存中存储纹理贴图。
    这样做的一个优点是当在四个纹素值之间执行双线性插值时,
    比起纹理按扫描线顺序排布,这些值更有极大可能在内存中相邻。
    反过来,这有利于纹理缓存性能。}。
\begin{figure}[htbp]
    \centering\input{Pictures/chap04/MortonBasic.tex}
    \caption{沿莫顿曲线访问点的顺序。沿$x$和$y$轴的坐标值用二进制表示。
        如果我们按其莫顿索引的顺序连接整数坐标点,我们可以看见莫顿曲线沿分层级的“z”形路径访问这些点。}
    \label{fig:4.7}
\end{figure}

莫顿编码值也编码了关于点所表示的位置的有用信息。
考虑2D中4位坐标值的情况:$x$和$y$坐标是$[0,15]$内的整数且
莫顿码有8位:$y_3x_3y_2x_2y_1x_1y_0x_0$.
编码会得出许多整数性质;一些例子包括:
\begin{itemize}
    \item 对于莫顿编码8位值中高位$y_3$置位的,我们就知道
          其本身的$y$坐标高位置位且因此$y\ge8$(\reffig{4.8}(a))。
    \item 下一位值$x_3$在中间划分$x$轴(\reffig{4.8}(b))。
          例如如果$y_3$置位而$x_3$没有,则相应的点一定位于\reffig{4.8}(c)的阴影区。
          具有许多高位匹配的点通常位于由匹配位值决定的幂2边长的对齐空间区域。
    \item $y_2$值将$y$轴划分为四个区域(\reffig{4.8}(d))。
\end{itemize}
\begin{figure}[htb]
    \centering\input{Pictures/chap04/MortonBinary.tex}\\
    \input{Pictures/chap04/MortonSplitHoriz.tex}
    \input{Pictures/chap04/MortonSplitVert.tex}
    \input{Pictures/chap04/MortonShadedRegion.tex}\\
    \input{Pictures/chap04/MortonSplitHoriz3.tex}
    \caption{莫顿编码的实现。莫顿值中不同位的值表示相应坐标所在的空间区域。
        (a)在2D中,一个点坐标的莫顿编码值高位定义了沿$y$轴中点的划分平面。
        如果该高位置位,则点在平面以上。(b)同样,莫顿值的第二高位在中间划分$x$轴。
        (c)如果高$y$位是1且高$x$位是0,则该点一定在阴影区域内。
        (d)第二高$y$位把$y$轴划分为四个区域。}
    \label{fig:4.8}
\end{figure}

另一个解释这些基于数位的性质的方法是按莫顿编码值来。
例如,\reffig{4.8}(a)对应在范围$[8,15]$内的索引,
\reffig{4.8}(c)对应$[8,11]$.因此,给定一组排序了的莫顿索引,
我们可以通过二分搜索找到对应于像\reffig{4.8}(c)区域那样的点的范围,
以在数组中找到每个终点。

LBVH是通过用每个空间区域中点处的划分平面
(即等价于之前定义的路径\refvar[Middle]{SplitMethod::Middle}{})划分图元而构建的BVH。
因为它基于上述的莫顿编码性质,所以划分非常高效。

仅仅以不同方式复现\refvar{Middle}{}并不有趣,所以这里的实现中我们将
构建\keyindex{分层线性包围盒层次}{hierarchical linear bounding volume hierarchy}{linear bounding volume hierarchy线性包围盒层次}(HLBVH)。
通过该方法首先用基于莫顿曲线的聚类为低层级构建树(以下称“小树”\sidenote{译者注:原文treelet。}),
然后用表面积启发法创建高层级树。
方法\refvar{HLBVHBuild}{()}实现了该方法并返回所得树的根节点。
\begin{lstlisting}
`\refcode{BVHAccel Method Definitions}{+=}\lastnext{BVHAccelMethodDefinitions}`
`\refvar{BVHBuildNode}{}` *`\refvar{BVHAccel}{}`::`\initvar{HLBVHBuild}{}`(`\refvar{MemoryArena}{}` &arena, 
        const std::vector<`\refvar{BVHPrimitiveInfo}{}`> &primitiveInfo,
        int *totalNodes,
        std::vector<std::shared_ptr<`\refvar{Primitive}{}`>> &orderedPrims) const {
    `\refcode{Compute bounding box of all primitive centroids}{}`
    `\refcode{Compute Morton indices of primitives}{}`
    `\refcode{Radix sort primitive Morton indices}{}`
    `\refcode{Create LBVH treelets at bottom of BVH}{}`
    `\refcode{Create and return SAH BVH from LBVH treelets}{}`
}
\end{lstlisting}

只用图元边界框形心构建BVH来对其排序——
它不考虑每个图元的实际空间范围。
该简化对HLBVH提供的性能很关键,
但它也意味着对于一些占有很大尺寸范围的场景而言,
构建的树不会像基于SAH的树那样考虑这些特例。

因为莫顿编码在整数坐标上操作,我们首先需要定界所有图元的形心,
这样我们就可以量化相对于整体边界的形心位置。
\begin{lstlisting}
`\initcode{Compute bounding box of all primitive centroids}{=}`
`\refvar{Bounds3f}{}` bounds;
for (const `\refvar{BVHPrimitiveInfo}{}` &pi : primitiveInfo)
    bounds = `\refvar[Union2]{Union}{}`(bounds, pi.`\refvar{centroid}{}`);
\end{lstlisting}

有了整体边界,我们现在可以为每个图元计算莫顿码了。
这是非常轻量的计算,但假定可能有数百万图元,则它是值得并行化的。
注意下面把512的循环块尺寸传入\refvar{ParallelFor}{()};
这让工作线程每次要处理512组图元而不是默认的一组。
因为每个图元计算莫顿码要执行的工作量相对较小,
这粒度更好地把分布式任务的开销分摊给了工作线程。
\begin{lstlisting}
`\initcode{Compute Morton indices of primitives}{=}`
std::vector<`\refvar{MortonPrimitive}{}`> mortonPrims(primitiveInfo.size());
`\refvar{ParallelFor}{}`(
    [&](int i) {
        `\refcode{Initialize mortonPrims[i] for ith primitive}{}`
    }, primitiveInfo.size(), 512);
\end{lstlisting}

为每个图元都创建一个实例\refvar{MortonPrimitive}{};
它保存了图元在数组\newline{\ttfamily primitiveInfo}中的图元索引及其莫顿码。
\begin{lstlisting}
`\refcode{BVHAccel Local Declarations}{+=}\lastnext{BVHAccelLocalDeclarations}`
struct `\initvar{MortonPrimitive}{}` {
    int `\initvar{primitiveIndex}{}`;
    uint32_t `\initvar{mortonCode}{}`;
};
\end{lstlisting}

我们为每个$x$、$y$和$z$坐标用10位数,这样莫顿码一共30位。
该粒度允许值与单个32位变量相容。
边界框内的浮点形心偏移量在$[0,1]$内,
所以我们用$2^{10}$缩放它们得到10位表示的整数坐标
(对于恰等于1的边界情况,可能得到出界的量化值1024;
这种情况在即将介绍的函数\refvar{LeftShift3}{()}中处理)。
\begin{lstlisting}
`\initcode{Initialize mortonPrims[i] for ith primitive}{=}`
constexpr int mortonBits = 10;
constexpr int mortonScale = 1 << mortonBits;
mortonPrims[i].`\refvar{primitiveIndex}{}` = primitiveInfo[i].`\refvar{primitiveNumber}{}`;
`\refvar{Vector3f}{}` centroidOffset = bounds.`\refvar{Offset}{}`(primitiveInfo[i].`\refvar{centroid}{}`);
mortonPrims[i].`\refvar{mortonCode}{}` = `\refvar{EncodeMorton3}{}`(centroidOffset * mortonScale);
\end{lstlisting}

为计算3D莫顿码,首先我们定义一个辅助函数:\refvar{LeftShift3}{()}接收
一个32位值并返回把第$i$位移到第$3i$位的结果,剩下位为零。
\reffig{4.9}说明了该操作。
\begin{figure}[htbp]
    \centering\includegraphics[width=0.75\linewidth]{chap04/LeftShift3.eps}
    \caption{移位以计算3D莫顿码。函数\refvar{LeftShift3}{()}接收32位整数值,
        且对于低10位,把第$i$位移到第$3i$位的位置——换句话说,向左移$2i$位。剩下全部位设为零。}
    \label{fig:4.9}
\end{figure}

实现该操作最明显的方法,即单独移动每一位,并不是最高效的
(它需要总共9次位移以及逻辑或来计算最终值)。
取而代之的是,我们可以把每位的移动分解为多个幂2尺寸的移位且一并把数位值移到其最终位置。
然后,所有需要移动给定幂2数位的位可以一并移动。
函数\refvar{LeftShift3}{()}实现了该计算,\reffig{4.10}展示了它如何工作的
\sidenote{译者注:原图第4行“9”的位置有误,此处已修正。}。
\begin{figure}[htbp]
    \centering\includegraphics[width=0.75\linewidth]{Pictures/chap04/Mortonpow2decomposition.eps}
    \caption{莫顿移位的幂2分解。通过一系列幂2尺寸的移动来为每个3D坐标计算莫顿码执行移位。
        首先,位8和9向左移16位,这把位8放在了其最终位置。接着位4到7移动8位。
        移动4和2位后(适当掩模使每位最终都移动适当数位),所有数位都在正确的位置上。
        该计算由函数\refvar{LeftShift3}{()}实现。}
    \label{fig:4.10}
\end{figure}

\begin{lstlisting}
`\initcode{BVHAccel Utility Functions}{=}\initnext{BVHAccelUtilityFunctions}`
inline uint32_t `\initvar{LeftShift3}{}`(uint32_t x) {
    if (x == (1 << 10)) --x;
    x = (x | (x << 16)) & 0b00000011000000000000000011111111;
    x = (x | (x <<  8)) & 0b00000011000000001111000000001111;
    x = (x | (x <<  4)) & 0b00000011000011000011000011000011;
    x = (x | (x <<  2)) & 0b00001001001001001001001001001001;
    return x;
}
\end{lstlisting}

\refvar{EncodeMorton3}{()}接收每个分量都是$0$到$2^{10}$之间浮点值的3D坐标值。
它把这些值转换为整数然后通过展开三个10位量化值使第$i$位在位置$3i$上来计算莫顿码,
然后$y$位再移一位,$z$位再移两位,全部结果求或(\reffig{4.11})。
\begin{figure}[htbp]
    \centering\includegraphics[width=0.75\linewidth]{chap04/Mortonxyzinterleave.eps}
    \caption{最终交错坐标值。有了\refvar{LeftShift3}{()}为$x$、$y$和$z$计算的交错值,
        最终莫顿编码值通过分别移动$y$和$z$一位和两位然后全部结果求或算得。}
    \label{fig:4.11}
\end{figure}

\begin{lstlisting}
`\refcode{BVHAccel Utility Functions}{+=}\lastnext{BVHAccelUtilityFunctions}`
inline uint32_t `\initvar{EncodeMorton3}{}`(const `\refvar{Vector3f}{}` &v) {
    return (`\refvar{LeftShift3}{}`(v.z) << 2) | (`\refvar{LeftShift3}{}`(v.y) << 1) |
            `\refvar{LeftShift3}{}`(v.x);
}
\end{lstlisting}

一旦计算了莫顿索引,我们将用\keyindex{基数排序}{radix sort}{}按莫顿索引值对
\newline{\ttfamily mortonPrims}排序。
我们已经发现对于BVH的构建,我们的基数排序实现明显快于使用我们系统标准库的{\ttfamily std::sort()}
(它是\keyindex{快速排序}{quicksort}{}和\keyindex{插入排序}{insertion sort}{}的混合)。
\begin{lstlisting}
`\initcode{Radix sort primitive Morton indices}{=}`
`\refvar{RadixSort}{}`(&mortonPrims);
\end{lstlisting}

回想基数排序和大多数排序算法的不同在于它不是基于比较一对值
而是基于依赖一些键值的桶子项。基数排序可用于排序整数值,
它从最右边数位到最左边每次排序一个数码。
它尤其值得每次对二进制值排序多个数码;
这样做减少了传递数据的次数。
这里的实现中,{\ttfamily bitsPerPass}设置每次传递处理的位数;
取值6后,我们有5次传递来排序30位。
\begin{lstlisting}
`\refcode{BVHAccel Utility Functions}{+=}\lastcode{BVHAccelUtilityFunctions}`
static void `\initvar{RadixSort}{}`(std::vector<`\refvar{MortonPrimitive}{}`> *v) {
    std::vector<`\refvar{MortonPrimitive}{}`> tempVector(v->size());
    constexpr int bitsPerPass = 6;
    constexpr int nBits = 30;
    constexpr int nPasses = nBits / bitsPerPass;
    for (int pass = 0; pass < nPasses; ++pass) {
        `\refcode{Perform one pass of radix sort, sorting bitsPerPass bits}{}`
    }
    `\refcode{Copy final result from tempVector, if needed}{}`
}
\end{lstlisting}

当前传递会排序{\ttfamily bitsPerPass}位,从{\ttfamily lowBit}开始。
\begin{lstlisting}
`\initcode{Perform one pass of radix sort, sorting bitsPerPass bits}{=}`
int lowBit = pass * bitsPerPass;
`\refcode{Set in and out vector pointers for radix sort pass}{}`
`\refcode{Count number of zero bits in array for current radix sort bit}{}`
`\refcode{Compute starting index in output array for each bucket}{}`
`\refcode{Store sorted values in output array}{}`
\end{lstlisting}

{\ttfamily in}和{\ttfamily out}引用分别对应于要排序的向量以及保存排序值的向量。
每次通过循环的传递都在输入向量{\ttfamily *v}和临时向量之间交替。
\begin{lstlisting}
`\initcode{Set in and out vector pointers for radix sort pass}{=}`
std::vector<`\refvar{MortonPrimitive}{}`> &in  = (pass & 1) ? tempVector : *v;
std::vector<`\refvar{MortonPrimitive}{}`> &out = (pass & 1) ? *v : tempVector;
\end{lstlisting}

如果我们每次传递中排序$n$位,则每个值可能落入的桶有$2^n$个。
我们先计数每个桶中会落入多少个值;这能让我们确定在输出数组中的何处保存排序值。
为了给当前值计算桶索引,该实现对索引移位使得在索引{\ttfamily lowBit}上的位
位于0号位,再掩模低处{\ttfamily bitsPerPass}个位。
\begin{lstlisting}
`\initcode{Count number of zero bits in array for current radix sort bit}{=}`
constexpr int nBuckets = 1 << bitsPerPass;
int bucketCount[nBuckets] = { 0 };
constexpr int bitMask = (1 << bitsPerPass) - 1;
for (const `\refvar{MortonPrimitive}{}` &mp : in) {
    int bucket = (mp.`\refvar{mortonCode}{}` >> lowBit) & bitMask;
    ++bucketCount[bucket];
}
\end{lstlisting}

有了每个桶中落入值的数量,我们就可以计算在输出数组中
每个桶的值开始处的偏移量;这就是之前的桶中落入值数量的和。
\begin{lstlisting}
`\initcode{Compute starting index in output array for each bucket}{=}`
int outIndex[nBuckets];
outIndex[0] = 0;
for (int i = 1; i < nBuckets; ++i)
    outIndex[i] = outIndex[i - 1] + bucketCount[i - 1];
\end{lstlisting}

现在我们知道每个桶在何处开始排序值了,
可以接收另一次图元传递来重算每个图元所在的桶并
将其\refvar{MortonPrimitive}{}保存在输出数组中。
这为当前这组数位完成了排序传递。
\begin{lstlisting}
`\initcode{Store sorted values in output array}{=}`
for (const `\refvar{MortonPrimitive}{}` &mp : in) {
    int bucket = (mp.`\refvar{mortonCode}{}` >> lowBit) & bitMask;
    out[outIndex[bucket]++] = mp;
}
\end{lstlisting}

当完成排序时,如果执行了奇数次基数排序传递,则最终排序值需要
从临时向量复制到原来传入\refvar{RadixSort}{()}的输出向量。
\begin{lstlisting}
`\initcode{Copy final result from tempVector, if needed}{}`
if (nPasses & 1)
    std::swap(*v, tempVector);
\end{lstlisting}

有了图元排序数组,我们将用附近的形心求得图元群集,
再在每个群集内创建图元上的LBVH。
这一步很适合并行化,因为通常有很多群集且每个群集都可以独立处理。
\begin{lstlisting}
`\initcode{Create LBVH treelets at bottom of BVH}{=}`
`\refcode{Find intervals of primitives for each treelet}{}`
`\refcode{Create LBVHs for treelets in parallel}{}`
\end{lstlisting}

每个图元群集都表示为一个\refvar{LBVHTreelet}{}。
它对群集中第一个图元在{\ttfamily mortonPrims}数组中的索引
以及后续图元的数量进行编码(见\reffig{4.12})。
\begin{lstlisting}
`\refcode{BVHAccel Local Declarations}{+=}\lastnext{BVHAccelLocalDeclarations}`
struct `\initvar{LBVHTreelet}{}` {
   int `\initvar{startIndex}{}`, `\initvar[LBVHTreelet:nPrimitives]{nPrimitives}{}`;
   `\refvar{BVHBuildNode}{}` *`\initvar{buildNodes}{}`;
};
\end{lstlisting}

\begin{figure}[htbp]
    \centering\input{Pictures/chap04/LBVHtreeletclusters.tex}
    \caption{LBVH小树的图元群集。图元形心聚类到覆盖其边界的均匀网格内。
        为每个格子中位于排序后的莫顿索引值连续段内的图元群集创建LBVH。}
    \label{fig:4.12}
\end{figure}

回想\reffig{4.8}中,莫顿码高位值匹配的点集位于原始盒中一个幂2对齐和幂2边长的子集内。
因为我们已经用莫顿编码值对数组{\ttfamily mortonPrims}排过序了,
所以高位值匹配的图元已经位于一段连续数组中。

这里我们将求30位莫顿码中有相同的高12位对应的图元集。
通过线性浏览数组{\ttfamily mortonPrims}来求得群集并
找到这12位中有任何一位发生变化的地方。
这对应了在$2^{12}=4096$个规范网格中对图元聚类,
每维有$2^4=16$个格子。
实践中,许多网格是空的,
尽管这里我们仍然希望求得许多独立的群集。
\begin{lstlisting}
`\initcode{Find intervals of primitives for each treelet}{=}`
std::vector<`\refvar{LBVHTreelet}{}`> treeletsToBuild;
for (int start = 0, end = 1; end <= (int)mortonPrims.size(); ++end) {
    uint32_t mask = 0b00111111111111000000000000000000;
    if (end == (int)mortonPrims.size() ||
        ((mortonPrims[start].`\refvar{mortonCode}{}` & mask) !=
         (mortonPrims[end].`\refvar{mortonCode}{}` & mask))) {
        `\refcode{Add entry to treeletsToBuild for this treelet}{}`
        start = end;
    }
}
\end{lstlisting}

当为一个小树求得一个图元群集后,就立即为它分配\refvar{BVHBuildNode}{}
(回想一个BVH中的节点数量不超过叶子节点数量的两倍,而后者又不超过图元数量)。
在现在的串行执行阶段中预分配这些内存比在LBVH的并行构建中分配更简单。

这里的一个重要细节是传入\refvar[MemoryArena:Alloc2]{MemoryArena::Alloc}{()}的{\ttfamily false}值;
它表示不要执行正在分配的底层对象的构造函数。
令人惊讶的是,运行\refvar{BVHBuildNode}{}构造函数会引入很大开销
并明显降低构建HLBVH时的整体性能。
因为在后面的代码中\refvar{BVHBuildNode}{}的所有成员都会初始化,
所以无论如何这里都没必要用构造函数执行初始化。

\begin{lstlisting}
`\initcode{Add entry to treeletsToBuild for this treelet}{=}`
int nPrimitives = end - start;
int maxBVHNodes = 2 * nPrimitives - 1;
`\refvar{BVHBuildNode}{}` *nodes = arena.`\refvar[MemoryArena:Alloc2]{Alloc}{}`<`\refvar{BVHBuildNode}{}`>(maxBVHNodes, false);
treeletsToBuild.push_back({start, nPrimitives, nodes});
\end{lstlisting}

一旦每个小树的图元都定好了,我们就可以并行地为它们创建LBVH了。
当构建完成后,每个\refvar{LBVHTreelet}{}的指针\refvar{buildNodes}{}会指向相应LBVH的根。

工作线程在构建LBVH时有两个地方必须相互配合。
第一,需要计算所有LBVH中的节点总数并通过传入\refvar{HLBVHBuild}{()}的
指针{\ttfamily totalNode}返回。
第二,当为LBVH创建叶子节点时,
需要数组{\ttfamily orderedPrims}的连续一段来
记录叶子节点中图元的索引。
我们的实现为两者都使用了原子变量——
{\ttfamily atomicTotal}跟踪节点数目,
{\ttfamily orderedPrimsOffset}跟踪{\ttfamily orderedPrims}下一有效项的索引。
\begin{lstlisting}
`\initcode{Create LBVHs for treelets in parallel}{=}`
std::atomic<int> atomicTotal(0), orderedPrimsOffset(0);
orderedPrims.resize(primitives.size());
`\refvar{ParallelFor}{}`(
    [&](int i) {
        `\refcode{Generate ith LBVH treelet}{}`
    }, treeletsToBuild.size());
*totalNodes = atomicTotal;
\end{lstlisting}

构建小树的工作由\refvar{emitLBVH}{()}执行,
它取用形心位于某空间区域内的图元并不断用划分平面
沿着三轴之一上的区域中心将当前空间区域划分为两半。

注意到\refvar{emitLBVH}{()}没有用
指向原子变量{\ttfamily atomicTotal}的指针来计数创建的节点,
而是更新一个非原子局部变量。
然后当每棵小树建成时这里的代码片只对每棵小树更新一次{\ttfamily atomicTotal}。
该方法比让工作线程在其执行过程中频繁修改{\ttfamily atomicTotal}具有明显更好的性能
(见附录\refsub{内存连续模型与性能}关于多核内存连续模型开销的讨论)。
\begin{lstlisting}
`\initcode{Generate ith LBVH treelet}{=}`
int nodesCreated = 0;
const int firstBitIndex = 29 - 12;
`\refvar{LBVHTreelet}{}` &tr = treeletsToBuild[i];
tr.`\refvar{buildNodes}{}` = 
    `\refvar{emitLBVH}{}`(tr.`\refvar{buildNodes}{}`, primitiveInfo, &mortonPrims[tr.`\refvar{startIndex}{}`],
             tr.`\refvar[LBVHTreelet:nPrimitives]{nPrimitives}{}`, &nodesCreated, orderedPrims,
             &orderedPrimsOffset, firstBitIndex);
atomicTotal += nodesCreated;
\end{lstlisting}

幸好有莫顿编码,当前空间区域不需要在\refvar{emitLBVH}{()}中显式表示:
传入的有序{\ttfamily mortonPrims}有一些匹配的高位,反过来对应了一个空间框。
对于莫顿码中剩下的每一位,该函数尝试沿着对应于数位{\ttfamily bitIndex}的平面来划分图元
(回想\reffig{4.8}(d)),然后再递归地调用自己。
尝试用来划分的下一位索引被作为该函数的最后一个参数传入:它最初为$29-12$,
因为29是从零计起的第30位索引,而我们之前用了高12位莫顿码值来聚类图元;
所以,我们知道那些数位都一定匹配该群集。
\begin{lstlisting}
`\refcode{BVHAccel Method Definitions}{+=}\lastnext{BVHAccelMethodDefinitions}`
`\refvar{BVHBuildNode}{}` *`\refvar{BVHAccel}{}`::`\initvar{emitLBVH}{}`(`\refvar{BVHBuildNode}{}` *&buildNodes,
        const std::vector<`\refvar{BVHPrimitiveInfo}{}`> &primitiveInfo,
        `\refvar{MortonPrimitive}{}` *mortonPrims, int nPrimitives, int *totalNodes,
        std::vector<std::shared_ptr<`\refvar{Primitive}{}`>> &orderedPrims,
        std::atomic<int> *orderedPrimsOffset, int bitIndex) const {
    if (bitIndex == -1 || nPrimitives < maxPrimsInNode) {
        `\refcode{Create and return leaf node of LBVH treelet}{}`
    } else {
        int mask = 1 << bitIndex;
        `\refcode{Advance to next subtree level if there's no LBVH split for this bit}{}`
        `\refcode{Find LBVH split point for this dimension}{}`
        `\refcode{Create and return interior LBVH node}{}`
    }
}
\end{lstlisting}

在\refvar{emitLBVH}{()}用最后低位划分完图元后,
不可能再有更多划分了,就创建一个叶子节点。
或者它也可以在只有很少的图元时就停止并创建叶子节点。

回想{\ttfamily orderedPrimsOffset}是数组{\ttfamily orderedPrims}中
下一有效元素的偏移量。
这里,对{\ttfamily fetch\_add()}的调用原子地将{\ttfamily nPrimitives}的值加到\newline
{\ttfamily orderedPrimsOffset}上并返回相加前它的旧值。
因为这些计算是原子的,多个LBVH构建线程可以
同时分割数组{\ttfamily orderedPrims}中的空间且没有数据竞争。
有了数组中的空间,叶子的构建和之前\refcode{Create leaf BVHBuildNode}{}中实现的方法一样。
\begin{lstlisting}
`\initcode{Create and return leaf node of LBVH treelet}{=}`
(*totalNodes)++;
`\refvar{BVHBuildNode}{}` *node = buildNodes++;
`\refvar{Bounds3f}{}` bounds;
int firstPrimOffset = orderedPrimsOffset->fetch_add(nPrimitives);
for (int i = 0; i < nPrimitives; ++i) {
    int primitiveIndex = mortonPrims[i].`\refvar{primitiveIndex}{}`;
    orderedPrims[firstPrimOffset + i] = primitives[primitiveIndex];
    bounds = `\refvar[Union2]{Union}{}`(bounds, primitiveInfo[primitiveIndex].`\refvar[BVHPrimitiveInfo::bounds]{bounds}{}`);
}
node->`\refvar[BVHBuildNode::InitLeaf]{InitLeaf}{}`(firstPrimOffset, nPrimitives, bounds);
return node;
\end{lstlisting}

可能有所有图元都位于划分平面同一侧的情况;因为图元按其莫顿索引排序,
可以通过看该范围内第一个和最后一个图元对于该平面是否有相同的数位来高效检测该情况。
这种情况下,\refvar{emitLBVH}{()}行进到下一位而不会无用地创建一个节点。
\begin{lstlisting}
`\initcode{Advance to next subtree level if there's no LBVH split for this bit}{=}`
if ((mortonPrims[0].`\refvar{mortonCode}{}` & mask) ==
    (mortonPrims[nPrimitives - 1].`\refvar{mortonCode}{}` & mask))
    return `\refvar{emitLBVH}{}`(buildNodes, primitiveInfo, mortonPrims, nPrimitives,
                    totalNodes, orderedPrims, orderedPrimsOffset,
                    bitIndex - 1);
\end{lstlisting}

如果划分平面两侧都有图元,则二分搜索会高效地找到
当前图元集中第{\ttfamily bitIndex}位从0变为1的划分点。
\begin{lstlisting}
`\initcode{Find LBVH split point for this dimension}{=}`
int searchStart = 0, searchEnd = nPrimitives - 1;
while (searchStart + 1 != searchEnd) {
    int mid = (searchStart + searchEnd) / 2;
    if ((mortonPrims[searchStart].`\refvar{mortonCode}{}` & mask) ==
        (mortonPrims[mid].`\refvar{mortonCode}{}` & mask))
        searchStart = mid;
    else
        searchEnd = mid;
}
int splitOffset = searchEnd;
\end{lstlisting}

有了划分偏移量,该方法现在可以声明一个节点用作内部节点
并为划出的两个图元集递归地构建LBVH了。
注意到莫顿编码还带来了一个方便:
不需要为了划分而复制或记录数组{\ttfamily mortonPrims}中的项:
因为它们都按莫顿码值排序了且从高到低处理数位,
两部分图元已经在划分平面的正确一侧。
\begin{lstlisting}
`\initcode{Create and return interior LBVH node}{=}`
(*totalNodes)++;
`\refvar{BVHBuildNode}{}` *node = buildNodes++;
`\refvar{BVHBuildNode}{}` *lbvh[2] = {
    `\refvar{emitLBVH}{}`(buildNodes, primitiveInfo, mortonPrims, splitOffset,
             totalNodes, orderedPrims, orderedPrimsOffset, bitIndex - 1),
    `\refvar{emitLBVH}{}`(buildNodes, primitiveInfo, &mortonPrims[splitOffset],
             nPrimitives - splitOffset, totalNodes, orderedPrims,
             orderedPrimsOffset, bitIndex - 1)
};
int axis = bitIndex % 3;
node->`\refvar[BVHBuildNode::InitInterior]{InitInterior}{}`(axis, lbvh[0], lbvh[1]);
return node;
\end{lstlisting}

一旦创建好所有LBVH小树,
\refvar{buildUpperSAH}{()}就为所有小树创建一个BVH。
因为它们通常只有几十个或几百个(无论如何不超过4096),
该步骤花的时间很少。
\begin{lstlisting}
`\initcode{Create and return SAH BVH from LBVH treelets}{=}`
std::vector<`\refvar{BVHBuildNode}{}` *> finishedTreelets;
for (`\refvar{LBVHTreelet}{}` &treelet : treeletsToBuild)
    finishedTreelets.push_back(treelet.`\refvar{buildNodes}{}`);
return `\refvar{buildUpperSAH}{}`(arena, finishedTreelets, 0,
                     finishedTreelets.size(), totalNodes);
\end{lstlisting}

这里不介绍该方法的实现,
因为它和完全基于SAH的BVH构建遵循相同的方法,
只不过是在小树根节点上而不是在场景图元上进行。
\begin{lstlisting}
`\initcode{BVHAccel Private Methods}{=}`
`\refvar{BVHBuildNode}{}` *`\initvar{buildUpperSAH}{}`(`\refvar{MemoryArena}{}` &arena,
    std::vector<`\refvar{BVHBuildNode}{}` *> &treeletRoots, int start, int end,
    int *totalNodes) const;
\end{lstlisting}

\subsection{为遍历而压实的BVH}\label{sub:为遍历而压实的BVH}
一旦建好BVH树,最后一步就是将其转换为紧凑的\sidenote{译者注:原文compact。}表达——
这样做能提升缓存、内存以及整个系统的性能。
最后的BVH存于内存中的一个线性数组内。
原始树的节点按\keyindex{深度优先}{depth-first}{}顺序排布,
即意味着在内存中每个内部节点的第一个孩子会立刻排在该节点之后。
这种情况下,只需要显式保存每个内部节点第二个孩子的偏移量。
见\reffig{4.13}关于树的拓扑与内存中节点顺序之间关系的图示。
\begin{figure}[htbp]
    \centering\input{Pictures/chap04/BVHlinearization.tex}
    \caption{BVH在内存中的线性排布。BVH的节点(左)按深度优先顺序(右)存储于内存中。
        因此,对于该树的任意中间节点(例如该例中的A和B),
        第一个孩子可在内存中父节点之后立刻找到。
        第二个孩子则通过偏移指针找到,这里用带箭头的线表示。
        树的叶子结点(D、E和C)没有孩子。}
    \label{fig:4.13}
\end{figure}

结构体\refvar{LinearBVHNode}{}保存遍历BVH所需的信息。
除了每个节点的边界框,它还为每个叶子节点保存偏移量和该节点内的图元数量。
对于内部节点,它保存了第二个孩子的偏移量以及构建层次时
是沿哪个坐标轴划分图元的;
这些信息用于下面的遍历例程以沿着光线按从前往后的顺序访问节点。
\begin{lstlisting}
`\refcode{BVHAccel Local Declarations}{+=}\lastcode{BVHAccelLocalDeclarations}`
struct `\initvar{LinearBVHNode}{}` {
    `\refvar{Bounds3f}{}` `\initvar[LinearBVHNode::bounds]{bounds}{}`;
    union {
        int `\initvar{primitivesOffset}{}`;    // leaf
        int `\initvar{secondChildOffset}{}`;   // interior
    };
    uint16_t `\initvar[LinearBVHNode::nPrimitives]{nPrimitives}{}`;  // 0 -> interior node
    uint8_t `\initvar[LinearBVHNode::axis]{axis}{}`;          // interior node: xyz
    uint8_t `\initvar[LinearBVHNode::pad]{pad}{}`[1];        // ensure 32 byte total size
};
\end{lstlisting}

该结构体被填充了以保证是32字节大小。
这样做保证了如果分配节点时第一个节点是对齐\keyindex{缓存行}{cache line}{}的,
则后续节点不会跨越缓存行
(只要缓存行大小至少为32字节,即现代CPU架构的情况)。

建好的树被方法\refvar{flattenBVHTree}{()}变换为\refvar{LinearBVHNode}{}表示,
它执行深度优先遍历并在内存中按线性顺序存储节点。
\begin{lstlisting}
`\initcode{Compute representation of depth-first traversal of BVH tree}{=}`
nodes = AllocAligned<`\refvar{LinearBVHNode}{}`>(totalNodes);
int offset = 0;
`\refvar{flattenBVHTree}{}`(root, &offset);
\end{lstlisting}

指向\refvar{LinearBVHNode}{}数组的指针保存为\refvar{BVHAccel}{}的一个成员变量,
所以它可以在\refvar{BVHAccel}{}的析构函数中释放。
\begin{lstlisting}
`\refcode{BVHAccel Private Data}{+=}\lastcode{BVHAccelPrivateData}`
`\refvar{LinearBVHNode}{}` *`\initvar[BVHAccel::nodes]{nodes}{}` = nullptr;
\end{lstlisting}

将树展平为线性表示很简单;
参数{\ttfamily *offset}跟踪当前在数组\refvar{BVHAccel::nodes}{}
中的偏移量。注意在递归调用处理其孩子之前
(如果该节点是内部节点)要把当前节点添加到该数组中。
\begin{lstlisting}
`\refcode{BVHAccel Method Definitions}{+=}\lastnext{BVHAccelMethodDefinitions}`
int `\refvar{BVHAccel}{}`::`\initvar{flattenBVHTree}{}`(`\refvar{BVHBuildNode}{}` *node, int *offset) {
    `\refvar{LinearBVHNode}{}` *linearNode = &`\refvar[BVHAccel::nodes]{nodes}{}`[*offset];
    linearNode->`\refvar[LinearBVHNode::bounds]{bounds}{}` = node->`\refvar[BVHBuildNode::bounds]{bounds}{}`;
    int myOffset = (*offset)++;
    if (node->`\refvar[BVHBuildNode:nPrimitives]{nPrimitives}{}` > 0) {
        linearNode->`\refvar{primitivesOffset}{}` = node->`\refvar{firstPrimOffset}{}`;
        linearNode->`\refvar[LinearBVHNode::nPrimitives]{nPrimitives}{}` = node->`\refvar[BVHBuildNode:nPrimitives]{nPrimitives}{}`;
    } else {
        `\refcode{Create interior flattened BVH node}{}`
    }
    return myOffset;
}
\end{lstlisting}

在内部节点时,递归调用会展平两棵子树。
第一棵如愿在数组中当前节点之后立刻结束,
而其递归调用\refvar{flattenBVHTree}{()}返回的第二棵偏移量
则保存在该节点的成员\refvar{secondChildOffset}{}中。
\begin{lstlisting}
`\initcode{Create interior flattened BVH node}{=}`
linearNode->`\refvar[LinearBVHNode::axis]{axis}{}` = node->`\refvar[BVHBuildNode::splitAxis]{splitAxis}{}`;
linearNode->`\refvar[LinearBVHNode::nPrimitives]{nPrimitives}{}` = 0;
`\refvar{flattenBVHTree}{}`(node->`\refvar[BVHBuildNode::children]{children}{}`[0], offset);
linearNode->`\refvar{secondChildOffset}{}` =
    `\refvar{flattenBVHTree}{}`(node->`\refvar[BVHBuildNode::children]{children}{}`[1], offset);
\end{lstlisting}

\subsection{遍历}\label{sub:遍历}
BVH的遍历代码非常简单——没有递归调用,只有少量数据用来维护当前遍历的状态。
方法\refvar[BVHAccel::Intersect]{Intersect}{()}从
预先计算一些与将要反复用到的光线相关的值开始。
\begin{lstlisting}
`\refcode{BVHAccel Method Definitions}{+=}\lastcode{BVHAccelMethodDefinitions}`
bool `\refvar{BVHAccel}{}`::`\initvar[BVHAccel::Intersect]{Intersect}{}`(const `\refvar{Ray}{}` &ray,
        `\refvar{SurfaceInteraction}{}` *isect) const {
    bool hit = false;
    `\refvar{Vector3f}{}` invDir(1 / ray.d.x, 1 / ray.d.y, 1 / ray.d.z);
    int dirIsNeg[3] = { invDir.x < 0, invDir.y < 0, invDir.z < 0 };
    `\refcode{Follow ray through BVH nodes to find primitive intersections}{}`
    return hit;
}
\end{lstlisting}

每当\refvar[BVHAccel::Intersect]{Intersect}{()}中的{\ttfamily while}循环开始一次迭代时,
{\ttfamily currentNodeIndex}保有将要访问的节点在数组\refvar[BVHAccel::nodes]{nodes}{}中的偏移量。
它起始于0值,表示树根。仍需要访问的节点存于数组{\ttfamily nodesToVisit[]}中,即充当一个栈;
{\ttfamily toVisitOffset}存有栈中下一个可弹出元素的偏移量。
\begin{lstlisting}
`\initcode{Follow ray through BVH nodes to find primitive intersections}{=}`
int toVisitOffset = 0, currentNodeIndex = 0;
int nodesToVisit[64];
while (true) {
    const `\refvar{LinearBVHNode}{}` *node = &`\refvar[BVHAccel::nodes]{nodes}{}`[currentNodeIndex];
    `\refcode{Check ray against BVH node}{}`
}
\end{lstlisting}

对于每个节点,我们都检查光线是否与节点的边界框相交(或从其内部发射)。
如果是我们就访问该节点,如果它是叶子节点就对其图元做相交测试,
如果是内部节点就处理它的孩子。
如果发现没有相交,就从{\ttfamily nodesToVisit[]}取出下一个将要访问的节点的偏移量
(或者如果栈空了就完成遍历了)。
\begin{lstlisting}
`\initcode{Check ray against BVH node}{=}`
if (node->`\refvar[LinearBVHNode::bounds]{bounds}{}`.`\refvar[Bounds3::IntersectP2]{IntersectP}{}`(ray, invDir, dirIsNeg)) {
    if (node->`\refvar[LinearBVHNode::nPrimitives]{nPrimitives}{}` > 0) {
        `\refcode{Intersect ray with primitives in leaf BVH node}{}`
    } else {
        `\refcode{Put far BVH node on nodesToVisit stack, advance to near node}{}`
    }
} else {
    if (toVisitOffset == 0) break;
    currentNodeIndex = nodesToVisit[--toVisitOffset];
}
\end{lstlisting}

如果当前节点是叶子,则该光线必须和它里面的图元做相交测试。
然后从栈{\ttfamily nodesToVisit}中找到下一个要访问的节点;
即使当前节点求得了交点,也必须访问剩下的节点,
以防万一它们中有一个给出更近的交点。
然而,如果求得一个交点,则该光线的\refvar{tMax}{}值将会更新为该相交距离;
这样可以更高效地丢弃剩下的节点中任何比该距离更远的部分。
\begin{lstlisting}
`\initcode{Intersect ray with primitives in leaf BVH node}{=}`
for (int i = 0; i < node->`\refvar[LinearBVHNode::nPrimitives]{nPrimitives}{}`; ++i)
    if (`\refvar[BVHAccel::primitives]{primitives}{}`[node->`\refvar{primitivesOffset}{}` + i]->`\refvar[Primitive::Intersect]{Intersect}{}`(ray, isect))
        hit = true;
if (toVisitOffset == 0) break;
currentNodeIndex = nodesToVisit[--toVisitOffset];
\end{lstlisting}

对于光线命中的内部节点,需要访问它的两个孩子。
如上所述,万一与光线相交的图元在第一个里面,
则访问光线穿过的第一个孩子再访问第二个是可取的,
这样光线的\refvar{tMax}{}值就能更新,
进而缩减光线的范围以及与之相交的边界框节点数目。

一个执行从前往后遍历而不会因光线与两个子节点相交
以及比较距离而带来开销的高效方法是使用光线方向向量
在该节点划分图元时所沿的坐标轴上的符号:
如果符号为负,我们应该先访问第二个孩子再访问第一个孩子,
因为进入第二棵子树的图元在划分点的上面一侧
(正号方向则反过来)。
这样做很简单:先要访问的节点的偏移量被复制到{\ttfamily currentNodeIndex},
另一个节点的偏移量被加到栈{\ttfamily nodesToVisit}中
(回想因为内存中节点按深度优先排列,第一个孩子正好在当前节点之后)。
\begin{lstlisting}
`\initcode{Put far BVH node on nodesToVisit stack, advance to near node}{=}`
if (dirIsNeg[node->`\refvar[LinearBVHNode::axis]{axis}{}`]) {
   nodesToVisit[toVisitOffset++] = currentNodeIndex + 1;
   currentNodeIndex = node->`\refvar{secondChildOffset}{}`;
} else {
   nodesToVisit[toVisitOffset++] = node->`\refvar{secondChildOffset}{}`;
   currentNodeIndex = currentNodeIndex + 1;
}
\end{lstlisting}

方法{\initvar{BVHAccel::IntersectP}{()}}本质上和常规相交方法一样,有两处区别是,它调用了\refvar{Primitive}{}的
方法\refvar[Primitive::IntersectP]{IntersectP}{()}而不是\refvar[Primitive::Intersect]{Intersect}{()},
以及当找到任何交点时就立刻停止遍历。


\section{kd树加速器}\label{sec:kd树加速器}

\keyindex{二叉空间划分}{binary space partitioning}{}(BSP)树用平面自适应地细分空间。
一个BSP树从包含整个场景的边界框开始。
如果框内图元的数量大于某个阈值,则用平面将该框分为两半。
然后图元和与之重合的任意一半关联,
同时位于两半里的图元就都与它们关联
(相比之下,BVH中划分后每个图元只能分配到两个组中的一个)。

划分过程递归进行,直到结果树中的每个叶子区域都包含足够少的图元或者达到最大深度。
因为划分平面可以放置于整个框内的任意位置,
且3D空间的不同部分可以精确到不同程度,
所以BSP易于处理分布不均的几何体。

BSP树的两个变种是\keyindex{kd树}{kd-tree}{tree树}和\keyindex{八叉树}{octree}{tree树}。
kd树\sidenote{译者注:“kd”是k个维度的缩写。}简单地限制划分平面垂直于一个坐标轴;
这让树的遍历和构建都更高效,而在如何划分空间上牺牲一些灵活性。
八叉树每一步用三个垂直于轴的平面同时将该框分为八个区域
(通常在每个方向沿范围中心划分)。

本节中,我们将在类\refvar{KdTreeAccel}{}中为光线相交加速实现一个kd树。
该类源码可在文件\href{https://github.com/mmp/pbrt-v3/tree/master/src/accelerators/kdtreeaccel.h}{\ttfamily accelerators/kdtreeaccel.h}
和\href{https://github.com/mmp/pbrt-v3/tree/master/src/accelerators/kdtreeaccel.cpp}{\ttfamily accelerators/kdtreeaccel.cpp}
中找到。

\begin{lstlisting}
`\initcode{KdTreeAccel Declarations}{=}\initnext{KdTreeAccelDeclarations}`
class `\initvar{KdTreeAccel}{}` : public `\refvar{Aggregate}{}` {
public:
    `\refcode{KdTreeAccel Public Methods}{}`
private:
    `\refcode{KdTreeAccel Private Methods}{}`
    `\refcode{KdTreeAccel Private Data}{}`
};
\end{lstlisting}

除了要保存的图元外,\refvar{KdTreeAccel}{}构造函数
还接收一些参数用于在构建树时指导要作出的决定;
这些参数存于成员变量中(\refvar{isectCost}{}、\refvar{traversalCost}{}、
\refvar{maxPrims}{}、{\ttfamily maxDepth}和\refvar{emptyBonus}{})留待后用。
见\reffig{4.14}中构建树的图示。
\begin{figure}[htbp]
    \centering\input{Pictures/chap04/kdtreesplits.tex}
    \caption{通过沿坐标轴之一递归地划分场景几何边界框来构建kd树。这里,第一次划分沿$x$轴;
        它摆放后使三角形刚好单独在右边区域而其余图元则在左边。
        然后再用轴对齐的划分平面细化若干次左边的区域。
        细化标准的细节——每一步用哪个轴划分空间、沿轴上哪个位置放置平面
        以及何时结束细分——在实践中均会极大影响树的性能。}
    \label{fig:4.14}
\end{figure}

\begin{lstlisting}
`\initcode{KdTreeAccel Method Definitions}{=}\initnext{KdTreeAccelMethodDefinitions}`
`\refvar{KdTreeAccel}{}`::`\refvar{KdTreeAccel}{}`(
        const std::vector<std::shared_ptr<`\refvar{Primitive}{}`>> &p,
        int isectCost, int traversalCost, `\refvar{Float}{}` emptyBonus,
        int maxPrims, int maxDepth)
    : `\refvar{isectCost}{}`(isectCost), `\refvar{traversalCost}{}`(traversalCost),
      `\refvar{maxPrims}{}`(maxPrims), `\refvar{emptyBonus}{}`(emptyBonus), `\refvar[KdTreeAccel::primitives]{primitives}{}`(p) {
    `\refcode{Build kd-tree for accelerator}{}`
}
\end{lstlisting}

\begin{lstlisting}
`\initcode{KdTreeAccel Private Data}{=}\initnext{KdTreeAccelPrivateData}`
const int `\initvar{isectCost}{}`, `\initvar{traversalCost}{}`, `\initvar{maxPrims}{}`;
const `\refvar{Float}{}` `\initvar{emptyBonus}{}`;
std::vector<std::shared_ptr<`\refvar{Primitive}{}`>> `\initvar[KdTreeAccel::primitives]{primitives}{}`;
\end{lstlisting}

\subsection{树状表示}\label{sub:树状表示}
kd树是二叉树,每个内部节点总是有两个孩子且树的叶子存有与之重合的图元。
每个内部节点必须提供三块信息的访问渠道:
\begin{itemize}
    \item 划分轴:该节点划分了$x,y$和$z$中的哪一个轴;
    \item 划分位置:划分平面沿该轴的位置;
    \item 孩子:关于如何到达其下两个子节点的信息。
\end{itemize}
每个叶子节点只需要记录哪个图元与之重合。

为了保证所有内部节点和许多叶子节点只用8字节内存
(假设\refvar{Float}{}占4字节)而麻烦一下是值得的,
因为这样做保证了八个节点契合一个64字节的缓存行。
因为树中经常有许多节点且每条光线通常都要访问许多节点,
最小化节点表示的大小能极大提高缓存性能。
我们最初的实现使用了16字节节点表示;
当我们把大小减少到8字节时我们得到了几乎20\%的提速。

叶子和内部节点都用下面的结构体\refvar{KdAccelNode}{}表示。
每个{\ttfamily union}成员后的注释都说明了特定域是用于内部节点、叶子节点还是两者都是。
\begin{lstlisting}
`\initcode{KdTreeAccel Local Declarations}{=}\initnext{KdTreeAccelLocalDeclarations}`
struct `\initvar{KdAccelNode}{}` {
    `\refcode{KdAccelNode Methods}{}`
    union {
        `\refvar{Float}{}` `\initvar[KdAccelNode::split]{split}{}`;                  // Interior
        int `\initvar{onePrimitive}{}`;             // Leaf
        int `\initvar{primitiveIndicesOffset}{}`;   // Leaf
    };
    union {
        int `\initvar[KdAccelNode::flags]{flags}{}`;         // Both
        int `\initvar{nPrims}{}`;        // Leaf
        int `\initvar{aboveChild}{}`;    // Interior
    };
};
\end{lstlisting}

变量\refvar{KdAccelNode::flags}{}的低两位用于区分用$x,y$和$z$划分的内部节点
(这些数位分别取值0,1和2)以及叶子节点(这些数位取值3)。
在8字节中保存叶子节点相对简单:\refvar{KdAccelNode::flags}{}的低2位
用于表示这是一个叶子,\refvar[nPrims]{KdAccelNode::nPrims}{}的高30位
可用于记录有多少个图元与之重合。
然后,如果只有一个图元与\refvar{KdAccelNode}{}叶子重合,
则指向数组\refvar{KdTreeAccel::primitives}{}
的整数索引会指出该\refvar{Primitive}{}。如果重合的图元多于一个,
则它们的索引保存于\refvar[primitiveIndices]{KdTreeAccel::primitiveIndices}{}的一段中。
该叶子第一个索引的偏移量存于\refvar[primitiveIndicesOffset]{KdAccelNode::primitiveIndicesOffset}{}且后面直接跟着剩下的索引。
\begin{lstlisting}
`\refcode{KdTreeAccel Private Data}{+=}\lastnext{KdTreeAccelPrivateData}`
std::vector<int> `\initvar{primitiveIndices}{}`;
\end{lstlisting}

叶子节点很容易初始化,不过我们要注意细节,
因为\refvar[KdAccelNode::flags]{flags}{}和\refvar{nPrims}{}共享同一存储;
我们需要注意在初始化其中一个时不要搞乱了另一个。
此外,在保存图元数量前必须向左移两位,
这样\refvar{KdAccelNode::flags}{}的低两位可以都设为1以表示这是一个叶子节点。
\begin{lstlisting}
`\refcode{KdTreeAccel Method Definitions}{+=}\lastnext{KdTreeAccelMethodDefinitions}`
void `\refvar{KdAccelNode}{}`::`\initvar[KdAccelNode::InitLeaf]{InitLeaf}{}`(int *primNums, int np,
        std::vector<int> *primitiveIndices) {
    `\refvar[KdAccelNode::flags]{flags}{}` = 3;
    `\refvar{nPrims}{}` |= (np << 2);
    `\refcode{Store primitive ids for leaf node}{}`
}
\end{lstlisting}

对于有零或一个重合图元的叶子节点,
因为有\refvar[onePrimitive]{KdAccelNode::onePrimitive}{}
域了,所以不再需要额外分配内存。
对于有多个重合图元的情况,则在数组{\ttfamily primitiveIndices}中分配存储。
\begin{lstlisting}
`\initcode{Store primitive ids for leaf node}{=}`
if (np == 0)
    `\refvar{onePrimitive}{}` = 0;
else if (np == 1)
    `\refvar{onePrimitive}{}` = primNums[0];
else {
    `\refvar{primitiveIndicesOffset}{}` = primitiveIndices->size();
    for (int i = 0; i < np; ++i)
        primitiveIndices->push_back(primNums[i]);
}
\end{lstlisting}

让内部节点减少到8字节也相当简单。
一个\refvar{Float}{}(当\refvar{Float}{}定义为{\ttfamily float}时其大小为32位)
保存了节点沿所选划分轴分割空间的位置,并且如之前所述,
\refvar{KdAccelNode::flags}{}低两位用于记录该节点是沿哪个轴划分的。
剩下的就是存储足够的信息使我们遍历树时能找到该节点的两个孩子。

我们排布节点的方式是只存储一个孩子指针,而不是存储两个指针或偏移量:
所有节点都分配到单个连续内存块,
内部节点的对应划分平面下方空间的孩子在数组中的保存位置总是紧跟其父亲
(通过在内存中保持至少一个孩子挨着其父亲,这样的排布也提高了缓存性能)。
另一个对应于划分平面上方的孩子,则在数组其他某处出现;
单个整数偏移量\refvar[aboveChild]{KdAccelNode::aboveChild}{}保存了它在节点数组中的位置。
该表示和\refsub{为遍历而压实的BVH}中BVH节点用的类似。

有了所有这些约定,初始化内部节点的代码就很简单了。
就像方法\refvar[KdAccelNode::InitLeaf]{InitLeaf}{()}
那样,在设置\refvar{aboveChild}{}前为\refvar[KdAccelNode::flags]{flags}{}赋值、
计算移位的\refvar{aboveChild}{}逻辑或值很重要,
这样才不会搞乱保存在\refvar[KdAccelNode::flags]{flags}{}中的数位。
\begin{lstlisting}
`\initcode{KdAccelNode Methods}{=}\initnext{KdAccelNodeMethods}`
void `\initvar[KdAccelNode::InitInterior]{InitInterior}{}`(int axis, int ac, `\refvar{Float}{}` s) {
    `\refvar[KdAccelNode::split]{split}{}` = s;
    `\refvar[KdAccelNode::flags]{flags}{}` = axis;
    `\refvar{aboveChild}{}` |= (ac << 2);
}
\end{lstlisting}

最后,我们将提供一些方法从节点中提取各种值,
这样调用者就不需要了解其内存表示的细节了。
\begin{lstlisting}
`\refcode{KdAccelNode Methods}{+=}\lastcode{KdAccelNodeMethods}`
`\refvar{Float}{}` `\initvar{SplitPos}{()}` const { return `\refvar[KdAccelNode::split]{split}{}`; }
int `\initvar[KdAccelNode::nPrimitives]{nPrimitives}{()}` const { return `\refvar{nPrims}{}` >> 2; }
int `\initvar[KdAccelNode::SplitAxis]{SplitAxis}{()}` const { return `\refvar[KdAccelNode::flags]{flags}{}` & 3; }
bool `\initvar{IsLeaf}{()}` const { return (`\refvar[KdAccelNode::flags]{flags}{}` & 3) == 3; }
int `\initvar{AboveChild}{()}` const { return `\refvar{aboveChild}{}` >> 2; }
\end{lstlisting}

\subsection{树的构建}\label{sub:树的构建}
kd树是用递归自顶向下算法构建的。
每一步中,我们有一个轴对齐空间区域和与该区域重合的图元集。
要么该区域分为两个子区域且转化为内部节点,
要么用重合的图元创建一个叶子节点,结束递归。

正如讨论\refvar{KdAccelNode}{}时所提到的,
所有树节点都保存于连续数组中。\newline
\refvar[nextFreeNode]{KdTreeAccel::nextFreeNode}{}记录了该数组中下一个有效节点,
\refvar[nAllocedNodes]{KdTreeAccel::\newline nAllocedNodes}{}记录了已经分配的总数。
通过一开始设置两者为0且不分配任何节点,这里的实现保证了当初始化树的第一个节点时能立即完成分配。

如果没有为构造函数给定,则还有必要确定树的最大深度。
尽管树的构建过程通常会自然地在合理的深度结束,
但限制最大深度很重要,这样极端情况下树所用的内存数量才不会无限增长。
我们已经发现值$8+1.3\log_2N$为大量场景给出了合理的最大深度。

\begin{lstlisting}
`\initcode{Build kd-tree for accelerator}{=}`
`\refvar{nextFreeNode}{}` = `\refvar{nAllocedNodes}{}` = 0;
if (maxDepth <= 0)
    maxDepth = std::round(8 + 1.3f * `\refvar{Log2Int}{}`(`\refvar[KdTreeAccel::primitives]{primitives}{}`.size()));
`\refcode{Compute bounds for kd-tree construction}{}`
`\refcode{Allocate working memory for kd-tree construction}{}`
`\refcode{Initialize primNums for kd-tree construction}{}`
`\refcode{Start recursive construction of kd-tree}{}`
\end{lstlisting}

\begin{lstlisting}
`\refcode{KdTreeAccel Private Data}{+=}\lastnext{KdTreeAccelPrivateData}`
`\refvar{KdAccelNode}{}` *`\initvar[KdTreeAccel::nodes]{nodes}{}`;
int `\initvar{nAllocedNodes}{}`, `\initvar{nextFreeNode}{}`;
\end{lstlisting}

因为构建例程会一路重复使用图元边界框,
所以在开始构建树前它们被保存在{\ttfamily vector}中,
这样就不需重复调用可能更慢的方法\refvar{Primitive::WorldBound}{()}。
\begin{lstlisting}
`\initcode{Compute bounds for kd-tree construction}{=}`
std::vector<`\refvar{Bounds3f}{}`> primBounds;
for (const std::shared_ptr<`\refvar{Primitive}{}`> &prim : `\refvar[KdTreeAccel::primitives]{primitives}{}`) {
    `\refvar{Bounds3f}{}` b = prim->`\refvar[Primitive::WorldBound]{WorldBound}{}`();
    `\refvar[KdTreeAccel::bounds]{bounds}{}` = `\refvar[Union2]{Union}{}`(`\refvar[KdTreeAccel::bounds]{bounds}{}`, b);
    primBounds.push_back(b);
}
\end{lstlisting}

\begin{lstlisting}
`\refcode{KdTreeAccel Private Data}{+=}\lastcode{KdTreeAccelPrivateData}`
`\refvar{Bounds3f}{}` `\initvar[KdTreeAccel::bounds]{bounds}{}`;
\end{lstlisting}

树构建例程的参数之一是图元索引数组,表示哪个图元与当前节点重合。
因为(当递归开始时)所有图元都和根节点重合,
所以我们从初始化值为零到{\ttfamily primitives.size()-1}的数组开始。
\begin{lstlisting}
`\initcode{Initialize primNums for kd-tree construction}{=}`
std::unique_ptr<int[]> primNums(new int[`\refvar[KdTreeAccel::primitives]{primitives}{}`.size()]);
for (size_t i = 0; i < `\refvar[KdTreeAccel::primitives]{primitives}{}`.size(); ++i)
    primNums[i] = i;
\end{lstlisting}

每个树节点都会调用\refvar{KdTreeAccel::buildTree}{()}。
它负责决定该节点应该是内部节点还是叶子并适当更新数据结构。
最后三个参数{\ttfamily edges}、{\ttfamily prims0}、{\ttfamily prims1}是
指向分配于代码片\refcode{Allocate working memory for kd-tree construction}{}的数据的指针,
稍后几页会对此作定义和介绍。
\begin{lstlisting}
`\initcode{Start recursive construction of kd-tree}{=}`
`\refvar[KdTreeAccel::buildTree]{buildTree}{}`(0, `\refvar[KdTreeAccel::bounds]{bounds}{}`, primBounds, primNums.get(), `\refvar[KdTreeAccel::primitives]{primitives}{}`.size(), 
          maxDepth, edges, prims0.get(), prims1.get());
\end{lstlisting}

\refvar{KdTreeAccel::buildTree}{()}的主要参数是供创建的节点使用的相对于
\refvar{KdAccelNode}{}数组的偏移量{\ttfamily nodeNum}、
给出该节点覆盖的空间区域边界框的{\ttfamily nodeBounds},
以及与之重合的图元索引{\ttfamily primNums}。
其余参数稍后在快用到它们时阐述。
\begin{lstlisting}
`\refcode{KdTreeAccel Method Definitions}{+=}\lastnext{KdTreeAccelMethodDefinitions}`
void `\refvar{KdTreeAccel}{}::\initvar[KdTreeAccel::buildTree]{buildTree}{}`(int nodeNum, const `\refvar{Bounds3f}{}` &nodeBounds,
        const std::vector<`\refvar{Bounds3f}{}`> &allPrimBounds, int *primNums,
        int nPrimitives, int depth,
        const std::unique_ptr<`\refvar{BoundEdge}{}`[]> edges[3], 
        int *prims0, int *prims1, int badRefines) {
    `\refcode{Get next free node from nodes array}{}`
    `\refcode{Initialize leaf node if termination criteria met}{}`
    `\refcode{Initialize interior node and continue recursion}{}`
}
\end{lstlisting}

如果所有分配的节点都已经用完了,则重新分配两倍数量的节点内存并复制旧值。
第一次调用\refvar{KdTreeAccel::buildTree}{()}时,
\refvar[nAllocedNodes]{KdTreeAccel::nAllocedNodes}{}
为0并分配树节点的一个初始块。
\begin{lstlisting}
`\initcode{Get next free node from nodes array}{=}`
if (`\refvar{nextFreeNode}{}` == `\refvar{nAllocedNodes}{}`) {
    int nNewAllocNodes = std::max(2 * `\refvar{nAllocedNodes}{}`, 512);
    `\refvar{KdAccelNode}{}` *n = `\refvar{AllocAligned}{}`<`\refvar{KdAccelNode}{}`>(nNewAllocNodes);
    if (`\refvar{nAllocedNodes}{}` > 0) {
        memcpy(n, `\refvar[KdTreeAccel::nodes]{nodes}{}`, `\refvar{nAllocedNodes}{}` * sizeof(`\refvar{KdAccelNode}{}`));
        `\refvar{FreeAligned}{}`(`\refvar[KdTreeAccel::nodes]{nodes}{}`);
    }
    `\refvar[KdTreeAccel::nodes]{nodes}{}` = n;
    `\refvar{nAllocedNodes}{}` = nNewAllocNodes;
}
++`\refvar{nextFreeNode}{}`;
\end{lstlisting}

当区域内有足够少量的图元或达到最大深度时就创建叶子节点(停止递归)。
参数{\ttfamily depth}一开始为树的最大深度,且每一层递减。
\begin{lstlisting}
`\initcode{Initialize leaf node if termination criteria met}{=}`
if (nPrimitives <= `\refvar{maxPrims}{}` || depth == 0) {
    `\refvar[KdTreeAccel::nodes]{nodes}{}`[nodeNum].`\refvar[KdAccelNode::InitLeaf]{InitLeaf}{}`(primNums, nPrimitives, &`\refvar{primitiveIndices}{}`);
    return;
}
\end{lstlisting}

若这是个内部节点,则需要选择一个划分平面,按该平面划分图元并递归。
\begin{lstlisting}
`\initcode{Initialize interior node and continue recursion}{=}`
`\refcode{Choose split axis position for interior node}{}`
`\refcode{Create leaf if no good splits were found}{}`
`\refcode{Classify primitives with respect to split}{}`
`\refcode{Recursively initialize children nodes}{}`
\end{lstlisting}

我们的实现选择用\refsub{表面积启发法}介绍的SAH来划分。
SAH适用于kd树和BVH;为节点中一系列候选划分平面计算估计的开销,
并选择给出最少开销的划分。

在这里的实现中,相交开销$t_{\text{isect}}$和遍历开销$t_{\text{trav}}$可由用户设置;
它们的默认值分别是80和1.
重要的是,这两个值的比例决定了树构建算法的表现
\footnote{该方法的许多其他实现似乎给这些开销使用了接近得多的值,
    有时甚至接近相等值(例如,见\citet{hurley2002fast})。
    在pbrt中这里所用的值为大量测试场景给出了最好的性能。
    我们怀疑这一矛盾是因为pbrt中光线-图元相交测试需要两次虚函数调用
    以及一次光线从世界到物体空间的变换这一事实,
    此外还有执行实际相交测试的开销。
    只支持三角图元的高度优化的光线追踪器不会有此类任何额外开销。
    见\refsub{只有三角形}关于这一平衡设计的更多讨论。}。
比起BVH所用的值,这些值之间更大的比例反映的事实是
访问kd树的节点比访问BVH节点的开销更少。

针对用于BVH树的SAH的一点修改是,对于kd树值得稍微偏好选择
使其中一个孩子没有与之重合的图元的划分,
因为光线穿过这些区域可以立即进行到下一个kd树节点而无需任何光线-图元相交测试。
因此,未划分和划分后区域的改进开销分别为
\begin{align*}
    t_{\text{isect}}N \quad \text{和} \quad t_{\text{trav}}+(1-b_{\mathrm{e}})(p_BN_Bt_{\text{isect}}+p_AN_At_{\text{isect}})\, ,
\end{align*}
其中$b_{\mathrm{e}}$是为零的“补贴”\sidenote{译者注:原文bonus。}值,
除非两个区域之一完全为空时取值0到1.

有了为开销模型计算概率的方法,唯一要解决的问题是
怎么生成候选划分位置以及怎么为每个候选者高效计算开销。
可以证明该模型最小开销能于在某一图元边界框的一个面上划分时取得——
不需要考虑在中间位置的划分(为了帮助你自己理解,
考虑一下开销函数在面的边界之间时的特性)。
这里,我们将考虑该区域内三个坐标轴之一或以上的所有边界框面。

利用精心构造的算法可以把检查所有这些候选者的开销维持在相对低的水平。
为了计算这些开销,我们将扫掠边界框在每个轴上的投影并追踪开销最低的那些(\reffig{4.15})。
每个边界框在每个轴上有两处边界,每处都用结构体\refvar{BoundEdge}{}的实例表示。
该结构体记录了边界沿轴的位置,它表示边界框的开始或结束
(沿轴从低到高),以及哪个图元与之关联。
\begin{figure}[htbp]
    \centering\input{Pictures/chap04/kdtreeprojectedbboxes.tex}
    \caption{给定我们要考虑的可能划分所沿的轴,图元的边界框被投影到该轴上,
    这带来了一个高效算法以追踪特定划分平面两侧会各有多少图元。
    例如这里,在$a_1$处划分会让$A$完全留在划分平面下方,$B$横跨之,而$C$完全在其上方。
    轴上每一个点$a_0,a_1,b_0,b_1,c_0$和$c_1$都由结构体\refvar{BoundEdge}{}的一个实例表示。}
    \label{fig:4.15}
\end{figure}
\begin{lstlisting}
`\refcode{KdTreeAccel Local Declarations}{+=}\lastnext{KdTreeAccelLocalDeclarations}`
enum class `\initvar{EdgeType}{}` { `\initvar[EdgeType::Start]{Start}{}`, `\initvar[EdgeType::End]{End}{}` };
\end{lstlisting}
\begin{lstlisting}
`\refcode{KdTreeAccel Local Declarations}{+=}\lastcode{KdTreeAccelLocalDeclarations}`
struct `\initvar{BoundEdge}{}` {
    `\refcode{BoundEdge Public Methods}{}`
    `\refvar{Float}{}` `\initvar[BoundEdge::t]{t}{}`;
    int `\initvar[BoundEdge::primNum]{primNum}{}`;
    `\refvar{EdgeType}{}` `\initvar[BoundEdge::type]{type}{}`;
};
\end{lstlisting}
\begin{lstlisting}
`\initcode{BoundEdge Public Methods}{=}`
`\refvar{BoundEdge}{}`(`\refvar{Float}{}` t, int primNum, bool starting)
    : `\refvar[BoundEdge::t]{t}{}`(t), `\refvar[BoundEdge::primNum]{primNum}{}`(primNum) {
    `\refvar[BoundEdge::type]{type}{}` = starting ? `\refvar{EdgeType::Start}{}` : `\refvar{EdgeType::End}{}`; 
}
\end{lstlisting}

对于任意树节点至多需要为{\ttfamily 2*\refvar[KdTreeAccel::primitives]{primitives}{}.size()}个\refvar{BoundEdge}{}计算开销,
所以一次分配全部三轴上所有边界的内存然后再为每个创建的节点复用。
\begin{lstlisting}
`\initcode{Allocate working memory for kd-tree construction}{=}\initnext{Allocateworkingmemoryforkdtreeconstruction}`
std::unique_ptr<`\refvar{BoundEdge}{}`[]> edges[3];
for (int i = 0; i < 3; ++i)
    edges[i].reset(new `\refvar{BoundEdge}{}`[2 * `\refvar[KdTreeAccel::primitives]{primitives}{}`.size()]);
\end{lstlisting}

在为创建的叶子确定估计的开销后,\refvar{KdTreeAccel::buildTree}{()}选择
一个轴尝试沿其划分并为每个候选划分计算开销函数。
{\ttfamily bestAxis}和{\ttfamily bestOffset}记录了该轴
以及目前给出最低开销{\ttfamily bestCost}的边界框边界索引。
{\ttfamily invTotalSA}初始化为节点表面积的倒数;
当计算光线穿过每个候选孩子节点的概率时会用到它的值。
\begin{lstlisting}
`\initcode{Choose split axis position for interior node}{=}`
int bestAxis = -1, bestOffset = -1;
`\refvar{Float}{}` bestCost = `\refvar{Infinity}{}`;
`\refvar{Float}{}` oldCost = `\refvar{isectCost}{}` * `\refvar{Float}{}`(nPrimitives);
`\refvar{Float}{}` totalSA = nodeBounds.`\refvar{SurfaceArea}{}`();
`\refvar{Float}{}` invTotalSA = 1 / totalSA;
`\refvar{Vector3f}{}` d = nodeBounds.`\refvar{pMax}{}` - nodeBounds.`\refvar{pMin}{}`;
`\refcode{Choose which axis to split along}{}`
int retries = 0;
retrySplit:
`\refcode{Initialize edges for axis}{}`
`\refcode{Compute cost of all splits for axis to find best}{}`
\end{lstlisting}

该方法首先尝试沿具有最大空间范围的轴寻找一个划分;
如果成功,该选项则有助于给出形状上趋于方形的空间区域。
这在直觉上是合理的方法。
如果沿该轴没有成功找到好的划分,则回退并依次尝试其他的。
\begin{lstlisting}
`\initcode{Choose which axis to split along}{=}`
int axis = nodeBounds.`\refvar{MaximumExtent}{}`();
\end{lstlisting}

首先用重合图元的边界框初始化该轴的数组{\ttfamily edges}。
然后该数组沿该轴从低到高存储,这样它就能从头到尾扫掠框的边界。
\begin{lstlisting}
`\initcode{Initialize edges for axis}{=}`
for (int i = 0; i < nPrimitives; ++i) {
    int pn = primNums[i];
    const `\refvar{Bounds3f}{}` &bounds = allPrimBounds[pn];
    edges[axis][2 * i] =     `\refvar{BoundEdge}{}`(bounds.`\refvar{pMin}{}`[axis], pn, true);
    edges[axis][2 * i + 1] = `\refvar{BoundEdge}{}`(bounds.`\refvar{pMax}{}`[axis], pn, false);
}
`\refcode{Sort edges for axis}{}`
\end{lstlisting}

C++标准库例程{\ttfamily sort()}要求被排序的结构要定义顺序;
这用值\refvar{BoundEdge::t}{}
来完成。然而,一个细微之处是如果值\refvar{BoundEdge::t}{}相同,
则需要通过比较节点类型来打破平局;
这是必要的,因为{\ttfamily sort()}所取决的事实是
{\ttfamily a < b}和{\ttfamily b < a}都为{\ttfamily false}的唯一时刻是{\ttfamily a == b}。
\begin{lstlisting}
`\initcode{Sort edges for axis}{=}`
std::sort(&edges[axis][0], &edges[axis][2*nPrimitives],
    [](const `\refvar{BoundEdge}{}` &e0, const `\refvar{BoundEdge}{}` &e1) -> bool {
        if (e0.`\refvar[BoundEdge::t]{t}{}` == e1.`\refvar[BoundEdge::t]{t}{}`)
            return (int)e0.`\refvar[BoundEdge::type]{type}{}` < (int)e1.`\refvar[BoundEdge::type]{type}{}`;
        else return e0.`\refvar[BoundEdge::t]{t}{}` < e1.`\refvar[BoundEdge::t]{t}{}`; 
    });
\end{lstlisting}

有了排好序的边界数组,我们想为它们中的每一处划分快速计算开销函数。
光线穿过每个孩子节点的概率很容易用其表面积计算,
划分处每侧的图元数量由变量{\ttfamily nBelow}和{\ttfamily nAbove}跟踪。
我们想保持它们的值是最新的,这样如果我们在某次循环中选择在{\ttfamily edgeT}处划分,
{\ttfamily nBelow}会给出最终在划分平面之下的图元数量而{\ttfamily nAbove}则给出之上的数量
\footnote{当多个边界框面投影到轴上同一点时,这一特性在这些点处可能不成立。
然而此处的实现只会高估数量,而且更重要的是,
它会于在这些点的每一个上进行的多次循环中的某一次取到正确的值,
所以无论如何最终算法的功能是正确的。}。

在第一个边界处,依据定义所有图元必须在该边界之上,
所以{\ttfamily nAbove}初始化为{\ttfamily nPrimitives}且{\ttfamily nBelow}设为0.
当循环考虑在边界框范围尾部的划分时,
{\ttfamily nAbove}需要递减,因为该框以前一定在划分平面之上,
如果在该点完成划分则它不会再在平面之上。
同样,计算划分开销后,如果划分候选项在边界框范围的起点处,
则所有后续划分中该框都会在下侧。
在循环体开头和结尾的测试为这两种情况更新了图元数量。
\begin{lstlisting}
`\initcode{Compute cost of all splits for axis to find best}{=}`
int nBelow = 0, nAbove = nPrimitives;
for (int i = 0; i < 2 * nPrimitives; ++i) {
    if (edges[axis][i].`\refvar[BoundEdge::type]{type}{}` == `\refvar{EdgeType::End}{}`) --nAbove;
    `\refvar{Float}{}` edgeT = edges[axis][i].`\refvar[BoundEdge::t]{t}{}`;
    if (edgeT > nodeBounds.`\refvar{pMin}{}`[axis] &&
        edgeT < nodeBounds.`\refvar{pMax}{}`[axis]) {
        `\refcode{Compute cost for split at ith edge}{}`
    }
    if (edges[axis][i].`\refvar[BoundEdge::type]{type}{}` == `\refvar{EdgeType::Start}{}`) ++nBelow;
}
\end{lstlisting}

{\ttfamily belowSA}和{\ttfamily aboveSA}持有两个候选孩子边框的表面积;
通过把六个面的面积加在一起很容易将其算出来。
\begin{lstlisting}
`\initcode{Compute child surface areas for split at edgeT}{=}`
int otherAxis0 = (axis + 1) % 3, otherAxis1 = (axis + 2) % 3;
`\refvar{Float}{}` belowSA = 2 * (d[otherAxis0] * d[otherAxis1] +
                     (edgeT - nodeBounds.`\refvar{pMin}{}`[axis]) * 
                     (d[otherAxis0] + d[otherAxis1]));
`\refvar{Float}{}` aboveSA = 2 * (d[otherAxis0] * d[otherAxis1] +
                     (nodeBounds.`\refvar{pMax}{}`[axis] - edgeT) * 
                     (d[otherAxis0] + d[otherAxis1]));
\end{lstlisting}

有了所有这些信息,就可以计算特定划分的开销了。
\begin{lstlisting}
`\initcode{Compute cost for split at ith edge}{=}`
`\refcode{Compute child surface areas for split at edgeT}{}`
`\refvar{Float}{}` pBelow = belowSA * invTotalSA; 
`\refvar{Float}{}` pAbove = aboveSA * invTotalSA;
`\refvar{Float}{}` eb = (nAbove == 0 || nBelow == 0) ? emptyBonus : 0;
`\refvar{Float}{}` cost = `\refvar{traversalCost}{}` + 
             `\refvar{isectCost}{}` * (1 - eb) * (pBelow * nBelow + pAbove * nAbove);
`\refcode{Update best split if this is lowest cost so far}{}`
\end{lstlisting}

如果为该候选划分计算的开销是目前最好的,则记录该划分的细节。
\begin{lstlisting}
`\initcode{Update best split if this is lowest cost so far}{=}`
if (cost < bestCost)  {
    bestCost = cost;
    bestAxis = axis;
    bestOffset = i;
}
\end{lstlisting}

可能在之前的测试中没有找到可行的划分(\reffig{4.16}展示了一种可能发生的情况)。
这种情况下,沿当前轴不存在可以把该节点划分开的单个候选位置。
这时,依次尝试另外两轴的划分。
(当{\ttfamily retries}等于2时)如果它们也没有找到划分,
则没有有用的方式细化该节点,因为两个孩子都仍会有同样多的重合图元。
当这种条件发生时,所能做的就是放弃并构建一个叶子节点。
\begin{figure}[htbp]
    \centering\input{Pictures/chap04/Overlappingbboxes.tex}
    \caption{如果多个边界框(虚线)如上所示与一个kd树节点(实线)重合,
    则不可能有划分位置能让其两侧的图元比总和更少。}
    \label{fig:4.16}
\end{figure}

也可能最佳划分的开销仍高于根本不划分该节点的开销。
如果它差得多且图元也不太多,就立即创建叶子节点。
否则,{\ttfamily badRefines}保持追踪目前在树的当前节点以上已经做了多少次不良划分。
允许稍微差些的细化是值得的,因为要考虑的图元子集更小后,之后的划分可能会找到更好的结果。
\begin{lstlisting}
`\initcode{Create leaf if no good splits were found}{=}`
if (bestAxis == -1 && retries < 2) {
    ++retries;
    axis = (axis + 1) % 3;
    goto retrySplit;
}
if (bestCost > oldCost) ++badRefines;
if ((bestCost > 4 * oldCost && nPrimitives < 16) || 
    bestAxis == -1 || badRefines == 3) {
    nodes[nodeNum].`\refvar[KdAccelNode::InitLeaf]{InitLeaf}{}`(primNums, nPrimitives, &`\refvar{primitiveIndices}{}`);
    return; 
}
\end{lstlisting}

选好划分位置后,按照之前代码中跟踪{\ttfamily nBelow}和{\ttfamily nAbove}的同样方式,
边界框的边界可用于把图元分为在划分处上方、下方或同在两侧。
注意下面的循环中跳过了数组中的项{\ttfamily bestOffset};
这是必要的,这样边界框被用作划分处的图元不会被错误地分类为同时位于划分处的两侧。
\begin{lstlisting}
`\initcode{Classify primitives with respect to split}{=}`
int n0 = 0, n1 = 0;
for (int i = 0; i < bestOffset; ++i)
    if (edges[bestAxis][i].`\refvar[BoundEdge::type]{type}{}` == `\refvar{EdgeType::Start}{}`)
        prims0[n0++] = edges[bestAxis][i].`\refvar[BoundEdge::primNum]{primNum}{}`;
for (int i = bestOffset + 1; i < 2 * nPrimitives; ++i)
    if (edges[bestAxis][i].`\refvar[BoundEdge::type]{type}{}` == `\refvar{EdgeType::End}{}`)
        prims1[n1++] = edges[bestAxis][i].`\refvar[BoundEdge::primNum]{primNum}{}`;
\end{lstlisting}

回想在kd树节点数组中该节点的“下方”孩子的节点序数是当前节点序数加一。
在递归从树的这一侧返回后,偏移量\refvar{nextFreeNode}{}被用于“上方”孩子。
这里唯一的重要细节是内存{\ttfamily prims0}被直接传入给两个孩子复用,
而{\ttfamily prims1}指针则首先向前推进
\sidenote{译者注:这段内容较难,笔者的理解是:由于\refvar{KdTreeAccel::buildTree}{()}在构建
过程中会原位修改传入的{\ttfamily prims0}和{\ttfamily prims1},所以需要保护现场。
左子树构建完成后,{\ttfamily prims0}的内容就没有用处了,可在右子树构建时被覆盖;
但构建左子树时不能变动还未构建的右子树所需的{\ttfamily prims1},
所以需要挪到新存储位置{\ttfamily prims1 + nPrimitives}。}。
这是必要的,因为当前对\refvar{KdTreeAccel::buildTree}{()}的调用取决于
下文中它首次递归调用\refvar{KdTreeAccel::buildTree}{()}时贮藏的{\ttfamily prims1}值,
毕竟它必须作为参数传给第二次调用。
然而,在首次递归调用立即使用之后就没有相应的必要贮藏{\ttfamily edges}值或贮藏{\ttfamily prims0}了。
\begin{lstlisting}
`\initcode{Recursively initialize children nodes}{=}`
`\refvar{Float}{}` tSplit = edges[bestAxis][bestOffset].`\refvar[BoundEdge::t]{t}{}`;
`\refvar{Bounds3f}{}` bounds0 = nodeBounds, bounds1 = nodeBounds;
bounds0.`\refvar{pMax}{}`[bestAxis] = bounds1.`\refvar{pMin}{}`[bestAxis] = tSplit;
`\refvar[KdTreeAccel::buildTree]{buildTree}{}`(nodeNum + 1, bounds0, allPrimBounds, prims0, n0,
          depth - 1, edges, prims0, prims1 + nPrimitives, badRefines);
int aboveChild = `\refvar{nextFreeNode}{}`;
nodes[nodeNum].`\refvar[KdAccelNode::InitInterior]{InitInterior}{}`(bestAxis, aboveChild, tSplit);
`\refvar[KdTreeAccel::buildTree]{buildTree}{}`(aboveChild, bounds1, allPrimBounds, prims1, n1, 
          depth - 1, edges, prims0, prims1 + nPrimitives, badRefines);
\end{lstlisting}

因此,整数数组{\ttfamily prims1}比数组{\ttfamily prims0}需要
多得多的空间存储最坏情况下重合图元可能的数目,
后者只需要一次处理单个层级的图元。
\begin{lstlisting}
`\refcode{Allocate working memory for kd-tree construction}{+=}\lastcode{Allocateworkingmemoryforkdtreeconstruction}`
std::unique_ptr<int[]> prims0(new int[`\refvar[KdTreeAccel::primitives]{primitives}{}`.size()]);
std::unique_ptr<int[]> prims1(new int[(maxDepth+1) * `\refvar[KdTreeAccel::primitives]{primitives}{}`.size()]);
\end{lstlisting}

\subsection{遍历}\label{sub:遍历2}
\reffig{4.17}展示了光线遍历树的基本过程
\sidenote{译者注:原文中该图题注将左右孩子写反了,已修正。}。
让光线与树的整体边框相交给出了初始的{\ttfamily tMin}和{\ttfamily tMax}值,即图中标出的点。
像本章的\refvar{BVHAccel}{}那样,如果光线错开了整体图元边框,
则该方法可立即返回{\ttfamily false}。
否则,它从根开始下沉到树中。
在每个内部节点处,它确定光线首先进入两个孩子中的哪个并按顺序处理两个孩子。
当光线退出树或者找到最近相交处时遍历结束。
\begin{figure}[htbp]
    \centering\input{Pictures/chap04/kdraytraversal.tex}
    \caption{光线遍历穿过kd树。(a)光线与树的边框相交,
    给出了要考虑的初始参数范围$[t_{\min},t_{\max}]$.
    (b)因为该范围非空,所以这里需要考虑根节点的两个孩子。
    光线首先进入左侧孩子,标记为“near”,有参数范围$[t_{\min},t_{\text{split}}]$.
    如果近处节点是含有图元的叶子节点,则执行光线-图元相交测试;
    否则处理其孩子节点。(c)如果该节点内没有找到命中处,或者找到的命中处
    超出了$[t_{\min},t_{\text{split}}]$,则处理右边的远处节点。
    (d)继续该过程——按深度优先处理树节点,从前往后遍历——直到
    求得最近相交处或者光线退出该树。}
    \label{fig:4.17}
\end{figure}
\begin{lstlisting}
`\refcode{KdTreeAccel Method Definitions}{+=}\lastcode{KdTreeAccelMethodDefinitions}`
bool `\refvar{KdTreeAccel}{}`::`\initvar[KdTreeAccel::Intersect]{Intersect}{}`(const `\refvar{Ray}{}` &ray,
        `\refvar{SurfaceInteraction}{}` *isect) const {
    `\refcode{Compute initial parametric range of ray inside kd-tree extent}{}`
    `\refcode{Prepare to traverse kd-tree for ray}{}`
    `\refcode{Traverse kd-tree nodes in order for ray}{}`
}
\end{lstlisting}

算法从寻找光线与树重合的整体参数范围$[t_{\min},t_{\max}]$开始,
如果没有重合则立即退出。
\begin{lstlisting}
`\initcode{Compute initial parametric range of ray inside kd-tree extent}{=}`
`\refvar{Float}{}` tMin, tMax;
if (!bounds.`\refvar[Bounds3::IntersectP]{IntersectP}{}`(ray, &tMin, &tMax)) 
    return false;
\end{lstlisting}

结构体\refvar{KdToDo}{}数组用于记录该光线目前要处理的节点;
它排了序使得数组中最后一个活跃项是应该考虑的下一个节点。
该数组中需要的最大项数是kd树的最大深度;
下文所用的数组大小在实践中应该是绰绰有余的。
\begin{lstlisting}
`\initcode{Prepare to traverse kd-tree for ray}{=}`
`\refvar{Vector3f}{}` invDir(1 / ray.`\refvar[Ray::d]{d}{}`.x, 1 / ray.`\refvar[Ray::d]{d}{}`.y, 1 / ray.`\refvar[Ray::d]{d}{}`.z);
constexpr int maxTodo = 64;
`\refvar{KdToDo}{}` todo[maxTodo];
int todoPos = 0;
\end{lstlisting}
\begin{lstlisting}
`\refcode{KdTreeAccel Declarations}{+=}\lastcode{KdTreeAccelDeclarations}`
struct `\initvar{KdToDo}{}` {
    const `\refvar{KdAccelNode}{}` *`\initvar[KdToDo::node]{node}{}`;
    `\refvar{Float}{}` `\initvar[KdToDo::tMin]{tMin}{}`, `\initvar[KdToDo::tMax]{tMax}{}`;
};
\end{lstlisting}

遍历继续穿过节点,循环中每次处理单个叶子或内部节点。
值\refvar[KdToDo::tMin]{tMin}{}和\refvar[KdToDo::tMax]{tMax}{}总是持有
光线与当前节点重合的参数范围。
\begin{lstlisting}
`\initcode{Traverse kd-tree nodes in order for ray}{=}`
bool hit = false;
const `\refvar{KdAccelNode}{}` *node = &`\refvar[KdTreeAccel::nodes]{nodes}{}`[0];
while (node != nullptr) {
    `\refcode{Bail out if we found a hit closer than the current node}{}`
    if (!node->`\refvar{IsLeaf}{}`()) {
        `\refcode{Process kd-tree interior node}{}`
    } else {
        `\refcode{Check for intersections inside leaf node}{}`
        `\refcode{Grab next node to process from todo list}{}`
    }
}
return hit;
\end{lstlisting}

对于和多个节点重合的图元可能以前就找到相交处了。
首次检测到时如果相交处在当前节点之外,
则有必要继续遍历树直到我们遇到一个节点的{\ttfamily tMin}超过相交处。
只有这时才能确定和其他图元不会再有更近的相交处了。
\begin{lstlisting}
`\initcode{Bail out if we found a hit closer than the current node}{=}`
if (ray.`\refvar{tMax}{}` < tMin) break;
\end{lstlisting}

对于内部节点要做的第一件事是让光线与节点的划分平面相交;
有了交点后,我们可以确定需要处理一个还是两个孩子节点以及光线穿过它们的顺序。
\begin{lstlisting}
`\initcode{Process kd-tree interior node}{=}`
`\refcode{Compute parametric distance along ray to split plane}{}`
`\refcode{Get node children pointers for ray}{}`
`\refcode{Advance to next child node, possibly enqueue other child}{}`
\end{lstlisting}

按照在光线-边界框测试中和计算光线与轴对齐平面相交一样的方式计算到划分平面的参数距离。
循环中我们每次用预先计算的值{\ttfamily invDir}保存除数。
\begin{lstlisting}
`\initcode{Compute parametric distance along ray to split plane}{=}`
int axis = node->`\refvar[KdAccelNode::SplitAxis]{SplitAxis}{}`();
`\refvar{Float}{}` tPlane = (node->`\refvar{SplitPos}{}`() - ray.`\refvar[Ray::o]{o}{}`[axis]) * invDir[axis];
\end{lstlisting}

现在需要确定光线遇到孩子节点的顺序使得是沿光线按从前往后的顺序遍历树的。
\reffig{4.18}展示了该计算的几何结构。
射线端点关于划分平面的位置足够区分两种情况,
现在忽略光线实际上没有穿过两节点之一的情况。
射线端点位于划分平面上的罕见情况需仔细处理,
需要改用它的方向来区分两种情况。
\begin{figure}[htbp]
    \centering\input{Pictures/chap04/Raybelowabove.tex}
    \caption{射线端点关于划分平面的位置可用于确定该首先处理该节点的哪个孩子。
    如果射线如$\bm r_1$的端点在划分平面的“下方”一侧,
    则我们在处理上方孩子前应该先处理下方孩子,反之亦然。}
    \label{fig:4.18}
\end{figure}
\begin{lstlisting}
`\initcode{Get node children pointers for ray}{=}`
const `\refvar{KdAccelNode}{}` *firstChild, *secondChild;
int belowFirst = (ray.`\refvar[Ray::o]{o}{}`[axis] <  node->`\refvar{SplitPos}{}`()) ||
                 (ray.`\refvar[Ray::o]{o}{}`[axis] == node->`\refvar{SplitPos}{}`() && ray.`\refvar[Ray::d]{d}{}`[axis] <= 0);
if (belowFirst) {
    firstChild = node + 1;
    secondChild = &`\refvar[KdTreeAccel::nodes]{nodes}{}`[node->`\refvar{AboveChild}{}`()];
} else {
    firstChild = &`\refvar[KdTreeAccel::nodes]{nodes}{}`[node->`\refvar{AboveChild}{}`()];
    secondChild = node + 1;
}
\end{lstlisting}

该节点的两个孩子可能没有必要都处理。
\reffig{4.19}展示了一些光线只穿过一个孩子的配置。
光线绝不会同时错过两个孩子,因为否则当前内部节点就不该被访问到。
\begin{figure}[htbp]
    \centering\input{Pictures/chap04/kdskipanode.tex}
    \caption{节点的两个孩子不需要都处理的两种情况,因为光线没有与之重合。
    (a)上方光线与划分平面的相交超出了光线的$t_{\max}$位置,
    因此没有进入更远的孩子。下方光线背对划分平面,这由负的$t_{\text{split}}$值表示。
    (b)光线在其$t_{\min}$值之前与平面相交,意味着不需要处理近处孩子。}
    \label{fig:4.19}
\end{figure}

下面的代码中第一个{\ttfamily if}测试与\reffig{4.19}(a)对应:
如果可以证明由于光线背对平面或因$t_{\text{split}}>t_{\max}$没与节点重合,
即光线没有与远处节点重合,则只有近处节点需要处理。
\reffig{4.19}(b)展示了第二个{\ttfamily if}测试中类似的情况:
如果光线没与之重合,则可能不需要处理近处节点。
否则,{\ttfamily else}语句负责两个孩子都需要处理的情况;
接着会处理近处节点,而远处节点列入列表{\ttfamily todo}。
\begin{lstlisting}
`\initcode{Advance to next child node, possibly enqueue other child}{=}`
if (tPlane > tMax || tPlane <= 0)
    node = firstChild;
else if (tPlane < tMin)
    node = secondChild;
else {
    `\refcode{Enqueue secondChild in todo list}{}`
    node = firstChild;
    tMax = tPlane;
}
\end{lstlisting}
\begin{lstlisting}
`\initcode{Enqueue secondChild in todo list}{=}`
todo[todoPos].`\refvar[KdToDo::node]{node}{}` = secondChild;
todo[todoPos].`\refvar[KdToDo::tMin]{tMin}{}` = tPlane;
todo[todoPos].`\refvar[KdToDo::tMax]{tMax}{}` = tMax;
++todoPos;
\end{lstlisting}

如果当前节点是叶子,则对叶子里的图元执行相交测试。
\begin{lstlisting}
`\initcode{Check for intersections inside leaf node}{=}`
int nPrimitives = node->`\refvar[KdAccelNode::nPrimitives]{nPrimitives}{}`();
if (nPrimitives == 1) {
    const std::shared_ptr<`\refvar{Primitive}{}`> &p = `\refvar[KdTreeAccel::primitives]{primitives}{}`[node->`\refvar{onePrimitive}{}`];
    `\refcode{Check one primitive inside leaf node}{}`
} else {
    for (int i = 0; i < nPrimitives; ++i) {
        int index = `\refvar{primitiveIndices}{}`[node->`\refvar{primitiveIndicesOffset}{}` + i];
        const std::shared_ptr<`\refvar{Primitive}{}`> &p = `\refvar[KdTreeAccel::primitives]{primitives}{}`[index];
        `\refcode{Check one primitive inside leaf node}{}`
    }
}
\end{lstlisting}

处理单个图元就是把相交请求传给图元的事。
\begin{lstlisting}
`\initcode{Check one primitive inside leaf node}{=}`
if (p->`\refvar[Primitive::Intersect]{Intersect}{}`(ray, isect)) 
    hit = true;
\end{lstlisting}

在叶子节点做完相交测试后,从数组{\ttfamily todo}加载下一个要处理的节点。
如果没有剩下更多节点了,则光线穿过该树没有命中任何东西。
\begin{lstlisting}
`\initcode{Grab next node to process from todo list}{=}`
if (todoPos > 0) {
    --todoPos;
    node = todo[todoPos].`\refvar[KdToDo::node]{node}{}`;
    tMin = todo[todoPos].`\refvar[KdToDo::tMin]{tMin}{}`;
    tMax = todo[todoPos].`\refvar[KdToDo::tMax]{tMax}{}`;
}
else
    break;
\end{lstlisting}

像\refvar{BVHAccel}{}那样,此处没有展示\refvar{KdTreeAccel}{}对于
阴影射线的特殊化相交方法。它和方法\refvar[KdTreeAccel::Intersect]{Intersect}{()}类似
但调用的是方法\refvar{Primitive::IntersectP}{()}且一旦
找到任何相交处就返回{\ttfamily true}而不担心是否找到了最近的那个。
\begin{lstlisting}
`\initcode{KdTreeAccel Public Methods}{=}`
bool `\initvar[KdTreeAccel::IntersectP]{IntersectP}{}`(const `\refvar{Ray}{}` &ray) const;
\end{lstlisting}

\section{扩展阅读}\label{sec:扩展阅读04}
引入光线追踪算法之后,涌现了大量尝试寻找高效方法对其加速的研究,
主要是通过开发改进的光线追踪加速结构。
《\citetitle{10.5555/94788}》\citep{10.5555/94788}中Arvo和Kirk的章节
总结了1989年最新进展并为区分不同光线相交加速方法提供了优秀的分类方案。

\citet{Kirk88theray}引入了\keyindex{元层次}{meta-hierarchies}{}的统一原则。
它们证明了通过让实现的加速数据结构与场景图元遵照相同的接口,
很容易混合与匹配不同的相交加速框架。
pbrt遵循这一模型,因为\refvar{Aggregate}{}继承自基类\refvar{Primitive}{}。

\subsection{网格}\label{sub:网格}
\citet{4056861}引入了均匀网格,
即把场景边界分解为等长网格的空间细分方法。
\citet{10.2312:egtp.19871000}
以及\citet{Cleary1988}描述了更高效的网格遍历方法。
\citet{10.1145/37401.37417}描述了
对该方法的大量改进并证明了网格对于渲染极其复杂场景的用处。
\citet{Jevans1989:23}引入了层次化网格,
即含有许多图元的网格自我细化为小格。
\citet{cazals1995filtering}以及
\citet{576857}为层次化网格开发了更复杂的技术。

\citet{4061545}为网格的并行创建开发了高效算法。
他们的有趣发现之一是随着所用处理核数量的增长,
网格创建性能很快被有效内存带宽所限制。

选择最优网格分辨率对于从网格中获得优异性能很重要。
\citet{4342587}有该话题的优秀论文,
为完全自动化选择分辨率以及在使用层次化网格时决定何时细化为子网格提供了坚实基础。
他们用大量简化假设推导出理论结果,然后证明了这些结果渲染真实世界场景的适用性。
他们的论文也包括对该领域前人工作很好的筛选引用。

\citet{lagae2008compact}基于
哈希法\sidenote{译者注:即hashing,也称散列法。}为均匀网格
描述了一种新颖的表示,它具有的优良性质是不仅每个图元
拥有对网格的单个索引,而且每个网格也只有单个图元索引。
他们证明了该表示有很低的内存使用量且仍然非常高效。

\citet{4634613}证明在透视空间中构建网格,
即投影中心是相机或光源时,能让追踪相机或光源发出的光线高效得多。
尽管该方法需要多种加速结构,但从为不同种类光线专门设计的多种结构中获得的性能提升可以很高。
他们的方法也因在某种意义上是栅格化和光线追踪的中间地带而令人瞩目。

\subsection{包围盒层次}\label{sub:包围盒层次}
\citet{10.1145/360349.360354}首先建议为标准可见曲面确定算法使用包围盒来剔除物体集。
在此基础上,\citet{10.1145/800250.807479}首先为快速光线追踪
的场景表示开发了层次化数据结构,尽管他们的方法依赖于用户去定义层次。
\citet{10.1145/15922.15916}基于用厚板集定界物体实现了最早之一的实用物体细分方法。
\citet{4057175}描述了自动计算包围盒层次的首个算法。
尽管他们的算法是基于依据盒的表面积来估计光线与包围盒相交的概率,
但它比现代SAH BVH算法低效得多。

本章的\refvar{BVHAccel}{}实现基于\citet{4342588}以及\citet{4342598}描述的构建算法。
边界框测试则是\citet{10.1145/1198555.1198748}引入的。
\citet{10.1080/2151237X.2007.10129248}开发了甚至更高效的边界框测试,
当同一光线对许多边界框做相交测试时它进行额外的预计算以换取更高的性能;
我们把实现他们的方法留作习题。

pbrt中用的BVH遍历算法由多位研究者同时开发出来;
见\citet{bouloshaines2006}的批注了解更多细节和背景。
树遍历的另一选项是\citet{10.1145/15922.15916};
他们维护一个按光线距离排序的节点堆。
在单片存储\sidenote{译者注:原文on-chip memory。}数量相对有限的GPU上,
为每条光线维护一个将要访问的节点的栈可能会有极其高的内存开销。
\citet{10.1145/1071866.1071869}引入了
“无栈”\sidenote{译者注:原文stackless。}kd树遍历算法,
它周期性地从树根开始回溯和搜索以找到下一个要访问的节点而不是显式保存所有要访问的节点。
\citet{10.5555/1921479.1921496}对该方法做了大量改进,
减少了从树根重新遍历的频率并将该方法应用于BVH。

许多研究者已经为构建BVH后提升其质量开发了许多技术。
\citet{10.2312:EGWR:EGSR07:073-084}和\citet{4634624}提出了
对BVH做局部调整的算法,\citet{10.1145/2159616.2159649}在
一个动画的多个坐标系上复用BVH,通过更新包围运动物体的部分来保持其质量。
也见\citet{BittnerFast2013}、\citet{10.1145/2492045.2492055}
以及\citet{BITTNER2015135}了解该领域的最新工作。

当前大多数构建BVH方法都基于自顶向下的树构建,
首先创建树节点然后将图元划分到孩子中并继续递归。
\citet{4634626}证明了另一个方法,
他说明自底向上的构建即首先创建叶子然后聚为父亲节点是可行选项。
\citet{10.1145/2492045.2492054}开发了该方法高效得多的实现并
证明了其对并行实现的适应性。

BVH的一个缺点是即便少量与边界框重合的相对较大图元也会极大降低BVH的效率:
仅因为下沉到叶子的几何体的重合边界框\sidenote{译者注:此句翻译不确定。},
许多树节点就会重合。\citet{4342593}提出
“分割剪裁”\sidenote{译者注:原文split clipping。}的办法;
提出了树中每个图元只出现一次的约束,
且巨大输入图元的边界框被细分为更紧致的子框集再用于树的构建。
\citet{4634636}观察到有问题的图元是
那些相对于其表面积在其边界框内有大量空白空间的,
所以它们细分了最异常的三角形并报告有巨大性能提升。
\citet{10.1145/1572769.1572771}开发了
在BVH构建期间划分图元的方法,使得当发现SAH开销下降时可以只划分图元。
也见\citet{10.1145/1572769.1572772}理论上
优化BVH划分算法及其与之前方法的关系,
以及\citet{10.1145/2492045.2492055}改进
决定何时分割三角形的准则。
\citet{10.5555/2980009.2980014}开发了一种方法为长细几何体如毛发等构建BVH;
因为这类几何体相对于其边界框体积来说非常细,
在大多数加速结构上它一般都有很差的性能。

BVH的内存要求可以非常大。在我们的实现中,每个节点为32字节。
场景中每个图元至多需要2个BVH树节点,每个图元的总开销可以高达64字节。
\citet{10.5555/2383894.2383909}建议为BVH节点使用更紧实的表示,牺牲一些效率。
首先,他们量化了每个节点中保存的边界框,用8或16字节来编码其相对于节点父亲边界框的位置。
然后,他们使用了{\itshape 隐式索引}\sidenote{译者注:原文implicit indexing。},
其中节点$i$的孩子在节点数组中的位置为$2i$和$2i+1$(假设分支系数为$2\times$)。
他们证明节约了大量内存,而性能影响适中。
\citet{10.2312:PE:VMV:VMV10:227-234}开发了另一种在空间上高效的BVH表示。
也见\citet{10.5555/1839214.1839242}开发的BVH节点与三角网格的紧实表示。

\citet{10.1111/j.1467-8659.2006.00970.x}为
缓存高效的BVH和kd树布局提出了算法并展示了来自它们的性能提升。
也见\citet{10.5555/1121584}的书籍了解该话题的广泛讨论。

\citet{10.1111/j.1467-8659.2009.01377.x}引入了线性BVH。
\citet{10.5555/1921479.1921493}在树的上层用SAH开发了HLBVH推广型。
他们还注意到莫顿编码值的高位可用于高效寻找图元群集——两种思想都用于我们HLBVH的开发。
\citet{10.1145/2018323.2018333}对HLBVH引入了更多改进,
它们多数都针对GPU实现。

不像HLBVH路线,这里\refvar{BVHAccel}{}中的BVH构建实现没有并行化。
详见\citet{5669303}了解始终利用SAH进行高性能并行BVH构建的方法。

\subsection{kd树}\label{sub:kd树}
\citet{6429331}为光线相交计算引入使用了八叉树。
\citet{kaplan1985use}\sidenote{译者注:未能找到该文献信息。}首先提出了为光线追踪使用kd树。
\citeauthor{kaplan1985use}的树构建算法总是从中间划分节点;
\citet{MacDonald1990}引入了SAH方法,
用相对表面积估计光线-节点遍历概率。
\citet{Naylor1993:27}也写过关于构建优良kd树的一般问题。
\citet{HavranImproving2002}回顾了许多这些问题并介绍了有用的改进。
\citet{hurley2002fast}提出为完全为空的树节点
添加补贴\sidenote{译者注:原文bonus。}因子到SAH中,就像我们的实现中做的那样。
见\citet{Havran2000:PhD}的博士论文了解对于高性能kd树构建和遍历算法的出色综述。


\citet{10.1007/978-3-642-71071-1_4}首次为kd树开发了高效的光线遍历算法。
\citet{ArvoRay1988}也研究了该问题并在《{\itshape\citefield{ArvoRay1988}{journaltitle}}》中一篇笔记里中作了讨论。
\citet{SUNG1992271}为BSP树加速器描述了一种光线遍历算法的实现;
我们的\refvar{KdTreeAccel}{}遍历代码一定程度上是基于它们的。

pbrt中kd树构建算法的渐进复杂度是$O(n\log^2n)$.
\citet{4061547}证明了用一些额外繁琐的实现可以在$O(n\log n)$时间内构建kd-树;
它们为典型场景报告了$2$到$3\times$的构建时间加速。

光线追踪最好的kd树是用“完美划分”\sidenote{译者注:原文perfect splits。}构建的,
每一步中剪裁正插入到树中的图元以适应当前节点的边界。
这避免了一个问题,例如一个物体的边界框可能与节点的边界框相交因而被存储于其中,
然而该物体自己并没有与该节点的边界框相交。
该方法由\citet{HavranImproving2002}提出并
由\citet{hurley2002fast}以及\citet{4061547}进一步讨论。
也见\citet{4634623}。
即便用完美划分,大型图元仍可能存于许多个kd树叶子中;
\citet{10.1111/cgf.12241}建议在内部节点中存储一些图元以解决该问题。

kd树构建往往比BVH构建慢得多(尤其是如果使用了“完美划分”),
所以并行构建算法特别有意义。该领域的最新工作包括
\citet{10.1111/j.1467-8659.2007.01062.x}和
\citet{10.5555/1921479.1921492},
他们提出了对多处理器有良好扩展性的高效并行kd树构建算法。

\subsection{表面积启发法}\label{sub:表面积启发法2}
自\citet{MacDonald1990}把SAH引入到
光线追踪以来许多研究者已经钻研了对SAH的改进。
\citet{10.2312:egs.20091046}派生出的一个版本是把
光线在空间中均匀分布的假设替换为光线的起点均匀分布于场景的边界框中。
\citet{4634614}引入了新的SAH,
它导致事实上光线一般并不均匀分布但是它们许多都从单个点
或一组相邻点(分别为相机和光源)发出。\citet{4634625}展示了当使用
“邮箱”\sidenote{译者注:原文mailboxing。}优化时应该怎样修改SAH,
而\citet{VINKLER2012283}用关于图元可见性的假设来调整它们的SAH开销。
\citet{10.1111/j.1467-8659.2011.01861.x}派生出
“光线终止表面积启发法”\sidenote{译者注:原文ray termination surface area heuristic。}(RTSAH),
它们用其来为阴影射线调整BVH遍历顺序以更快找到与遮挡物的相交处。
也见\citet{10.2312:sre.20151164}调整SAH以在kd树
遍历时对正被遮挡的阴影射线负责。

计算SAH会开销很大,尤其是当考虑许多不同的划分或图元分割时。
该问题的一个办法是只在候选点子集处计算它——例如,
沿着pbrt中\refvar{BVHAccel}{}里用的桶方法的直线来。
\citet{hurley2002fast}为构建kd树推荐该方法,
而\citet{4061550}详细讨论了它。
\citet{10.1111/j.1467-8659.2007.01062.x}引入了
将三角形全范围而不仅仅是其形心归入统计\sidenote{译者注:原文binning。}的改进。

\citet{4061549}注意到例如若你只需在一点计算SAH,
则你不需要对图元排序而只需对它们做线性扫描以计算图元数量和该点的边界框。
他们证明了用基于其在许多独立点上的值得到的分段二次式来逼近SAH并
用它选择良好划分会得到高效的树。
\citet{4061550}用了类似的近似。

尽管SAH能得到非常高效的kd树和BVH,但明确的是它并不完美:
许多研究者已经注意到遇到有更高SAH估计开销的kd树或BVH比
具有最低估计开销的树给出更好性能的情况并不罕见。
\citet{10.1145/2492045.2492056}调查了他们的一些结果并
提出两个额外启发法帮助解决之;一个考虑了事实上大多数光线始于曲面——
光线起点实际上并不在场景中随机分布,另一个考虑了当多条光线一起穿过层级时
的SIMD\sidenote{译者注:single instruction multiple data,单指令流多数据流。}分散度。
尽管这些新层次在解释为什么给定的树提供了这样的性能上很有效,
但至今也不知道怎样将其与树构建算法搭配。

\subsection{加速结构的其他话题}\label{sub:加速结构的其他话题}
\citet{10.1145/357332.357335}讨论了
为包围盒使用不同形状的权衡方法并建议把物体投影到屏幕上
再使用$z$-缓存区渲染来为相机光线寻找相交处加速。

许多研究者已经研究了划分平面不必是轴对齐时一般BSP树的适用性,就像kd树的那样。
\citet{4342591}用预选的候选划分平面集来构建树,
然而因为比kd树更慢的构建阶段和更慢的遍历,他们的结果在实际中只接近kd树的性能。
\citet{4634637}展示了比现代kd树更快渲染场景的BSP实现但会花非常长的构建时间。

有很多让一组光线一起而不是每次一个遍历加速结构的技术。
该方法(“包追踪”\sidenote{译者注:原文packet tracing。})是高性能光线追踪的重要组成;
\refsub{包追踪}会更深入地讨论它。

动画图元给光线追踪器带来两个挑战:
第一,如果物体在移动则尝试在多帧动画上复用加速结构的渲染器必须更新加速结构。
\citet{10.2312:egst.20071056}展示了这种情况下怎样渐进更新BVH,
\citet{10.1111/j.1467-8659.2009.01497.x}建议创建相邻图元群集
然后构建这些群集的BVH(因此减轻BVH构建算法的负担)。
第二个问题是对于快速移动的图元,在它们在帧时间上整个运动的边界框可能非常大,
导致有许多不必要的光线-图元相交测试。
关于该问题的著名工作包括\citet{504}加上时间把光线追踪(以及加速的八叉树)推广到四维。
最近Gr\"{u}nschlo\ss{}等\parencite*{10.1145/2018323.2018334}
\sidenote{译者注:参考文献列表中该作者名字中的德文字母“\ss{}”错误显示为“SS”,
目前无法解决,下同,请读者见谅。同时欢迎提供解决办法!}
为运动图元开发了针对BVH的改进。
也见\citet{10.2312:egst.20071056}关于光线追踪动画场景的综述论文。

\citet{10.1145/37401.37409}提出了加速结构的新型方法,
他们引入了5D数据结构来同时基于3D空间和2D光线方向进行细分。
\citet{10.1111/j.1467-8659.2008.01269.x}为
使用三角网格描述的场景提出了另一个有趣的方法:
他们计算一个受约束的四面体网格划分\sidenote{译者注:原文tetrahedralization。},
其中模型的所有三角形面都在四面体网格划分中表示。
然后光线逐步穿过四面体直到它们与来自场景描述的三角形相交。
该方法仍慢于最新kd树和BVH数倍,却是思考该问题的一个新的有趣方式。

在kd树和BVH之间有个有趣的中间地带,树节点为每个子节点持有划分平面而不是只有单个划分平面。
例如,这一改进使得能在像kd树那样的加速结构中作物体划分,
把每个图元放入仅一个子树并允许子树重合,
但仍保留了kd树高效遍历的许多好处。
\citet{10.1007/978-3-642-72617-0_17}\sidenote{译者注:未找到原文引用文献的信息,
译文改用同年同名同第一作者但发表位置不同的文献。}首次
将该改进引入到kd树中用于排序空间数据,命名为“空间kd树”
\sidenote{译者注:原文spatial kd-tree。}(skd-tree)。
skd树最近已经被许多研究者应用到光线追踪,包括
\citet{10.1145/585740.585761}、\citet{10.1145/1283900.1283912}、
\citet{10.5555/2383894.2383912}、
\citet{4061548}以及\citet{2151237X.2006.10129224}。

当使用像网格或kd树的空间划分方法时,图元可能与结构的多个节点重合,
光线可能在其穿过该结构时多次与同一个图元做相交测试。
\citet{Arnaldi1987}以及
\citet{10.2312:egtp.19871000}开发了
“邮箱”技术来解决这个问题:每条光线都给定唯一的整数标识符,
每个图元都记录与之测试的最后一条光线的id。
如果该id匹配,则相交测试是不必要的,可以跳过它。

尽管很高效,但该方法不适用于多线程光线追踪器。
为了解决该问题,\citet{Benthin_2006}建议排序
每条光线的小哈希表以记录最近相交的图元。
\citet{shevtsov2007ray}维护了
最后$n$个相交图元id的小数组并在执行相交测试前线性搜索它。
尽管两种方法仍可能多次检出一些图元,但它们通常剔除了大多数冗余测试。

\input{content/chap0406.tex}


\part{成像过程}
\input{content/chap05.tex}

\chapterimage{Pictures/chap06/landscape-dof-960x1920.png}
\chapter{相机模型}\label{chap:相机模型}
\setcounter{sidenote}{1}

在第\refchap{绪论}中,我们介绍了计算机图形学中常用的针孔相机模型。
该模型很容易描述和模拟,但它忽略了真实相机中透镜对于穿过的光线具有的重要效应。
例如,针孔相机渲染的所有东西都是清晰合焦的——真实透镜系统不可能达到这种状态。
这样的图像常常看得出来是计算机生成的。
更一般地,离开透镜系统的辐射分布和进入它的分布有很大区别;
对透镜的这种效应建模对于准确模拟成像的辐射度量非常重要。

相机透镜系统也会引入各种影响其所构建图像的\keyindex{像差}{aberration}{};
例如,因为能到达胶片或传感器边缘的光比中心处更少,
\keyindex{渐晕}{vignetting}{}导致图像边缘变暗。
透镜也可以造成\keyindex{枕状畸变}{pincushion distortion}{distortion畸变}或\keyindex{桶状畸变}{barrel distortion}{distortion畸变},
即让直线成像为曲线。
尽管透镜设计者尽力在其设计中最小化像差,但它们仍可对图像有明显作用。

像第\refchap{形状}的\refvar{Shape}{}那样,pbrt中的相机也表示为抽象基类。
本章介绍类\refvar{Camera}{}及其两个关键方法\refvar[GenerateRay]{Camera::GenerateRay}{()}
和\refvar[GenerateRayDifferential]{Camera::GenerateRayDifferential}{()}。
第一个方法计算对应于胶片平面上样本位置的世界空间光线。
通过基于不同成像模型的不同方法生成这些光线,pbrt中的相机可以创建同一3D场景的多种图像。
第二种方法不仅生成该光线还计算采样该光线的图像区域的信息;
例如该信息用于第\refchap{纹理}的抗锯齿计算。
在\refsub{采样相机2}将介绍一些额外的\refvar{Camera}{}方法以支持双向光传输算法。

本章中,我们将展示\refvar{Camera}{}接口的一些实现,
从实现具有一定普遍性的理想针孔模型开始,
到和真实世界相机一样能模拟光穿过一组玻璃透镜元件成像的逼真模型结束。

\input{content/chap0601.tex}

\section{投影相机模型}\label{sec:投影相机模型}

三维计算机图形学中的一个基本问题是{\itshape 3D视见问题}:
如何将三维场景投影到二维图像上进行显示。
大多数经典方法都可以用$4\times4$的\keyindex{投影变换}{projective transformation}{transformation变换}矩阵来表达。
因此,我们将引入一个投影矩阵相机类\refvar{ProjectiveCamera}{},
然后在此基础上定义两个相机模型。
第一个实现了\keyindex{正交投影}{orthographic projection}{projection投影},
另一个实现了\keyindex{透视投影}{perspective projection}{projection投影}——
两种经典且广泛使用的\keyindex{投影}{projection}{}。
\begin{lstlisting}
`\initcode{Camera Declarations}{+=}\lastcode{CameraDeclarations}`
class `\initvar{ProjectiveCamera}{}` : public `\refvar{Camera}{}` {
public:
    `\refcode{ProjectiveCamera Public Methods}{}`
protected:
    `\refcode{ProjectiveCamera Protected Data}{}`
};
\end{lstlisting}

还有三个坐标系(总结于\reffig{6.1}中)对定义和讨论投影相机很有用。
\begin{itemize}
    \item \keyindex{屏幕空间}{screen space}{}:
          屏幕空间定义在胶片平面上。相机将相机空间中的物体投影到胶片平面上;
          \keyindex{屏幕窗口}{screen window}{}内的部分在生成的图像中是可见的。
          屏幕空间的深度$z$值从0变到1,分别对应近处和远处截平面的点。
          注意,虽然这称为“屏幕”空间,但它仍然是一个三维坐标系,因为$z$值是有意义的。
    \item \keyindex{规范化设备坐标}{normalized device coordinate}{}(NDC){\sffamily 空间}:
          这是被渲染的实际图像的坐标系。对于$x$和$y$,该空间范围从$(0,0)$变到$(1,1)$,
          其中$(0,0)$是图像的左上角。深度值与屏幕空间中的相同,线性变换将屏幕空间转换为NDC空间。
    \item \keyindex{栅格空间}{raster space}{}\sidenote{译者注:也称光栅空间。}:
          这与NDC空间几乎相同,除了$x$和$y$坐标从$(0,0)$变到(resolution.x, resolution.y)
          \sidenote{译者注:resolution指分辨率。}。
\end{itemize}

投影相机用$4\times4$矩阵在所有这些空间之间进行转换,
但具有特殊成像特性的相机不必用矩阵表示所有这些转换。
\begin{figure}[htbp]
    \centering\includegraphics[width=0.75\linewidth]{chap06/Cameracoordinatespaces.eps}
    \put(-280,0){\small 相机空间:$(0,0,0)$}
    \put(-270,60){\small NDC:$(0,0,0)$}
    \put(-220,110){\small NDC:$(0,0,1)$}
    \put(-160,20){\small $z=\text{near}$}
    \put(-160,10){\small NDC:$(1,1,0)$}
    \put(-160,0){\small 栅格:$(\text{res}.x,\text{res}.y,0)$}
    \put(-70,35){\small $z=\text{far}$}
    \put(-70,25){\small NDC:$(1,1,1)$}
    \put(-70,15){\small 栅格:$(\text{res}.x,\text{res}.y,1)$}
    \caption{几个与相机相关的坐标空间常用于简化\protect\refvar{Camera}{}的实现。
        相机类持有它们之间的变换。世界空间中的场景物体由相机查看,它位于相机空间原点,并指向$+z$轴。
        近处和远处平面之间的物体被投影到相机空间中的胶片平面$z=\text{near}$上。
        胶片平面在栅格空间中$z=0$处,其中$x$和$y$范围从$(0,0)$变到(resolution.x, resolution.y)。
        规范化设备坐标(NDC)空间将栅格空间归一化,因此$x$和$y$范围从$(0,0)$变到$(1,1)$.}
    \label{fig:6.1}
\end{figure}

除了基类\refvar{Camera}{}要求的参数外,\refvar{ProjectiveCamera}{}还接收投影变换矩阵、
图像的屏幕空间范围以及与景深有关的额外参数。
\keyindex{景深}{depth of field}{}将在本节末尾介绍和实现,
它模拟了真实透镜系统中出现的失焦物体的模糊性。
\begin{lstlisting}
`\initcode{ProjectiveCamera Public Methods}{=}`
`\refvar{ProjectiveCamera}{}`(const `\refvar{AnimatedTransform}{}` &CameraToWorld, 
        const `\refvar{Transform}{}` &CameraToScreen, const `\refvar{Bounds2f}{}` &screenWindow,
        `\refvar{Float}{}` shutterOpen, `\refvar{Float}{}` shutterClose, `\refvar{Float}{}` lensr, `\refvar{Float}{}` focald,
        `\refvar{Film}{}` *film, const `\refvar{Medium}{}` *medium)
    : `\refvar{Camera}{}`(CameraToWorld, shutterOpen, shutterClose, film, medium),
      `\refvar{CameraToScreen}{}`(CameraToScreen) {
    `\refcode{Initialize depth of field parameters}{}`
    `\refcode{Compute projective camera transformations}{}`
}
\end{lstlisting}

\refvar{ProjectiveCamera}{}的实现将投影变换传递给这里展示的基类构造函数。
该变换给出了相机到屏幕的投影;
由此,构造函数能轻松算出从栅格空间到相机空间一路所需的其他变换。
\begin{lstlisting}
`\initcode{Compute projective camera transformations}{=}`
`\refcode{Compute projective camera screen transformations}{}`
`\refvar{RasterToCamera}{}` = `\refvar[Transform::Inverse]{Inverse}{}`(CameraToScreen) * `\refvar{RasterToScreen}{}`;
\end{lstlisting}
\begin{lstlisting}
`\initcode{ProjectiveCamera Protected Data}{=}\initnext{ProjectiveCameraProtectedData}`
`\refvar{Transform}{}` `\initvar{CameraToScreen}{}`, `\initvar{RasterToCamera}{}`;
\end{lstlisting}

在构造函数中唯一要计算的重要变换是屏幕到栅格的投影。
在下面的代码中,请注意变换的组成(从下往上看),
我们从屏幕空间的一个点开始,先平移使得屏幕左上角位于原点,
然后用屏幕宽度和高度的倒数进行缩放,
得到一个$x$和$y$坐标在0到1之间的点(这些是NDC坐标)。
最后,我们用栅格化分辨率进行缩放,这样我们最终就能完全覆盖
从$(0,0)$直到整个栅格分辨率的栅格范围。
这里一个重要细节是$y$坐标被该变换倒置了;
这是必要的,因为增加的$y$值在屏幕坐标中是向上移动但在栅格坐标中是向下的。
\begin{lstlisting}
`\initcode{Compute projective camera screen transformations}{=}`
`\refvar{ScreenToRaster}{}` = `\refvar{Scale}{}`(film->`\refvar{fullResolution}{}`.x, 
                       film->`\refvar{fullResolution}{}`.y, 1) *
    `\refvar{Scale}{}`(1 / (screenWindow.`\refvar{pMax}{}`.x - screenWindow.`\refvar{pMin}{}`.x),
          1 / (screenWindow.`\refvar{pMin}{}`.y - screenWindow.`\refvar{pMax}{}`.y), 1) *
    `\refvar{Translate}{}`(`\refvar{Vector3f}{}`(-screenWindow.`\refvar{pMin}{}`.x, -screenWindow.`\refvar{pMax}{}`.y, 0));
`\refvar{RasterToScreen}{}` = `\refvar[Transform::Inverse]{Inverse}{}`(`\refvar{ScreenToRaster}{}`);
\end{lstlisting}
\begin{lstlisting}
`\refcode{ProjectiveCamera Protected Data}{+=}\lastnext{ProjectiveCameraProtectedData}`
`\refvar{Transform}{}` `\initvar{ScreenToRaster}{}`, `\initvar{RasterToScreen}{}`;
\end{lstlisting}

\subsection{正交相机}\label{sub:正交相机}
\begin{lstlisting}
`\initcode{OrthographicCamera Declarations}{=}`
class `\initvar{OrthographicCamera}{}` : public `\refvar{ProjectiveCamera}{}` {
public:
    `\refcode{OrthographicCamera Public Methods}{}`
private:
    `\refcode{OrthographicCamera Private Data}{}`
};
\end{lstlisting}

定义在文件\href{https://github.com/mmp/pbrt-v3/blob/master/src/cameras/orthographic.h}{\ttfamily cameras/orthographic.h}和
\href{https://github.com/mmp/pbrt-v3/tree/master/src/cameras/orthographic.cpp}{\ttfamily cameras/orthographic.cpp}中的\keyindex{正交相机}{orthographic camera}{camera相机},
是基于正交投影变换的。
正交变换取场景中的一块矩形区域并将其投影到定义该区域之框的前方一面。
它不具有\keyindex{前缩}{foreshortening}{}效应——当物体远离时它们在成像平面上变小——
但它让平行线依然平行,并保留物体间的相对距离。
\reffig{6.2}{}展示了该立方体是如何定义场景可见区域的。
\begin{figure}[htbp]
    \centering\includegraphics[width=0.33\linewidth]{chap06/Orthoviewingvolume.eps}
    \caption{正交视见体是相机空间中的轴对齐框,
        其定义使该区域内的物体投影到该框$z=\text{near}$的一面上。}
    \label{fig:6.2}
\end{figure}

\reffig{6.3}比较了用正交投影和下节定义的透视投影来渲染的结果
\sidenote{译者注:原图为exr格式,此处转换为png格式以便制作插图,
    图像细节和色域可能发生细微变化。后续均作此处理,读者可到原书官网查看原图。}。
\begin{figure}[htbp]
    \centering
    \subfloat[正交]{\includegraphics[width=\linewidth]{chap06/car-ortho.png}\label{fig:6.3.1}}\\
    \subfloat[透视]{\includegraphics[width=\linewidth]{chap06/car-perspective.png}\label{fig:6.3.2}}
    \caption{用不同相机模型渲染的汽车模型。用(a)正交和(b)透视相机从同一视点渲染汽车。
        缺少前缩使得正交视角看起来深度更少,但它保留了平行线,是很有用的特性。}
    \label{fig:6.3}
\end{figure}

正交相机构造函数用稍后定义的函数\refvar{Orthographic}{()}生成正交变换矩阵。
\begin{lstlisting}
`\initcode{OrthographicCamera Public Methods}{=}`
`\refvar{OrthographicCamera}{}`(const `\refvar{AnimatedTransform}{}` &CameraToWorld,
        const `\refvar{Bounds2f}{}` &screenWindow, `\refvar{Float}{}` shutterOpen,
        `\refvar{Float}{}` shutterClose, `\refvar{Float}{}` lensRadius, `\refvar{Float}{}` focalDistance,
        `\refvar{Film}{}` *film, const `\refvar{Medium}{}` *medium)
    : `\refvar{ProjectiveCamera}{}`(CameraToWorld, `\refvar{Orthographic}{}`(0, 1),
                       screenWindow, shutterOpen, shutterClose,
                       lensRadius, focalDistance, film, medium) {
    `\refcode{Compute differential changes in origin for orthographic camera rays}{}`
}
\end{lstlisting}

正交视角变换保持$x$和$y$坐标不变但将近处平面的$z$值映射为0而远处平面的$z$值映射为1.
为此,场景先沿$z$轴平移使得近处平面对齐到$z=0$.
然后,场景按$z$缩放使得远处平面映射为$z=1$.
这两个变换合成得到整个变换。(对于像pbrt那样的光线追踪器,
我们想让近处平面位于0处,这样光线就会从穿过相机位置的平面上发出;
远处平面偏移量不是很重要。)
\begin{lstlisting}
`\refcode{Transform Method Definitions}{+=}\lastnext{TransformMethodDefinitions}`
`\refvar{Transform}{}` `\initvar{Orthographic}{}`(`\refvar{Float}{}` zNear, `\refvar{Float}{}` zFar) {
    return `\refvar{Scale}{}`(1, 1, 1 / (zFar - zNear)) *
           `\refvar{Translate}{}`(`\refvar{Vector3f}{}`(0, 0, -zNear));
}
\end{lstlisting}

幸亏正交投影很简单,在方法\refvar[OrthographicCamera::GenerateRayDifferential]{GenerateRayDifferential}{()}中
很容易直接计算$x$和$y$方向的差分射线。
差分射线的方向将和主射线一样(对于同一个正交相机生成的所有光线都是这样),
且端点差异对于所有射线也会一样。
因此,这里的构造函数预先计算射线端点因胶片平面上在$x$和$y$方向移动单个像素而
在相机空间坐标中移动了多少。
\begin{lstlisting}
`\initcode{Compute differential changes in origin for orthographic camera rays}{=}`
`\refvar[OrthographicCamera::dxCamera]{dxCamera}{}` = `\refvar{RasterToCamera}{}`(`\refvar{Vector3f}{}`(1, 0, 0));
`\refvar[OrthographicCamera::dyCamera]{dyCamera}{}` = `\refvar{RasterToCamera}{}`(`\refvar{Vector3f}{}`(0, 1, 0));
\end{lstlisting}
\begin{lstlisting}
`\initcode{OrthographicCamera Private Data}{=}`
`\refvar{Vector3f}{}` `\initvar[OrthographicCamera::dxCamera]{dxCamera}{}`, `\initvar[OrthographicCamera::dyCamera]{dyCamera}{}`;
\end{lstlisting}

我们现在可以执行代码取栅格空间中的一个样本点并将其变为相机光线。
\reffig{6.4}总结了该过程。首先,栅格空间样本位置变换为相机空间的一点,
即给出近处平面上一点作为相机光线的端点。
因为相机空间观察方向沿$z$轴指出,相机空间光线方向为$(0,0,1)$.

\begin{figure}[htbp]
    \centering\includegraphics[width=0.8\linewidth]{chap06/Orthogenerateray.eps}
    \caption{为了用正交相机创建光线,胶片平面上的栅格空间位置被变换到相机空间,
        给出近处平面上的射线端点。相机空间中光线的方向为$(0,0,1)$,沿$z$轴。}
    \label{fig:6.4}
\end{figure}

若为该场景启用景深,则修改射线端点和方向来模拟景深。本节将稍后解释景深。
光线的时间值通过按偏移量\refvar{CameraSample::time}{}(在范围$[0,1)$内)
在快门开启和关闭间线性插值来设置。最后,光线在被返回前变换到世界空间。
\begin{lstlisting}
`\initcode{OrthographicCamera Definitions}{=}\initnext{OrthographicCameraDefinitions}`
`\refvar{Float}{}` `\refvar{OrthographicCamera}{}`::`\initvar[OrthographicCamera::GenerateRay]{\refvar{GenerateRay}{}}{}`(const `\refvar{CameraSample}{}` &sample,
        `\refvar{Ray}{}` *ray) const {
    `\refcode{Compute raster and camera sample positions}{}`
    *ray = `\refvar{Ray}{}`(pCamera, `\refvar{Vector3f}{}`(0, 0, 1));
    `\refcode{Modify ray for depth of field}{}`
    ray->`\refvar[Ray::time]{time}{}` = `\refvar{Lerp}{}`(sample.`\refvar[CameraSample::time]{time}{}`, `\refvar{shutterOpen}{}`, `\refvar{shutterClose}{}`);
    ray->`\refvar[Ray::medium]{medium}{}` = `\refvar[Camera::medium]{medium}{}`;
    *ray = `\refvar{CameraToWorld}{}`(*ray);
    return 1;
}
\end{lstlisting}

一旦设置好所有变换矩阵,就很容易将栅格空间样本变换到相机空间。
\begin{lstlisting}
`\initcode{Compute raster and camera sample positions}{=}`
`\refvar{Point3f}{}` pFilm = `\refvar{Point3f}{}`(sample.`\refvar{pFilm}{}`.x, sample.`\refvar{pFilm}{}`.y, 0);
`\refvar{Point3f}{}` pCamera = `\refvar{RasterToCamera}{}`(pFilm);
\end{lstlisting}

\refvar[OrthographicCamera::GenerateRayDifferential]{GenerateRayDifferential}{()}的实现执行一样的计算来生成相机主光线。
差分射线端点用在\refvar{OrthographicCamera}{}构造函数中算得的偏移量求出,
然后将整个差分射线变换到世界空间。
\begin{lstlisting}
`\refcode{OrthographicCamera Definitions}{+=}\lastcode{OrthographicCameraDefinitions}`
`\refvar{Float}{}` `\refvar{OrthographicCamera}{}`::`\initvar[OrthographicCamera::GenerateRayDifferential]{\refvar{GenerateRayDifferential}{}}{}`(
        const `\refvar{CameraSample}{}` &sample, `\refvar{RayDifferential}{}` *ray) const {
    `\refcode{Compute main orthographic viewing ray}{}`
    `\refcode{Compute ray differentials for OrthographicCamera}{}`
    ray->`\refvar[Ray::time]{time}{}` = `\refvar{Lerp}{}`(sample.`\refvar[CameraSample::time]{time}{}`, `\refvar{shutterOpen}{}`, `\refvar{shutterClose}{}`);
    ray->`\refvar{hasDifferentials}{}` = true;
    ray->`\refvar[Ray::medium]{medium}{}` = `\refvar[Camera::medium]{medium}{}`;
    *ray = `\refvar{CameraToWorld}{}`(*ray);
    return 1;
}
\end{lstlisting}
\begin{lstlisting}
`\initcode{Compute main orthographic viewing ray}{=}`
`\refcode{Compute raster and camera sample positions}{}`
*ray = `\refvar{RayDifferential}{}`(pCamera, `\refvar{Vector3f}{}`(0, 0, 1));
`\refcode{Modify ray for depth of field}{}`
\end{lstlisting}
\begin{lstlisting}
`\initcode{Compute ray differentials for OrthographicCamera}{=}`
if (`\refvar{lensRadius}{}` > 0) {
    `\refcode{Compute OrthographicCamera ray differentials accounting for lens}{}`
} else {
    ray->`\refvar{rxOrigin}{}` = ray->`\refvar[Ray::o]{o}{}` + `\refvar[OrthographicCamera::dxCamera]{dxCamera}{}`;
    ray->`\refvar{ryOrigin}{}` = ray->`\refvar[Ray::o]{o}{}` + `\refvar[OrthographicCamera::dyCamera]{dyCamera}{}`;
    ray->`\refvar{rxDirection}{}` = ray->`\refvar{ryDirection}{}` = ray->`\refvar[Ray::d]{d}{}`;
}
\end{lstlisting}
\begin{lstlisting}
`\initcode{Compute OrthographicCamera ray differentials accounting for lens}{=}`
`\refcode{Sample point on lens}{}`
`\refvar{Float}{}` ft = `\refvar{focalDistance}{}` / ray->`\refvar[Ray::d]{d}{}`.z;

`\refvar{Point3f}{}` pFocus = pCamera + `\refvar[OrthographicCamera::dxCamera]{dxCamera}{}` + (ft * `\refvar{Vector3f}{}`(0, 0, 1));
ray->`\refvar{rxOrigin}{}` = `\refvar{Point3f}{}`(pLens.x, pLens.y, 0);
ray->`\refvar{rxDirection}{}` = `\refvar{Normalize}{}`(pFocus - ray->`\refvar{rxOrigin}{}`);

pFocus = pCamera + `\refvar[OrthographicCamera::dyCamera]{dyCamera}{}` + (ft * `\refvar{Vector3f}{}`(0, 0, 1));
ray->`\refvar{ryOrigin}{}` = `\refvar{Point3f}{}`(pLens.x, pLens.y, 0);
ray->`\refvar{ryDirection}{}` = `\refvar{Normalize}{}`(pFocus - ray->`\refvar{ryOrigin}{}`);
\end{lstlisting}

\subsection{透视相机}\label{sub:透视相机}
透视投影和正交投影相似,也把一个空间体投影到2D胶片平面上。
然而,它包含前缩效应:远处物体比近处相同尺寸的物体投影得更小。
不像正交投影那样,透视投影不保持距离和角度,平行线也不再保持平行。
透视投影非常符合眼睛或相机透镜生成3D世界图像的方式。
投影相机实现于文件\href{https://github.com/mmp/pbrt-v3/blob/master/src/cameras/perspective.h}{\ttfamily cameras/perspective.h}
和\href{https://github.com/mmp/pbrt-v3/blob/master/src/cameras/perspective.cpp}{\ttfamily cameras/perspective.cpp}中。
\begin{lstlisting}
`\initcode{PerspectiveCamera Declarations}{=}`
class `\initvar{PerspectiveCamera}{}` : public `\refvar{ProjectiveCamera}{}` {
public:
    `\refcode{PerspectiveCamera Public Methods}{}`
private:
    `\refcode{PerspectiveCamera Private Data}{}`
};
\end{lstlisting}
\begin{lstlisting}
`\initcode{PerspectiveCamera Method Definitions}{=}\initnext{PerspectiveCameraMethodDefinitions}`
`\refvar{PerspectiveCamera}{}`::`\refvar{PerspectiveCamera}{}`(
        const `\refvar{AnimatedTransform}{}` &CameraToWorld,
        const `\refvar{Bounds2f}{}` &screenWindow, `\refvar{Float}{}` shutterOpen,
        `\refvar{Float}{}` shutterClose, `\refvar{Float}{}` lensRadius, `\refvar{Float}{}` focalDistance,
        `\refvar{Float}{}` fov, `\refvar{Film}{}` *film, const `\refvar{Medium}{}` *medium)
    : `\refvar{ProjectiveCamera}{}`(CameraToWorld, `\refvar{Perspective}{}`(fov, 1e-2f, 1000.f),
                       screenWindow, shutterOpen, shutterClose,
                       lensRadius, focalDistance, film, medium) {
    `\refcode{Compute differential changes in origin for perspective camera rays}{}`
    `\refcode{Compute image plane bounds at z=1 for PerspectiveCamera}{}`
}
\end{lstlisting}

透视投影描述了场景的透视图。
场景中的点投影到垂直于$z$轴的视平面上。
函数\refvar{Perspective}{()}计算该变换;
它接收视场角度{\ttfamily fov}以及到近处$z$平面和远处$z$平面的距离。
在透视投影后,近处$z$平面上的点映射为$z=0$,远处平面上的点则有$z=1$(\reffig{6.5})。
对于基于栅格化的渲染系统,仔细设置这些平面的位置很重要;
它们决定了要渲染的场景的$z$范围,但将它们取值的量级设置得相差过大可能导致数值精度误差。
对于像pbrt的光线追踪器,可以按其位置任意设置它们。
\begin{figure}[htbp]
    \centering\includegraphics[width=0.5\linewidth]{chap06/Perspectivetransformationmatrix.eps}
    \caption{投影变换矩阵将相机空间里的点投影到胶片平面。
        被投影点的$x'$和$y'$坐标等于投影前$x$和$y$坐标除以$z$坐标。
        投影后的$z'$坐标的计算使得近处平面的点映射为$z'=0$而
        远处平面的点映射为$z'=1$.}
    \label{fig:6.5}
\end{figure}

\begin{lstlisting}
`\refcode{Transform Method Definitions}{+=}\lastcode{TransformMethodDefinitions}`
`\refvar{Transform}{}` `\initvar{Perspective}{}`(`\refvar{Float}{}` fov, `\refvar{Float}{}` n, `\refvar{Float}{}` f) {
    `\refcode{Perform projective divide for perspective projection}{}`
    `\refcode{Scale canonical perspective view to specified field of view}{}`
}
\end{lstlisting}
该变换最容易理解,分两个步骤:
\begin{enumerate}
    \item 相机空间的点$\bm p$被投影到视平面上。
          一点代数计算证明视平面上投影后的$x'$和$y'$坐标可计算为$x$和$y$除以点的$z$坐标值。
          投影后的深度$z$被重新映射使近处平面的$z$值为0而远处平面的$z$值为1.
          我们要计算\sidenote{译者注:原本投影后的点都落在投影平面上,
          $z$坐标变为同一个定值,与$x,y$无关。但为了深度测试等用途,
          我们还是希望保留$z$的排序关系,所以为其指定一个与$x,y$无关的可逆映射。
          本文给出的是最常用的形式,其他渲染器可能只是具体系数有一些差别。}
          \begin{align*}
              x' & =\frac{x}{z}\, ,           \\
              y' & =\frac{y}{z}\, ,           \\
              z' & =\frac{f(z-n)}{z(f-n)}\, .
          \end{align*}
          整个该计算可用齐次坐标编码为$4\times4$矩阵:
          \begin{align*}
              \left[\begin{array}{cccc}
                      1 & 0 & 0             & 0               \\
                      0 & 1 & 0             & 0               \\
                      0 & 0 & \frac{f}{f-n} & -\frac{fn}{f-n} \\
                      0 & 0 & 1             & 0
                  \end{array}\right]
          \end{align*}
          \begin{lstlisting}
`\initcode{Perform projective divide for perspective projection}{=}`
`\refvar{Matrix4x4}{}` persp(1, 0,           0,              0,
                0, 1,           0,              0,
                0, 0, f / (f - n), -f*n / (f - n),
                0, 0,           1,              0);
\end{lstlisting}
    \item 用户指定的视场角({\ttfamily fov})通过缩放投影平面上的$(x,y)$值
          使得视场内的点投影到视平面上坐标$[-1,1]$内来实现。
          对于正方形图像,屏幕空间内$x$和$y$都在$[-1,1]$内。
          否则,图像更窄的那个方向映射到$[-1,1]$,
          更宽的方向映射到成比例的更大屏幕空间值范围。
          回想正切等于直角三角形对边与邻边之比。
          这里邻边长为1,所以对边长为$\tan\frac{\text{\ttfamily fov}}{2}$.
          用该长度倒数缩放将视场映射到$[-1,1]$内的范围。
\end{enumerate}
\begin{lstlisting}
`\initcode{Scale canonical perspective view to specified field of view}{=}`
`\refvar{Float}{}` invTanAng = 1 / std::tan(`\refvar{Radians}{}`(fov) / 2);
return `\refvar{Scale}{}`(invTanAng, invTanAng, 1) * `\refvar{Transform}{}`(persp);
\end{lstlisting}

类似于\refvar{OrthographicCamera}{},关于\refvar{PerspectiveCamera}{}生成的
相机光线如何随我们移动胶片平面上的像素而改变的信息可在构造函数中预先算出
\sidenote{译者注:注意栅格空间与相机空间之间的变换仍然是可以用齐次变换矩阵表示的
    线性变换,所以从栅格空间到相机空间的差分计算与起点和终点的位置无关,
    所以正文中取用$(0,0,0),(1,0,0),(0,1,0)$三点即可。}。
这里我们计算相机空间近处投影平面上的位置随像素位置移动而发生的变化。
\begin{lstlisting}
`\initcode{Compute differential changes in origin for perspective camera rays}{=}`
`\refvar[PerspectiveCamera::dxCamera]{dxCamera}{}` = (`\refvar{RasterToCamera}{}`(`\refvar{Point3f}{}`(1, 0, 0)) -
            `\refvar{RasterToCamera}{}`(`\refvar{Point3f}{}`(0, 0, 0)));
`\refvar[PerspectiveCamera::dyCamera]{dyCamera}{}` = (`\refvar{RasterToCamera}{}`(`\refvar{Point3f}{}`(0, 1, 0)) -
            `\refvar{RasterToCamera}{}`(`\refvar{Point3f}{}`(0, 0, 0)));
\end{lstlisting}
\begin{lstlisting}
`\initcode{PerspectiveCamera Private Data}{=}\initnext{PerspectiveCameraPrivateData}`
`\refvar{Vector3f}{}` `\initvar[PerspectiveCamera::dxCamera]{dxCamera}{}`, `\initvar[PerspectiveCamera::dyCamera]{dyCamera}{}`;
\end{lstlisting}

用透视投影时,所有光线都从相机空间原点$(0,0,0)$发出。
光线的方向由从原点指向近处平面上的点{\ttfamily pCamera}的向量给出,
该点对应提供的\refvar{CameraSample}{}的{\ttfamily pFilm}位置。
换句话说,该光线方向向量的每个分量等于该点的位置,
所以不需做无用减法来计算该方向,我们只需直接用点{\ttfamily pCamera}来初始化该方向。
\begin{lstlisting}
`\refcode{PerspectiveCamera Method Definitions}{+=}\lastnext{PerspectiveCameraMethodDefinitions}`
`\refvar{Float}{}` `\refvar{PerspectiveCamera}{}`::`\initvar[PerspectiveCamera::GenerateRay]{\refvar{GenerateRay}{}}{}`(const `\refvar{CameraSample}{}` &sample,
        `\refvar{Ray}{}` *ray) const {
    `\refcode{Compute raster and camera sample positions}{}`
    *ray = `\refvar{Ray}{}`(`\refvar{Point3f}{}`(0, 0, 0), `\refvar{Normalize}{}`(`\refvar{Vector3f}{}`(pCamera)));
    `\refcode{Modify ray for depth of field}{}`
    ray->`\refvar[Ray::time]{time}{}` = `\refvar{Lerp}{}`(sample.`\refvar[CameraSample::time]{time}{}`, `\refvar{shutterOpen}{}`, `\refvar{shutterClose}{}`);
    ray->`\refvar[Ray::medium]{medium}{}` = `\refvar[Camera::medium]{medium}{}`;
    *ray = `\refvar{CameraToWorld}{}`(*ray);
    return 1;
}
\end{lstlisting}

方法\refvar[PerspectiveCamera::GenerateRayDifferential]{GenerateRayDifferential}{()}遵
循\refvar[PerspectiveCamera::GenerateRay]{GenerateRay}{()}的实现,只是多了计算差分射线的代码片。
\begin{lstlisting}
`\initcode{PerspectiveCamera Public Methods}{=}`
`\refvar{Float}{}` `\initvar[PerspectiveCamera::GenerateRayDifferential]{\refvar{GenerateRayDifferential}{}}{}`(const `\refvar{CameraSample}{}` &sample,
                              `\refvar{RayDifferential}{}` *ray) const;
\end{lstlisting}
\begin{lstlisting}
`\initcode{Compute offset rays for PerspectiveCamera ray differentials}{=}`
if (`\refvar{lensRadius}{}` > 0) {
    `\refcode{Compute PerspectiveCamera ray differentials accounting for lens}{}`
} else {
    ray->`\refvar{rxOrigin}{}` = ray->`\refvar{ryOrigin}{}` = ray->`\refvar[Ray::o]{o}{}`;
    ray->`\refvar{rxDirection}{}` = `\refvar{Normalize}{}`(`\refvar{Vector3f}{}`(pCamera) + `\refvar[PerspectiveCamera::dxCamera]{dxCamera}{}`);
    ray->`\refvar{ryDirection}{}` = `\refvar{Normalize}{}`(`\refvar{Vector3f}{}`(pCamera) + `\refvar[PerspectiveCamera::dyCamera]{dyCamera}{}`);
}
\end{lstlisting}
\begin{lstlisting}
`\initcode{Compute PerspectiveCamera ray differentials accounting for lens}{=}`
`\refcode{Sample point on lens}{}`
`\refvar{Vector3f}{}` dx = `\refvar{Normalize}{}`(`\refvar{Vector3f}{}`(pCamera + `\refvar[PerspectiveCamera::dxCamera]{dxCamera}{}`));
`\refvar{Float}{}` ft = `\refvar{focalDistance}{}` / dx.z;
`\refvar{Point3f}{}` pFocus = `\refvar{Point3f}{}`(0, 0, 0) + (ft * dx);
ray->`\refvar{rxOrigin}{}` = `\refvar{Point3f}{}`(pLens.x, pLens.y, 0);
ray->`\refvar{rxDirection}{}` = `\refvar{Normalize}{}`(pFocus - ray->`\refvar{rxOrigin}{}`);

`\refvar{Vector3f}{}` dy = `\refvar{Normalize}{}`(`\refvar{Vector3f}{}`(pCamera + `\refvar[PerspectiveCamera::dyCamera]{dyCamera}{}`));
ft = `\refvar{focalDistance}{}` / dy.z;
pFocus = `\refvar{Point3f}{}`(0, 0, 0) + (ft * dy);
ray->`\refvar{ryOrigin}{}` = `\refvar{Point3f}{}`(pLens.x, pLens.y, 0);
ray->`\refvar{ryDirection}{}` = `\refvar{Normalize}{}`(pFocus - ray->`\refvar{ryOrigin}{}`);
\end{lstlisting}

\subsection{薄透镜模型与景深}\label{sub:薄透镜模型与景深}
只允许光线穿过单个点后到达胶片的理想针孔相机是不可物理实现的;
尽管可以用具有极小\keyindex{光圈}{aperture}{}的相机近似该情况,但小光圈只允许相对很少的光到达胶片传感器。
用小光圈时需要较长曝光时间来捕获足够的光子以准确获取图像,
这反过来会导致场景中在相机快门开启时正运动的物体变模糊。

真实相机具有透镜系统,它将光聚焦穿过大小有限的光圈后打到胶片平面上。
相机设计者(以及使用可调光圈相机的摄影师)面临权衡:
光圈越大,到达胶片的光越多,需要的曝光越短。
然而,透镜只能聚焦于单个平面(\keyindex{焦平面}{focal plane}{}),
场景中的物体离该平面越远就越模糊。
光圈越大,该效应就越显著:处在与透镜系统聚焦深度不同处的物体变得越发模糊。

\refsec{逼真相机}的相机模型对真实相机中的透镜系统实现了相当准确的模拟。
对于目前介绍的简单相机模型,我们可以运用光学中的经典近似,\keyindex{薄透镜近似}{thin lens approximation}{},
借助传统计算机图形学投影模型来对有限光圈的效应建模。
薄透镜近似将光学系统建模为球形剖面的单个透镜,
该透镜的厚度相比于透镜的\keyindex{曲率半径}{radius of curvature}{}很小
(\refsub{厚透镜近似}介绍更一般的厚透镜近似,它假设透镜的厚度是不可忽略的)。

在薄透镜近似下,平行于\keyindex{光轴}{optical axis}{}的入射光线穿过透镜后
聚在透镜后方称为\keyindex{焦点}{focal point}{}的点上。
若胶片平面置于透镜后方距离等于\keyindex{焦距}{focal length}{}的地方,
则无限远的物体将对好焦,它们成像为胶片上的单个点。

\reffig{6.6}展示了基本设置。这里我们遵循典型透镜坐标系统的惯例
让透镜垂直于$z$轴并位于$z=0$,场景在$-z$
(注意这个坐标系统和我们给相机空间用的看向$+z$的不同)。
透镜场景一侧的距离记为无撇的变量$z$,透镜胶片一侧的距离(正的$z$)记为有撇的$z'$.
\begin{figure}[htbp]
    \centering\includegraphics[width=0.6\linewidth]{chap06/Thinlens.eps}
    \caption{沿$z$轴置于$z=0$处的薄透镜。平行于光轴的入射光线穿过薄透镜后(虚线)
        均通过点$\bm p$,即焦点。透镜与焦点的距离$f$为透镜的焦距。}
    \label{fig:6.6}
\end{figure}

对于场景中与焦距为$f$的薄透镜距离为$z$的点
\sidenote{译者注:注意这里$z<0$.},
\keyindex{高斯透镜方程}{Gaussian lens equation}{}将
物体与透镜的距离和透镜与像点的距离联系起来
\sidenote{译者注:高斯透镜方程的推导过程可参考译者补充的\refsub{透镜}。}:
\begin{align}\label{eq:6.1}
    \frac{1}{z'}-\frac{1}{z}=\frac{1}{f}\, .
\end{align}
注意对于$z=-\infty$,正如预期那样我们有$z'=f$.

我们可以用高斯透镜方程求解透镜与胶片的距离即\keyindex{对焦距离}{focal distance}{}
\sidenote{译者注:笔者查阅到的一些资料中“对焦距离”的含义与本书不同,
    故此处“focal distance”译为“对焦距离”是折中做法。
    有可能学界对这一概念的定义并不统一。
    此外译者认为代码中的用法与此处的定义并不完全对应,
    详见\url{https://github.com/mmp/pbrt-v4/issues/232}。欢迎读者提供帮助。}
来为某个$z$设置焦平面(\reffig{6.7}):
\begin{align}\label{eq:6.2}
    z'=\frac{fz}{f+z}\, .
\end{align}

\begin{figure}[htbp]
    \centering\includegraphics[width=0.6\linewidth]{chap06/Focusthinlens.eps}
    \caption{为了让薄透镜聚焦场景中的深度$z$处,
        \refeq{6.2}可用于计算$z$处的点聚焦到的透镜胶片一侧的距离$z'$.
        通过调整透镜和胶片平面的距离执行对焦。}
    \label{fig:6.7}
\end{figure}

不在焦平面上的点在胶片平面上成像为圆斑,而不是单个点。
该斑的边界称为\keyindex{弥散圆}{circle of confusion}{}。
弥散圆的大小受到光线穿过的光圈\keyindex{直径}{diameter}{}、
对焦距离以及物体与透镜间距离的影响。
\reffig{6.8}展示了具有一系列龙模型副本的场景中的这种效应,即景深。
随着透镜光圈尺寸增大,离焦平面越远的点模糊度越大。
注意在所有图像中从右边数第二个龙都保持对准焦,
因为焦平面就放置在其深度处。
\begin{figure}[htbp]
    \centering
    \subfloat[无景深]{\includegraphics[width=0.85\linewidth]{chap06/dragons-nodof.png}\label{fig:6.8.1}}\\
    \subfloat[小光圈]{\includegraphics[width=0.85\linewidth]{chap06/dragons-small-dof.png}\label{fig:6.8.2}}\\
    \subfloat[中光圈]{\includegraphics[width=0.85\linewidth]{chap06/dragons-med-dof.png}\label{fig:6.8.3}}\\
    \subfloat[大光圈]{\includegraphics[width=0.85\linewidth]{chap06/dragons-large-dof.png}\label{fig:6.8.4}}
    \caption{(a)无景深渲染场景,(b)相对较小透镜光圈导致的景深,让失焦区域只有少量模糊,
        (c)和(d)中随着透镜光圈尺寸增大,失焦区域弥散圆的尺寸也增大,使胶片平面上的模糊程度更大。}
    \label{fig:6.8}
\end{figure}

\reffig{6.9}展示了用于渲染景观场景的景深。
注意该效应是如何把观察者的视线吸引到图像中间合焦的小草上的。
\begin{figure}[htbp]
    \centering\includegraphics[width=\linewidth]{chap06/landscape-dof.png}
    \caption{景深赋予这部分景观场景很强的纵深感({\itshape 感谢Laubwerk提供场景})。}
    \label{fig:6.9}
\end{figure}

实践中,物体不必刚好在焦平面上才能清晰合焦显示;
只要弥散圆大致小于胶片传感器上的一个像素,物体就会清晰显示。
能让物体清晰显示的到透镜的距离范围称为透镜的\keyindex{景深}{depth of field}{}。

高斯透镜方程也让我们能计算弥散圆大小;
令焦距为$f$的透镜聚焦距离$z_{\mathrm{f}}$处,胶片平面位于$z'_{\mathrm{f}}$处。
给定另一个在深度$z$处的点,高斯透镜方程给出到透镜对焦的点的距离$z'$.
该点要么在胶片平面前方,要么在后方;\reffig{6.10}(a)展示了在后方的情况。
\begin{figure}[htbp]
    \centering\includegraphics[width=0.7\linewidth]{chap06/Circleofconfusiondiameter.eps}
    \caption{(a)若焦距为$f$的薄透镜对焦某个深度$z_{\mathrm{f}}$,
    则高斯透镜方程给出从透镜到胶片平面的距离为$z'_{\mathrm{f}}$.
    场景中在深度$z\neq z_{\mathrm{f}}$处的点将被投影为胶片平面上的圆;
    这里$z$对焦到胶片平面之后的$z'$.(b)为了计算弥散圆的直径,
    我们可应用相似三角形:透镜直径$d_{\mathrm{l}}$与$z'$之比一定和
    弥散圆直径$d_{\mathrm{c}}$与$z'-z'_{\mathrm{f}}$之比相同。}
    \label{fig:6.10}
\end{figure}

弥散圆直径由$z'$与透镜间的锥体和胶片平面相交给出。
如果我们知道透镜直径$d_{\mathrm{l}}$,则我们可用
相似三角形来求解弥散圆的直径$d_{\mathrm{c}}$(\reffig{6.10}(b)):
\begin{align*}
    \frac{d_{\mathrm{l}}}{z'}=\frac{d_{\mathrm{c}}}{|z'-z'_{\mathrm{f}}|}\, .
\end{align*}
求解$d_{\mathrm{c}}$,我们有
\begin{align*}
    d_{\mathrm{c}}=\left|\frac{d_{\mathrm{l}}(z'-z'_{\mathrm{f}})}{z'}\right|\, .
\end{align*}
运用高斯透镜方程以场景深度来表示结果,我们可得
\begin{align*}
    d_{\mathrm{c}}=\left|\frac{d_{\mathrm{l}}f(z-z_{\mathrm{f}})}{z(f+z_{\mathrm{f}})}\right|\, .
\end{align*}
注意弥散圆直径正比于透镜直径。
透镜直径常表示为透镜的\keyindex{F值}{$f$-number}{}
\sidenote{译者注:也称焦比、光圈系数。}$n$,
它将直径表示为焦距的分数,$\displaystyle d_{\mathrm{l}}=\frac{f}{n}$.

\reffig{6.11}展示了50mm焦距和25mm光圈的透镜对焦$z_{\mathrm{f}}=1$m时该函数的图像。
注意模糊度关于焦平面两边的深度是不对称的,
且焦平面前方物体比后方增长得快得多。
\begin{figure}[htbp]
    \centering\input{Pictures/chap06/circle-confusion-diameter.tex}
    \caption{50mm焦距25mm光圈的透镜对焦1米处时弥散圆直径关于深度的函数。}
    \label{fig:6.11}
\end{figure}

在光线追踪器中建模薄透镜非常简单:只需要选择透镜上一点
并找到起始于透镜上该点的合适光线使得焦平面上的物体能在胶片上合焦(\reffig{6.12})。
因此,投影相机为景深接收两个额外参数:一个设置透镜光圈大小,另一个设置对焦距离。
\begin{figure}[htbp]
    \centering\includegraphics[width=0.4\linewidth]{chap06/Thinlenschooseray.eps}
    \caption{(a)对于针孔相机模型,胶片平面上每个点(实心圆)
        都关联单条相机光线,该点由穿过针孔透镜单个点(空心圆)的光线给出。
        (b)对于具有有限光圈的相机模型,我们在圆盘形透镜上为每条光线采样一点(实心圆)。
        然后我们计算穿过透镜中心(对应于针孔模型)的光线以及它与焦平面(实线)相交的点。
        我们知道无论透镜样本位置在哪里,所有在焦平面上的物体都一定对准焦。
        因此,对应于透镜位置样本的光线(虚线)由起始于透镜样本点并穿过算出的焦平面上交点的光线给出。}
    \label{fig:6.12}
\end{figure}

\begin{lstlisting}
`\refcode{ProjectiveCamera Protected Data}{+=}\lastcode{ProjectiveCameraProtectedData}`
`\refvar{Float}{}` `\initvar{lensRadius}{}`, `\initvar{focalDistance}{}`;
\end{lstlisting}
\begin{lstlisting}
`\initcode{Initialize depth of field parameters}{=}`
`\refvar{lensRadius}{}` = lensr;
`\refvar{focalDistance}{}` = focald;
\end{lstlisting}

通常每个图像像素都需要追踪许多光线以充分采样透镜得到平滑景深。
\reffig{6.13}展示了来自\reffig{6.9}的每个像素只有四个样本的景观场景
(\reffig{6.9}每个像素有2048个)。
\begin{figure}[htbp]
    \centering\includegraphics[width=\linewidth]{chap06/landscape-dof-4spp.png}
    \caption{每个像素只有样本的景深景观场景:景深是欠采样的且图像呈颗粒状({\itshape 感谢Laubwerk提供场景})。}
    \label{fig:6.13}
\end{figure}

\begin{lstlisting}
`\initcode{Modify ray for depth of field}{=}`
if (`\refvar{lensRadius}{}` > 0) {
    `\refcode{Sample point on lens}{}`
    `\refcode{Compute point on plane of focus}{}`
    `\refcode{Update ray for effect of lens}{}`
}
\end{lstlisting}

第\refchap{蒙特卡罗积分}定义的函数\refvar{ConcentricSampleDisk}{()}接收
$[0,1)^2$中的样本位置$(u,v)$并将其映射到中心位于原点$(0,0)$处的2D单位圆盘。
为了将其转化为透镜上的一点,用透镜半径缩放这些坐标。
类\refvar{CameraSample}{}在成员变量\refvar{pLens}{}中提供透镜采样参数$(u,v)$.
\begin{lstlisting}
`\initcode{Sample point on lens}{=}`
`\refvar{Point2f}{}` pLens = `\refvar{lensRadius}{}` * `\refvar{ConcentricSampleDisk}{}`(sample.`\refvar{pLens}{}`);
\end{lstlisting}

光线的起点即透镜上的该点。现在需要为新光线确定合适的方向。
我们知道来自该给定图像样本并穿过透镜的\emph{所有}光线一定都
汇聚于焦平面上的同一点。 而且我们知道穿过透镜中心的光线不改变方向,
所以寻找合适的汇聚点就是将针孔模型中未受扰动的光线与焦平面相交
然后将新光线的方向设为从透镜上的点指向该交点的向量。

对于该简单模型,焦平面垂直于$z$轴且光线起始于原点,
所以让穿过透镜中心的光线与焦平面相交很简单。
相交处$t$值为
\begin{align*}
    t=\frac{\text{\ttfamily focalDistance}}{{\bm d}_z}\, .
\end{align*}
\begin{lstlisting}
`\initcode{Compute point on plane of focus}{=}`
`\refvar{Float}{}` ft = `\refvar{focalDistance}{}` / ray->`\refvar[Ray::d]{d}{}`.z;
`\refvar{Point3f}{}` pFocus = (*ray)(ft);
\end{lstlisting}

现在可以初始化光线了。端点设为透镜上的采样点,
而方向设置使得光线穿过焦平面上的点{\ttfamily pFocus}。
\begin{lstlisting}
`\initcode{Update ray for effect of lens}{=}`
ray->`\refvar[Ray::o]{o}{}` = `\refvar{Point3f}{}`(pLens.x, pLens.y, 0);
ray->`\refvar[Ray::d]{d}{}` = `\refvar{Normalize}{}`(pFocus - ray->`\refvar[Ray::o]{o}{}`);
\end{lstlisting}

为了计算薄透镜的光线差分,代码片\refcode{Update ray for effect of lens}{}中
用的方法也用到在胶片平面上于$x$和$y$方向偏移一像素的光线上。
这里不再介绍此处实现的代码片\refcode{Compute OrthographicCamera ray differentials accounting for lens}{}和
\refcode{Compute PerspectiveCamera ray differentials accounting for lens}{}了
\sidenote{译者注:我补充回来了。}。

\input{content/chap0603.tex}

\section{逼真相机}\label{sec:逼真相机}
\begin{remark}
    本节含有高级内容,第一次阅读时可以跳过。
\end{remark}

薄透镜模型使得能渲染因景深而模糊的图像,
但它只是对多个\keyindex{透镜元件}{lens element}{lens透镜}构成的
真实相机透镜系统非常粗糙的近似,而每个透镜元件都会改变穿过它的辐射分布
(\reffig{6.15}展示了具有8个元件的22mm焦距\keyindex{广角}{wide-angle}{}镜头横截面)。
即使基本的手机相机也趋于有五个左右独立的透镜元件,
而\keyindex{数码单镜头反光相机}{digital single-lens reflex camera}{camera相机}
(数码单反相机,DSLR)镜头可能有十个或更多。
通常,具备更大数量透镜元件的更复杂透镜系统能
比更简单的透镜系统创建更高质量的图像。
\begin{figure}[htbp]
    \centering\input{Pictures/chap06/wide22-cross-section.tex}
    \caption{广角透镜系统的横截面(在pbrt发行版的{\ttfamily scenes/lenses/wide.22mm.dat}
    中)。透镜坐标系统让胶片平面垂直于$z$轴且位于$z=0$处。
    透镜在左边负z轴上,然后场景在透镜左侧。透镜系统中部表示为粗黑线的光圈阻挡命中它的光线。
    在许多透镜系统中,可以调整光圈大小以在更短曝光时间(大光圈)和更大景深(小光圈)间权衡。}
    \label{fig:6.15}
\end{figure}

本节讨论\refvar{RealisticCamera}{}的实现,
它模拟光穿过像\reffig{6.15}那样的透镜系统后聚焦并渲染像\reffig{6.16}那样的图像。
其实现基于光线追踪,即相机追随光路穿过透镜元件,
并考虑具有不同折射率的介质(空气,各类玻璃)间界面的折射,
直到光路要么射出光学系统要么被光圈或镜头罩吸收。
离开前端镜头元件的光线代表相机响应曲线,可用于估计
沿任意光线入射辐亮度的积分器,例如\refvar{SamplerIntegrator}{}。
\refvar{RealisticCamera}{}的实现在文件\href{https://github.com/mmp/pbrt-v3/tree/master/src/cameras/realistic.h}{\ttfamily cameras/realistic.h}
和\href{https://github.com/mmp/pbrt-v3/tree/master/src/cameras/realistic.cpp}{\ttfamily cameras/realistic.cpp}中。
\begin{figure}[htbp]
    \centering\includegraphics[width=0.6\linewidth]{chap06/sanmiguel-fisheye.png}
    \caption{用鱼眼透镜和很宽视场渲染的图像。注意边缘暗处是
        准确模拟成像辐射度量(\refsub{相机测量方程})所致,
        而直线扭曲为曲线则是许多广角镜头的特点,但在用投影矩阵表示透镜投影模型时没有考虑。}
    \label{fig:6.16}
\end{figure}
\begin{lstlisting}
`\initcode{RealisticCamera Declarations}{=}`
class `\initvar{RealisticCamera}{}` : public `\refvar{Camera}{}` {
public:
    `\refcode{RealisticCamera Public Methods}{}`
private:
    `\refcode{RealisticCamera Private Declarations}{}`
    `\refcode{RealisticCamera Private Data}{}`
    `\refcode{RealisticCamera Private Methods}{}`
};
\end{lstlisting}

除了把相机放置于场景中的常见变换、\refvar{Film}{}以及快门打开和关闭的时间外,
\refvar{RealisticCamera}{}构造函数还接收透镜系统描述文件的文件名、
到期望的焦平面的距离以及光圈直径。之后有了第\refchap{蒙特卡罗积分}蒙特卡罗积分与
\refsub{相机测量方程}成像辐射度量的预备知识后,
将在\refsub{采样相机1}介绍参数{\ttfamily simpleWeighting}的作用。
\begin{lstlisting}
`\initcode{RealisticCamera Method Definitions}{=}\initnext{RealisticCameraMethodDefinitions}`
`\refvar{RealisticCamera}{}`::`\refvar{RealisticCamera}{}`(const `\refvar{AnimatedTransform}{}` &CameraToWorld,
        `\refvar{Float}{}` shutterOpen, `\refvar{Float}{}` shutterClose, `\refvar{Float}{}` apertureDiameter,
        `\refvar{Float}{}` focusDistance, bool simpleWeighting, const char *lensFile,
        `\refvar{Film}{}` *film, const `\refvar{Medium}{}` *medium)
    : `\refvar{Camera}{}`(CameraToWorld, shutterOpen, shutterClose, film, medium),
      simpleWeighting(simpleWeighting) {
    `\refcode{Load element data from lens description file}{}`
    `\refcode{Compute lens-film distance for given focus distance}{}`
    `\refcode{Compute exit pupil bounds at sampled points on the film}{}`
}
\end{lstlisting}
\begin{lstlisting}
`\initcode{Load element data from lens description file}{=}`
std::vector<`\refvar{Float}{}`> lensData;
if (ReadFloatFile(lensFile, &lensData) == false) {
    `\refvar{Error}{}`("Error reading lens specification file \"%s\".", lensFile);
    return;
}
if ((lensData.size() % 4) != 0) {
    `\refvar{Error}{}`("Excess values in lens specification file \"%s\"; "
          "must be multiple-of-four values, read %d.",
          lensFile, (int)lensData.size());
    return;
}
for (int i = 0; i < (int)lensData.size(); i += 4) {
    if (lensData[i] == 0) {
        if (apertureDiameter > lensData[i+3]) {
            `\refvar{Warning}{}`("Specified aperture diameter %f is greater than maximum "
                    "possible %f.  Clamping it.", apertureDiameter, lensData[i+3]);
        } else {
            lensData[i+3] = apertureDiameter;
        }
    }
    `\refvar{elementInterfaces}{}`.push_back((`\refvar{LensElementInterface}{}`)
        {lensData[i] * (`\refvar{Float}{}`).001, lensData[i+1] * (`\refvar{Float}{}`).001, lensData[i+2],
         lensData[i+3] * `\refvar{Float}{}`(.001) / `\refvar{Float}{}`(2.)});
}
\end{lstlisting}
\begin{lstlisting}
`\initcode{RealisticCamera Private Data}{=}\initnext{RealisticCameraPrivateData}`
const bool `\initvar{simpleWeighting}{}`;
\end{lstlisting}

在从磁盘加载透镜描述文件之后,构造函数调整透镜与
胶片间的距离使得焦平面位于期望的深度即{\ttfamily focusDistance},
然后预先计算一些关于离胶片最近透镜元件的哪部分面积让光从场景射到胶片的信息,
就像在胶片平面上各点看到的那样。在介绍完背景材料之后,
\refsub{对焦}和\refsub{出射瞳}将分别定义代码片
\refcode{Compute lens-film distance for given focus distance}{}
和\refcode{Compute exit pupil bounds at sampled points on the film}{}。

\subsection{透镜系统表示}\label{sub:透镜系统表示}
透镜系统由一系列透镜元件组成,每个元件通常是某种形制的玻璃。
透镜系统设计者的挑战是在有限空间、成本和生产难度下
设计一组能在胶片或传感器上高质量成像的元件
(例如为了让保持手机变薄,其相机厚度非常有限)。

最容易生产的是横截面为球形的透镜,
透镜系统通常是绕\keyindex{光轴}{optical axis}{}对称的,习惯记为$z$.
我们将假设这两个性质在本节下文中成立。
用胶片对齐到平面$z=0$且透镜在胶片左侧沿$-z$轴放置的坐标系统定义透镜系统。

透镜系统常表示为独立透镜元件(或空气)间的一系列界面,
而不是每个元件的显式表示。\reftab{6.1}展示了定义每个界面的量。
表中最后一项定义了最右边的界面,如\reffig{6.17}所示:
它是个半径等于曲率半径的球体块。元件的厚度是沿$z$到右边下一个
元件(或胶片平面)的距离,\keyindex{折射率}{index of refraction}{}是
对界面右边的介质而言的。元件在$z$轴上下的范围由光圈直径设置。
\begin{table}[htbp]
    \centering
    \begin{tabular}{SSSS}
        \toprule
        \ \ \ \ \textbf{曲率半径} & \ \ \ \ \textbf{厚度} & \ \ \textbf{折射率} & \textbf{光圈直径} \\
        \midrule
        35.98738                  & 1.21638               & 1.54                & 23.716            \\
        11.69718                  & 9.9957                & 1                   & 17.996            \\
        13.08714                  & 5.12622               & 1.772               & 12.364            \\
        -22.63294                 & 1.76924               & 1.617               & 9.812             \\
        71.05802                  & 0.8184                & 1                   & 9.152             \\
        0                         & 2.27766               & 0                   & 8.756             \\
        -9.58584                  & 2.43254               & 1.617               & 8.184             \\
        -11.28864                 & 0.11506               & 1                   & 9.152             \\
        -166.7765                 & 3.09606               & 1.713               & 10.648            \\
        -7.5911                   & 1.32682               & 1.805               & 11.44             \\
        -16.7662                  & 3.98068               & 1                   & 12.276            \\
        -7.70286                  & 1.21638               & 1.617               & 13.42             \\
        -11.97328                 & (取决于焦点)        & 1                   & 17.996            \\
        \bottomrule
    \end{tabular}
    \caption{\reffig{6.15}中透镜系统的表格化描述。每行描述了两个透镜元件间的界面、
        元件与空气间的界面或者光圈。第一行描述了最左边的界面。半径为0的元件对应光圈。
        距离单位为mm。}
    \label{tab:6.1}
\end{table}
\begin{figure}[htbp]
    \centering\includegraphics[width=0.6\linewidth]{chap06/Lenselement.eps}
    \caption{透镜界面(实曲线)与光轴相交于位置$z$.界面几何形状由
        表示其在光轴上下方范围的光圈半径以及元件的曲率半径$r$描述。
        如果元件有球形横截面,则它的轮廓由球心在光轴上距离$r$的球体给定,
        该球体也穿过$z$.如果$r$是负的,则元件界面就如从场景中看到那样是凹的
        (如图所示);否则就是\protect\keyindex{凸}{convex}{}的。透镜厚度给出了到
        右边下一个界面的距离,或者对于最右边的界面是到胶片平面的距离。}
    \label{fig:6.17}
\end{figure}

结构体\refvar{LensElementInterface}{}表示单个透镜元件界面。
\begin{lstlisting}
`\initcode{RealisticCamera Private Declarations}{=}`
struct `\initvar{LensElementInterface}{}` {
    `\refvar{Float}{}` `\initvar{curvatureRadius}{}`;
    `\refvar{Float}{}` `\initvar{thickness}{}`;
    `\refvar{Float}{}` `\initvar[LensElementInterface::eta]{eta}{}`;
    `\refvar{Float}{}` `\initvar{apertureRadius}{}`;
};
\end{lstlisting}

这里没有介绍的代码片\refcode{Load element data from lens description file}{}
\sidenote{译者注:我补充回来了。}读取透镜元件
并初始化数组\refvar[elementInterfaces]{RealisticCamera::elementInterfaces}{}。
见源代码中的注释了解该文件格式的细节,它并行化\reftab{6.1}中的结构,
并见pbrt发行版中的目录{\ttfamily scenes/lenses}了解大量透镜描述示例。

对从文件读取的值做了两个调整:第一,透镜系统传统上用毫米单位描述,
但pbrt假设场景单位用米。因此,除了折射率外的域都按1/1000缩小。
第二,元件直径被除以二;在下面的代码中半径是用起来更方便的量。
\begin{lstlisting}
`\refcode{RealisticCamera Private Data}{+=}\lastnext{RealisticCameraPrivateData}`
std::vector<`\refvar{LensElementInterface}{}`> `\initvar{elementInterfaces}{}`;
\end{lstlisting}

加载完透镜界面描述后,让一些关于透镜系统的值随时可得是很有用的。
\refvar{LensRearZ}{()}和\refvar{LensFrontZ}{()}分别返回
透镜系统尾部和头部元件的$z$深度\sidenote{译者注:靠近胶片的是尾部,远离胶片的是头部。}。
注意返回的$z$深度在相机空间中,而不是透镜空间中,所以为正值。
\begin{lstlisting}
`\initcode{RealisticCamera Private Methods}{=}\initnext{RealisticCameraPrivateMethods}`
`\refvar{Float}{}` `\initvar{LensRearZ}{}`() const {
    return `\refvar{elementInterfaces}{}`.back().`\refvar{thickness}{}`;
}
\end{lstlisting}

求头部元件$z$位置需要求所有元件厚度之和(见\reffig{6.18})。
任何位于系统性能敏感部分的代码都不需要该值,
所以在需要时重算它就行。如果该方法对性能有影响,
最好还是在\refvar{RealisticCamera}{}中缓存该值。
\begin{figure}[htbp]
    \centering\includegraphics[width=0.4\linewidth]{chap06/Elementthicknessandposition.eps}
    \caption{元件厚度与光轴上位置的关系。胶片平面位于$z=0$,尾部元件的厚度$t_3$给出
        了从胶片到其界面的距离;这里尾部界面与轴交于$z=-t_3$.下一个元件厚度为$t_2$且
        位于$z=-t_3-t_2$,以此类推。头部元件交$z$轴于$\sum_i-t_i$.}
    \label{fig:6.18}
\end{figure}
\begin{lstlisting}
`\refcode{RealisticCamera Private Methods}{+=}\lastnext{RealisticCameraPrivateMethods}`
`\refvar{Float}{}` `\initvar{LensFrontZ}{}`() const {
    `\refvar{Float}{}` zSum = 0;
    for (const `\refvar{LensElementInterface}{}` &element : `\refvar{elementInterfaces}{}`)
        zSum += element.`\refvar{thickness}{}`;
    return zSum;
}
\end{lstlisting}

\refvar{RearElementRadius}{()}按单位米返回尾部元件光圈半径。
\begin{lstlisting}
`\refcode{RealisticCamera Private Methods}{+=}\lastnext{RealisticCameraPrivateMethods}`
`\refvar{Float}{}` `\initvar{RearElementRadius}{}`() const {
    return `\refvar{elementInterfaces}{}`.back().`\refvar{apertureRadius}{}`;
}
\end{lstlisting}
\subsection{追踪穿过透镜的光线}\label{sub:追踪穿过透镜的光线}
给定起始于透镜系统胶片一侧的光线,\refvar{TraceLensesFromFilm}{()}依次
计算与每个元件的相交处,如果其路径在穿过透镜系统途中被挡住了就终结该光线并返回{\ttfamily false}。
否则它就返回{\ttfamily true}并用相机空间中退出的光线来初始化{\ttfamily *rOut}。
在遍历时,{\ttfamily elementZ}追踪当前透镜元件的$z$截距。
因为光线起始于胶片,所以按照和\refvar{elementInterfaces}{}存储的相反顺序遍历透镜。
\begin{lstlisting}
`\refcode{RealisticCamera Method Definitions}{+=}\lastnext{RealisticCameraMethodDefinitions}`
bool `\refvar{RealisticCamera}{}`::`\initvar{TraceLensesFromFilm}{}`(const `\refvar{Ray}{}` &rCamera,
        `\refvar{Ray}{}` *rOut) const {
    `\refvar{Float}{}` elementZ = 0;
    `\refcode{Transform rCamera from camera to lens system space}{}`
    for (int i = `\refvar{elementInterfaces}{}`.size() - 1; i >= 0; --i) {
        const `\refvar{LensElementInterface}{}` &element = `\refvar{elementInterfaces}{}`[i];
        `\refcode{Update ray from film accounting for interaction with element}{}`
    }
    `\refcode{Transform rLens from lens system space back to camera space}{}`
    return true;
}
\end{lstlisting}

因为在pbrt的相机空间中相机指向$+z$轴但透镜在$-z$轴,
所以射线端点和方向的$z$分量需要取反。
尽管这是个简单到可以直接施加的变换,
我们还是偏好用显式的\refvar{Transform}{}使目的更明确。
\begin{lstlisting}
`\initcode{Transform rCamera from camera to lens system space}{=}`
static const `\refvar{Transform}{}` CameraToLens = `\refvar{Scale}{}`(1, 1, -1);
`\refvar{Ray}{}` rLens = CameraToLens(rCamera);
\end{lstlisting}

回想\reffig{6.18}中怎样计算元件的$z$截距:
因为我们从后往前访问元件,所以在考虑该元件的作用前
必须从{\ttfamily elementZ}中减去元件的厚度来计算其$z$截距。
\begin{lstlisting}
`\initcode{Update ray from film accounting for interaction with element}{=}`
elementZ -= element.`\refvar{thickness}{}`;
`\refcode{Compute intersection of ray with lens element}{}`
`\refcode{Test intersection point against element aperture}{}`
`\refcode{Update ray path for element interface interaction}{}`
\end{lstlisting}

有了元件的$z$轴截距,下一步是计算沿光线与元件界面(或光圈平面)相交处的参数值$t$.
对于光圈,采用光线-平面测试(见\refsub{光线-边界相交})。
对于球形界面,\refvar{IntersectSphericalElement}{()}执行该测试
并且如果找到相交处则还返回曲面法线;计算折射光方向时将需要该法线。
\begin{lstlisting}
`\initcode{Compute intersection of ray with lens element}{=}`
`\refvar{Float}{}` t;
`\refvar{Normal3f}{}` n;
bool isStop = (element.`\refvar{curvatureRadius}{}` == 0);
if (isStop)
    t = (elementZ - rLens.`\refvar[Ray::o]{o}{}`.z) / rLens.`\refvar[Ray::d]{d}{}`.z;
else {
    `\refvar{Float}{}` radius = element.`\refvar{curvatureRadius}{}`;
    `\refvar{Float}{}` zCenter = elementZ + element.`\refvar{curvatureRadius}{}`;
    if (!`\refvar{IntersectSphericalElement}{}`(radius, zCenter, rLens, &t, &n))
        return false;
}
\end{lstlisting}

方法\refvar{IntersectSphericalElement}{()}大致
和\refvar{Sphere::Intersect}{()}一样,不过它专门针对
元件中心在$z$轴上(且因此中心的$x$和$y$分量为零)这一情况。
这里文中没有包含代码片\refcode{Compute t0 and t1 for ray-element intersection}{}
和\refcode{Compute surface normal of element at ray intersection point}{},因为它们和
\refvar{Sphere::Intersect}{()}的实现一样\sidenote{译者注:我补充回来了。}。
\begin{lstlisting}
`\refcode{RealisticCamera Method Definitions}{+=}\lastnext{RealisticCameraMethodDefinitions}`
bool `\refvar{RealisticCamera}{}`::`\initvar{IntersectSphericalElement}{}`(`\refvar{Float}{}` radius,
        `\refvar{Float}{}` zCenter, const `\refvar{Ray}{}` &ray, `\refvar{Float}{}` *t, `\refvar{Normal3f}{}` *n) {
    `\refcode{Compute t0 and t1 for ray-element intersection}{}`
    `\refcode{Select intersection  based on ray direction and element curvature}{}`
    `\refcode{Compute surface normal of element at ray intersection point}{}`
    return true;
}
\end{lstlisting}
\begin{lstlisting}
`\initcode{Compute t0 and t1 for ray-element intersection}{=}`
`\refvar{Point3f}{}` o = ray.`\refvar[Ray::o]{o}{}` - `\refvar{Vector3f}{}`(0, 0, zCenter);
`\refvar{Float}{}` A = ray.`\refvar[Ray::d]{d}{}`.x*ray.`\refvar[Ray::d]{d}{}`.x + ray.`\refvar[Ray::d]{d}{}`.y*ray.`\refvar[Ray::d]{d}{}`.y + ray.`\refvar[Ray::d]{d}{}`.z*ray.`\refvar[Ray::d]{d}{}`.z;
`\refvar{Float}{}` B = 2 * (ray.`\refvar[Ray::d]{d}{}`.x*o.x + ray.`\refvar[Ray::d]{d}{}`.y*o.y + ray.`\refvar[Ray::d]{d}{}`.z*o.z);
`\refvar{Float}{}` C = o.x*o.x + o.y*o.y + o.z*o.z - radius*radius;
`\refvar{Float}{}` t0, t1;
if (!`\refvar{Quadratic}{}`(A, B, C, &t0, &t1))
    return false;
\end{lstlisting}
\begin{lstlisting}
`\initcode{Compute surface normal of element at ray intersection point}{=}`
*n = `\refvar{Normal3f}{}`(`\refvar{Vector3f}{}`(o + *t * ray.`\refvar[Ray::d]{d}{}`));
*n = `\refvar{Faceforward}{}`(`\refvar{Normalize}{}`(*n), -ray.`\refvar[Ray::d]{d}{}`);
\end{lstlisting}

然而这里在选择返回哪个交点时有个微妙之处\sidenote{译者注:原文subtlety。}:
$t>0$的最近相交处不一定在元件界面上;
见\reffig{6.19}\footnote{“微妙之处”(subtlety)一般意味着作者花费好几个小时来调试它。}。
例如,对于自场景中接近并与(具有负曲率半径的)凹透镜相交的光线,
两个相交处中不管近处那个是否有$t>0$都该返回远处那个。
幸运的是,基于光线方向和曲率半径的简单逻辑可指明用哪个$t$值。
\begin{figure}[htbp]
    \centering\includegraphics[width=0.5\linewidth]{chap06/Lenscorrectintersection.eps}
    \caption{当计算光线与球形透镜元件的相交处时,光线与整球的首个相交处不一定是我们想要的。
        这里,第二个相交处才在真正的元件界面(粗线)上,而第一个应该被忽略。}
    \label{fig:6.19}
\end{figure}
\begin{lstlisting}
`\initcode{Select intersection  based on ray direction and element curvature}{=}`
bool useCloserT = (ray.`\refvar[Ray::d]{d}{}`.z > 0) ^ (radius < 0);
*t = useCloserT ? std::min(t0, t1) : std::max(t0, t1);
if (*t < 0)
    return false;
\end{lstlisting}

每个透镜元件都按某半径绕光轴扩展;如果与该元件的交点在该半径之外,
则该光线实际上将与镜头罩相交并终止。
类似地,如果光线与光圈相交,它也会终止。
因此,这里我们用当前元件的适用限制来测试交点,
要么终止该光线,要么它幸存下来并将其端点更新为当前交点。
\begin{lstlisting}
`\initcode{Test intersection point against element aperture}{=}`
`\refvar{Point3f}{}` pHit = rLens(t);
`\refvar{Float}{}` r2 = pHit.x * pHit.x + pHit.y * pHit.y;
if (r2 > element.`\refvar{apertureRadius}{}` * element.`\refvar{apertureRadius}{}`)
    return false;
rLens.`\refvar[Ray::o]{o}{}` = pHit;
\end{lstlisting}

如果当前元件是光圈,则光路在穿过元件界面时不受影响。
对于玻璃(或塑料)透镜元件,光线在从具有某个折射率的介质进入
到具有另一折射率的介质时在交界面会改变方向
(光线可能从空气进入玻璃、从玻璃进入空气,或者从
具有某个折射率的玻璃进入具有不同折射率的另一种玻璃)。

\refsec{镜面反射与透射}讨论了两种介质边界间折射率的变化
将怎样改变光线的方向及其携带的辐射量(这里的情况下
我们可以忽略辐射量的变化,因为如果光线在进入和退出透镜系统时
处于同一种介质中则这种效应会抵消掉——这里都是空气)。
函数\refvar{Refract}{()}定义在\refsub{镜面透射};
注意它预设入射方向指向远离曲面的方向,所以传入前要对光线方向取反。
该函数在出现\keyindex{全内反射}{total internal reflection}{reflection反射}时
返回{\ttfamily false},该情况下光路终止。
否则在{\ttfamily w}中返回折射方向。

通常,穿过这类界面时一些光被透射而另一些被反射。
这里我们忽略反射并假设完美传输。尽管这是种近似,但它是合理的:
制造透镜时一般用了设计的涂料把反射降低到光线所带辐射的0.25\%左右
(然而,对这少量的反射建模对于实现\keyindex{镜头光晕}{lens flare}{}会很重要)。
\begin{lstlisting}
`\initcode{Update ray path for element interface interaction}{=}`
if (!isStop) {
    `\refvar{Vector3f}{}` w;
    `\refvar{Float}{}` etaI = element.`\refvar[LensElementInterface::eta]{eta}{}`;
    `\refvar{Float}{}` etaT = (i > 0 && `\refvar{elementInterfaces}{}`[i - 1].`\refvar[LensElementInterface::eta]{eta}{}` != 0) ?
        `\refvar{elementInterfaces}{}`[i - 1].`\refvar[LensElementInterface::eta]{eta}{}` : 1;
    if (!`\refvar{Refract}{}`(`\refvar{Normalize}{}`(-rLens.`\refvar[Ray::d]{d}{}`), n, etaI / etaT, &w))
        return false;
    rLens.`\refvar[Ray::d]{d}{}` = w;
}
\end{lstlisting}

若光线成功从前端透镜元件射出,它只需要从透镜空间变换到相机空间。
\begin{lstlisting}
`\initcode{Transform rLens from lens system space back to camera space}{=}`
if (rOut != nullptr) {
    static const `\refvar{Transform}{}` LensToCamera = `\refvar{Scale}{}`(1, 1, -1);
    *rOut = LensToCamera(rLens);
}
\end{lstlisting}

方法\refvar{TraceLensesFromScene}{()}和\refvar{TraceLensesFromFilm}{()}非常相似,这里不再介绍。
主要差别在于它是从前往后而不是从后往前遍历元件。
注意它假设传入的光线已经在相机空间了;
如果光线始于世界空间则调用者应负责执行该变换。
返回的光线位于尾部透镜元件朝向胶片的相机空间中。
\begin{lstlisting}
`\refcode{RealisticCamera Private Methods}{+=}\lastcode{RealisticCameraPrivateMethods}`
bool `\initvar{TraceLensesFromScene}{}`(const `\refvar{Ray}{}` &rCamera, `\refvar{Ray}{}` *rOut) const;
\end{lstlisting}

\subsection{厚透镜近似}\label{sub:厚透镜近似}
\refsub{薄透镜模型与景深}中用的薄透镜近似是基于透镜系统沿光轴厚度为0的简化假设。
透镜系统的厚透镜近似因为考虑了透镜系统的$z$范围而更精确些。
这里在介绍厚透镜的基本概念之后,我们将用厚透镜近似来确定
要把透镜系统放在离胶片多远处来对焦\refsub{对焦}中想要的对焦深度。

厚透镜近似将透镜系统表示为两对沿光轴的距离——
\keyindex{焦点}{focal point}{}以及\keyindex{主平面}{principal plane}{};
一个透镜系统有两个\keyindex{主点}{cardinal point}{}。
如果经由一个理想透镜系统追踪平行于光轴的光线,
则所有这些光线将会交于光轴上同一点——这就是焦点
(实际中,真实透镜系统并不绝对理想,不同高度的入射光线
与光轴将相交于一个$z$值小范围——这就是\keyindex{球面像差}{spherical aberration}{aberration像差})。
有了特定的透镜系统,我们就能从每一侧追踪穿过它且与光轴平行的光线,
并计算它们与$z$轴的相交处以找到焦点(见\reffig{6.20})。
\begin{figure}[htbp]
    \centering\includegraphics[width=\linewidth]{chap06/Lenssystemcardinalpoints.eps}
    \caption{计算透镜系统的主点。文件{\ttfamily lenses/dgauss.dat}中
    描述的透镜系统,来自场景的入射光线平行于光轴(轴上方),
    来自胶片的光线平行于光轴(下方)。这些入射光线对应的
    离开透镜系统的光线与光轴的相交处给出了两个焦点$f'_z$(胶片一侧)
    和$f_z$(场景一侧)。每对入射和出射光线的延长线以及原始光线的相交处给定了
    主平面$z=p_z$和$z=p'_z$,这里表示为垂直于光轴的蓝线。}
    \label{fig:6.20}
\end{figure}

每个主平面通过延长平行于光轴的入射光线以及离开透镜的光线直至相交来求得;
相交处的$z$深度给出了对应主平面的深度。\reffig{6.20}展示了
一个透镜系统及其焦点$f_z$和$f'_z$以及位于$z$值$p_z$和$p'_z$的主平面
(正如\refsub{薄透镜模型与景深},有撇的变量表示透镜系统胶片一侧的点,
无撇的变量表示场景中要成像的点)。

给定离开透镜的光线,求焦点要先计算光线的$x$和$y$分量为零时的值$t_{\mathrm{f}}$.
若进入的光线只沿$x$偏移了光轴,则我们要求出
使$o_x+t_{\mathrm{f}}d_x=0$的$t_{\mathrm{f}}$.因此
\begin{align*}
    t_{\mathrm{f}}=\frac{-o_x}{d_x}\, .
\end{align*}
同样的方法,为了求得离开透镜的光线与原始光线有相同高度处
的主平面的$t_{\mathrm{p}}$,我们有$o_x+t_{\mathrm{p}}d_x=x$,因此
\begin{align*}
    t_{\mathrm{p}}=\frac{x-o_x}{d_x}\, .
\end{align*}
一旦算出这两个$t$值,射线方程就可用于求得对应点的$z$坐标。

方法\refvar{ComputeCardinalPoints}{()}为给定光线计算焦点和主平面的$z$深度。
注意它假设光线在相机空间中但返回透镜空间中沿光轴的$z$值。
\begin{lstlisting}
`\refcode{RealisticCamera Method Definitions}{+=}\lastnext{RealisticCameraMethodDefinitions}`
void `\refvar{RealisticCamera}{}`::`\initvar{ComputeCardinalPoints}{}`(const `\refvar{Ray}{}` &rIn,
        const `\refvar{Ray}{}` &rOut, `\refvar{Float}{}` *pz, `\refvar{Float}{}` *fz) {
    `\refvar{Float}{}` tf = -rOut.`\refvar[Ray::o]{o}{}`.x / rOut.`\refvar[Ray::d]{d}{}`.x;
    *fz = -rOut(tf).z;
    `\refvar{Float}{}` tp = (rIn.`\refvar[Ray::o]{o}{}`.x - rOut.`\refvar[Ray::o]{o}{}`.x) / rOut.`\refvar[Ray::d]{d}{}`.x;
    *pz = -rOut(tp).z;
}
\end{lstlisting}

方法\refvar{ComputeThickLensApproximation}{()}为透镜系统
计算两对主点。
\begin{lstlisting}
`\refcode{RealisticCamera Method Definitions}{+=}\lastnext{RealisticCameraMethodDefinitions}`
void `\refvar{RealisticCamera}{}`::`\initvar{ComputeThickLensApproximation}{}`(`\refvar{Float}{}` pz[2],
        `\refvar{Float}{}` fz[2]) const {
    `\refcode{Find height x from optical axis for parallel rays}{}`
    `\refcode{Compute cardinal points for film side of lens system}{}`
    `\refcode{Compute cardinal points for scene side of lens system}{}`
}
\end{lstlisting}

首先,我们必须为追踪的光线选择沿$x$轴的高度。
它应该离$x=0$足够远使得有足够数值精度来准确计算
离开透镜系统的光线与$z$轴相交于哪里,
但$x$轴上也不能太高以免在光线穿过透镜系统时命中光圈。
这里,我们采用胶片对角范围的很小比例;其效果一般很好,除非光圈非常小。
\begin{lstlisting}
`\initcode{Find height x from optical axis for parallel rays}{=}`
`\refvar{Float}{}` x = .001 * `\refvar{film}{}`->`\refvar{diagonal}{}`;
\end{lstlisting}

为了构建从场景进入透镜系统的光线{\ttfamily rScene},
我们从透镜前端偏移一点(回想传入\refvar{TraceLensesFromScene}{()}的
光线应该在相机空间)。
\begin{lstlisting}
`\initcode{Compute cardinal points for film side of lens system}{=}`
`\refvar{Ray}{}` rScene(`\refvar{Point3f}{}`(x, 0, `\refvar{LensFrontZ}{}`() + 1), `\refvar{Vector3f}{}`(0, 0, -1));
`\refvar{Ray}{}` rFilm;
`\refvar{TraceLensesFromScene}{}`(rScene, &rFilm);
`\refvar{ComputeCardinalPoints}{}`(rScene, rFilm, &pz[0], &fz[0]);
\end{lstlisting}

从透镜系统胶片一侧起始的等价过程为我们给出了另外两个主点。
\begin{lstlisting}
`\initcode{Compute cardinal points for scene side of lens system}{=}`
rFilm = `\refvar{Ray}{}`(`\refvar{Point3f}{}`(x, 0, `\refvar{LensRearZ}{}`() - 1), `\refvar{Vector3f}{}`(0, 0, 1));
`\refvar{TraceLensesFromFilm}{}`(rFilm, &rScene);
`\refvar{ComputeCardinalPoints}{}`(rFilm, rScene, &pz[1], &fz[1]);
\end{lstlisting}

\subsection{对焦}\label{sub:对焦}
透镜系统可以通过相对于胶片移动来对焦场景中的指定深度,
这样在想要的对焦深度处的点成像为胶片平面上的一点。
高斯透镜方程\refeq{6.3}给了我们一个可解的关系来对焦厚透镜。

对于厚透镜,高斯透镜方程把到场景中位于$z$处的点的距离
和它对焦到$z'$的点用下式联系起来:
\begin{align}\label{eq:6.3}
    \frac{1}{z'-p'_z}-\frac{1}{z-p_z}=\frac{1}{f}\, .
\end{align}
对于薄透镜,$p_z=p'_z=0$,即推出\refeq{6.1}。

如果我们知道主平面的位置$p_z$和$p'_z$以及透镜的焦距$f$
并想要对焦沿光轴的某个深度$z$,则我们需确定应将系统平移多远$\delta$使得
\begin{align*}
    \frac{1}{z'-p'_z+\delta}-\frac{1}{z-p_z+\delta}=\frac{1}{f}\, .
\end{align*}

胶片一侧的焦点应在胶片上,所以$z'=0$,且$z=z_{\mathrm{f}}$即给定的对焦深度。
唯一未知的是$\delta$,一些代数处理为我们给出
\begin{align}\label{eq:6.4}
    \delta=\frac{1}{2}\left(p_z-z_{\mathrm{f}}+p'_z-\sqrt{(p_z-z_{\mathrm{f}}-p'_z)(p_z-z_{\mathrm{f}}-4f-p'_z)}\right)\, .
\end{align}
(实际上有两个解,但两个中更近的这个解给出了对透镜位置较小的调整,因此是合适的那个。)

\refvar{FocusThickLens}{()}用该近似对焦透镜系统。
计算$\delta$后,它返回应放置透镜系统处沿$z$轴离胶片的偏移量。
\begin{lstlisting}
`\refcode{RealisticCamera Method Definitions}{+=}\lastnext{RealisticCameraMethodDefinitions}`
`\refvar{Float}{}` `\refvar{RealisticCamera}{}`::`\initvar{FocusThickLens}{}`(`\refvar{Float}{}` focusDistance) {
    `\refvar{Float}{}` pz[2], fz[2];
    `\refvar{ComputeThickLensApproximation}{}`(pz, fz);
    `\refcode{Compute translation of lens, delta, to focus at focusDistance}{}`
    return `\refvar{elementInterfaces}{}`.back().`\refvar{thickness}{}` + delta;
}
\end{lstlisting}

\refeq{6.4}给出了偏移量$\delta$.透镜的焦距$f$是主点$f'_z$和$p'_z$间的距离。
还要注意对于$z$用的是对焦距离
\sidenote{译者注:和\refsec{投影相机模型}中的“focal distance”不同,这里的原文是“focus distance”。}
的相反数,因为光轴指向负的$z$.
\begin{lstlisting}
`\initcode{Compute translation of lens, delta, to focus at focusDistance}{=}`
`\refvar{Float}{}` f = fz[0] - pz[0];
`\refvar{Float}{}` z = -focusDistance;
`\refvar{Float}{}` delta = 0.5f * (pz[1] - z + pz[0] -
    std::sqrt((pz[1] - z - pz[0]) * (pz[1] - z - 4 * f - pz[0])));
\end{lstlisting}

我们现在终于可以实现\refvar{RealisticCamera}{}构造函数中
对焦透镜系统的代码片了(回想最尾部元件界面的厚度是该界面到胶片的距离)。
\begin{lstlisting}
`\initcode{Compute lens-film distance for given focus distance}{=}`
`\refvar{elementInterfaces}{}`.back().`\refvar{thickness}{}` = `\refvar{FocusThickLens}{}`(focusDistance);
\end{lstlisting}

\subsection{出射瞳}\label{sub:出射瞳}
不是所有从胶片平面上给定点起始朝向尾部透镜元件的光线都能成功射出透镜系统;
一些会被光圈阻挡或者与透镜系统外壳相交。
反过来,尾部透镜元件上不是所有点都能把辐射传输到胶片上的点。
尾部元件上确实能带着光通过透镜系统的点集称为\keyindex{出射瞳}{exit pupil}{pupil瞳};
它的尺寸和位置随着胶片平面上的视点变化
(类似地,\keyindex{入射瞳}{entrance pupil}{pupil瞳}是场景中给定点起始
且能到达胶片的光线所穿过的前端透镜元件区域)
\sidenote{译者注:这和后文补充材料\refsub{光圈}中对出射瞳与入射瞳的定义有一些出入,请读者酌情采信。}。

\reffig{6.21}展示了广角镜头从胶片平面上两点看到的出射瞳。
当点靠近胶片边缘时出射瞳更小。这类收缩的一个结果是暗角。
\begin{figure}[htbp]
    \centering
    \subfloat[]{\includegraphics[width=0.5\linewidth]{chap06/wide22-rays-edge.eps}\label{fig:6.21.1}}\qquad
    \subfloat[]{\includegraphics[width=0.2\linewidth]{chap06/wide22-exit-edge.eps}\label{fig:6.21.2}}\\
    \subfloat[]{\includegraphics[width=0.5\linewidth]{chap06/wide22-rays-middle.eps}\label{fig:6.21.3}}\qquad
    \subfloat[]{\includegraphics[width=0.2\linewidth]{chap06/wide22-exit-middle.eps}\label{fig:6.21.4}}
    \caption{22mm广角镜头在5.5mm光圈(f/4)下的出射瞳。
        (a)起始于胶片平面边缘一点的光线进入尾部透镜元件的多个点。
        虚线表示被阻挡而没有射出透镜系统的光线。
        (b)从(a)中的有利位置看到的出射瞳的像。尾部透镜元件是黑色的,而出射瞳表示为灰色。
        (c)在胶片中心,不同的出射瞳区域允许光线射出到场景中。
        (d)从胶片中心看到的出射瞳的像。}
    \label{fig:6.21}
\end{figure}

当追踪起始于胶片的光线时,我们想避免追踪太多没能穿过透镜系统的光线;
因此值得把采样限制在出射瞳本身及其周围很小区域内,
而不是浪费地在尾部透镜元件整个区域上采样。

在追踪光线前计算胶片上每点的出射瞳会开销极大;相反\refvar{RealisticCamera}{}
的实现预先计算沿胶片平面上的直线段对应的出射瞳边界。
因为我们假设透镜系统是绕光轴径向对称的,所以出射瞳边界也会是径向对称的,
且胶片平面上任意点的边界都可以通过适当旋转这些线段边界来求得(\reffig{6.22})
\sidenote{译者注:\protect\reffig{6.22}题注最后一句翻译不确定是否正确。}。
然后这些边界用于为特定胶片采样位置高效求取出射瞳边界。
\begin{figure}[htbp]
    \centering\includegraphics[width=0.7\linewidth]{chap06/Exitpupilboundsfilm.eps}
    \caption{预先计算出射瞳边界。(a)\refvar{RealisticCamera}{}为胶片平面上
        沿$x$轴的一系列线段计算出射瞳的边界,直到达到胶片中心到角点的距离$r$.
        (b)鉴于径向对称的假设,我们可以通过计算点与$x$轴间的夹角$\theta$来
        为胶片上的任意点(实心点)求取出射瞳边界。如果初始出射瞳边界内的一点被采样
        再旋转$\theta$,则我们就有了原始点处出射瞳内的一点。}
    \label{fig:6.22}
\end{figure}

需要意识到的一个重要细节是,由于透镜系统通过沿光轴平移来对焦,
所以当调整透镜系统的对焦时出射瞳的形状和位置会变化。
因此在对焦后才计算这些边界十分重要
\footnote{作者也是调试几个小时后才痛苦地吸取这一教训。}。
\begin{lstlisting}
`\initcode{Compute exit pupil bounds at sampled points on the film}{=}`
int nSamples = 64;
`\refvar{exitPupilBounds}{}`.resize(nSamples);
`\refvar{ParallelFor}{}`(
    [&](int i) {
        `\refvar{Float}{}` r0 = (`\refvar{Float}{}`)i / nSamples * `\refvar{film}{}`->`\refvar{diagonal}{}` / 2;
        `\refvar{Float}{}` r1 = (`\refvar{Float}{}`)(i + 1) / nSamples * `\refvar{film}{}`->`\refvar{diagonal}{}` / 2;
        `\refvar{exitPupilBounds}{}`[i] = `\refvar{BoundExitPupil}{}`(r0, r1);
    }, nSamples);
\end{lstlisting}
\begin{lstlisting}
`\refcode{RealisticCamera Private Data}{+=}\lastcode{RealisticCameraPrivateData}`
std::vector<`\refvar{Bounds2f}{}`> `\initvar{exitPupilBounds}{}`;
\end{lstlisting}

方法\refvar{BoundExitPupil}{()}计算从胶片平面上一段点所见出射瞳的2D边界框。
通过尝试追踪在尾部透镜元件切平面的一组点上穿过透镜系统的光线来求取该边界框。
成功穿过透镜系统的光线的边界框便给出了出射瞳的大致边界——见\reffig{6.23}
\sidenote{译者注:\reffig{6.23}题注最后两句的意思是:尾部透镜元件在其切平面
    上的投影范围作为光线方向向量(归一化前)终点的采样范围,胶片上的区间作为光线起点的采样范围;
    采样许多这样的光线,保留其中成功穿过透镜系统的光线,这些光线穿过尾部透镜切平面时
    的位置即大致给出了出射瞳范围。}。
\begin{figure}[htbp]
    \centering\includegraphics[width=0.5\linewidth]{chap06/Sampleexitpupil.eps}
    \caption{怎样计算出射瞳边界的2D图示。\refvar{BoundExitPupil}{()}接收
        胶片上沿$x$轴的区间。它沿区间采样一系列点(图的下方)。
        对于每个点,它还采样尾部透镜元件在与其尾部相切的平面上相应范围边框内的一点。
        它计算所有起始于区间的点且成功穿过透镜系统的光线在切平面上的边界框。}
    \label{fig:6.23}
\end{figure}
\begin{lstlisting}
`\refcode{RealisticCamera Method Definitions}{+=}\lastnext{RealisticCameraMethodDefinitions}`
`\refvar{Bounds2f}{}` `\refvar{RealisticCamera}{}`::`\initvar{BoundExitPupil}{}`(`\refvar{Float}{}` pFilmX0,
        `\refvar{Float}{}` pFilmX1) const {
    `\refvar{Bounds2f}{}` pupilBounds;
    `\refcode{Sample a collection of points on the rear lens to find exit pupil}{}`
    `\refcode{Return entire element bounds if no rays made it through the lens system}{}`
    `\refcode{Expand bounds to account for sample spacing}{}`
    return pupilBounds;
}
\end{lstlisting}

该实现非常密集地采样出射瞳——每段共有$1024^2$个点。
我们发现实践中该采样率能提供很好的出射瞳边界。
\begin{lstlisting}
`\initcode{Sample a collection of points on the rear lens to find exit pupil}{=}`
const int nSamples = 1024 * 1024;
int nExitingRays = 0;
`\refcode{Compute bounding box of projection of rear element on sampling plane}{}`
for (int i = 0; i < nSamples; ++i) {
    `\refcode{Find location of sample points on x segment and rear lens element}{}`
    `\refcode{Expand pupil bounds if ray makes it through the lens system}{}`
}
\end{lstlisting}

尾部元件在与之相切\sidenote{译者注:原文误写为垂直,已修正。}的平面上
的边界框不足以成为出射瞳在该平面上的投影的保守边界框;
因为元件一般是曲形的,穿过该平面边界之外的光线可能自己与尾部透镜元件有效范围相交。
比起计算精确边界,我们则是大幅增大边框。结果是用来计算出射瞳边界而采样的许多点会被浪费;
实践中,这是很小的代价,因为通常这些样本在透镜的光线追踪阶段会很快被终止。
\begin{lstlisting}
`\initcode{Compute bounding box of projection of rear element on sampling plane}{=}`
`\refvar{Float}{}` rearRadius = `\refvar{RearElementRadius}{}`();
`\refvar{Bounds2f}{}` projRearBounds(`\refvar{Point2f}{}`(-1.5f * rearRadius, -1.5f * rearRadius),
                        `\refvar{Point2f}{}`( 1.5f * rearRadius,  1.5f * rearRadius));
\end{lstlisting}

通过在$x$区间端点间线性插值来求得胶片上的$x$样本点。
稍后\refsub{Hammersley和Halton序列}{}定义的函数\refvar{RadicalInverse}{()}用于
为出射瞳边界框内的采样点计算插值偏移量。我们将看到
这里实现的采样策略对应于在3D中使用Hammersley点;
得到的点集会最小化整个3D域覆盖情况的差异,反过来保证了准确估计出射瞳边界。
\begin{lstlisting}
`\initcode{Find location of sample points on x segment and rear lens element}{=}`
`\refvar{Point3f}{}` pFilm(`\refvar{Lerp}{}`((i + 0.5f) / nSamples, pFilmX0, pFilmX1), 0, 0);
`\refvar{Float}{}` u[2] = { `\refvar{RadicalInverse}{}`(0, i), `\refvar{RadicalInverse}{}`(1, i) };
`\refvar{Point3f}{}` pRear(`\refvar{Lerp}{}`(u[0], projRearBounds.pMin.x, projRearBounds.pMax.x),
              `\refvar{Lerp}{}`(u[1], projRearBounds.pMin.y, projRearBounds.pMax.y),
              `\refvar{LensRearZ}{}`());
\end{lstlisting}

现在我们可构造从{\ttfamily pFilm}到{\ttfamily pRear}的光线并
通过看它是否成功射出透镜系统前端来确定它是否在出射瞳内。
如果是,则拓展出射瞳边界以包含该点。如果采样点已经在
目前算出的出射瞳边界框内,则我们可以跳过光线追踪步骤以节约些无用功。
\begin{lstlisting}
`\initcode{Expand pupil bounds if ray makes it through the lens system}{=}`
if (`\refvar{Inside}{}`(`\refvar{Point2f}{}`(pRear.x, pRear.y), pupilBounds) ||
    `\refvar{TraceLensesFromFilm}{}`(`\refvar{Ray}{}`(pFilm, pRear - pFilm), nullptr)) {
    pupilBounds = `\refvar[Union1]{Union}{}`(pupilBounds, `\refvar{Point2f}{}`(pRear.x, pRear.y));
    ++nExitingRays;
}
\end{lstlisting}

可能没有样本光线能成功穿过透镜系统;例如一些在胶片范围边缘处出射瞳会消失的
超大广角镜头完全可能发生这种情况。这种情况下边界并不重要,
\refvar{BoundExitPupil}{()}返回包括了整个尾部透镜元件的边界。
\begin{lstlisting}
`\initcode{Return entire element bounds if no rays made it through the lens system}{=}`
if (nExitingRays == 0) 
    return projRearBounds;
\end{lstlisting}

当一个样本成功穿过透镜系统而它的一个相邻样本没有时,
有可能与该邻居很相近的另一个样本实际上能成功穿过。
因此,最终边界在每个方向上再大致按样本间距拓展以考虑这一不确定性。
\begin{lstlisting}
`\initcode{Expand bounds to account for sample spacing}{=}`
pupilBounds = `\refvar{Expand}{}`(pupilBounds,
                     2 * projRearBounds.`\refvar{Diagonal}{}`().`\refvar{Length}{}`() /
                     std::sqrt(nSamples));
\end{lstlisting}

有了预先计算并存储在\refvar[exitPupilBounds]{RealisticCamera::exitPupilBounds}{}中的边界,
方法\refvar{SampleExitPupil}{()}就能很高效地为胶片平面上的给定点求得出射瞳边界。
为了准确地对成像辐射度量学建模,下列代码需要知道该边界框的面积,
所以它通过{\ttfamily sampleBoundsArea}返回。
\begin{lstlisting}
`\refcode{RealisticCamera Method Definitions}{+=}\lastnext{RealisticCameraMethodDefinitions}` 
`\refvar{Point3f}{}` `\refvar{RealisticCamera}{}`::`\initvar{SampleExitPupil}{}`(const `\refvar{Point2f}{}` &pFilm,
        const `\refvar{Point2f}{}` &lensSample, `\refvar{Float}{}` *sampleBoundsArea) const {
    `\refcode{Find exit pupil bound for sample distance from film center}{}`
    `\refcode{Generate sample point inside exit pupil bound}{}`
    `\refcode{Return sample point rotated by angle of pFilm with +x axis}{}`
}
\end{lstlisting}
\begin{lstlisting}
`\initcode{Find exit pupil bound for sample distance from film center}{=}`
`\refvar{Float}{}` rFilm = std::sqrt(pFilm.x * pFilm.x + pFilm.y * pFilm.y);
int rIndex = rFilm / (`\refvar{film}{}`->`\refvar{diagonal}{}` / 2) * `\refvar{exitPupilBounds}{}`.size();
rIndex = std::min((int)`\refvar{exitPupilBounds}{}`.size() - 1, rIndex);
`\refvar{Bounds2f}{}` pupilBounds = `\refvar{exitPupilBounds}{}`[rIndex];
if (sampleBoundsArea) *sampleBoundsArea = pupilBounds.Area();
\end{lstlisting}

给定瞳的边界框后,通过对提供的位于$[0,1)^2$的值{\ttfamily lensSample}线性插值
来采样边界框中的一点。
\begin{lstlisting}
`\initcode{Generate sample point inside exit pupil bound}{=}`
`\refvar{Point2f}{}` pLens = pupilBounds.`\refvar[Bounds3::Lerp]{Lerp}{}`(lensSample);
\end{lstlisting}

因为出射瞳边界是从胶片上一点沿$+x$轴算得的,
但点{\ttfamily pFilm}是胶片上的任意点,
故出射瞳边框内的样本点必须像{\ttfamily pFilm}对$+x$轴那样旋转相同角度。
\begin{lstlisting}
`\initcode{Return sample point rotated by angle of pFilm with +x axis}{=}`
`\refvar{Float}{}` sinTheta = (rFilm != 0) ? pFilm.y / rFilm : 0;
`\refvar{Float}{}` cosTheta = (rFilm != 0) ? pFilm.x / rFilm : 1;
return `\refvar{Point3f}{}`(cosTheta * pLens.x - sinTheta * pLens.y,
               sinTheta * pLens.x + cosTheta * pLens.y,
               `\refvar{LensRearZ}{}`());
\end{lstlisting}

\subsection{生成光线}\label{sub:生成光线}
现在我们有了机制去追踪穿过透镜系统的光线并采样由胶片平面上的点得来的出射瞳边界内的点,
将\refvar{CameraSample}{}转化为离开相机的光线非常简单:
我们需要计算样本在胶片平面上的位置并生成起始于该点射向尾部透镜元件的光线,
然后通过透镜系统追踪它。
\begin{lstlisting}
`\refcode{RealisticCamera Method Definitions}{+=}\lastcode{RealisticCameraMethodDefinitions}`
`\refvar{Float}{}` `\refvar{RealisticCamera}{}`::`\initvar[RealisticCamera::GenerateRay]{\refvar{GenerateRay}{}}{}`(const `\refvar{CameraSample}{}` &sample,
        `\refvar{Ray}{}` *ray) const {
    `\refcode{Find point on film, pFilm, corresponding to sample.pFilm}{}`
    `\refcode{Trace ray from pFilm through lens system}{}`
    `\refcode{Finish initialization of RealisticCamera ray}{}`
    `\refcode{Return weighting for RealisticCamera ray}{}`
}
\end{lstlisting}

值\refvar[pFilm]{CameraSample::pFilm}{}与图像的整个像素分辨率有关。
这里我们用传感器的物理模型来操作,所以我们开始先将样本转化回$[0,1)^2$中。
接着,通过用该样本值在其区域上线性插值来求得胶片上的对应点
\sidenote{译者注:我暂未理解为何下面第5行代码中$x$值会取相反数,欢迎读者提供帮助。
    详见\url{https://github.com/mmp/pbrt-v4/issues/233}
    和\url{https://github.com/mmp/pbrt-v3/issues/289}。}。
\begin{lstlisting}
`\initcode{Find point on film, pFilm, corresponding to sample.pFilm}{=}`
`\refvar{Point2f}{}` s(sample.`\refvar{pFilm}{}`.x / `\refvar{film}{}`->`\refvar{fullResolution}{}`.x,
          sample.`\refvar{pFilm}{}`.y / `\refvar{film}{}`->`\refvar{fullResolution}{}`.y);
`\refvar{Point2f}{}` pFilm2 = `\refvar{film}{}`->`\refvar{GetPhysicalExtent}{}`().`\refvar[Bounds3::Lerp]{Lerp}{}`(s);
`\refvar{Point3f}{}` pFilm(-pFilm2.x, pFilm2.y, 0);
\end{lstlisting}

然后\refvar{SampleExitPupil}{()}给我们尾部透镜元件切平面上的一点,
让我们确定了光线的方向。我们可追踪该光线穿过透镜系统。
若光线被光圈阻挡或没能成功穿过透镜系统,\refvar[RealisticCamera::GenerateRay]{GenerateRay}{()}返回权重0
(调用者应保证检查该情况)。
\begin{lstlisting}
`\initcode{Trace ray from pFilm through lens system}{=}`
`\refvar{Float}{}` exitPupilBoundsArea;
`\refvar{Point3f}{}` pRear = `\refvar{SampleExitPupil}{}`(`\refvar{Point2f}{}`(pFilm.x, pFilm.y), sample.`\refvar{pLens}{}`,
                                &exitPupilBoundsArea);
`\refvar{Ray}{}` rFilm(pFilm, pRear - pFilm, `\refvar{Infinity}{}`,
          `\refvar{Lerp}{}`(sample.`\refvar[CameraSample::time]{time}{}`, `\refvar{shutterOpen}{}`, `\refvar{shutterClose}{}`));
if (!`\refvar{TraceLensesFromFilm}{}`(rFilm, ray))
    return 0;
\end{lstlisting}

如果光线成功射出透镜系统,则需要处理常规细节以完成其初始化。
\begin{lstlisting}
`\initcode{Finish initialization of RealisticCamera ray}{=}`
*ray = `\refvar{CameraToWorld}{}`(*ray);
ray->`\refvar[Ray::d]{d}{}` = `\refvar{Normalize}{}`(ray->`\refvar[Ray::d]{d}{}`);
ray->`\refvar[Ray::medium]{medium}{}` = `\refvar[Camera::medium]{medium}{}`;
\end{lstlisting}

在介绍了蒙特卡罗积分一些必要的背景后,
代码片\refcode{Return weighting for RealisticCamera ray}{}将在\refsub{采样相机1}介绍。

\subsection{相机测量方程}\label{sub:相机测量方程}
有了上述对真实成像过程更精确的模拟,
更仔细地定义胶片或相机传感器的辐射度量是值得的。
从出射瞳到胶片的光线携带着来自场景的辐射;
因此从胶片平面上一点来考虑,有一组对应的辐射入射方向。
离开出射瞳的辐射分布会被胶片上该点所见的失焦模糊程度影响——
\reffig{6.24}展示了从胶片上两点看到的出射瞳辐射的渲染图
\sidenote{译者注:我认为原文\reffig{6.24}的题注将“清晰合焦”和“失焦”写反了,此处已修改。}。
\begin{figure}[htbp]
    \centering
    \includegraphics[width=0.4\linewidth]{chap06/exitpupil-a.png}\quad
    \includegraphics[width=0.4\linewidth]{chap06/exitpupil-b.png}
    \caption{从\reffig{6.16}中胶片平面上两点看到的出射瞳。
        (一)失焦的点上看到的出射瞳;入射辐射在其区域上实际上是常量。
        (二)清晰合焦区域中一像素处所见的出射瞳是部分场景的一小幅图像,
        可能有急剧变化的辐射。}
    \label{fig:6.24}
\end{figure}

给定入射辐射函数,我们可以定义胶片平面上一点的辐射照度。
如果我们从利用辐射亮度定义辐射照度的\refeq{5.4}开始,
则我们可以用\refeq{5.6}把在立体角上的积分转化为面积
(这种情况下即定界出射瞳的尾部透镜元件的切平面面积$A_{\mathrm{e}}$)上的积分。
这给出了胶片平面上一点$\bm p$的辐射照度:
\begin{align*}
    E({\bm p})=\int_{A_{\mathrm{e}}}{L_{\mathrm{i}}({\bm p},{\bm p}')
    \frac{|\cos\theta\cos\theta'|}{\|{\bm p}'-{\bm p}\|^2}\mathrm{d}A_{\mathrm{e}}}\, .
\end{align*}

\reffig{6.25}展示了该情况的几何结构。
\begin{figure}[htbp]
    \centering\includegraphics[width=0.5\linewidth]{chap06/Camerameasurementequation.eps}
    \caption{\refeq{6.5}辐射照度度量方程的几何设置。当辐射从
        尾部透镜元件切平面上的点$\bm p'$穿过到胶片平面上的点$\bm p$时可以被度量。
        $z$是从胶片平面到尾部元件切平面的轴向距离\refvar[LensRearZ]{RealisticCamera::LensRearZ}{()},
        而$\theta$是从$\bm p'$到$\bm p$的向量与光轴的夹角。}
    \label{fig:6.25}
\end{figure}

因为胶片平面平行
\sidenote{译者注:原文误写为垂直,已修改。}
于出射瞳平面,所以$\theta=\theta'$.
我们可进一步利用$\bm p$和$\bm p'$间的距离等于从胶片平面到出射瞳轴向距离
(这里我们表示为$z$)除以$\cos\theta$的事实。
将这些全部结合起来,我们有
\begin{align}\label{eq:6.5}
    E({\bm p})=\frac{1}{z^2}\int_{A_{\mathrm{e}}}{L_{\mathrm{i}}({\bm p},{\bm p}')|\cos^4\theta|\mathrm{d}A_{\mathrm{e}}}\, .
\end{align}

对于胶片范围相对于距离$z$较大的相机,项$\cos^4\theta$会
明显降低入射辐射照度——该因素也促成了暗角。
大多数现代数字相机都按照会增加传感器边缘像素值的预设校正因子来校正该效应。

胶片上一点的辐射照度对快门开启时间的积分
给出了\keyindex{注量}{fluence}{}
\sidenote{译者注:即\keyindex{辐射曝光量}{radiant exposure}{}。},
即单位面积上的辐射能量,单位$\text{J/m}^2$.
\begin{align}\label{eq:6.6}
    H({\bm p})=\frac{1}{z^2}\int_{t_0}^{t_1}\int_{A_{\mathrm{e}}}
    L_{\mathrm{i}}({\bm p},{\bm p}',t')|\cos^4\theta|\mathrm{d}A_{\mathrm{e}}\mathrm{d}t'\, .
\end{align}

度量一点的注量刻画了胶片平面接收的能量大小与相机快门开启的时长部分相关的效应。

摄影胶片(或者数字相机内的CCD\sidenote{译者注:即\keyindex{电荷耦合器件}{charge-coupled device}{}。}
或CMOS\sidenote{译者注:即\keyindex{互补式金属氧化物半导体}{complementary metal-oxide-semiconductor}{}。})
实际上是度量一小片面积上的辐射能量
\footnote{2015年代手机数字相机中像素的典型尺寸是每边1.5微米。}。
取\refeq{6.6}并在传感器像素面积$A_{\mathrm{p}}$上积分,我们有
\begin{align}\label{eq:6.7}
    J=\frac{1}{z^2}\int_{A_{\mathrm{p}}}\int_{t_0}^{t_1}\int_{A_{\mathrm{e}}}
    L_{\mathrm{i}}({\bm p},{\bm p}',t')|\cos^4\theta|\mathrm{d}A_{\mathrm{e}}\mathrm{d}t'\mathrm{d}A_{\mathrm{p}}\, ,
\end{align}
焦耳到达一个像素。

\refsec{蒙特卡罗估计器}中,我们将看到蒙特卡罗可以怎样运用于估计这些各种积分的值。
然后\refsub{采样相机1}中我们将定义\refvar{RealisticCamera::GenerateRay}{()}中的
代码片\refcode{Return weighting for RealisticCamera ray}{};
各种计算权重的方法允许我们计算这些量中的每一个。
\refsub{采样相机2}定义了相机模型的\keyindex{重要性函数}{importance function}{},
它表征了其对沿不同光线到达的入射光照的敏感性。

\section{扩展阅读}\label{sec:扩展阅读06}

在他的开创性画板系统中,\citet{10.1145/1461551.1461591}是
第一个为计算机图形学使用投影矩阵的人。
\citet{10.1201/9781315365459}\sidenote{译者注:该书已有第四版\citep{10.1201/b22086}。}给出了
写得非常好的正交和透视投影矩阵的推导。
关于投影的其他优秀参考有\citet{10.5555/63448}的《\citetitle{10.5555/63448}》,
以及\citet{EBERLY2007}关于游戏引擎设计的书籍\sidenote{译者注:原文引用第一版,此处改为引用第二版。}。

\citet{4056910}使用独特的投影方法为Omnimax\textsuperscript{\textregistered}影院
\sidenote{译者注:即现在的IMAX影院。}生成图像。
本章的\refvar{EnvironmentCamera}{}与\citet{KENTON1992288}描述的相机模型相同。

\citet{10.1145/800224.806818,10.1145/357299.357300,10.1145/800059.801169}做了
计算机图形学中景深和运动模糊的早期工作。
Cook和合作者们基于薄透镜模型为这些效应开发了更精确的模型;
它即\refsub{薄透镜模型与景深}中用于景深计算的方法\citep{10.1145/800031.808590,10.1145/7529.8927}。
见\citet{10.2312:EGWR:EGSR07:121-126}关于
具有非针孔光圈的相机可采用的辐射度量类别的广泛分析。

\citet{10.1145/218380.218463}展示了怎样用光线追踪
模拟复杂相机透镜以建模真实相机的成像效应;
\refsec{逼真相机}中的\refvar{RealisticCamera}{}就基于他们的方法。
\citet{10.1111/j.1467-8659.2011.01851.x}改进了该模拟的大量细节,
合并了依赖波长的效应并一起实现衍射和眩光。
\refsub{出射光瞳}中我们模拟出射光瞳的方法和他们的一样。
见\citet{0321188780}和\citet{9780071476874}的书籍了解
对光学和透镜系统的精彩介绍。

\citet{10.1111/j.1467-8659.2012.03132.x}用多项式来
建模透镜对穿过它们的光线的影响;
它们可以从单个透镜的多项式近似中构造模拟整个透镜系统的多项式。
该方法节约了追踪穿过透镜的光线的计算成本,
不过对于复杂场景其开销相比于剩余的渲染计算量而言一般可以忽略。
\citet{10.1111/cgf.12301}提升了该方法的准确性并展示了怎样将该方法与双向路径追踪结合。

\citet{5280315}介绍了几乎完整模拟数字相机的实现,
包括模数转换\sidenote{译者注:原文analog-to-digital conversion。}和
该过程固有的像素度量值噪声。

\section{习题}\label{sec:习题06}
\begin{enumerate}
      \item \circletwo 一些类型的相机通过在胶片上滑动矩形狭缝
      来曝光胶片。当物体移动方向与曝光狭缝不同时
      会产生有趣效应\citep{761554,10.1080/2151237X.2007.10129235}。
      而且,大多数数字相机在几毫秒时段内从连续扫描线中读出像素值;
      这会造成具有相似视觉特性的\keyindex{卷帘快门}{rolling shutter}{}痕迹。
      在本章的一个或多个相机实现中修改生成时间样本的方式对该效应建模。
      渲染能清楚展示考虑该问题所带来的影响的运动物体图像。
      \item \circletwo 编写一个应用加载由\refvar{EnvironmentCamera}{}渲染的图像,
      并利用纹理映射将其应用在球心位于视点处的球体上,
      使得可以交互地观察它们。用户应能够自由改变观察方向。
      如果为生成球上的纹理坐标使用了正确的纹理映射函数,
      则该应用生成的图像看起来像是观察者处于渲染时相机在场景中的位置,
      因此给予用户交互地环顾场景的能力。
      \item \circletwo \refvar{RealisticCamera}{}中的光圈
      建模为完美的圆;对于可调光圈相机,其光圈通常由
      直边可动叶片构成,因此是$n$边形。修改\refvar{RealisticCamera}{}以
      建模更真实的光圈形状并渲染能展示你模型差异的图像。
      你可能会发现渲染小、亮、失焦物体的场景(例如镜面高光)
      对展示其差异很有用。
      \item \circletwo 计算机图形学中景深的标准模型把
      弥散圆建模为将场景中一点成像为均匀强度的圆盘,
      然而许多真实透镜产生的弥散圆有非线性的变化,例如高斯分布。
      该效应称为\keyindex{散景}{bokeh}{}\citep{10.1145/1242073.1242155}。
      例如,当失焦时观察小光点,反射折射式\sidenote{译者注:原文catadioptric。}(镜面)
      透镜会产生环形高光。修改\refvar{RealisticCamera}{}中
      景深的实现(例如通过偏置透镜样本位置的分布)以生成具有该效应的图像。
      渲染能展示其与标准模型间差异的图像。
      \item \circletwo \keyindex{焦点堆栈渲染}{focal stack rendering}{}:
      焦点堆栈是一固定场景的一系列图像,每张图像的相机对焦距离不同。
      \citet{5740919}与\citet{Jacobs2012}介绍了焦点堆栈的许多应用,
      包括任意的景深,即用户可以指定任意深度对焦,达到传统光学不可能有的效果。
      用pbrt渲染焦点堆栈并编写交互工具以用其控制对焦效应。
      \item \circlethree \keyindex{光场相机}{light field camera}{camera相机}:
      \citet{ng:hal-02551481}讨论了一种相机的物理设计和应用,
      它能捕获胶片上各出射光瞳的微小图像,而不是像常规相机那样
      对每一像素在整个出射光瞳上的辐射求平均。这样的相机会获取光场的表示——
      到达相机传感器的辐射在空间和方向上变化着的分布。
      通过获取光场,就能实现许多有趣操作,包括拍摄相片后重新对焦。
      阅读\citeauthor{ng:hal-02551481}的论文并在pbrt中实现一个
      获取场景光场的\refvar{Camera}{}。编写工具以允许用户交互地重新对焦这些光场。
      \item \circlethree \refvar{RealisticCamera}{}的实现将胶片置于
      光轴中心并与之垂直。尽管这是真实相机最常见的配置,但通过调整胶片
      相对于透镜系统的放置可以得到有趣的效应。例如,当前实现的焦平面
      总是垂直于光轴;如果胶片平面(或透镜系统)倾斜使得胶片不垂直于光轴,
      则焦平面不再垂直于光轴(这对风景摄影很有用,例如让焦平面
      和地平面对齐时甚至用更大光圈也能有更大景深)。或者可以平移胶片平面
      使得它不在光轴中心;例如可利用该平移保持焦平面和极高物体对齐。
      修改\refvar{RealisticCamera}{}以允许这一或两种调整并渲染展示结果的图像。
      注意当前实现的许多地方(例如出射光瞳计算)都有这些修改会违背的假设需要你解决。
\end{enumerate}

\section{译者补充:几何光学}\label{sec:译者补充:几何光学}

\begin{remark}
    本节内容不是原书内容,而是译者根据《Optics》\citep{hecht2016optics}
    补充的,请酌情参考和斧正。然而本节仍需要读者对光与波的基本物理概念有所了解,
    这些内容一般可在中学和大学物理教材中找到。
\end{remark}

\subsection{光学背景知识}\label{sub:光学背景知识}
光是人眼可见频段的电磁波,可在真空与介质中传播。
光在真空中的传播速度最大,在介质中的传播速度与介质类型有关。
\begin{definition}
    三维中的\keyindex{波前}{wavefront}{}是某一时刻波\keyindex{相位}{phase}{}相同的点构成的面。
\end{definition}
\begin{definition}
    电磁波在真空中的传播速率$c$与在介质中的传播速率$v$的比值
    定义为\keyindex{绝对折射率}{absolute index of refraction}{index of refraction折射率},
    可简称折射率:
    \begin{align}
        n=\frac{c}{v}\, .
    \end{align}
\end{definition}
\begin{corollary}
    任意介质的绝对折射率都大于1.
\end{corollary}

光传播时在两种介质交界面上会发生\keyindex{反射}{reflection}{}
和\keyindex{折射}{refraction}{},如\reffig{6.26}所示:
其中上层与下层介质的绝对折射率分别为$n_i$和$n_t (n_i<n_t)$;
入射光线、反射光线、折射光线与界面法线的夹角$\theta_i, \theta_r, \theta_t$分别称为
\keyindex{入射角}{angle of incidence}{}、
\keyindex{反射角}{angle of reflection}{}、
\keyindex{折射角}{angle of refraction}{}。
入射光线和界面法线确定的平面称为\keyindex{入射平面}{plane of incidence}{}。
\begin{figure}[htbp]
    \centering\includegraphics[width=0.5\linewidth]{chap06/ReflectionAndRefraction.eps}
    \caption{反射与折射分别遵循反射定律和折射定律。}
    \label{fig:6.26}
\end{figure}

\begin{proposition}
    \keyindex{反射定律}{law of reflection}{}:反射光线在入射平面内;
    反射光线和入射光线分居界面法线两侧;反射角等于入射角,即
    \begin{align}
        \theta_r=\theta_i \, .
    \end{align}
\end{proposition}
\begin{proposition}
    \keyindex{折射定律}{law of refraction}{}:折射光线在入射平面内;
    折射光线和入射光线分居界面法线两侧;折射角与入射角遵循\keyindex{斯涅尔定律}{Snell's law}{},即
    \begin{align}
        n_i\sin\theta_i=n_t\sin\theta_t \, .
    \end{align}
\end{proposition}
\begin{corollary}
    光线进入更大折射率的介质时会更偏向界面法线,
    进入更小折射率介质时则会更偏离法线。
\end{corollary}
\begin{definition}
    两种介质的绝对折射率之比定义为\keyindex{相对折射率}{relative index of refraction}{index of refraction折射率}:
    \begin{align}
        n_{ti}=\frac{n_t}{n_i}\, .
    \end{align}
    由此斯涅尔定律也可写作:
    \begin{align}
        \frac{\sin\theta_i}{\sin\theta_t}=n_{ti}\, .
    \end{align}
\end{definition}

我们考虑\reffig{6.27}的情景:一束光线从点$S$起依次穿过$m$层介质并发生折射,
最终到达点$P$.设在每种介质中传播时对应的介质绝对折射率、路程、传播速度
分别为$n_j, s_j, v_j (j=1,\ldots,m)$.于是光线传播的总时间为
\begin{align}
    t=\sum\limits_{j=1}^{m}{\frac{s_j}{v_j}}\, ,
\end{align}
利用绝对折射率的定义可得
\begin{align}
    t=\frac{1}{c}\sum\limits_{j=1}^{m}{n_js_j}\, .
\end{align}
相对于空间路程$\sum\limits_{j=1}^{m}{s_j}$,
我们定义上式中的$\sum\limits_{j=1}^{m}{n_js_j}$
为\keyindex{光程}{optical path length}{}(OPL)。
更一般地,在非均匀介质中折射率是位置的函数,因此有
\begin{align}
    OPL=\int_S^P {n(s)\mathrm{d}s}\, .
\end{align}
\begin{figure}[htbp]
    \centering\includegraphics[width=0.5\linewidth]{chap06/PropagatingThroughLayeredMaterial.eps}
    \caption{在多层介质中传播的光线。}
    \label{fig:6.27}
\end{figure}

事实上,包括反射与折射在内,光的一般传播规律遵循费马原理。
它可推导出光线在真空中沿直线传播的性质以及反射定律和折射定律。
\begin{proposition}
    \keyindex{费马原理}{Fermat's principle}{}的现代形式是:
    光线从点$S$到点$P$的传播路径一定是对光程\keyindex{稳定}{stationary}{}的。
\end{proposition}

\begin{figure}[htbp]
    \centering\includegraphics[width=0.5\linewidth]{chap06/FermatsPrincipleAppliedToRefraction.eps}
    \caption{费马原理运用于折射。}
    \label{fig:6.28}
\end{figure}

如\reffig{6.28}所示,我们利用费马原理来推导斯涅尔定律:
\begin{prove}
    考虑从点$S$到点$P$的光线,它在界面上点$O$处发生折射,
    相应量已标在图中。其光程为
    \begin{align}
        OPL=n_i\overline{SO}+n_t\overline{OP}=n_i\sqrt{x^2+h^2}+n_t\sqrt{b^2+(a-x)^2}\, .
    \end{align}

    依据费马原理,我们令$\displaystyle\frac{\mathrm{d}OPL}{\mathrm{d}x}=0$,可得
    \begin{align}
        \frac{n_ix}{\sqrt{x^2+h^2}}-\frac{n_t(a-x)}{\sqrt{b^2+(a-x)^2}}=0\, .
    \end{align}
    注意到$\displaystyle\sin\theta_i=\frac{x}{\sqrt{x^2+h^2}}$以及
    $\displaystyle\sin\theta_t=\frac{(a-x)}{\sqrt{b^2+(a-x)^2}}$,带入得
    \begin{align}
        n_i\sin\theta_i=n_t\sin\theta_t\, .
    \end{align}
    即得到斯涅尔定律。
\end{prove}

\begin{figure}[htbp]
    \centering
    \subfloat[日常的反射光可视作由无数原子级散射复合而成,例如这张人脸。]
    {\includegraphics[width=0.25\linewidth]{chap06/PersonFace.eps}\label{fig:6.29.1}}\,
    \subfloat[光学系统的共轭点]{\includegraphics[width=0.7\linewidth]{chap06/ConjugateFoci.eps}\label{fig:6.29.2}}
    \caption{光的发散与汇聚。}
    \label{fig:6.29}
\end{figure}

如\reffig{6.29.1},自身发光或被照亮的物体可以视作是大量辐射点\keyindex{源}{source}{}构成的。
每个点源都在发出\keyindex{球面波}{spherical wave}{}。
此时我们说光线从给定点源\keyindex{发散}{diverge}{}。
反之,若球面波坍缩到一点,我们说光线\keyindex{汇聚}{converge}{}到该点。
实际中我们往往只处理\keyindex{波前}{wavefront}{}的一部分。
在\keyindex{几何光学}{geometrical optics}{optics光学}中,
我们研究如何通过插入反射和折射体来控制改变波前(光线)并忽略所有衍射效应。
如\reffig{6.29.2}所示,当点$S$发散的一部分波前经过\keyindex{光学系统}{optical system}{}后
汇聚到点$P$时,依据光路可逆性原理,在点$P$处发散的波前可以反向经过该系统处理后汇聚到点$S$.
我们称点$S$和点$P$互为\keyindex{共轭点}{conjugate point}{}。
对于理想光学系统,三维空间中的任意区域都会完美成像为另一区域,
前者称为\keyindex{物空间}{object space}{},后者称为\keyindex{像空间}{image space}{}。

\subsection{透镜}\label{sub:透镜}
\begin{definition}
    \keyindex{透镜}{lens}{}是重新配置能量传播分布的折射器件,
    即意味着现有介质的不连续性。
\end{definition}

如\reffig{6.30},最常见的是中间厚边缘薄的\keyindex{凸透镜}{convex lens}{lens透镜},
也称\keyindex{汇聚透镜}{converging lens}{lens透镜}、\keyindex{正透镜}{positive lens}{lens透镜},
以及中间薄边缘厚的\keyindex{凹透镜}{concave lens}{lens透镜},
也称\keyindex{发散透镜}{diverging lens}{lens透镜}、\keyindex{负透镜}{negative lens}{lens透镜}。
当一束平行光穿过凸透镜(或凹透镜)时,
光线汇聚的点(或从之发散的点)称为透镜的一个\keyindex{焦点}{focal point}{}。
对于\reffig{6.30}(b),将点源置于透镜前方光轴上的$F_1$处,光线汇聚到共轭点$F_2$.
此时若在$F_2$处放置一个光屏,则屏上会有相应发亮的像,称该像是\keyindex{实}{real}{}的。
对于\reffig{6.30}(c),将点源置于透镜前方无穷远处,从透镜射出的光是发散的,
且看起来好像是从点$F_2$发散。但在该位置的光屏上不会出现发亮的像,所以称该像是\keyindex{虚}{virtual}{}的。
\begin{figure}[htbp]
    \centering\includegraphics[width=0.8\linewidth]{chap06/HyperbolicLenses.eps}
    \caption{几种双曲面透镜横截面。其中(a)和(b)是凸透镜,(c)是凹透镜。}
    \label{fig:6.30}
\end{figure}

\subsubsection{球面透镜的折射}
现实生活中绝大多数透镜的表面是一片球面,因为它的制造难度比非球面透镜低得多。
尽管这种透镜会有\keyindex{像差}{aberration}{},但现代制造技术可以把它控制在极小范围内。
以下我们将探讨球面透镜的成像规律,并约定后文中的透镜都是关于光轴旋转对称的。
\begin{figure}[htbp]
    \centering
    \subfloat[球形界面上的折射。]{\includegraphics[width=0.48\linewidth]{chap06/RefractionSphericalInterface.eps}\label{fig:6.31.1}}\,
    \subfloat[球形界面上相同入射角的光线。]{\includegraphics[width=0.5\linewidth]{chap06/IncidentSameAngle.eps}\label{fig:6.31.2}}
    \caption{球面透镜的折射。}
    \label{fig:6.31}
\end{figure}

\reffig{6.31}展示了位于点$S$处的点源发出的波遇到
以点$C$为中心且半径为$R$的球面透镜界面后的折射情况。
外部介质和透镜的折射率分别为$n_1$和$n_2 (n_1<n_2)$.
点$V$称为界面的\keyindex{顶点}{vertex}{}。
长度$s_o=\overline{SV}$称为\keyindex{物距}{object distance}{}。
光线$\overline{SA}$在界面上点$A$发生折射后偏向局部法线,也偏向了光轴,
设于光轴交于点$P$,入射角为$\theta_i$,折射角为$\theta_t$.
注意最终过点$P$的折射光线都有相同的入射角(\reffig{6.31.2})。
长度$s_i=\overline{VP}$称为\keyindex{像距}{image distance}{}。
连线$\overline{CA}$与光轴的夹角为$\varphi$,并记
长度$\mathcal{l}_o=\overline{SA}, \mathcal{l}_i=\overline{AP}$.
因此光线从点$S$经$A$到$P$的光程为
\begin{align}
    OPL=n_1\mathcal{l}_o+n_2\mathcal{l}_i\, ,
\end{align}
其中$\mathcal{l}_o$和$\mathcal{l}_i$可分别在$\triangle SAC$和$\triangle ACP$中应用余弦定理得到:
\begin{align}
    \mathcal{l}_o & =\sqrt{R^2+(s_o+R)^2-2R(s_o+R)\cos\varphi}\, ,       \\
    \mathcal{l}_i & =\sqrt{R^2+(s_i-R)^2-2R(s_i-R)\cos(\pi-\varphi)}\, .
\end{align}

考虑到点$A$的位置由$\varphi$完全决定,而$n_1, n_2, s_o, s_i, R$等量是定值,
所以光程是关于$\varphi$的函数。
依据费马原理,令光程对$\varphi$的导数为零,得到
\begin{align}
    \frac{n_1R(s_o+R)\sin\varphi}{\mathcal{l}_o}-\frac{n_2R(s_i-R)\sin\varphi}{\mathcal{l}_i}=0\, ,
\end{align}
整理后为
\begin{align}
    \frac{n_1}{\mathcal{l}_o}+\frac{n_2}{\mathcal{l}_i}=\frac{1}{R}\left(\frac{n_2s_i}{\mathcal{l}_i}-\frac{n_1s_o}{\mathcal{l}_o}\right)\, .
\end{align}

也就是说,从点$S$到点$P$的折射光线必定遵循上述数量规律。
事实上该规律不要求$s_o$等量必须为正数。具体来说:
\begin{itemize}
    \item 当点$S$在点$V$左边时,$s_o$取正,反之取负;
    \item 当点$P$在点$V$右边时,$s_i$取正,反之取负;
    \item 当点$C$在点$V$右边时,$R$取正,反之取负。
\end{itemize}

按照这样的约定确定符号时,该规律对球面凹透镜同样适用。
然而它形式仍较复杂,我们考虑进一步化简它:当$\varphi$非常接近0时,点$A$十分接近点$V$,
且此时运用一阶近似$\sin\varphi\approx\varphi$和$\cos\varphi\approx 1$易
知$\mathcal{l}_o\approx s_o, \mathcal{l}_i\approx s_i$,此时近似有
\begin{align}\label{eq:6.24}
    \frac{n_1}{s_o}+\frac{n_2}{s_i}=\frac{n_2-n_1}{R}\, ,
\end{align}

我们称这样与光轴所成角度很小的光线为\keyindex{近轴光线}{paraxial ray}{ray光线}。
注意\refeq{6.24}在关于光轴对称的很小一片区域上与点$A$的位置无关,
我们称该区域为\keyindex{近轴区域}{paraxial region}{}。
1841年,高斯在上述近似下系统阐述了成像规律,因此也称该结论为
一阶、近轴或\keyindex{高斯光学}{Gaussian optics}{optics光学}。

如\reffig{6.32.1},当光轴上点$F_o$的像在无穷远时,$s_i=\infty$,此时的物距$s_o$称为
\keyindex{第一焦距}{first focal length}{focal length焦距}或
\keyindex{物焦距}{object focal length}{focal length焦距}$f_o$,即
\begin{align}
    f_o=\frac{n_1}{n_2-n_1}R\, .
\end{align}
点$F_o$称为\keyindex{第一焦点}{first focus}{focus焦点}或\keyindex{物焦点}{object focus}{focus焦点}。
当$f_o>0$时,焦点$F_o$在顶点$V$左侧,反之在右侧。
类似地如\reffig{6.32.2},当$s_o=\infty$时,光轴上对应成像的点$F_i$
称为\keyindex{第二焦点}{second focus}{focus焦点}或\keyindex{像焦点}{image focus}{focus焦点}。
此时的像距$s_i$称为\keyindex{第二焦距}{second focal length}{focal length焦距}或\keyindex{像焦距}{image focal length}{focal length焦距},即
\begin{align}
    f_i=\frac{n_2}{n_2-n_1}R\, .
\end{align}
且当$f_i>0$时,焦点$F_i$在顶点$V$右侧,反之在左侧。
\begin{figure}[htbp]
    \centering
    \subfloat[物焦点与物焦距]{\includegraphics[scale=1.25]{chap06/ObjectFocus.eps}\label{fig:6.32.1}}\,\,
    \subfloat[像焦点与像焦距]{\includegraphics[scale=1.25]{chap06/ImageFocus.eps}\label{fig:6.32.2}}
    \caption{球面透镜的焦距与焦点}
    \label{fig:6.32}
\end{figure}

之前我们说过,当出射光线是发散时我们称该像是虚的(\reffig{6.33.1}),
此时$s_i<0$,像点$F_i$在顶点$V$的左边;类似地,当入射光线是汇聚时我们称物是虚的
(\reffig{6.33.2}),此时$s_o<0$,物点$F_o$在顶点$V$右边。
\begin{figure}[htbp]
    \centering
    \subfloat[虚像点]{\vspace{-5cm}\includegraphics[scale=1]{chap06/VirtualImagePoint.eps}\label{fig:6.33.1}}\,\,\,
    \subfloat[虚物点]{\includegraphics[scale=1]{chap06/VirtualObjectPoint.eps}\label{fig:6.33.2}}
    \caption{虚像点和虚物点}
    \label{fig:6.33}
\end{figure}

\subsubsection{薄透镜}
只有一个元件(即两个折射界面)的称为\keyindex{简单透镜}{simple lens}{lens透镜},
多于一个元件的称为\keyindex{复合透镜}{compound lens}{lens透镜}。
此外按照透镜厚度能否忽略还可以分为\keyindex{薄透镜}{thin lens}{lens透镜}和\keyindex{厚透镜}{thick lens}{lens透镜}。
以下我们讨论简单薄透镜的成像规律。
\begin{figure}[htbp]
    \centering\includegraphics[width=0.75\linewidth]{chap06/SphericalLens.eps}
    \caption{球面透镜的成像规律}
    \label{fig:6.34}
\end{figure}

\reffig{6.34}展示了两个界面都是球面的透镜是如何折射光线的。
这里外部介质折射率为$n_m$,透镜折射率为$n_l, (n_l>n_m)$,
左边(第一)界面的球心为$C_1$,半径为$R_1>0$,顶点为$V_1$;
右边(第二)界面的球心为$C_2$,半径为$R_2<0$,顶点为$V_2$.
两个顶点距离为$d$.在近轴近似下,光轴上的点源$S$发出的光线经过
第一界面折射后的像点为$P'$,对应有物距$s_{o1}=\overline{SV_1}>0$,
像距$s_{i1}=\overline{V_1P'}<0$;
同时$P'$也是第二次折射的物点,折射后为光轴上的像点$P$,
并有物距$s_{o2}=\overline{P'V_2}>0$,像距$s_{i2}=\overline{V_2P}>0$.
根据\refeq{6.24}的结论,两次折射分别满足:
\begin{align}
    \frac{n_m}{s_{o1}}+\frac{n_l}{s_{i1}} & =\frac{n_l-n_m}{R_1}\, , \\
    \frac{n_l}{s_{o2}}+\frac{n_m}{s_{i2}} & =\frac{n_m-n_l}{R_2}\, .
\end{align}
考虑到$|s_{o2}|=|s_{i1}|+d$,结合各量正负性带入上式整理得
\begin{align}
    \frac{n_m}{s_{o1}}+\frac{n_m}{s_{i2}}=(n_l-n_m)\left(\frac{1}{R_1}-\frac{1}{R_2}\right)+\frac{n_ld}{(s_{i1}-d)s_{i1}}\, .
\end{align}
注意到对于整个透镜而言,其物距为$s_o=s_{o1}$,像距为$s_i=s_{i2}$.
当透镜很薄时,$d\approx 0$,点$V_1$和$V_2$十分靠近,所以上式右边最后一项可以舍去。
同时假设外部介质是空气,即$n_m\approx 1$,
于是我们得到:
\begin{proposition}
    \keyindex{薄透镜方程}{thin-lens equation}{}:
    \begin{align}
        \frac{1}{s_o}+\frac{1}{s_i}=(n_l-1)\left(\frac{1}{R_1}-\frac{1}{R_2}\right)\, .
    \end{align}
\end{proposition}

事实上该结论对各量取正负的情况都成立。分别令物距和像距取无穷远,
可以得到透镜对应的像焦距$f_i=\lim\limits_{s_o\to\infty}s_i$
和物焦距$f_o=\lim\limits_{s_i\to\infty}s_o$.
显然$f_i=f_o$,我们去掉下标把焦距记为$f$,则有
\begin{align}
    \frac{1}{f}=(n_l-1)\left(\frac{1}{R_1}-\frac{1}{R_2}\right)\, .
\end{align}
由此得到
\begin{proposition}
    \keyindex{高斯透镜公式}{Gaussian lens formula}{}:
    \begin{align}
        \frac{1}{s_o}+\frac{1}{s_i}=\frac{1}{f}\, .
    \end{align}
\end{proposition}

\begin{figure}[htbp]
    \centering\includegraphics[width=0.75\linewidth]{chap06/OpticalCenter.eps}
    \caption{球面透镜的光心}
    \label{fig:6.35}
\end{figure}

对于球面透镜而言,过光心的光线不改变传播方向。
如\reffig{6.35}所示,我们取一对平行平面分别与两界面相切于点$A$和点$B$.
分别过点$A$和点$B$作平面的垂线,由于这是球面透镜,它们必和光轴交于
对应球心$C_1$和$C_2$,于是$AC_1$平行于$BC_2$,且易知$A, B, C_1, C_2$四点共面。
设$AB$与$C_1C_2$交于点$O$,则有$\triangle AOC_1\sim\triangle BOC_2$,于是
\begin{align}
    \frac{|R_1|}{|R_2|}=\frac{\overline{OC_1}}{\overline{OC_2}}\, .
\end{align}
考虑到$R_1, R_2, \overline{C_1C_2}$均为定值,所以$O$是定点。
又因为$\angle OAC_1=\angle OBC_2$,所以第一界面的折射角等于第二界面的入射角,
两次折射介质相反,所以第一界面的入射角等于第二界面的折射角,即入射光线与最终出射光线同向。
也就是说,当且仅当光线经过球面透镜的定点$O$时不改变方向,称定点$O$为透镜的\keyindex{光心}{optical center}{}。
此时光线的横向偏移量正比于透镜厚度。对于薄透镜而言,我们可以近似地把整条过光心的光线视作直线。

我们已经知道平行近轴光线会被球形界面聚焦到一点。
因此如\reffig{6.36}所示,若干束与光轴倾角很小、范围很小、且垂直于界面入射的光线
对应的焦点将分布在球面$\sigma$上,其球心即界面的球心$C$.
在光线范围很小的情况下,我们可把$\sigma$视作与透镜光轴垂直的平面,
称之为\keyindex{焦平面}{focal plane}{}。
如\reffig{6.37},在近轴假设下,所有平行光束都会被球面透镜聚焦
到\keyindex{第二焦平面}{second focal plane}{focal plane焦平面}上,
也称\keyindex{后焦平面}{back focal plane}{focal plane焦平面}。
类似地还有与物焦点对应的\keyindex{第一焦平面}{first focal plane}{focal plane焦平面},
也称\keyindex{前焦平面}{front focal plane}{focal plane焦平面}。
\begin{figure}[htbp]
    \centering\includegraphics[width=0.75\linewidth]{chap06/FocusingSeveralRayBundles.eps}
    \caption{几束窄范围平行光的聚焦。}
    \label{fig:6.36}
\end{figure}
\begin{figure}[htbp]
    \centering\includegraphics[width=0.75\linewidth]{chap06/FocalPlane.eps}
    \caption{透镜的焦平面。}
    \label{fig:6.37}
\end{figure}

对于透镜有三种特殊光线(包括延长线)帮助我们确定成像情况。
如\reffig{6.38}所示,它们分别是过光心而不改变方向的光线1、
平行于光轴入射而过焦点的光线2、过焦点入射而平行于光轴出射的光线3.
出射光线的交点即定出成像的位置、大小、倒立情况。
\begin{figure}[htbp]
    \centering\includegraphics[width=\linewidth]{chap06/RealObjectAndPositiveLensAndNegativeLens.eps}
    \caption{利用三种特殊光线确定凸透镜和凹透镜对实物的成像。}
    \label{fig:6.38}
\end{figure}

我们接下来讨论薄透镜的具体成像规律。如\reffig{6.39}所示,
以焦距为$f$的凸透镜为例。分别记物和像偏移光轴的横向
\sidenote{横向指垂直于光轴的方向,下同。}距离为$y_o, y_i$,
在轴上方取正,下方取负。所以这里图中有$y_o>0, y_i<0$.
当$y_o$与$y_i$异号时,我们说像是\keyindex{倒}{inverted}{}的,
反之则是\keyindex{正}{erect}{}的。
记物到物焦点$F_o$的轴向距离为$x_o=\overline{S_1F_o}=s_o-f$,
当$S_1$在$F_o$左边时$x_o$取正,反之取负;
记像到像焦点$F_i$的轴向距离为$x_i=\overline{F_iP_1}=s_i-f$,
当$P_1$在$F_i$右边时$x_i$取正,反之取负。
\begin{figure}[htbp]
    \centering\includegraphics[width=0.75\linewidth]{chap06/ObjectAndImageLocation.eps}
    \caption{薄透镜成像时物和像的位置。}
    \label{fig:6.39}
\end{figure}

因为易知$\triangle AF_iO\sim\triangle P_2F_iP_1$,所以有
\begin{align}
    \frac{y_o}{|y_i|}=\frac{f}{x_i}\, .
\end{align}
类似地,因为$\triangle S_1F_oS_2\sim\triangle OF_oB$,所以有
\begin{align}
    \frac{y_o}{|y_i|}=\frac{x_o}{f}\, .
\end{align}
结合上面两式得到
\begin{proposition}
    薄透镜方程的\keyindex{牛顿形式}{Newtonian form}{}:
    \begin{align}
        x_ox_i=f^2\, .
    \end{align}
\end{proposition}
1704年牛顿首次在他的《Opticks》一书中阐述了该规律。
上式还说明,$x_o$与$x_i$一定同号,因此有
\begin{corollary}
    薄透镜的物和像一定在各自相应焦点的对侧。
\end{corollary}

\begin{definition}
    光学系统最终成像的横向维度与物的对应维度之比
    称为\keyindex{横向放大率}{transverse magnification}{},记作
    \begin{align}
        M_T=\frac{y_i}{y_o}\, .
    \end{align}
\end{definition}
不难发现,横向放大率还满足
\begin{align}
    M_T=-\frac{s_i}{s_o}=-\frac{x_i}{f}=-\frac{f}{x_o}\, .
\end{align}

\begin{corollary}
    $M_T>0$对应正像,$M_T<0$对应倒像。
\end{corollary}
\begin{corollary}
    单个薄透镜元件所成实像必是倒像。
\end{corollary}

\reftab{6.2}列出了薄透镜成像的规律,其中凸透镜部分与\reffig{6.40}对应。
\begin{table}[htbp]
    \centering
    \begin{tabular}{c|c|cccc}
        \toprule
        \multirow{2}{*}{\textbf{透镜类型}} & \multirow{2}{*}{\textbf{实物位置}} & \multicolumn{4}{c}{\textbf{像}}                                                                \\
        \cline{3-6}
                                           &                                    & \textbf{类型}                   & \textbf{位置}            & \textbf{朝向} & \textbf{相对大小} \\
        \midrule
        \multirow{5}{*}{凸透镜}            & $\infty>s_o>2f$                    & 实                              & $f<s_i<2f$               & 倒            & 缩小              \\
                                           & $s_o=2f$                           & 实                              & $s_i=2f$                 & 倒            & 相等              \\
                                           & $f<s_o<2f$                         & 实                              & $\infty>s_i>2f$          & 倒            & 放大              \\
                                           & $s_o=f$                            & -                               & $\pm\infty$              & -             & -                 \\
                                           & $s_o<f$                            & 虚                              & $|s_i|>s_o$              & 正            & 放大              \\
        \midrule
        凹透镜                             & 任意                               & 虚                              & $|s_i|<|f|$且$|s_i|<s_o$ & 正            & 缩小              \\
        \bottomrule
    \end{tabular}
    \caption{薄透镜对实物的成像。}
    \label{tab:6.2}
\end{table}

\begin{figure}
    \centering\includegraphics[width=0.75\linewidth]{chap06/ImageFormingBehavior.eps}
    \caption{薄凸透镜成像规律。}
    \label{fig:6.40}
\end{figure}

\begin{definition}
    光学系统轴向上像的变化量与物的变化量之比称为\keyindex{纵向放大率}{longitudinal magnification}{},即
    \begin{align}
        M_L=\frac{\mathrm{d}x_i}{\mathrm{d}x_o}\, .
    \end{align}
\end{definition}
根据薄透镜方程的牛顿形式可得
\begin{corollary}
    对于单种介质中焦距为$f$的薄透镜,其纵向放大率$M_L$与横向放大率$M_T$满足:
    \begin{align}
        M_L=-\frac{f^2}{x_o^2}=-M_T^2\, .
    \end{align}
\end{corollary}
显然恒有$M_L<0$,即指向透镜的方向会被成像为背对透镜的方向(\reffig{6.41})。
\begin{figure}[htbp]
    \centering\includegraphics[width=0.5\linewidth]{chap06/ImageOrientation.eps}
    \caption{薄透镜成像朝向。}
    \label{fig:6.41}
\end{figure}

还有一种特殊的成像情况:
即入射光线(的延长线)汇聚在透镜右侧,即物是虚的,$s_o<0$.
这种情况在复合透镜中十分常见。
\reffig{6.42}依次展示了虚物所成的正实像、倒虚像、放大正实像。
\begin{corollary}
    当物和像在透镜同侧时,两者必然是一实一虚的。
\end{corollary}
\begin{figure}[htbp]
    \centering
    \subfloat[虚物和正实像。]{\includegraphics[scale=1]{chap06/VirtualObjectAndRealUprightImage.eps}\label{fig:6.42.1}}\,
    \subfloat[虚物和倒虚像。]{\includegraphics[scale=1]{chap06/VirtualObjectAndVirtualInvertedImage.eps}\label{fig:6.42.2}}\\
    \subfloat[虚物和放大的正实像。]{\includegraphics[scale=1]{chap06/VirtualObjectAndRealEnlargedUprightImage.eps}\label{fig:6.42.3}}
    \caption{薄透镜的虚物成像。}
    \label{fig:6.42}
\end{figure}

\reffig{6.43}给出了利用焦平面来追踪光线折射方向的方法。
我们已经知道平行入射薄透镜的光线会交于焦平面上一点,反之亦然。
对于与焦平面交于点$A$的入射光线$AB$,我们作过光心的光线$AO$,则出射光线应平行于$AO$.
\begin{figure}[htbp]
    \centering\includegraphics[width=\linewidth]{chap06/FocalPlaneRayTracing.eps}
    \caption{焦平面光线追踪。}
    \label{fig:6.43}
\end{figure}

\reffig{6.44}展示了利用该方法在三个同轴薄透镜$L_1,L_2,L_3$中追踪光线的情况。
其中标出了$L_1$和$L_3$的第一焦平面和$L_2$的第二焦平面。
入射光线$A_1B_1$与$L_1$第一焦平面交于$A_1$,作过$L_1$光心$O_1$的光线$A_1O_1$,
再过点$B_1$作其平行线得到出射光线$B_1B_2$,并与$L_2$的第二焦平面交于$A_2$.
作过$L_2$光心$O_2$的光线$O_2A_2$,再过点$B_2$作其平行线得到出射光线$B_2B_3$,
并与$L_3$的第一焦平面交于点$A_3$.作过$L_3$光心$O_3$的光线$A_3O_3$,
再过点$B_3$作其平行线,得到最终出射光线。
\begin{figure}[htbp]
    \centering\includegraphics[width=0.75\linewidth]{chap06/ThreeLensesFocalPlane.eps}
    \caption{利用焦平面追踪穿过三个透镜的光线。}
    \label{fig:6.44}
\end{figure}

\subsubsection{复合透镜}
我们在每个透镜元件均是薄透镜的前提下讨论复合透镜的性质。
\reffig{6.45}展示了两个透镜元件间的距离小于其任意一个的焦距时组成的复合透镜,
即$d<f_1$且$d<f_2$.其中$P_1'$是第一个透镜的实像,同时也是第二个透镜的虚物。
$P_1$则是第二个透镜的实像,也是整个复合透镜的像。
图中还标出了其他量。依据薄透镜公式,有
\begin{align}
    \frac{1}{s_{o1}}+\frac{1}{s_{i1}} & =\frac{1}{f_1}\, , \\
    \frac{1}{s_{o2}}+\frac{1}{s_{i2}} & =\frac{1}{f_2}\, .
\end{align}
并注意到$s_{o2}=d-s_{i1}$,带入整理得像距为
\begin{align}
    s_{i2}=\frac{f_2\left(d-\displaystyle\frac{s_{o1}f_1}{s_{o1}-f_1}\right)}{d-f_2-\displaystyle\frac{s_{o1}f_1}{s_{o1}-f_1}}\, .
\end{align}

复合透镜的横向放大率$M_T$是各元件横向放大率$M_{T1}$与$M_{T2}$之积,即
\begin{align}
    M_T=M_{T1}M_{T2}=\left(-\frac{s_{i1}}{s_{o1}}\right)\left(-\frac{s_{i2}}{s_{o2}}\right)=\frac{f_1s_{i2}}{d(s_{o1}-f_1)-s_{o1}f_1}\, .
\end{align}
\begin{figure}[htbp]
    \centering\includegraphics[scale=1]{chap06/TwoThinLenses01.eps}
    \caption{两个透镜元件间的距离小于任意一个的焦距。}
    \label{fig:6.45}
\end{figure}

\reffig{6.46}展示了两个薄透镜距离大于两者焦距之和的情况,即$d>f_1+f_2$.
其有关结论是类似的。
\begin{figure}[htbp]
    \centering\includegraphics[width=\linewidth]{chap06/TwoThinLenses02.eps}
    \caption{两个透镜元件间的距离大于两者焦距之和。}
    \label{fig:6.46}
\end{figure}


\begin{definition}
    光学系统最后一个界面与将其视作整体时的第二焦点间的距离称为\keyindex{后焦距}{back focal length}{focal length焦距}(b.f.l.)。
    类似地,第一个界面的顶点到第一焦点的距离称为\keyindex{前焦距}{front focal length}{focal length焦距}(f.f.l.)。
\end{definition}

在前面介绍的两个薄透镜复合的情况下(\reffig{6.47}),容易得到
\begin{align}
    \text{f.f.l.} & =\lim\limits_{s_{i2}\rightarrow \infty}{s_{o1}}=\frac{f_1(d-f_2)}{d-(f_1+f_2)}\, , \\
    \text{b.f.l.} & =\lim\limits_{s_{o1}\rightarrow \infty}{s_{i2}}=\frac{f_2(d-f_1)}{d-(f_1+f_2)}\, .
\end{align}
\begin{figure}[htbp]
    \centering\includegraphics[scale=1]{chap06/backfocallength.eps}
    \caption{一个薄凹透镜与一个薄凸透镜复合后的后焦距。}
    \label{fig:6.47}
\end{figure}

当$d\rightarrow 0$时,意味着透镜\keyindex{紧密}{in contact}{}贴合在一起,
例如一些\keyindex{消色差双合透镜}{achromatic doublets}{}。此时有
\begin{align}
    \text{f.f.l.}=\text{b.f.l.}=\frac{f_1f_2}{f_1+f_2}\, .
\end{align}

\begin{proposition}
    两个焦距分别为$f_1$和$f_2$的薄透镜紧密贴合后(仍视作薄透镜)
    的\keyindex{有效焦距}{effective focal length}{focal length焦距}$f$为
    \begin{align}
        \frac{1}{f}=\frac{1}{f_1}+\frac{1}{f_2}\, .
    \end{align}
\end{proposition}
\begin{corollary}
    $N$个紧密贴合的薄透镜(仍视作薄透镜)的有效焦距$f$与每个透镜的焦距$f_i (i=1,2,\ldots,N)$的关系为
    \begin{align}
        \frac{1}{f}=\sum\limits_{i=1}^{N}{\frac{1}{f_i}}\, .
    \end{align}
\end{corollary}

\subsection{光圈}\label{sub:光圈}
\subsubsection{孔径光阑与视场光阑}
透镜的大小是有限的,这限制了能进入光学系统的光线范围。
此时透镜的有效直径就充当了\keyindex{光圈}{aperture}{}\sidenote{也可称作光阑。}。
如\reffig{6.48},我们称像这样决定成像光量的透镜边缘或单独的快门等为\keyindex{孔径光阑}{aperture stop}{aperture光圈}(A.S.)。
光学系统的孔径光阑是物理实体,它限制了从光轴上的物点发出且能进入系统的光束范围。
复杂相机中位于一部分元件后的可调叶片式快门通常就是孔径光阑(\reffig{6.52}),它决定了整组透镜聚集光线的能力。
限制系统所能成像的物体尺寸或角度范围的元件则称为\keyindex{视场光阑}{field stop}{aperture光圈}(F.S.),
它决定了视场范围(\reffig{6.48})。相机中一般由胶片或CCD传感器充当视场光阑来限制成像平面。
因此孔径光阑控制从物点到共轭像点的光量,而视场光阑则决定是否彻底阻挡光线。
调大孔径光阑可以增加进光能量使得像的每一点都有更大辐照度,
而调大视场光阑则使原来被挡住的物体也能成像了。
\begin{figure}[htbp]
    \centering\includegraphics{chap06/ApertureStopAndFieldStop.eps}
    \caption{孔径光阑与视场光阑。}
    \label{fig:6.48}
\end{figure}

\subsubsection{入射瞳与出射瞳}
孔径光阑的像称为\keyindex{瞳}{pupil}{},它决定给定光线能否通过整个光学系统。
\begin{definition}
    从光轴上的物点通过孔径光阑之前的元件看到的该光阑的像称为\keyindex{入射瞳}{entrance pupil}{pupil瞳}。
\end{definition}
\begin{definition}
    从光轴上的像点通过孔径光阑之后的元件看到的该光阑的像称为\keyindex{出射瞳}{exit pupil}{pupil瞳}。
\end{definition}

\reffig{6.49}展示了孔径光阑分别在透镜前方和后方时的入射瞳与出射瞳,
瞳的位置和大小均可以利用前文介绍的成像规律确定。
如果物与孔径光阑之间没有其他透镜,则孔径光阑本身就是入射瞳(\reffig{6.49.1});
如果孔径光阑与像之间没有其他透镜,则孔径光阑本身就是出射瞳(\reffig{6.49.2})。
入射瞳决定了能确实进入光学系统的光锥,出射瞳则决定离开系统的光锥。
在这任意一种光锥范围之外的任何点源发出的光线都无法到达像平面。此外,瞳和孔径光阑是共轭的。
当没有暗角时,任意进入入射瞳的发散光锥都能穿过孔径光阑并变为穿过出射瞳的汇聚光锥。

要强调的是,光轴上不同位置的物点可能对应着不同的瞳和孔径光阑。
例如\reffig{6.49.1}中,如果透镜更小一些,物体离透镜更近一些,
则通过光圈的光线可能超出了透镜有效直径范围,此时透镜本身就充当孔径光阑,瞳也相应变化。
反之,如果透镜更大一些,物体离透镜更远一些,则孔径光阑和瞳都没变。

在设计光学系统时,瞳的位置和尺寸十分重要。
例如使用一些观察设备时,人会把眼睛置于出射瞳中心。
考虑到人眼瞳孔大小根据照明水平一般在2mm至8mm间,
所以夜视望远镜的出射瞳通常至少为8mm,
而昼视设备使用3mm至4mm的出射瞳就足够了。
高能步枪的瞄准镜则会使用更大的出射瞳且在镜筒后端较远处,以避免后座力伤到眼睛。

我们把离轴物点发出并穿过孔径光阑中心的光线称为\keyindex{主光线}{chief ray}{ray光线}。
主光线进入光学系统时所沿直线会穿过入射瞳中心$E_{np}$,
且离开系统时所沿直线会穿过出射瞳中心$E_{xp}$.
它对于校正透镜设计的像差十分重要。
\begin{figure}[htbp]
    \centering
    \subfloat[孔径光阑在前方。]{\includegraphics[scale=1]{chap06/FrontApertureStop.eps}\label{fig:6.49.1}}\\
    \subfloat[孔径光阑在后方。]{\includegraphics[scale=1]{chap06/EntrancePupilAndExitPupil.eps}\label{fig:6.49.2}}
    \caption{入射瞳与出射瞳。}
    \label{fig:6.49}
\end{figure}

\reffig{6.50}则展示了三透镜系统的瞳和孔径光阑,
其中标注出了两种光线,一种是主光线,另一种则称为\keyindex{边缘光线}{marginal ray}{ray光线}。
边缘光线从光轴上的物点出发并射向入射瞳(或孔径光阑)的边缘。
\begin{figure}[htbp]
    \centering\includegraphics{chap06/PupilsAndStopsForAThree-LensSystem.eps}
    \caption{三透镜系统的瞳和孔径光阑。}
    \label{fig:6.50}
\end{figure}

当不清楚哪一个元件实际充当孔径光阑时,须让系统的每个元件被其左边的元件组合成像。
光轴上某一物点对应的张角范围最小的像即为入射瞳。
该像对应的元件即为该物点相应的孔径光阑。

注意\reffig{6.51}中,当物点在光轴上时,透镜$L_1$的边缘充当孔径光阑;
当物点远离光轴时,到达成像平面的对应光锥变窄,有效孔径光阑变小。
像这样图像靠近边缘的点会变淡(暗)的现象称为\keyindex{暗角}{vignetting}{}。
\begin{figure}[htbp]
    \centering\includegraphics[width=\linewidth]{chap06/Vignetting.eps}
    \caption{暗角。}
    \label{fig:6.51}
\end{figure}

\subsubsection{相对光圈与F值}
当我们使用透镜对较远处的物体成像时,其上某一极小区域发出
并能进入透镜的光能正比于透镜的面积,或说正比于入射瞳的面积。
不过如果光源是极窄的激光束,则这一结论不成立。
我们排除这类特殊情况,并忽略反射损耗、像差等,
则入射光能会分布到相应的成像区域(\reffig{6.52}),其辐照度反比于成像面积。
考虑到当入射瞳是圆形时,其面积正比于直径$D$.
而成像面积则是横向维度的平方,故正比于焦距的平方。
因此成像平面的辐照度正比于$\displaystyle\left(\frac{D}{f}\right)^2$.
我们称$\displaystyle\frac{D}{f}$为\keyindex{相对光圈}{relative aperture}{aperture光圈},
称其倒数为\keyindex{F值}{$f$-number}{}或\keyindex{焦比}{focal ratio}{},记作$f/\#$,即
\begin{align}
    f/\#=\frac{f}{D}\, .
\end{align}
例如光圈25mm、焦距50mm的透镜的F值为2,记作$f/2$.
F值越小进光量越大。\reffig{6.53}(a)是一个透镜后接可调叶片式光圈
时F值分别取2和4的情况。
\begin{figure}[htbp]
    \centering\includegraphics{chap06/ALarge-FormatCamera.eps}
    \caption{相机光学系统构造示意图。}
    \label{fig:6.52}
\end{figure}
\begin{figure}[htbp]
    \centering\includegraphics{chap06/F-number.eps}
    \caption{(a)遮挡透镜以改变F值。(b)镜头的可调光圈设置。}
    \label{fig:6.53}
\end{figure}

通常我们用焦距和最大光圈来指定透镜,例如镜筒上写着“50mm,$f/1.4$”。
因为摄影曝光时间正比于F值的平方,所以F值有时也称为透镜的\keyindex{速度}{speed}{}。
例如$f/1.4$的透镜是$f/2$的两倍快。
镜头光圈的F值常标为1、1.4、2、2.8、4、5.6、8、11、16和22等,此时它的最大相对光圈为$f/1$.
这些相邻光圈设置的F值依次扩大为$\sqrt{2}$倍(再舍入),因此辐照度依次减半。
所以相机用$f/1.4$曝光$\displaystyle\frac{1}{500}$秒、用$f/2$曝光$\displaystyle\frac{1}{250}$秒、
用$f/2.8$曝光$\displaystyle\frac{1}{125}$秒的进光量是相同的。


\chapterimage{Pictures/chap07/checkerboard-ref-465x930.png}
\chapter{采样与重构}\label{chap:采样与重构}
\setcounter{sidenote}{1}
尽管像pbrt那样的渲染器最终输出的是彩色像素的2D网格,
但实际上入射辐射是定义在胶片平面上的连续函数。
从该连续函数计算出离散像素值的方法会显著影响渲染器生成的最终图像的质量;
如果没有仔细执行该过程,则会出现伪影\sidenote{译者注:原文artifact。}。
反之,如果执行得很好,则为此进行相对少量的额外计算就能极大提升渲染图像的质量。

本章从介绍\emph{采样理论}开始——即从定义在连续域上的函数
取出离散样本值并用它们重建与原本类似的新函数的理论。
在采样理论和低偏差点集(一种均匀分布的样本点类型)思想的基础上,
本章定义的\refvar{Sampler}{}以不同方式生成$n$维样本向量
\footnote{回想上一章中\refvar{Camera}{}用\refvar{CameraSample}{}
    在胶片平面、透镜上以及时间域中取点——通过取用这些样本向量前几维来设定\refvar{CameraSample}{}值。}。
本章将介绍五种\refvar{Sampler}{}实现,涵盖了采样问题的各种方法。

本章以类\refvar{Filter}{}和\refvar{Film}{}作结。
\refvar{Filter}{}用于确定每个像素周围要融合多少倍样本量来计算最终像素值,
类\refvar{Film}{}则积累图像样本对图中像素的贡献量。

\section{采样理论}\label{sec:采样理论}
数字图像表示为一组像素值,通常对齐到矩形网格。
当在物理设备上展示数字图像时,这些值用于确定显示器上像素发射的光谱功率。
当考虑数字图像时,区分图像像素与显示器像素很重要,
前者表示一个函数在特定样本位置的值,后者是具有某个发光分布的物理对象
(例如对于LCD显示器,当以倾斜角度观察它时,颜色和亮度可能会极大变化)。
显示器用图像像素值在显示器表面构造新的图像函数。
该函数定义在显示器所有点位上,而不只是数字图像像素的无穷小点上。
这样取一组样本值并将其转换回连续函数的过程称为\keyindex{重建}{reconstruction}{}。

为了计算数字图像中的离散像素值,必须采样原始连续定义的图像函数。
在pbrt中,像大多数其他光线追踪渲染器那样,
获取图像函数有关信息的唯一方法就是通过追踪光线来对其采样。
例如,能计算胶片平面上两点间的图像函数变化边界的通用方法是不存在的。
尽管可以通过在像素位置上精确采样该函数来生成图像,
但通过在不同位置上取用更多样本并将这些关于图像函数的
额外信息融合到最终的像素值中能得到更好的结果。
实际上,为了有最佳质量的结果,计算像素值时应使得
在显示设备上重建的图像尽可能与虚拟相机胶片平面上的场景原始图像逼近。
注意这和希望显示器像素在其位置上取用图像函数实际值的目标有些微妙区别。
处理这一区别是本章实现的算法的主要目标
\footnote{本书中我们将忽略物理显示器像素特性相关问题并
    在显示器执行本节后面所述理想重建过程的假设下处理。
    该假设显然与真实显示器的工作方式不符,但这里它避免了不必要的复杂分析。
    \citet{GLASSNER1995}的第3章很好地处理了非理想显示设备
    及其对图像采样和重建过程的影响。}。

因为采样和重建过程涉及估值,所以它引入了称作\keyindex{混叠}{aliasing}{alias混叠}的误差,
并会以许多方式表现出来,包括锯齿状边缘或动画中的闪烁。
产生这些误差的原因是采样过程不能捕获来自连续定义的图像函数的全部信息。

作为这些思想的一个例子,考虑一个1D函数(我们也会称之为信号)即$f(x)$,
我们可以求函数定义域中任意期望位置$x'$处的值$f(x')$.
每个这样的$x'$称为\keyindex{样本位置}{sample position}{},
$f(x')$的值称为\keyindex{样本值}{sample value}{}。
\reffig{7.1}展示了光滑1D函数的样本集,以及逼近原始函数$f$的重建信号$\tilde{f}$.
本例中,$\tilde{f}$是分段线性函数,通过线性插值相邻样本值来逼近$f$
(已经熟悉采样理论的读者会认出这是用帽函数\sidenote{译者注:原文hat function。}重建的)。
因为关于$f$唯一可用的信息是来自在位置$x'$处的采样值,
且没有关于$f$在样本间特性的信息,所以$\tilde{f}$不能完全匹配$f$.
\begin{figure}[htbp]
    \centering
    \subfloat[]{\includegraphics[width=0.4\linewidth]{chap07/point-sampling.eps}}\,\,
    \subfloat[]{\includegraphics[width=0.4\linewidth]{chap07/linear-reconstruction.eps}}
    \caption{(a)通过取$f(x)$的\emph{样本点}集(标实心记),我们确定了函数在这些位置处的值。
        (b)样本值可用于\emph{重建}逼近$f(x)$的函数$\tilde{f}(x)$.
        \refsub{混叠}介绍的采样原理准确描述了关于$f(x)$的条件、
        所需样本的数目,以及使得$\tilde{f}(x)$和$f(x)$一模一样的重建技术。
        原始函数有时能只从样本点中完全重建的事实令人瞩目。}
    \label{fig:7.1}
\end{figure}

\keyindex{傅里叶分析}{Fourier analysis}{}可用于评估重建函数与原始函数间的匹配质量。
本节将用丰富细节来介绍一部分采样和重建过程中涉及的傅里叶分析主要思想,
但略去了许多性质的证明并跳过了与pbrt所用的采样算法没有直接关系的细节。
本章“扩展阅读”一节有关于这些话题详细信息的指引。

\subsection{频域与傅里叶变换}\label{sub:频域与傅里叶变换}
傅里叶分析的基础之一是\keyindex{傅里叶变换}{Fourier transform}{transform变换},
它在\keyindex{频域}{frequency domain}{}中来表示函数(我们称
通常的函数是在\keyindex{空域}{spatial domain}{}中表示的)。
考虑\reffig{7.2}中的两个函数。\reffig{7.2.1}中$x$的函数变化得相对较慢,
而\reffig{7.2.2}中的函数变化得迅速得多。称变化越慢的函数有越低频的分量。
\begin{figure}[htbp]
    \centering
    \subfloat[]{\includegraphics[width=0.32\linewidth]{chap07/func-lowfreq.eps}\label{fig:7.2.1}}\,\,\,\,
    \subfloat[]{\includegraphics[width=0.32\linewidth]{chap07/func-highfreq.eps}\label{fig:7.2.2}}
    \caption{(a)低频函数和(b)高频函数。粗略地说,函数频率越高,在给定区域内变化得越快。}
    \label{fig:7.2}
\end{figure}

\reffig{7.3}展示了这两个函数在频率空间的表示;低频函数的表示比高频函数更快变为0。
\begin{figure}[htbp]
    \centering
    \subfloat[]{\includegraphics[width=0.32\linewidth]{chap07/fourier-lowfreq.eps}}\,\,\,\,
    \subfloat[]{\includegraphics[width=0.32\linewidth]{chap07/fourier-highfreq.eps}}
    \caption{\reffig{7.2}中的函数的频率空间表示。本图展示了每个频率$\omega$对空域中每个函数的贡献。}
    \label{fig:7.3}
\end{figure}

许多函数可以分解为平移过的正弦曲线的加权和。
约瑟夫·傅里叶\sidenote{译者注:Jean-Baptiste Joseph Fourier,
    18至19世纪法国著名数学家和物理学家。其中文译名还常作“傅立叶”。}
首先描述了这一奇特事实,傅里叶变换即将函数转换为该表示。
函数的频率空间表示便于深入了解其一些特点——正弦函数的频率分布对应于原函数的频率分布。
使用该形式后,就能用傅里叶分析深入了解采样和重建过程引入的误差以及如何降低该误差带来的感知影响。

1D函数$f(x)$的傅里叶变换为\footnote{要告知读者的是,
    在不同领域中该积分前的常数并不总是一样的。例如一些作者(包括许多物理界的)
    更喜欢在两个积分前乘上$\frac{1}{\sqrt{2\pi}}$.}
\begin{align}\label{eq:7.1}
    F(\omega)=\int_{-\infty}^{\infty}f(x)\mathrm{e}^{-\mathrm{i}2\pi\omega x}\mathrm{d}x\, .
\end{align}
(回想$\mathrm{e}^{-\mathrm{i}x}=\cos x+\mathrm{i}\sin x$,其中$\mathrm{i}=\sqrt{-1}$.)
为了简便,这里我们将只考虑\keyindex{偶函数}{even function}{},
即$f(-x)=f(x)$,这种情况下$f$的傅里叶变换没有虚数项。
新函数$F$是\keyindex{频率}{frequency}{}$\omega$的函数
\footnote{本章中,我们将用符号$\omega$表示频率。在本书剩下部分中,$\omega$表示规范化的方向向量。
    这种记号的重复在使用它们的给定上下文中不应混淆。简单来说,当我们说函数的“频谱”(spectrum)时,
    我们是在说它在其频率空间表示中的频率分布,而不是和颜色相关的东西。}。
我们将用$\mathcal{F}$表示傅里叶变换运算
\sidenote{译者注:原文使用的符号是$\mathrm{F}$,这里译者换用更常用的花体$\mathcal{F}$,
    也更利于阅读时与其他符号区分。},即$\mathcal{F}\{f(x)\}=F(\omega)$.
$\mathcal{F}$显然是线性运算——即对任意标量$a$都有$\mathcal{F}\{af(x)\}=a\mathcal{F}\{f(x)\}$,
且$\mathcal{F}\{f(x)+g(x)\}=\mathcal{F}\{f(x)\}+\mathcal{F}\{g(x)\}$.

\refeq{7.1}称为\keyindex{傅里叶分析}{Fourier analysis}{}方程,有时简称\keyindex{傅里叶变换}{Fourier transform}{transform变换}。
我们也可用\keyindex{傅里叶合成}{Fourier synthesis}{}方程从频域变换回空域,
也称作\keyindex{傅里叶逆变换}{inverse Fourier transform}{transform变换}
\sidenote{译者注:大多数文献采用的傅里叶变换或逆变换定义中$\omega$是角频率,
    但本书的定义中$\omega$是频率。数值上角频率等于频率乘以$2\pi$.}:
\begin{align}\label{eq:7.2}
    f(x)=\int_{-\infty}^{\infty}F(\omega)\mathrm{e}^{\mathrm{i}2\pi\omega x}\mathrm{d}\omega\, .
\end{align}

\reftab{7.1}展示了许多重要函数及其频率空间表示
\sidenote{译者注:表中原文将频域函数写作$f(\omega)$,译者改为了$F(\omega)$.
    此外,表中原文对余弦函数和shah函数的频域表示混用了系数不同的傅里叶变换定义,
    导致与本书所采用的定义不符,译者已根据本书定义对其作了修正。
    具体推导过程可参考译者补充的\refsec{译者补充:信号处理}。}。
这些函数中许多都基于狄拉克$\delta$分布
\sidenote{译者注:也称作单位\keyindex{冲激函数}{impulse function}{}、脉冲函数。},
该空间函数的定义使得$\displaystyle\int_{-\infty}^{\infty}\delta(x)\mathrm{d}x=1$,且对任意$x\neq0$,都有$\delta(x)=0$.
这些性质的一个重要结论是\sidenote{译者注:我在原文基础上为该式加上了积分上下限以表明它是定积分。}
\begin{align*}
    \int_{-\infty}^{\infty} f(x)\delta(x)\mathrm{d}x=f(0)\, .
\end{align*}

$\delta$分布不能表示为标准数学函数,但通常可以视作以原点为中心且宽度逼近0的
单位面积矩形函数\sidenote{译者注:原文box function。}的极限。
\begin{table}[htbp]
    \centering\begin{tabular}{l p{170pt}}
        \toprule
        {\bfseries 空域}                                                   & {\bfseries 频率空间表示}                                                                     \\
        \midrule
        矩形函数:$f(x)=\left\{\begin{array}{ll}
                1, & \text{若}|x|<\frac{1}{2}, \\
                0, & \text{其他}.
            \end{array}\right.$           & sinc函数:$\displaystyle F(\omega)=\mathrm{sinc}(\omega)=\frac{\sin(\pi\omega)}{\pi\omega}$  \\
        \hline
        高斯函数:$f(x)=\mathrm{e}^{-\pi x^2}$                             & 高斯函数:$F(\omega)=\mathrm{e}^{-\pi \omega^2}$                                             \\
        \hline
        常函数:$f(x)=1$                                                   & $\delta$函数:$F(\omega)=\delta(\omega)$                                                     \\
        \hline
        余弦函数:$f(x)=\cos x$                                            & 平移的$\delta$函数:
        $F(\omega)=\frac{1}{2}(\delta(1-2\pi\omega)+\delta(1+2\pi\omega))$                                                                                                \\
        \hline
        shah函数:$\displaystyle f(x)=III_T(x)=T\sum\limits_k\delta(x-kT)$ & $\displaystyle F(\omega)=TIII_{\frac{1}{T}}(\omega)=\sum\limits_k\delta(\omega-\frac{k}{T})$ \\
        \bottomrule
    \end{tabular}
    \caption{傅里叶变换对。空域中的函数及其频率空间表示。
        因为傅里叶变换的对称性,如果左边一列被当作频率空间,
        则右边一列是这些函数的空间等价。}
    \label{tab:7.1}
\end{table}

\subsection{理想采样与重建}\label{sub:理想采样与重建}
利用频率空间分析,我们现在能正式研究采样的性质了。
回想采样过程要求我们选择一组等间隔样本位置并计算这些位置的函数值。
形式上,这对应于让该函数乘以“shah”——或称“冲激串”函数
\sidenote{译者注:也称狄拉克梳状函数。},即无数等间隔的$\delta$函数之和。
shah函数$III_T(x)$定义为\sidenote{译者注:为了和虚数单位$\mathrm{i}$更好区分,
    译者将式子中的下标$i$改为$k$,下同。}
\begin{align*}
    III_T(x)=T\sum\limits_{k=-\infty}^{\infty}\delta(x-kT)\, ,
\end{align*}
其中$T$定义了\keyindex{周期}{period}{},也称\keyindex{采样率}{sampling rate}{}。
\reffig{7.4}展示了采样的正式定义。相乘后得到函数在等间隔点处取值的无限序列:
\begin{align*}
    III_T(x)f(x)=T\sum\limits_k\delta(x-kT)f(kT)\, .
\end{align*}
\begin{figure}[htbp]
    \centering
    \subfloat[]{\includegraphics[width=0.32\linewidth]{chap07/func-to-sample.eps}}\,
    \subfloat[]{\includegraphics[width=0.32\linewidth]{chap07/shah-samples.eps}}\,
    \subfloat[]{\includegraphics[width=0.32\linewidth]{chap07/shah-sampled-function.eps}}
    \caption{形式化的采样过程。(a)函数$f(x)$乘以(b)shah函数$III_T(x)$,
        得到(c)表示其在每个样本点处取值的缩放后的$\delta$函数的无限序列。}
    \label{fig:7.4}
\end{figure}

通过选择重建滤波器函数$r(x)$并计算\keyindex{卷积}{convolution}{},
这些样本值可用于定义重建的函数$\tilde{f}$,即
\begin{align*}
    (III_T(x)f(x))\otimes r(x)\, ,
\end{align*}
其中卷积运算$\otimes$定义为
\begin{align*}
    f(x)\otimes g(x)=\int_{-\infty}^{\infty}f(x')g(x-x')\mathrm{d}x'\, .
\end{align*}

对于重建,卷积给出以样本点为中心并缩放后的重建滤波器实例加权和:
\begin{align*}
    \tilde{f}(x)=T\sum\limits_{k=-\infty}^{\infty}f(kT)r(x-kT)\, .
\end{align*}

例如\reffig{7.1}中使用了三角形重建滤波器$r(x)=\max(0,1-|x|)$.
\reffig{7.5}展示了为该例所用的缩放后的三角形函数。
\begin{figure}[htbp]
    \centering\includegraphics[width=0.6\linewidth]{chap07/func-tri-reconstruction.eps}
    \caption{虚线表示的三角形重建滤波器实例的和给出了实线表示的对原始函数的重建逼近。}
    \label{fig:7.5}
\end{figure}

为了得到直观的结果,我们经历了看似不用这么复杂的过程:
用一些方法在样本间插值也能得到重建函数$\tilde{f}(x)$.
然而通过仔细构建这些背景,傅里叶分析现在能更简单地用于该过程。

通过在频域分析采样函数,我们能更深入理解采样过程。
特别地,我们将能确定原始函数能从其在样本位置的取值中完全恢复的条件——一个很强的结论。
对于此处的讨论,我们现在假设函数$f(x)$是\keyindex{带限}{band limited}{}的——
存在某个频率$\omega_0$使得$f(x)$在大于$\omega_0$处不再包含任何频率。
根据定义,带限函数具有紧支撑\sidenote{译者注:compact support。}的频率空间表示,
即对于所有$|\omega|>\omega_0$都有$F(\omega)=0$.\reffig{7.3}中的两个频谱都是带限的。

傅里叶分析所用的一个重要思想是两个函数之积的傅里叶变换$\mathcal{F}\{f(x)g(x)\}$可
表示为它们各自傅里叶变换$F(\omega)$和$G(\omega)$的卷积:
\begin{align*}
    \mathcal{F}\{f(x)g(x)\}=F(\omega)\otimes G(\omega)\, .
\end{align*}

类似地,空域卷积等价于频域乘积:
\begin{align}\label{eq:7.3}
    \mathcal{F}\{f(x)\otimes g(x)\}=F(\omega)G(\omega)\, .
\end{align}

这些性质是傅里叶分析的标准参考文献中得来的。
利用这些思想可以发现,空域中原始的采样步骤,即shah函数与原始函数$f(x)$相乘,
可等价描述为频域中$F(\omega)$与另一shah函数的卷积。

从\reftab{7.1}中我们还知道shah函数$III_T(x)$的频谱;
周期为$T$的shah函数的傅里叶变换是另一个周期为$\displaystyle\frac{1}{T}$的shah函数。
牢记周期间的倒数关系很重要:它意味着如果样本在空域中隔得较远,
则它们在频域中离得更近。

因此采样信号的频域表示通过$F(\omega)$和新的shah函数的卷积给出。
让$\delta$函数与一个函数卷积得到该函数副本,故用shah函数卷积
得到原始函数副本的无限序列,间隔等于该shah的周期(\reffig{7.6})。
它是样本序列的频率空间表示。
\begin{figure}[htbp]
    \centering\includegraphics[width=0.45\linewidth]{chap07/func-convolve-shah.eps}
    \caption{$F(\omega)$与shah函数的卷积。结果是$F$的无数个副本。}
    \label{fig:7.6}
\end{figure}

现在我们有了该函数频谱副本的无限集,我们该怎样重建原始函数呢?
观察\reffig{7.6},答案很明显:只需要抹除除了以原点为中心外的所有频谱副本,就能得到原始的$F(\omega)$.
\begin{figure}[htbp]
    \centering
    \subfloat[]{\includegraphics[width=0.32\linewidth]{chap07/func-convolve-shah-a.eps}}\,
    \subfloat[]{\includegraphics[width=0.32\linewidth]{chap07/unit-box-filter.eps}}\,
    \subfloat[]{\includegraphics[width=0.32\linewidth]{chap07/single-func-after-box.eps}}
    \caption{(a)$F(\omega)$的副本序列和(b)合适的矩形函数相乘得到(c)原始频谱。}
    \label{fig:7.7}
\end{figure}

为了丢弃除了中间外的所有频谱副本,我们乘以具有合适宽度的矩形函数(\reffig{7.7})。
宽度为$T$的矩形函数$\textstyle\prod_T(x)$定义为
\begin{align*}
    {\textstyle\prod_T}(x)=\left\{\begin{array}{ll}
        \displaystyle\frac{1}{T}, & \text{若}\displaystyle|x|<\frac{T}{2}, \\
        0,                        & \text{其他}.
    \end{array}\right.
\end{align*}

该相乘步骤对应了用重建滤波器在空域做卷积。这是理想采样与重建过程。总结为:
\begin{align*}
    \tilde{F}=(F(\omega)\otimes III_{\frac{1}{T}}(\omega))\textstyle\prod_{\frac{1}{T}}(\omega)\, .
\end{align*}

这是个重要结论:仅仅通过采样一组均匀间隔的点,我们就能确定$f(x)$的精准频率空间表示。
除了知道该函数是带限的外,没有使用关于函数成分的额外信息。

在空域运用等价过程同样能完全恢复$f(x)$.因为矩形函数的傅里叶逆变换是sinc函数,
所以空域中的理想重建是\sidenote{译者注:原文该式继承了\reftab{7.1}的错误,
这里笔者补回了第一项的系数$\frac{1}{T}$.}
\begin{align*}
    \tilde{f}=\left(\frac{1}{T}f(x)III_T(x)\right)\otimes \mathrm{sinc}_T(x)\, ,
\end{align*}
其中$\mathrm{sinc}_T(x)=\mathrm{sinc}(Tx)$,因此\sidenote{译者注:原文该式
将$\mathrm{sinc}_T(x-kT)$误写为$\mathrm{sinc}(x-kT)$,已修正。}
\begin{align*}
    \tilde{f}(x)=\sum\limits_{k=-\infty}^{\infty}\mathrm{sinc}_T(x-kT)f(kT)\, .
\end{align*}

不幸的是,因为sinc函数有无限定义域,所以必须用所有采样值$f(kT)$来计算空域中$\tilde{f}(x)$的任一特定值。
实际实现中更爱用空间范围有限的滤波器,即使它们不能完美重建原始函数。

图形学常用的可选方法是用矩形函数做重建,即高效地对$x$附近区域内的全部样本值做平均。
通过考虑矩形滤波器的频域表现可以看到这是非常糟糕的选择:
该技术试图通过\emph{乘以sinc}来分离出函数频谱中间的副本,
这不仅在选出函数频谱中央副本方面做得很差,
还包含了无限序列中其他副本的高频贡献。

\subsection{混叠}\label{sub:混叠}
除了sinc函数无限定义域的问题外,理想采样与重建方法一个最严重的实际问题是它假设信号是带限的。
对于非带限信号,或者没能以足够高的采样率采样其频率成分的信号,
之前描述的过程会重建出与原始信号不同的函数。

成功重建的关键是用宽度合适的矩形函数相乘以完全恢复原始频谱$F(\omega)$的能力。
注意在\reffig{7.6}中,信号频谱的副本被空白空间分隔,所以能够被完美重建。
然而如果以更低的采样率采样原始函数,考虑一下会发生什么。
回想周期为$T$的shah函数$III_T$的傅里叶变换是周期为$\displaystyle\frac{1}{T}$的新shah函数。
这意味着如果空域中样本间的距离增大,频域的样本间隔会变小,
将频谱$F(\omega)$的副本挤在一起。如果副本挨得太近,它们就开始重叠。

因为副本被加在一起,所以得到的频谱看起来不再像许多原始的副本(\reffig{7.8})。
当该新频谱乘以矩形函数后,结果是相似但不等于原始$F(\omega)$的频谱:
原始信号的高频细节渗入到重建信号频谱的低频区域。
这些新的低频伪影称作\keyindex{混叠}{alias}{}(因为高频“伪装”成低频),
得到的信号被称是\keyindex{混叠的}{aliased}{alias混叠}。
\begin{figure}
    \centering
    \subfloat[]{\includegraphics[width=0.4\linewidth]{chap07/freq-space-overlap.eps}}\,
    \subfloat[]{\includegraphics[width=0.4\linewidth]{chap07/freq-space-aliasing.eps}}
    \caption{(a)采样率过低时,函数频谱副本会重叠,当执行重建时会导致(b)混叠。}
    \label{fig:7.8}
\end{figure}

\reffig{7.9}\sidenote{译者注:原文该图题注函数表达式有笔误,已修正。}
展示了欠采样并重建1D函数$f(x)=1+\cos(4\pi x^2)$时的混叠效应。
\begin{figure}[htbp]
    \centering
    \subfloat[]{\includegraphics[width=0.4\linewidth]{chap07/freq-increasing-func.eps}}\,
    \subfloat[]{\includegraphics[width=0.4\linewidth]{chap07/freq-increasing-aliased.eps}}
    \caption{函数$f(x)=1+\cos(4\pi x^2)$采样点的混叠。(a)该函数。
        (b)以0.125单位为样本间隔采样并用sinc滤波器执行完美重建后所重建出的函数。
        混叠造成原始函数的高频信息被丢失了并作为低频误差重新出现。}
    \label{fig:7.9}
\end{figure}

一种可能解决重叠频谱问题的办法是简单地增加采样率
直到频谱的副本隔得足够远而不再重叠,进而完全消除混叠。
事实上,\keyindex{采样定理}{sampling theorem}{}准确告诉我们所需的采样率。
该定理说只要均匀样本点的频率$\omega_s$大于信号中出现的最大频率$\omega_0$的两倍,
就能从样本中完美重建原始信号。该最小采样频率称为\keyindex{奈奎斯特频率}{Nyquist frequency}{frequency频率}
\sidenote{译者注:哈里·奈奎斯特(Harry Nyquist),19至20世纪瑞典裔美国著名物理学家,通讯理论的奠基者之一。}。

对于非带限信号($\omega_0=\infty$),不可能以足够高的采样率执行完美重建。
非带限信号有无限支撑的频谱,所以无论其频谱副本隔得多远(即无论我们用多高的采样率),
都总会有重叠。不幸的是,计算机图形学中要处理的函数很少是带限的。
特别地,任何不连续的函数都不是带限的,因此我们不能完美采样和重建它。
这是有道理的,因为两个样本间的函数连续性是不清楚的,样本没有提供不连续处的信息。
因此除了提高采样率外还必须用不同方法来消除混叠可能引入到渲染器结果中的误差。

\subsection{抗锯齿技术}\label{sub:抗锯齿技术}
如果不仔细对待采样和重建,则最终图像中可能出现大量伪影。
有时区分采样伪影与重建伪影很有用;确切地说,我们会称采样伪影为\keyindex{预混叠}{prealiasing}{alias混叠},
称重建伪影为\keyindex{后混叠}{postaliasing}{alias混叠}。任意想要修正这些误差的尝试都
大体划分为\keyindex{抗锯齿}{antialiasing}{}\sidenote{译者注:也称反混叠。}。
本节将回顾除了只增加采样率外的大量抗锯齿技术。

\subsubsection*{非均匀采样}
尽管知道我们要采样的图像函数有无穷的频率成分因而不能从样本点中完美重建,
但以非均匀的方式改变样本间隔有可能降低混叠的视觉影响。
如果$\xi$表示在0到1间的随机数,则基于冲激串的非均匀样本集为
\begin{align*}
    \sum\limits_{k=-\infty}^{\infty}\delta\left(x-(k+\frac{1}{2}-\xi)T\right)\, .
\end{align*}

对于不足以刻画该函数的固定采样率,均匀和非均匀采样都会得到不正确的重建信号。
然而非均匀采样倾向于将规则的混叠伪影转化为不容易引起人类视觉系统注意的噪声。
在频率空间,采样信号的副本最终也被随机平移,
所以当执行重建时结果是随机误差而不是有条理的混叠。

\subsubsection*{自适应采样}
另一个对抗混叠的建议方法是\keyindex{自适应超采样}{adaptive supersampling}{}:
如果我们能辨别出频率高于奈奎斯特上限的信号区域,
则我们可以在这些区域再取额外样本而不用承担在每处都增加采样频率所致的计算开销。
在实际中让该方法奏效是很困难的,因为寻找所有需要超采样的地方会很难。
大多数这样做的技术都基于测试相邻样本值并找到两值间有明显变化的地方;
然后假设该区域信号有较高频率。

通常相邻样本值不能确定地告诉我们它们之间到底发生了什么:
即使它们值相同,函数也可能在它们间有巨大变化。
或者相邻样本可能有相差很大的值但实际上并没有出现任何混叠。
例如,第\refchap{纹理}的纹理滤波算法全力消除场景中图像贴图和表面过程纹理造成的混叠;
我们不想让自适应采样例程在纹理值迅速变化但实际上没有出现过高频率的区域不必要地采额外样本。

\subsubsection*{预滤波}
采样理论提供的另一个消除混叠的方法是对原始函数滤波(即模糊)
使得所用采样率不能精确捕获的高频率不再保留下来。
该方法应用于第\refchap{纹理}的纹理函数。
该技术通过从被采样函数中移除信息来改变其特性,模糊一般不如混叠令人讨厌。

回想我们要用选定宽度的矩形滤波器与原始函数频谱相乘使得
奈奎斯特上限之上的频率被移除。在空域,这对应于原始函数与sinc滤波器做卷积,
\begin{align*}
    f(x)\otimes \mathrm{sinc}(2\omega_sx)\, .
\end{align*}

在实际中,我们也可以用范围有限的滤波器。该滤波器的频率空间表示能
帮助弄清它能有多逼近理想sinc滤波器的表现。

\reffig{7.10}展示了函数$1+\cos(4\pi x^2)$与\refsec{图像重建}
介绍的范围有限的sinc变种的卷积\sidenote{译者注:原文正文与插图
    题注的函数表达式均有笔误,与图示不符,译者已修改。}。
注意高频细节被消除了;该函数可用\reffig{7.9}的采样率采样和重建而无混叠。
\begin{figure}[htbp]
    \centering\includegraphics[width=0.4\linewidth]{chap07/highfreq-prefiltered.eps}
    \caption{函数$1+\cos(4\pi x^2)$与移除采样率$T=0.125$对应的
        奈奎斯特上限之外频率的滤波器的卷积。高频细节已从该函数移除掉,
        使得新函数至少能无混叠地被采样与重建。}
    \label{fig:7.10}
\end{figure}

\subsection{应用到图像合成}\label{sub:应用到图像合成}
这些思想应用到2D情况的渲染场景的图像采样和重建很简单:
我们有一张图像可视作2D图像位置$(x,y)$到辐亮度值$L$的函数:
\begin{align*}
    f(x,y)\rightarrow L\, .
\end{align*}

好消息是,有了我们的光线追踪器,我们能在我们所选的任意点$(x,y)$处求该函数的值。
坏消息是,一般不太可能在采样前对$f$预滤波来从中移除高频。
因此本章采样器使用两种策略,既将采样率提升至超过最终图像基础像素间隔,
也有非均匀分布样本以将混叠转化为噪声。

将场景函数的定义推广为也依赖于时间$t$以及采样处的透镜位置$(u,v)$的更高维函数很有用。
因为来自相机的光线基于这五个量,它们中任意一个变化都会得到不同的光线,
进而可能是不同的$f$值。对于特定的图像位置,该点的辐亮度一般
随时间(如果场景中有运动物体)和透镜上的位置(如果相机有光圈有限的透镜)变化。

更一般地,因为第\refchap{光传输I:表面反射}到第\refchap{光传输III:双向方法}
定义的许多积分器都用统计技术来估计沿给定光线的辐亮度,
所以当重复给定相同光线时它们可能返回不同的辐亮度值。
如果我们进一步将场景辐亮度函数扩展至包含积分器所用的样本值
(例如,为了照明计算而用于在面光源上选点的值),
我们就有甚至更高维的图像函数
\begin{align*}
    f(x,y,t,u,v,i_1,i_2,\ldots)\rightarrow L\, .
\end{align*}

采样好所有这些维度是高效生成高质量图像的重要一部分。
例如如果我们保证图像上位置$(x,y)$附近倾向于在透镜上有不同的$(u,v)$,
则得到的渲染图像会有更小的误差,因为每个样本更有可能考虑了其相邻样本没有考虑的场景信息。
后面几节的类\refvar{Sampler}{}会解决高效采样所有这些维度的问题。

\subsection{渲染中的混叠来源}\label{sub:渲染中的混叠来源}
几何体是在渲染图像中造成混叠的最常见因素之一。
当投影到图像平面时,物体的边界引入了\keyindex{阶跃函数}{step function}{}——
图像函数的值突然从一个值跳到另一个值。
不仅阶跃函数像前面所述那样有无穷的频率成分,
而且更糟糕的是,完美重建滤波器在运用于混叠样本时也会造成伪影:
重建函数中出现\keyindex{振铃}{ringing}{}伪影,
即称作\keyindex{吉布斯现象}{Gibbs phenomenon}{}的效应。
\reffig{7.11}为1D函数展示了该效应的例子。
正如我们将在本章后续所看到的,选择有效的重建滤波器来面对混叠需要科学、艺术以及个人品味。
\begin{figure}[htbp]
    \centering\includegraphics[width=0.5\linewidth]{chap07/gibbs-phenomenon.eps}
    \caption{吉布斯现象的图示。当没有以奈奎斯特采样率采样函数却又用
        sinc滤波器重建一组混叠的样本时,重建的函数会有“振铃”伪影,它在真实函数附近振荡。
        这里用0.125的样本间隔采样1D阶跃函数(虚线)。当用sinc重建时,振铃出现了(实线)。}
    \label{fig:7.11}
\end{figure}

场景中非常小的物体也会造成几何混叠。如果几何体小到落入图像平面样本之间,
则它会在一个动画的若干帧中不可预测地消失和重现。

混叠的另一个来源可能来自物体上的纹理和材质。没有被正确滤波的纹理贴图
(解决该问题是第\refchap{纹理}的主要内容)或光泽表面的小高光
可能造成\keyindex{着色混叠}{shading aliasing}{alias混叠}。
如果采样率没有高到足以充分采样这些特征,则会导致混叠。
此外,一个物体投射的尖锐阴影会在最终图像中引入另一个阶跃函数。
尽管有可能从图像平面上的几何边来辨别阶跃函数的位置,
但从阴影边界中检测阶跃函数则更加困难。

对于渲染图像中混叠的关键认识是,我们永远不可能移除所有这些来源,
所有我们必须开发技术来减轻其对最终图像质量的影响。

\subsection{理解像素}\label{sub:理解像素}
在本章剩余内容中牢记两个关于像素的观点很重要。
第一,一定记住构成图像的像素是图像函数在图像平面上离散位置的样本点;这样的像素没有相应的“面积”。
正如\citet{Smith95apixel}着重指出的,将像素视作具有有限面积的小方形是错误的认知模型,
会导致一系列问题。通过用信号处理的方法介绍本章话题,我们尝试为更准确的认知模型奠定基础。

第二个问题是最终图像中的像素是在像素网格上的离散整数坐标$(x,y)$处自然定义的,
但本章的\refvar{Sampler}{}是在连续的浮点位置$(x,y)$处生成图像样本的。
映射这两个域的自然方法是将连续坐标舍入到最近的离散坐标;
既然它把刚好和离散坐标有相同值的连续坐标就映射为那个离散坐标,该方法看起来不错。
然而结果是,给定覆盖范围$[x_0,x_1]$的离散坐标集,则连续坐标集覆盖范围为$\displaystyle\left[x_0-\frac{1}{2},x_1+\frac{1}{2}\right)$.
因此任何为给定离散像素范围生成连续样本位置的代码都被$\displaystyle\frac{1}{2}$的偏移量扰乱。
它们很容易被忘记并导致隐晦的错误。

如果我们改用
\begin{align*}
    d=\lfloor c\rfloor\, ,
\end{align*}
将连续坐标$c$截断为离散坐标$d$,并通过
\begin{align*}
    c=d+\frac{1}{2}\, ,
\end{align*}
将离散转换为连续,则离散范围$[x_0,x_1]$对应的连续坐标范围
自然是$[x_0,x_1+1)$且所得代码会简单得多\citep{HECKBERT1990246}。
\reffig{7.12}展示了我们将在pbrt中采用的这一转化。
\begin{figure}[htbp]
    \centering\includegraphics[width=0.5\linewidth]{chap07/Pixelsdiscretecontinuous.eps}
    \caption{图像中的像素可以解释为\emph{离散}或\emph{连续}坐标。
    离散图像五个像素的宽度覆盖了连续像素范围$[0,5)$.
    特定离散像素$d$的坐标在连续表示中为$\displaystyle d+\frac{1}{2}$.}
    \label{fig:7.12}
\end{figure}

\section{采样接口}\label{sec:采样接口}
正如先在\refsub{应用到图像合成}介绍的,
pbrt中实现的渲染方法包含了在图像平面的2D点之外的额外维度上选择样本点。
各种算法将用于生成这些点,但它们的所有实现都继承自定义其接口的抽象类\refvar{Sampler}{}。
核心采样声明和函数在文件\href{https://github.com/mmp/pbrt-v3/blob/master/src/core/sampler.h}{\ttfamily core/sampler.h}
和\href{https://github.com/mmp/pbrt-v3/blob/master/src/core/sampler.cpp}{\ttfamily core/sampler.cpp}中。
样本生成的每种实现都在目录{\ttfamily samplers/}下其自己的源文件内。

\refvar{Sampler}{}的任务是生成$[0,1)^n$中$n$维样本的序列,
其中每个图像样本都要为其生成这样的样本向量,且每个样本中的维数$n$可能会变,
这取决于光传输算法执行的计算(见\reffig{7.13})。
\begin{figure}[htbp]
    \centering\includegraphics[width=0.9\linewidth]{chap07/Samplerndimensional.eps}
    \caption{采样器为每个图像样本生成用来合成最终图像的$n$维样本向量。
        这里,像素$(3,8)$正被采样,且在该像素区域内有两个图像样本。
        样本向量的前两维给出样本在该像素内的偏移量$(x,y)$,
        接下来三维决定相应相机光线的时间和透镜位置。后续维度用于
        第\refchap{光传输I:表面反射}、\refchap{光传输II:体积渲染}和\refchap{光传输III:双向方法}中
        的蒙特卡罗光传输算法。这里,光传输算法已经请求了样本向量中的四个2D数组样本;
        例如,这些值可能用于选择面光源上的四个点来为图像样本计算辐亮度。}
    \label{fig:7.13}
\end{figure}

因为样本值必须严格小于1,所以定义一个常数\refvar{OneMinusEpsilon}{}很有用,
它表示小于1的最大可表示浮点常数。然后,我们会截断样本向量值使之不大于该值。
\begin{lstlisting}
`\initcode{Random Number Declarations}{=}`
#ifdef PBRT_FLOAT_IS_DOUBLE
static const `\refvar{Float}{}` `\initvar{OneMinusEpsilon}{}` = 0x1.fffffffffffffp-1;
#else
static const `\refvar{Float}{}` OneMinusEpsilon = 0x1.fffffep-1;
#endif
\end{lstlisting}

可能最简单的\refvar{Sampler}{}实现是当每次需要样本向量的额外分量时直接返回$[0,1)$内的均匀随机值。
这样的采样器可产生正确的图像但会需要非常多的样本(以及更多要追踪的光线与更多的时间)来
创建用更先进采样器所能取得的相同质量的图像。
使用更佳采样模式的运行时间开销大致和用诸如均匀随机数的低质量模式相同;
因为为每个图像样本计算辐亮度比计算样本的分量值会有大得多的开销,
所以这样做是有回报的(\reffig{7.14})。
\begin{figure}[htbp]
    \centering
    \subfloat[差的采样]{\includegraphics[width=\linewidth]{chap07/spheres-bad-sampler.png}\label{fig:7.14.1}}\\
    \subfloat[更好的采样]{\includegraphics[width=\linewidth]{chap07/spheres-better-sampler.png}\label{fig:7.14.2}}
    \caption{用(a)相对低效的采样器和(b)精心设计的采样器渲染的场景,
        它们用了同样多的样本。从高光边缘到光泽反射,图像质量的提升是明显的。}
    \label{fig:7.14}
\end{figure}

下面假设这些样本向量的一些特性:
\begin{itemize}
    \item \refvar{Sampler}{}生成的前五维通常由\refvar{Camera}{}使用。
          这种情况下,前两个专门用于选择图像上当前像素区域内的点;
          第三个用于计算应该取用该样本的时间;第四和五维为景深给出透镜位置$(u,v)$.
    \item 一些采样算法在样本向量的某些维度中生成的样本比其他维度更好。
          在系统其他地方,我们假设一般前面的维度具有放置得最好的样本值。
\end{itemize}

还要注意\refvar{Sampler}{}生成的$n$维样本通常不会整个显式表示或存储,
而是常常按照光传输算法的需要逐步生成。(然而,存储整个样本向量并对其分量逐渐作出调整
是\refsub{基本样本空间采样器}中\refvar{MLTSampler}{}的基础,
它用于\refsub{MLT积分器}的\refvar{MLTIntegrator}{}。)

\subsection{评估样本模式:偏差}\label{sub:评价样本模式:偏差}
\begin{remark}
    本节含有高级内容,第一次阅读时可以跳过。
\end{remark}

傅里叶分析给了我们一种方法来评估2D采样模式的质量,
但它只是让我们能够在可表示的带限频率方面量化增加更均匀间隔的样本所带来的提升。
考虑到图像中出现了来自边缘的无穷频率成分以及蒙特卡罗光传输算法对$(n>2)$维样本向量的需求,
傅里叶分析对于我们的需求而言是不够的。

给定一个渲染器和放置样本的候选算法,一种评估该算法效果的方法是用其
采样模式来渲染图像并计算它和用大量样本渲染的参考图像相比的误差。
本章后面我们将用该方法比较采样算法,不过它只告诉了我们该算法对于特定场景的表现如何,
且若没有经过渲染过程它将不能让我们感觉出样本点的质量。

除了傅里叶分析,数学家还发明了一个称作\keyindex{偏差}{discrepancy}{}的概念
用于评估$n$维样本位置模式的质量。分布良好(稍后形式化说明)的模式有低的偏差值,
且因此该样本模式生成问题可以考虑成寻找点的合适的\emph{低偏差}模式
\footnote{当然,这样使用偏差隐含假设了用于计算偏差的度量
    对于图像采样而言是与模式的质量有良好关联性的,这可能会有所区别,
    尤其是当人类视觉系统参与该过程时。}。
大量确定性技术已经被开发出来,甚至能在高维空间中生成低偏差点集
(本章后面所用的大多数采样算法都使用这些技术)。

偏差的基本思想是$n$维空间$[0,1)^n$中点集的质量可通过查看域$[0,1)^n$中的各区域、
数出每个区域内的点数并拿每个区域的体积和其内的样本点数作比较来评估。
通常,给定的某一占比体积内应该大致含有样本点总数的相同比例。
尽管不可能总是这种情况,但我们仍可尽量使用让实际体积与点估计的体积间的最大差异(即偏差)最小化的模式。
\reffig{7.15}展示了该思想在二维下的例子。
\begin{figure}[htbp]
    \centering\includegraphics[width=0.5\linewidth]{chap07/Boxdiscrepancy.eps}
    \caption{给定$[0,1)^2$中2D样本点集后矩形(阴影)的偏差。
    四个样本点中的一个在矩形内,所以该点集将把矩形的面积估为$\frac{1}{4}$.
    该矩形是真实面积是$0.3\times0.3=0.09$,所以该特定矩形的偏差
    为$0.25-0.09=0.16$.通常,我们关心的是找出所有可能的矩形(或某些其他形状)中的最大偏差。}
    \label{fig:7.15}
\end{figure}

为了计算点集的偏差,我们首先取作为$[0,1)^n$子集的一簇形状$B$.
例如常用一个角位于原点的方盒。其对应于
\begin{align*}
    B=\{[0,v_1]\times[0,v_2]\times\cdots\times[0,v_n]\}\, ,
\end{align*}
其中$0\le v_i<1$.给定样本点序列$P=x_1,\ldots,x_N$,
$P$关于$B$的偏差为\footnote{算符$\sup$,也称作\emph{上确界},给出了定义域内函数值的最紧上界。}
\begin{align}\label{eq:7.4}
    D_N(B,P)=\sup\limits_{b\in B}\left|\frac{\#\{x_i\in b\}}{N}-V(b)\right|\, ,
\end{align}
其中$\#\{x_i\in b\}$是$b$中的点数,$V(b)$是$b$的体积。

对于为什么\refeq{7.4}是合理的质量度量的直观解释是,
值$\displaystyle\frac{\#\{x_i\in b\}}{N}$是
由特定点集$P$给出的方盒$b$体积的近似。
因此,偏差是所有可能的方盒用这种办法逼近其体积时的最差误差。
当形状集$B$是一个角在原点的方盒集时,
该值称为\keyindex{星偏差}{star discrepancy}{discrepancy偏差}
\sidenote{译者注:也称“均匀性偏差”。},$D^*_N(P)$.
对于$B$的另一个流行选择是全体轴对齐框的集合,即去掉了一个角在原点的限制。

对于一些特定点集,可以解析计算偏差。例如考虑一维中的点集
\begin{align*}
    x_i=\frac{i}{N}\, .
\end{align*}
我们可以看到$x_i$的星偏差为\sidenote{译者注:原文将$x_N$误写为$x_n$,已修正。}
\begin{align*}
    D^*_N(x_1,\ldots,x_N)=\frac{1}{N}\, .
\end{align*}
例如,取区间$b=\displaystyle\left[0,\frac{1}{N}\right)$.则$V(b)=\displaystyle\frac{1}{N}$,
但$\#\{x_i\in b\}=0$.该区间(以及区间$\displaystyle\left[0,\frac{2}{N}\right)$等)
的体积与体积内所见点的比例有最大的差异。

该序列的星偏差可通过稍微对其改动来改进:
\begin{align}\label{eq:7.5}
    x_i=\frac{i-\frac{1}{2}}{N}\, .
\end{align}
则
\begin{align*}
    D^*_N(x_i)=\frac{1}{2N}\, .
\end{align*}
说明一维点序列的星偏差边界为
\sidenote{译者注:这里我简单推导了该式:假设$P=\{x_i\}_{i=1}^{N}$已经按照
升序排列。构造一个新序列$Q=\left\{\frac{j}{N}\right\}_{j=0}^{N}$,然后让$P$和$Q$的元素交错排列构造
新序列$U=\{u_k\}_{k=1}^{2N+1}=\left\{0,x_1,\frac{1}{N},\ldots,x_i,\frac{i}{N},\ldots,x_N,1\right\}$.
则依据偏差的定义可以证明,$D^*_N(x_i)=\max\limits_{1\le k\le 2N}{|u_{k+1}-u_k|}=\max\limits_{1\le i\le N}{d_i}$,
其中$d_i=\max{\left\{\left|x_i-\frac{i-1}{N}\right|,\left|x_i-\frac{i}{N}\right|\right\}}$.
又注意到$d_i=\frac{1}{2N}+\left|x_i-\frac{2i-1}{2N}\right|$,于是得到文中该式。}
\begin{align*}
    D^*_N(x_i)=\frac{1}{2N}+\max\limits_{1\le i\le N}\left|x_i-\frac{2i-1}{2N}\right|\, .
\end{align*}
因此,之前\refeq{7.5}中的序列具有1D序列中所能取到的最低偏差。
通常,分析和计算1D序列的偏差边界比高维简单得多。
对于构造更复杂的点序列、高维序列以及比方盒更不规则的形状,
通常必须通过构造大量形状$b$、计算其偏差并报告找到的最大值来数值地估计偏差。

聪明的读者会注意到根据低偏差度量,1D中该均匀序列是最优的,
但本章前面我们说过,对于2D中的图像采样,不规则的抖动模式优于均匀模式,
因为它们将混叠替换为噪声。在这一框架下,均匀样本显然不是最好的。
幸运的是,更高维的低偏差模式比在一维中更不均匀得多,
因此实际中通常能作为样本模式工作得很好。
然而,其根本的均匀性意味着低偏差模式比伪随机变化的模式更可能倾向于视觉上令人讨厌的混叠。

只靠偏差并不一定是好的度量:一些低偏差点集表现出样本的聚集性,
其中两个或以上样本可能靠得很近。\refsec{Sobol采样器}的Sobol采样器
尤其困扰于该问题——见\reffig{7.36},它展示了其前两维的图示。
直觉上,靠得太近的样本不能很好地利用采样资源:
一个样本离另一个越近,它就越不可能给出关于被采样函数的新信息。
因此,计算点集中任意两个样本间的最小距离也已被证明是一种有用的样本模式质量度量;最小距离越大越好。

有各种算法用来生成在该度量下得分不错的\keyindex{泊松圆盘}{Poisson disk}{}采样模式。
通过构造,泊松圆盘模式内没有两个点比某一距离$d$更近。
研究已表明眼睛的视杆细胞和视锥细胞也按该方式分布,
这进一步验证了该分布适合用来成像的观点。
实际中,我们发现泊松圆盘模式对于采样2D图像工作得很好,
但对于更复杂的渲染情形中的更高维采样会比更好的低偏差模式更低效;
见“扩展阅读”一节了解更多信息。

\subsection{基本采样器接口}\label{sub:基本采样器接口}
基类\refvar{Sampler}{}不仅定义了采样器的接口,
还提供了一些通用功能供\refvar{Sampler}{}
的实现使用。
\begin{lstlisting}
`\initcode{Sampler Declarations}{=}\initnext{SamplerDeclarations}`
class `\initvar{Sampler}{}` {
public:
    `\refcode{Sampler Interface}{}`
    `\refcode{Sampler Public Data}{}`
protected:
    `\refcode{Sampler Protected Data}{}`
private:
    `\refcode{Sampler Private Data}{}`
};
\end{lstlisting}

所有\refvar{Sampler}{}实现必须提供指定了要为最终图像中每个像素生成的样本数量的构造函数。
在罕见情况下,将胶片建模为只有单个覆盖整个可视区域的“像素”可能对系统有用
(这种过载的像素定义有点夸张,但我们允许它简化某些实现方面)。
既然该“像素”可能有数十亿个样本,我们就用64位精度的变量来存储样本数量。
\begin{lstlisting}
`\initcode{Sampler Method Definitions}{=}\initnext{SamplerMethodDefinitions}`
`\refvar{Sampler}{}`::`\refvar{Sampler}{}`(int64_t samplesPerPixel)
: `\refvar{samplesPerPixel}{}`(samplesPerPixel) { }
\end{lstlisting}
\begin{lstlisting}
`\initcode{Sampler Public Data}{=}`
const int64_t `\initvar{samplesPerPixel}{}`;
\end{lstlisting}

当渲染算法准备好在给定像素上工作时,它通过提供
该像素在图像中的坐标并调用\refvar{StartPixel}{()}来开始。
一些\refvar{Sampler}{}实现利用哪个像素正被采样的知识来提升
其为该像素生成的样本整体分布,而其他的则忽略该信息。
\begin{lstlisting}
`\initcode{Sampler Interface}{=}\initnext{SamplerInterface}`
virtual void `\initvar{StartPixel}{}`(const `\refvar{Point2i}{}` &p);
\end{lstlisting}

方法\refvar{Get1D}{()}为当前样本向量的下一维返回样本值,
\refvar{Get2D}{()}则为下两维返回样本值。
尽管能通过使用调取一对\refvar{Get1D}{()}返回的值来构造2D样本值,
但一些采样器如果知道两维会一起用时能够生成更好的点分布。
\begin{lstlisting}
`\refcode{Sampler Interface}{+=}\lastnext{SamplerInterface}`
virtual `\refvar{Float}{}` `\initvar{Get1D}{}`() = 0;
virtual `\refvar{Point2f}{}` `\initvar{Get2D}{}`() = 0;
\end{lstlisting}

在pbrt中,我们不支持从采样器中获取3D或更高维度的样本值,
因为它们一般对于这里实现的渲染算法类型而言是非必需的。
如果需要,可用来自低维分量的多个值来构造高维样本点。

这些接口的一个显著特点是必须仔细编写使用样本值的代码使其
总是以同样的顺序获取样本维度。考虑下列代码:\\
{\ttfamily
sampler->StartPixel(p);\\
do \{\\
\indent Float v = a(sampler->Get1D());\\
\indent if (v > 0)\\
\indent \indent v += b(sampler->Get1D());\\
\indent v += c(sampler->Get1D());\\
\} while (sampler->StartNextSample());
}

情况下,样本向量的第一维总会传给函数{\ttfamily a()};
当执行调用{\ttfamily b()}的代码路径时,{\ttfamily b()}会收到第二维。
然而,若{\ttfamily if}测试并不总是为真或假,则{\ttfamily c()}有时
会从样本向量的第二维收到样本,否则从第三维收到样本。
因此,采样器为了提供在每个维度评估都分布良好的样本点所作的努力就白费了。
故需要仔细编写使用\refvar{Sampler}{}的代码使得它始终如一地用掉样本向量维度以避免该问题。

为了方便,基类\refvar{Sampler}{}提供了为给定像素初始化\refvar{CameraSample}{}的方法。
\begin{lstlisting}
`\refcode{Sampler Method Definitions}{+=}\lastnext{SamplerMethodDefinitions}`
`\refvar{CameraSample}{}` `\refvar{Sampler}{}`::`\initvar{GetCameraSample}{}`(const `\refvar{Point2i}{}` &pRaster) {
    `\refvar{CameraSample}{}` cs;
    cs.`\refvar{pFilm}{}` = (`\refvar{Point2f}{}`)pRaster + `\refvar{Get2D}{}`();
    cs.`\refvar[CameraSample::time]{time}{}` = `\refvar{Get1D}{}`();
    cs.`\refvar{pLens}{}` = `\refvar{Get2D}{}`();
    return cs;
}
\end{lstlisting}

一些渲染算法为其采样的某些维度利用了样本值数组;
比起生成一系列单独的样本,大多数样本生成算法通过考虑
数组所有元素上以及一个像素内所有样本上的样本值分布可生成更高质量的样本数组。

如果需要样本数组,则必须在渲染开始前请求之。
在渲染开始前——例如在重写了方法\refvar{SamplerIntegrator::Preprocess}{()}的方法中,
应该为每个这样维度的数组调用方法\refvar[Request1DArray]{Request[12]DArray}{()}。
例如,在具有两个面光源的场景中,当积分器追踪了四条阴影射线到第一个光源,
八条到第二个光源时,积分器会为每个图像样本请求两个2D样本数组,
每个分别有四个和八个样本(需要2D数组是因为需要两个维度来参数化光源表面)。
在\refsec{俄罗斯轮盘赌与划分}中,我们将看到使用样本数组会怎样对应于
用“划分”\sidenote{译者注:原文splitting。}的蒙特卡罗技术更密集地采样光传输积分的某些维度。
\begin{lstlisting}
`\refcode{Sampler Interface}{+=}\lastnext{SamplerInterface}`
void `\initvar{Request1DArray}{}`(int n);
void `\initvar{Request2DArray}{}`(int n);
\end{lstlisting}

大多数\refvar{Sampler}{}能更好地生成某些特定大小的数组。
例如,在数量为2的幂时,来自\refvar{ZeroTwoSequenceSampler}{}的样本
则分布得好得多。方法\linebreak
\refvar[RoundCount]{Sampler::RoundCount}{()}
帮助传达该信息。需要样本数组的代码应该用想要取用的样本数目调用该方法,
以给\refvar{Sampler}{}机会把样本数目调整到更好的值。
然后应该用返回的值作为从\refvar{Sampler}{}实际请求样本的数目。
默认实现返回不变的给定数目。
\begin{lstlisting}
`\refcode{Sampler Interface}{+=}\lastnext{SamplerInterface}`
virtual int `\initvar{RoundCount}{}`(int n) const {
    return n;
}
\end{lstlisting}

在渲染时,可调用方法\refvar[Get1DArray]{Get[12]DArray}{()}获取
指向之前请求的当前维度下样本数组起点的指针。
在\refvar{Get1DArray}{()}和\refvar{Get2DArray}{()}的代码行中
\sidenote{译者注:原文疑似笔误写成了\refvar{Get1D}{()}和\refvar{Get2D}{()},已修正。},
它们返回指向样本数组的指针,数组大小由参数{\ttfamily n}传给
初始化时对\refvar[Request1DArray]{Request[12]DArray}{()}
的相应调用。调用者也必须提供数组大小以“获取”方法用于验证返回的缓冲区具有期望大小。
\begin{lstlisting}
`\refcode{Sampler Interface}{+=}\lastnext{SamplerInterface}`  
const `\refvar{Float}{}` *`\initvar{Get1DArray}{}`(int n);
const `\refvar{Point2f}{}` *`\initvar{Get2DArray}{}`(int n);
\end{lstlisting}

为一个样本完成这些工作后,积分器调用\refvar{StartNextSample}{()}。
该调用为当前像素通知\refvar{Sampler}{}针对
样本分量的后续请求应该返回下一个样本从第一维起的值。
该方法返回{\ttfamily true},直到已经为每个像素生成了请求的原始数目样本
(此时调用者应该要么在另一个像素上开始工作要么停止试图使用更多样本)。
\begin{lstlisting}
`\refcode{Sampler Interface}{+=}\lastnext{SamplerInterface}`
virtual bool `\initvar{StartNextSample}{}`();
\end{lstlisting}

\refvar{Sampler}{}的实现存储了关于当前样本的各种状态:
正在采样哪个像素,用了该样本的多少维度等等。
因此对于多个线程同时使用的单个\refvar{Sampler}{}而言自然是不安全的。
方法\refvar{Clone}{()}生成一个初始\refvar{Sampler}{}的新实例给渲染线程使用;
它为采样器的随机数生成器(如果有)接收一个种子值,这样不同线程会有不同的随机数序列。
在多个图块间复用相同伪随机数序列可能导致微妙的图像伪影,例如重复的噪声模式。

方法\refvar{Clone}{()}的各种实现一般并不有趣,所以这里文中没有包含它们。
\begin{lstlisting}
`\refcode{Sampler Interface}{+=}\lastnext{SamplerInterface}`
virtual std::unique_ptr<`\refvar{Sampler}{}`> `\initvar{Clone}{}`(int seed) = 0;
\end{lstlisting}

一些光传输算法(特别是\refsec{随机渐进光子映射}的随机渐进光子映射)
在进行到下一像素前并不使用当前像素内的所有样本,
而是跳跃到周围的像素,每个里面每次取一个样本。
方法\refvar{SetSampleNumber}{()}允许积分器在当前像素内设置样本的索引以生成下一个。
一旦{\ttfamily sampleNum}大于或等于每个像素请求的原始样本数目该方法就返回{\ttfamily false}。
\begin{lstlisting}
`\refcode{Sampler Interface}{+=}\lastcode{SamplerInterface}`
virtual bool `\initvar{SetSampleNumber}{}`(int64_t sampleNum);
\end{lstlisting}

\subsection{采样器实现}\label{sub:采样器实现}
基类\refvar{Sampler}{}在其接口内提供了一些方法的实现。
首先,方法\refvar[Sampler::StartPixel]{StartPixel}{()}的实现记录当前正被采样的像素坐标
并置零\refvar{currentPixelSampleIndex}{}即像素中当前正被生成的样本数量。
注意这是有一个实现的虚方法;重载该方法的子类需要显式调用\refvar{Sampler::StartPixel}{()}。
\begin{lstlisting}
`\refcode{Sampler Method Definitions}{+=}\lastnext{SamplerMethodDefinitions}`
void `\initvar[Sampler::StartPixel]{\refvar{Sampler}{}::\refvar{StartPixel}{}}{}`(const `\refvar{Point2i}{}` &p) {
    `\refvar{currentPixel}{}` = p;
    `\refvar{currentPixelSampleIndex}{}` = 0;
    `\refcode{Reset array offsets for next pixel sample}{}`
}
\end{lstlisting}

\refvar{Sampler}{}子类可获取当前像素坐标和像素内的样本数量,
但它们应当将其作为只读值对待。
\begin{lstlisting}
`\initcode{Sampler Protected Data}{=}\initnext{SamplerProtectedData}`
`\refvar{Point2i}{}` `\initvar{currentPixel}{}`;
int64_t `\initvar{currentPixelSampleIndex}{}`;
\end{lstlisting}

当像素样本被更新或显式设置时,\refvar{currentPixelSampleIndex}{}也随之更新。
像\refvar[Sampler::StartPixel]{StartPixel}{()}那样,
方法\refvar[Sampler::StartNextSample]{StartNextSample}{()}和\refvar[Sampler::SetSampleNumber]{SetSampleNumber}{()}也
都是虚实现;这些实现也必须由\refvar{Sampler}{}子类中重载它们的实现来显式调用。
\begin{lstlisting}
`\refcode{Sampler Method Definitions}{+=}\lastnext{SamplerMethodDefinitions}`
bool `\initvar[Sampler::StartNextSample]{\refvar{Sampler}{}::\refvar{StartNextSample}{}}{}`() {
    `\refcode{Reset array offsets for next pixel sample}{}`
    return ++`\refvar{currentPixelSampleIndex}{}` < `\refvar{samplesPerPixel}{}`;
}
\end{lstlisting}
\begin{lstlisting}
`\refcode{Sampler Method Definitions}{+=}\lastnext{SamplerMethodDefinitions}`
bool `\initvar[Sampler::SetSampleNumber]{\refvar{Sampler}{}::\refvar{SetSampleNumber}{}}{}`(int64_t sampleNum) {
    `\refcode{Reset array offsets for next pixel sample}{}`
    `\refvar{currentPixelSampleIndex}{}` = sampleNum;
    return `\refvar{currentPixelSampleIndex}{}` < `\refvar{samplesPerPixel}{}`;
}
\end{lstlisting}

基类\refvar{Sampler}{}的实现也仔细记录
对样本分量数组的请求并为这些值分配存储空间。
所需的样本数组大小存于\refvar{samples1DArraySizes}{}和\refvar{samples2DArraySizes}{},
整个像素的样本数组值的内存分配于\refvar{sampleArray1D}{}和\refvar{sampleArray2D}{}。
每份分配中前{\ttfamily n}个值用于像素中首个样本的相应数组,以此类推。
\begin{lstlisting}
`\refcode{Sampler Method Definitions}{+=}\lastnext{SamplerMethodDefinitions}`
void `\initvar[Sampler::Request1DArray]{\refvar{Sampler}{}::\refvar{Request1DArray}{}}{}`(int n) {
    `\refvar{samples1DArraySizes}{}`.push_back(n);
    `\refvar{sampleArray1D}{}`.push_back(std::vector<`\refvar{Float}{}`>(n * `\refvar{samplesPerPixel}{}`));
}
\end{lstlisting}
\begin{lstlisting}
`\refcode{Sampler Method Definitions}{+=}\lastnext{SamplerMethodDefinitions}`
void `\initvar[Sampler::Request2DArray]{\refvar{Sampler}{}::\refvar{Request2DArray}{}}{}`(int n) {
    `\refvar{samples2DArraySizes}{}`.push_back(n);
    `\refvar{sampleArray2D}{}`.push_back(std::vector<`\refvar{Point2f}{}`>(n * `\refvar{samplesPerPixel}{}`));
}
\end{lstlisting}
\begin{lstlisting}
`\refcode{Sampler Protected Data}{+=}\lastcode{SamplerProtectedData}`
std::vector<int> `\initvar{samples1DArraySizes}{}`, `\initvar{samples2DArraySizes}{}`;
std::vector<std::vector<`\refvar{Float}{}`>> `\initvar{sampleArray1D}{}`;
std::vector<std::vector<`\refvar{Point2f}{}`>> `\initvar{sampleArray2D}{}`;
\end{lstlisting}

像方法\refvar[Get1DArray]{Get[12]DArray}{()}获取当前样本内的数组那样,\refvar{array1DOffset}{}和
\refvar{array2DOffset}{}被更新成将为样本向量返回的下一数组的索引。
\begin{lstlisting}
`\initcode{Sampler Private Data}{=}`
size_t `\initvar{array1DOffset}{}`, `\initvar{array2DOffset}{}`;
\end{lstlisting}
当处理新像素或当前像素中样本数量改变时,这些数组偏移量必须重置为0.
\begin{lstlisting}
`\initcode{Reset array offsets for next pixel sample}{=}`
`\refvar{array1DOffset}{}` = `\refvar{array2DOffset}{}` = 0;
\end{lstlisting}

要返回合适的数组指针,首先要基于当前样本向量内已经消耗了多少来选择合适的数组,
然后基于当前像素样本索引返回其合适的实例。
\begin{lstlisting}
`\refcode{Sampler Method Definitions}{+=}\lastnext{SamplerMethodDefinitions}`
const `\refvar{Float}{}` *`\initvar[Sampler::Get1DArray]{\refvar{Sampler}{}::\refvar{Get1DArray}{}}{}`(int n) {
    if (`\refvar{array1DOffset}{}` == `\refvar{sampleArray1D}{}`.size())
        return nullptr;
    return &`\refvar{sampleArray1D}{}`[`\refvar{array1DOffset}{}`++][`\refvar{currentPixelSampleIndex}{}` * n];
}
\end{lstlisting}
\begin{lstlisting}
`\refcode{Sampler Method Definitions}{+=}\lastnext{SamplerMethodDefinitions}`
const `\refvar{Point2f}{}` *`\initvar[Sampler::Get2DArray]{\refvar{Sampler}{}::\refvar{Get2DArray}{}}{}`(int n) {
    if (`\refvar{array2DOffset}{}` == `\refvar{sampleArray2D}{}`.size())
        return nullptr;
    return &`\refvar{sampleArray2D}{}`[`\refvar{array2DOffset}{}`++][`\refvar{currentPixelSampleIndex}{}` * n];
}
\end{lstlisting}

\subsection{像素采样器}\label{sub:像素采样器}
尽管一些采样算法很容易递进生成每个样本向量的元素,但其他算法会更自然地为一个像素同时生成
所有样本向量所有维度上的样本值。类\refvar{PixelSampler}{}
实现了一些对该类采样器的实现有用的功能。
\begin{lstlisting}
`\refcode{Sampler Declarations}{+=}\lastnext{SamplerDeclarations}`
class `\initvar{PixelSampler}{}` : public `\refvar{Sampler}{}` {
public:
    `\refcode{PixelSampler Public Methods}{}`
protected:
    `\refcode{PixelSampler Protected Data}{}`
};
\end{lstlisting}
\begin{lstlisting}
`\initcode{PixelSampler Public Methods}{=}`
`\refvar{PixelSampler}{}`(int64_t samplesPerPixel, int nSampledDimensions);
bool `\refvar[PixelSampler::StartNextSample]{StartNextSample}{}`();
bool `\refvar[PixelSampler::SetSampleNumber]{SetSampleNumber}{}`(int64_t);
`\refvar{Float}{}` `\refvar[PixelSampler::Get1D]{Get1D}{}`();
`\refvar{Point2f}{}` `\refvar[PixelSampler::Get2D]{Get2D}{}`();
\end{lstlisting}

渲染算法要用的样本向量维数是不能提前知道的
(确实,它只隐式取决于调用\refvar{Get1D}{()}和\refvar{Get2D}{()}的次数
以及请求的数组)。因此,\refvar{PixelSampler}{}构造函数
接收\refvar{Sampler}{}要计算的非数组样本值的最大维数。
如果所有这些分量维度都用掉了,则\refvar{PixelSampler}{}直接为额外维度返回均匀随机值。

对于每个预先计算的维度,构造函数都分配一个{\ttfamily vector}来存储样本值,
像素内的每个样本对应一个值。这些向量按{\ttfamily\refvar{samples1D}{}[dim][pixelSample]}来索引
\sidenote{译者注:原文将\refvar{samples1D}{}误写为{\ttfamily sample1D},已修正。};
尽管交换这些索引的顺序可能看起来更合理,但现在这样的内存排布——
对于给定维度,所有样本的所有样本分量值在内存中是连续的
\sidenote{译者注:指这些值的内存地址是连续的。}——
对于生成这些值的代码而言变得更方便了。
\begin{lstlisting}
`\refcode{Sampler Method Definitions}{+=}\lastnext{SamplerMethodDefinitions}`
`\refvar{PixelSampler}{}`::`\refvar{PixelSampler}{}`(int64_t samplesPerPixel,
        int nSampledDimensions)
    : `\refvar{Sampler}{}`(samplesPerPixel) {
    for (int i = 0; i < nSampledDimensions; ++i) {
        `\refvar{samples1D}{}`.push_back(std::vector<`\refvar{Float}{}`>(samplesPerPixel));
        `\refvar{samples2D}{}`.push_back(std::vector<`\refvar{Point2f}{}`>(samplesPerPixel));
    }
}
\end{lstlisting}

继承自\refvar{PixelSampler}{}的\refvar{Sampler}{}实现的
关键责任接着是在其方法\refvar{StartPixel}{()}
中填充数组\refvar{samples1D}{}和\refvar{samples2D}{}
(以及\refvar{sampleArray1D}{}和\refvar{sampleArray2D}{})。

\refvar{current1DDimension}{}和\refvar{current2DDimension}{}保存了
当前像素样本针对对应数组的偏移量。在开始处理每个新样本前必须将它们重置为0.
\begin{lstlisting}
`\initcode{PixelSampler Protected Data}{=}\initnext{PixelSamplerProtectedData}`
std::vector<std::vector<`\refvar{Float}{}`>> `\initvar{samples1D}{}`;
std::vector<std::vector<`\refvar{Point2f}{}`>> `\initvar{samples2D}{}`;
int `\initvar{current1DDimension}{}` = 0, `\initvar{current2DDimension}{}` = 0;
\end{lstlisting}
\begin{lstlisting}
`\refcode{Sampler Method Definitions}{+=}\lastnext{SamplerMethodDefinitions}`
bool `\initvar[PixelSampler::StartNextSample]{\refvar{PixelSampler}{}::\refvar{StartNextSample}{}}`() {
    `\refvar{current1DDimension}{}` = `\refvar{current2DDimension}{}` = 0;
    return `\refvar{Sampler}{}::\refvar[Sampler::StartNextSample]{StartNextSample}{}`();
}
\end{lstlisting}
\begin{lstlisting}
`\refcode{Sampler Method Definitions}{+=}\lastnext{SamplerMethodDefinitions}`
bool `\initvar[PixelSampler::SetSampleNumber]{\refvar{PixelSampler}{}::\refvar{SetSampleNumber}{}}{}`(int64_t sampleNum) {
    `\refvar{current1DDimension}{}` = `\refvar{current2DDimension}{}` = 0;
    return `\refvar{Sampler}{}::\refvar[Sampler::SetSampleNumber]{SetSampleNumber}{}`(sampleNum);
}
\end{lstlisting}

有了子类\refvar{PixelSampler}{}计算的数组中的样本值,
实现\refvar{Get1D}{()}只需依维度返回值直到算出的
所有维度都已被用掉,此时返回均匀随机值。
\begin{lstlisting}
`\refcode{Sampler Method Definitions}{+=}\lastnext{SamplerMethodDefinitions}`
`\refvar{Float}{}` `\initvar[PixelSampler::Get1D]{\refvar{PixelSampler}{}::\refvar{Get1D}{}}{}`() {
    if (`\refvar{current1DDimension}{}` < `\refvar{samples1D}{}`.size())
        return `\refvar{samples1D}{}`[`\refvar{current1DDimension}{}`++][`\refvar{currentPixelSampleIndex}{}`];
    else
        return `\refvar[PixelSampler::rng]{rng}{}`.`\refvar{UniformFloat}{}`();
}
\end{lstlisting}

{\initvar{PixelSampler::Get2D}{()}}同理,所以这里不再介绍。

\refvar{PixelSampler}{}用的随机数生成器是{\ttfamily protected}的
而不是{\ttfamily private}的。这对于其一些也需要随机数
来初始化\refvar{samples1D}{}和\refvar{samples2D}{}的子类会很方便。
\begin{lstlisting}
`\refcode{PixelSampler Protected Data}{+=}\lastcode{PixelSamplerProtectedData}`
`\refvar{RNG}{}` `\initvar[PixelSampler::rng]{rng}{}`;
\end{lstlisting}

\subsection{全局采样器}\label{sub:全局采样器}
其他生成样本的算法很少基于像素而是自然地生成分布于整幅图像的连续样本,
连续访问完全不同的像素(许多这样的采样器会高效地放置每个追加的样本
使其填充$n$维样本空间中的最大空洞,这自然导致后续样本在不同像素内)。
这些采样算法对于目前描述的\refvar{Sampler}{}接口有点问题:
例如考虑一个为前两维生成如\reftab{7.2}中间一列所示的一系列样本值的采样器。
这些样本值乘以图像每维分辨率得到图像平面中的样本位置
(这里我们为了简化考虑一幅$2\times3$的图像)。
注意对于这里的采样器(其实是\refvar{HaltonSampler}{}),
每六个样本就访问每个像素。若我们正渲染的图像每个像素用三个样本,
则为了给像素$(0,0)$生成所有的样本,我们需要生成索引为0、6和12的样本,以此类推。
\begin{table}[htb]
    \centering
    \begin{tabular}{lll}
        \toprule
        样本索引 & $[0,1)^2$的样本坐标   & 像素样本坐标          \\
        \midrule
        0        & $(0.000000,0.000000)$ & $(0.000000,0.000000)$ \\
        1        & $(0.500000,0.333333)$ & $(1.000000,1.000000)$ \\
        2        & $(0.250000,0.666667)$ & $(0.500000,2.000000)$ \\
        3        & $(0.750000,0.111111)$ & $(1.500000,0.333333)$ \\
        4        & $(0.125000,0.444444)$ & $(0.250000,1.333333)$ \\
        5        & $(0.625000,0.777778)$ & $(1.250000,2.333333)$ \\
        6        & $(0.375000,0.222222)$ & $(0.750000,0.666667)$ \\
        7        & $(0.875000,0.555556)$ & $(1.750000,1.666667)$ \\
        8        & $(0.062500,0.888889)$ & $(0.125000,2.666667)$ \\
        9        & $(0.562500,0.037037)$ & $(1.125000,0.111111)$ \\
        10       & $(0.312500,0.370370)$ & $(0.625000,1.111111)$ \\
        11       & $(0.812500,0.703704)$ & $(1.625000,2.111111)$ \\
        12       & $(0.187500,0.148148)$ & $(0.375000,0.444444)$ \\
        $\vdots$ &                       &                       \\
        \bottomrule
    \end{tabular}
    \caption{\refvar{HaltonSampler}{}生成中间一列坐标的前两维。
        因为它是个\refvar{GlobalSampler}{},所以它必须定义从像素坐标到样本索引的逆映射;
        这里,它通过将第一维坐标放大2倍、第二维坐标放大3倍
        以在$2\times3$像素的图像上放置样本,得到右边一列的像素样本坐标。}
    \label{tab:7.2}
\end{table}

若有了这样的采样器,我们就能定义\refvar{Sampler}{}接口使得
它为每个样本指定正在渲染的像素而不是相反(即告诉\refvar{Sampler}{}要渲染哪个像素)。

然而,采用目前的设计也有很好的理由:该方法更易把胶片分解为
小的图块以供多线程渲染,每个线程计算一个可高效并入最终图像的局部区域内的像素。
因此,我们必须要求这样的采样器能无序生成样本,使得每个像素的全部样本是连续生成的。

\refvar{GlobalSampler}{}帮助沟通\refvar{Sampler}{}接口的要求
与这类采样器的合理操作。它提供了\refvar{Sampler}{}所有纯虚方法的实现,
即代之以其子类必须实现的两个新的纯虚方法。
\begin{lstlisting}
`\refcode{Sampler Declarations}{+=}\lastcode{SamplerDeclarations}`
class `\initvar{GlobalSampler}{}` : public `\refvar{Sampler}{}` {
public:
    `\refcode{GlobalSampler Public Methods}{}`
private:
    `\refcode{GlobalSampler Private Data}{}`
};
\end{lstlisting}
\begin{lstlisting}
`\initcode{GlobalSampler Public Methods}{=}\initnext{GlobalSamplerPublicMethods}`
`\refvar{GlobalSampler}{}`(int64_t samplesPerPixel) : `\refvar{Sampler}{}`(samplesPerPixel) { }
\end{lstlisting}

有两个方法是实现必须提供的。第一个是\refvar{GetIndexForSample}{()},
它执行从当前像素和给定样本索引到样本向量全集中全局索引的逆映射。
例如,对于生成\reftab{7.2}中值的\refvar{Sampler}{},
如果\refvar{currentPixel}{}是$(0,2)$,则\refvar{GetIndexForSample}{(0)}会返回2,
因为样本索引2相应的像素样本坐标$(0.5,2)$对应着该像素区域中的首个样本
\sidenote{译者注:原文写的坐标值是$(0.25,0.666667)$,疑是笔误,已修改。}。
\begin{lstlisting}
`\refcode{GlobalSampler Public Methods}{+=}\lastnext{GlobalSamplerPublicMethods}`
virtual int64_t `\initvar{GetIndexForSample}{}`(int64_t sampleNum) const = 0;
\end{lstlisting}

紧密相关的\refvar{SampleDimension}{()}为
序列中第{\ttfamily index}个样本向量的给定维度返回样本值。
因为前两维用于偏移到当前像素,所以它们要做特殊处理:
该方法的实现返回的值应该是当前像素内的样本偏移量,
而不是原始的$[0,1)^2$样本值。例如\reftab{7.2}中,
\refvar{SampleDimension}{(4,1)}中会返回0.333333,
因为索引为4的样本的第二维相对于像素$(0,1)$偏移了这么多。
\begin{lstlisting}
`\refcode{GlobalSampler Public Methods}{+=}\lastcode{GlobalSamplerPublicMethods}`
virtual `\refvar{Float}{}` `\initvar{SampleDimension}{}`(int64_t index, int dimension) const = 0;
\end{lstlisting}

当开始为一个像素生成样本时,必须重置样本的维度并找到像素内首个样本的索引。
像所有采样器那样,接下来生成样本数组的所有值。
\begin{lstlisting}
`\refcode{Sampler Method Definitions}{+=}\lastnext{SamplerMethodDefinitions}`
void `\initvar[GlobalSampler::StartPixel]{\refvar{GlobalSampler}{}::\refvar{StartPixel}{}}{}`(const `\refvar{Point2i}{}` &p) {
    `\refvar{Sampler}{}`::`\refvar[Sampler::StartPixel]{StartPixel}{}`(p);
    `\refvar[GlobalSampler::dimension]{dimension}{}` = 0;
    `\refvar{intervalSampleIndex}{}` = `\refvar{GetIndexForSample}{}`(0);
    `\refcode{Compute arrayEndDim for dimensions used for array samples}{}`
    `\refcode{Compute 1D array samples for GlobalSampler}{}`
    `\refcode{Compute 2D array samples for GlobalSampler}{}`
}
\end{lstlisting}

成员变量\refvar[GlobalSampler::dimension]{dimension}{}跟踪
采样器实现将被要求生成的样本值的下一维;
当调用\refvar[GlobalSampler::Get1D]{Get1D}{()}和
\refvar[GlobalSampler::Get2D]{Get2D}{()}时它是递增的。
\refvar{intervalSampleIndex}{}记录当前像素内当前样本$s_i$对应的样本索引。
\begin{lstlisting}
`\initcode{GlobalSampler Private Data}{=}\initnext{GlobalSamplerPrivateData}`
int `\initvar[GlobalSampler::dimension]{dimension}{}`;
int64_t `\initvar{intervalSampleIndex}{}`;
\end{lstlisting}

必须决定为数组样本使用样本向量的哪些维度。
在靠前的维度比后面的维度质量更好的假设下,
为\refvar{CameraSample}{}留出前几个维度很重要,
因为这些样本值的质量经常对最终图像质量有很大影响。

因此,\refvar{arrayStartDim}{}前的维度用于常规的1D和2D样本,
而后续维度用于先1D再2D的数组样本。最后,起始于\refvar{arrayEndDim}{}的更高维
进一步用于非数组的1D和2D样本。当\refvar{GlobalSampler}{}构造函数运行时
不可能计算\refvar{arrayEndDim}{},因为目前还没有积分器请求数组样本。
因此,该值在方法\refvar[GlobalSampler::StartPixel]{StartPixel}{()}中
(重复且冗余地)计算。
\begin{lstlisting}
`\refcode{GlobalSampler Private Data}{+=}\lastcode{GlobalSamplerPrivateData}`
static const int `\initvar{arrayStartDim}{}` = 5;
int `\initvar{arrayEndDim}{}`;
\end{lstlisting}

所有像素样本的数组样本总数由像素样本数量与请求的样本数组尺寸的乘积给出。
\begin{lstlisting}
`\initcode{Compute arrayEndDim for dimensions used for array samples}{=}`
`\refvar{arrayEndDim}{}` = `\refvar{arrayStartDim}{}` +
              `\refvar{sampleArray1D}{}`.size() + 2 * `\refvar{sampleArray2D}{}`.size();
\end{lstlisting}

实际生成数组样本只需计算当前样本维度内所需值的数量。
\begin{lstlisting}
`\initcode{Compute 1D array samples for GlobalSampler}{=}`
for (size_t i = 0; i < `\refvar{samples1DArraySizes}{}`.size(); ++i) {
    int nSamples = `\refvar{samples1DArraySizes}{}`[i] * `\refvar{samplesPerPixel}{}`;
    for (int j = 0; j < nSamples; ++j) {
        int64_t index = `\refvar{GetIndexForSample}{}`(j);
        `\refvar{sampleArray1D}{}`[i][j] =
            `\refvar{SampleDimension}{}`(index, `\refvar{arrayStartDim}{}` + i);
    }
}
\end{lstlisting}

2D样本数组的生成类似;这里不再介绍
代码片\refcode{Compute 2D array samples for GlobalSampler}{}
\sidenote{译者注:我补充回来了。}。
\begin{lstlisting}
`\initcode{Compute 2D array samples for GlobalSampler}{=}`
int dim = `\refvar{arrayStartDim}{}` + `\refvar{samples1DArraySizes}{}`.size();
for (size_t i = 0; i < `\refvar{samples2DArraySizes}{}`.size(); ++i) {
    int nSamples = `\refvar{samples2DArraySizes}{}`[i] * `\refvar{samplesPerPixel}{}`;
    for (int j = 0; j < nSamples; ++j) {
        int64_t idx = `\refvar{GetIndexForSample}{}`(j);
        `\refvar{sampleArray2D}{}`[i][j].x = `\refvar{SampleDimension}{}`(idx, dim);
        `\refvar{sampleArray2D}{}`[i][j].y = `\refvar{SampleDimension}{}`(idx, dim+1);
    }
    dim += 2;
}
`\refvar{Assert}{}`(dim == `\refvar{arrayEndDim}{}`);
\end{lstlisting}

当像素样本变化时,必须重置当前样本维度计数器并计算像素内下一样本的样本索引。
\begin{lstlisting}
`\refcode{Sampler Method Definitions}{+=}\lastnext{SamplerMethodDefinitions}`
bool `\initvar[GlobalSampler::StartNextSample]{\refvar{GlobalSampler}{}::\refvar{StartNextSample}{}}{}`() {
    `\refvar[GlobalSampler::dimension]{dimension}{}` = 0;
    `\refvar{intervalSampleIndex}{}` = `\refvar{GetIndexForSample}{}`(`\refvar{currentPixelSampleIndex}{}` + 1);
    return `\refvar{Sampler}{}`::`\refvar[Sampler::StartNextSample]{StartNextSample}{}`();
}
\end{lstlisting}
\begin{lstlisting}
`\refcode{Sampler Method Definitions}{+=}\lastnext{SamplerMethodDefinitions}`
bool `\initvar[GlobalSampler::SetSampleNumber]{\refvar{GlobalSampler}{}::\refvar{SetSampleNumber}{}}{}`(int64_t sampleNum) {
    `\refvar[GlobalSampler::dimension]{dimension}{}` = 0;
    `\refvar{intervalSampleIndex}{}` = `\refvar{GetIndexForSample}{}`(sampleNum);
    return `\refvar{Sampler}{}`::`\refvar[Sampler::SetSampleNumber]{SetSampleNumber}{}`(sampleNum);
}
\end{lstlisting}

有了该机制,获取常规1D样本值只需跳过分配给数组样本的维度
并把当前样本索引和维度传给实现的方法\refvar{SampleDimension}{()}。
\begin{lstlisting}
`\refcode{Sampler Method Definitions}{+=}\lastnext{SamplerMethodDefinitions}`
`\refvar{Float}{}` `\initvar[GlobalSampler::Get1D]{\refvar{GlobalSampler}{}::\refvar{Get1D}{}}{}`() {
    if (`\refvar[GlobalSampler::dimension]{dimension}{}` >= `\refvar{arrayStartDim}{}` && `\refvar[GlobalSampler::dimension]{dimension}{}` < `\refvar{arrayEndDim}{}`)
        `\refvar[GlobalSampler::dimension]{dimension}{}` = `\refvar{arrayEndDim}{}`;
    return `\refvar{SampleDimension}{}`(`\refvar{intervalSampleIndex}{}`, `\refvar[GlobalSampler::dimension]{dimension}{}`++);
}
\end{lstlisting}

2D样本样本同理。
\begin{lstlisting}
`\refcode{Sampler Method Definitions}{+=}\lastcode{SamplerMethodDefinitions}`
`\refvar{Point2f}{}` `\initvar[GlobalSampler::Get2D]{\refvar{GlobalSampler}{}::\refvar{Get2D}{}}{}`() {
    if (`\refvar[GlobalSampler::dimension]{dimension}{}` + 1 >= `\refvar{arrayStartDim}{}` && `\refvar[GlobalSampler::dimension]{dimension}{}` < `\refvar{arrayEndDim}{}`)
        `\refvar[GlobalSampler::dimension]{dimension}{}` = `\refvar{arrayEndDim}{}`;
    `\refvar{Point2f}{}` p(`\refvar{SampleDimension}{}`(`\refvar{intervalSampleIndex}{}`, `\refvar[GlobalSampler::dimension]{dimension}{}`),
              `\refvar{SampleDimension}{}`(`\refvar{intervalSampleIndex}{}`, `\refvar[GlobalSampler::dimension]{dimension}{}` + 1));
    `\refvar[GlobalSampler::dimension]{dimension}{}` += 2;
    return p;
}
\end{lstlisting}

\section{分层采样}\label{sec:分层采样}
我们将要介绍的首个\refvar{Sampler}{}实现会把
像素区域细分为矩形区域并在每个区域内生成单个样本。
这些区域常称为\keyindex{层}{strata}{},
而该采样器称为\refvar{StratifiedSampler}{}。
分层背后的关键思想是通过把采样域细分为不重叠区域
并从每个中取单个样本,我们更不可能错失整个图像的重要特征,
因为保证了样本不会全都挨在一起。换句话说,
如果许多样本都从样本空间中的邻近点取得则对我们没有好处,
因为每个新样本不能增加许多关于图像函数特性的新信息。
从信号处理的角度看,我们在隐式定义整体采样率,
它使得层级越小,我们拥有的层数就越多,因此采样率就越高。

分层采样器通过对层中心点施加一个至多为层的一半宽高的
随机\keyindex{扰动}{jitter}{}量来将每个样本置于每层中的随机点处。
如\refsec{采样理论}讨论的,该扰动引起的非均匀性帮助把混叠转化为噪声。
采样器还提供了非扰动模式,给出层中的均匀采样;
比起用于渲染高质量图像,该模式对于比较不同采样技术才最有用。

直接对高维采样应用分层法很快导致样本量巨大。
例如,如果我们在每个维度上把5D的图像、透镜和时间样本空间划分为四层,
则每个像素的样本总量将是$4^5=1024$。
我们可以通过在某些维度取更少的样本(或者不分层某些维度,实际上使用单层)来降低该影响,
但我们将会失去在这些维度拥有分层良好的样本的好处。
该分层问题称为\keyindex{维度灾难}{curse of dimensionality}{}。

通过为域的维度子集计算低维分层模式然后随机联合每个维度集合中的样本,
我们可以获取分层的大多数好处而不用为过多采样总量付出代价
(该过程有时称为\keyindex{填充}{padding}{})。
\reffig{7.16}展示了基本思想:我们可能只想每个像素取四个样本,
但仍得在所有维度上对样本分层。我们独立生成四个2D分层图像样本,
四个1D分层时间样本,以及四个2D分层透镜样本。
然后我们随机地为每个图像样本联合一个时间以及透镜样本值。
结果是每个像素拥有合起来良好覆盖样本空间的样本。
\begin{figure}[htbp]
    \centering\includegraphics[width=0.8\linewidth]{chap07/Samplepadding.eps}
    \caption{我们可以生成良好的样本模式并获得分层的好处而不要求
        同时对所有采样维度分层。这里,我们已把$(x,y)$图像位置、
        时间$t$以及$(u,v)$透镜位置分为独立的层,每个都有四个区域。
        每个都是独立采样的,然后每个图像样本都随机关联一个时间样本
        和一个透镜样本。我们保留了在每个单独维度上分层的好处而不用指数级地增加样本总量。}
    \label{fig:7.16}
\end{figure}

\reffig{7.17}展示了在渲染景深时使用分层的透镜样本和
使用不分层的随机样本相比图像质量的提升。
\begin{figure}[htbp]
    \subfloat[参考]{\includegraphics[width=0.49\linewidth]{chap07/dof-ref.png}\label{fig:7.17.1}}\,
    \subfloat[随机采样]{\includegraphics[width=0.49\linewidth]{chap07/dof-random.png}\label{fig:7.17.2}}\\
    \subfloat[分层采样]{\includegraphics[width=0.49\linewidth]{chap07/dof-stratified.png}\label{fig:7.17.3}}
    \caption{渲染有景深的紫色球体时采样模式的影响。
        (a)模糊球体的高质量参考图像。(b)在每个像素中随机采样而无分层所生成的图像。
        (c)用同样数量的样本生成的图像,但用的是\refvar{StratifiedSampler}{},
        它分层了图像样本以及对该图更重要的透镜样本。对于该情形分层法做出了很大改善。}
    \label{fig:7.17}
\end{figure}

\reffig{7.18}展示比较了几种采样模式。
第一种是完全随机的模式:我们生成大量样本而完全不使用分层。
其结果很差;一些区域只有几个样本而另一些区域有好几团样本。
第二种是均匀分层模式。最后,均匀模式被扰动,
随机偏移量被加到每个样本的位置上,但仍将其保留在格子中。
这给出了比纯随机模式更好的整体分布而又保留了分层的好处,
尽管仍有一些样本团以及欠采样的区域。
\begin{figure}[htbp]
    \subfloat[]{\includegraphics[width=0.49\linewidth]{chap07/random-point-samples.eps}\label{fig:7.18.1}}\,
    \subfloat[]{\includegraphics[width=0.49\linewidth]{chap07/uniform-point-samples.eps}\label{fig:7.18.2}}\\
    \subfloat[]{\includegraphics[width=0.49\linewidth]{chap07/jittered-point-samples.eps}\label{fig:7.18.3}}
    \caption{三种2D采样模式。(a)随机模式是无效模式,许多样本团让大片图像没有好好采样。
        (b)均匀分层模式的分布更好但会加剧混叠伪影。
        (c)分层扰动模式将来自均匀模式的混叠转化为高频噪声而仍保留了分层的好处。}
    \label{fig:7.18}
\end{figure}

\reffig{7.19}展示了用\refvar{StratifiedSampler}{}渲染的图像,
并展示了扰动的样本位置怎样将混叠伪影转化为不那么讨厌的噪声。
\begin{figure}[htbp]
    \centering
    \subfloat[参考]{\includegraphics[width=0.8\linewidth]{chap07/checkerboard-ref.png}\label{fig:7.19.1}}\\
    \subfloat[1个均匀样本]{\includegraphics[width=0.8\linewidth]{chap07/checkerboard-unif-1spp.png}\label{fig:7.19.2}}\\
    \subfloat[1个扰动样本]{\includegraphics[width=0.8\linewidth]{chap07/checkerboard-jitter-1spp.png}\label{fig:7.19.3}}\\
    \subfloat[4个扰动样本]{\includegraphics[width=0.8\linewidth]{chap07/checkerboard-jitter-4spp.png}\label{fig:7.19.4}}
    \caption{用棋盘纹理比较图像采样方法。这是幅很难渲染好的图像,
        因为当我们接近地平线时棋盘格关于像素间隔的频率趋于无穷。
        (a)参考图像,每个像素用256个样本渲染,展示了接近理想结果的样子。
        (b)每个像素只用一个样本渲染的图像,没有扰动。注意前景中格子边缘的锯齿伪影。
        还注意棋盘格函数在样本之间经历了许多周期的距离处的伪影;
        如之前介绍过的信号处理理论所料,细节错误地重复表现为低频混叠。
        (c)扰动图像样本的结果,每个像素还是只有一个样本。
        第二幅图像规则的混叠已经被替换为不那么讨厌的噪声伪影。
        (d)每个像素用四个扰动样本的结果仍然不如参考图像,但明显优于之前的结果。}
    \label{fig:7.19}
\end{figure}

\begin{lstlisting}
`\initcode{StratifiedSampler Declarations}{=}`
class `\initvar{StratifiedSampler}{}` : public `\refvar{PixelSampler}{}` {
public:
    `\refcode{StratifiedSampler Public Methods}{}`
private:
    `\refcode{StratifiedSampler Private Data}{}`
};
\end{lstlisting}
\begin{lstlisting}
`\initcode{StratifiedSampler Public Methods}{=}`
`\refvar{StratifiedSampler}{}`(int xPixelSamples, int yPixelSamples,
        bool jitterSamples, int nSampledDimensions)
    : `\refvar{PixelSampler}{}`(xPixelSamples * yPixelSamples, nSampledDimensions),
      `\refvar{xPixelSamples}{}`(xPixelSamples), `\refvar{yPixelSamples}{}`(yPixelSamples),
      `\refvar{jitterSamples}{}`(jitterSamples) { }
\end{lstlisting}
\begin{lstlisting}
`\initcode{StratifiedSampler Private Data}{=}`
const int `\initvar{xPixelSamples}{}`, `\initvar{yPixelSamples}{}`;
const bool `\initvar{jitterSamples}{}`;
\end{lstlisting}

作为\refvar{PixelSampler}{}的子类,
\refvar[StratifiedSampler::StartPixel]{StartPixel}{()}的实现必须按照传给\refvar{PixelSampler}{}
构造函数的维数{\ttfamily nSampledDimensions}一起生成1D和2D样本以及请求的数组样本。
\begin{lstlisting}
`\initcode{StratifiedSampler Method Definitions}{=}`
void `\refvar{StratifiedSampler}{}`::`\initvar[StratifiedSampler::StartPixel]{StartPixel}{}`(const `\refvar{Point2i}{}` &p) {
    `\refcode{Generate single stratified samples for the pixel}{}`
    `\refcode{Generate arrays of stratified samples for the pixel}{}`
    `\refvar{PixelSampler}{}`::StartPixel(p);
}
\end{lstlisting}

生成初始的分层样本后,它们被随机打乱;这是本节开头描述的填充方法。
如果没有进行打乱,则样本维度的值可能以某种方式相关而引发图像中的错误——
例如,用于选择胶片位置的首个2D样本和首个2D透镜样本会总是都在相邻于原点的左下方那层。
\begin{lstlisting}
`\initcode{Generate single stratified samples for the pixel}{=}`
for (size_t i = 0; i < `\refvar{samples1D}{}`.size(); ++i) {
    `\refvar{StratifiedSample1D}{}`(&`\refvar{samples1D}{}`[i][0], `\refvar{xPixelSamples}{}` * `\refvar{yPixelSamples}{}`,
                       `\refvar[PixelSampler::rng]{rng}{}`, `\refvar{jitterSamples}{}`);
    `\refvar{Shuffle}{}`(&`\refvar{samples1D}{}`[i][0], `\refvar{xPixelSamples}{}` * `\refvar{yPixelSamples}{}`, 1, `\refvar[PixelSampler::rng]{rng}{}`);
}
for (size_t i = 0; i < `\refvar{samples2D}{}`.size(); ++i) {
    `\refvar{StratifiedSample2D}{}`(&`\refvar{samples2D}{}`[i][0], `\refvar{xPixelSamples}{}`, `\refvar{yPixelSamples}{}`,
                       `\refvar[PixelSampler::rng]{rng}{}`, `\refvar{jitterSamples}{}`);
    `\refvar{Shuffle}{}`(&`\refvar{samples2D}{}`[i][0], `\refvar{xPixelSamples}{}` * `\refvar{yPixelSamples}{}`, 1, `\refvar[PixelSampler::rng]{rng}{}`);
}
\end{lstlisting}

1D和2D分层采样例程实现为实用函数。
两个都在域中给定层数上循环并在每个里面放置一个样本点。
\begin{lstlisting}
`\initcode{Sampling Function Definitions}{=}\initnext{SamplingFunctionDefinitions}`
void `\initvar{StratifiedSample1D}{}`(`\refvar{Float}{}` *samp, int nSamples, `\refvar{RNG}{}` &rng,
        bool jitter) {
    `\refvar{Float}{}` invNSamples = (`\refvar{Float}{}`)1 / nSamples;
    for (int i = 0; i < nSamples; ++i) {
        `\refvar{Float}{}` delta = jitter ? rng.`\refvar{UniformFloat}{}`() : 0.5f;
        samp[i] = std::min((i + delta) * invNSamples, `\refvar{OneMinusEpsilon}{}`);
    }
}
\end{lstlisting}

\refvar{StratifiedSample2D}{()}同样生成范围$[0,1)^2$中的样本。
\begin{lstlisting}
`\refcode{Sampling Function Definitions}{+=}\lastnext{SamplingFunctionDefinitions}`
void `\initvar{StratifiedSample2D}{}`(`\refvar{Point2f}{}` *samp, int nx, int ny, `\refvar{RNG}{}` &rng,
        bool jitter) {
    `\refvar{Float}{}` dx = (`\refvar{Float}{}`)1 / nx, dy = (`\refvar{Float}{}`)1 / ny;
    for (int y = 0; y < ny; ++y)
        for (int x = 0; x < nx; ++x) {
            `\refvar{Float}{}` jx = jitter ? rng.`\refvar{UniformFloat}{}`() : 0.5f;
            `\refvar{Float}{}` jy = jitter ? rng.`\refvar{UniformFloat}{}`() : 0.5f;
            samp->x = std::min((x + jx) * dx, `\refvar{OneMinusEpsilon}{}`);
            samp->y = std::min((y + jy) * dy, `\refvar{OneMinusEpsilon}{}`);
            ++samp;
        }
}
\end{lstlisting}

函数\refvar{Shuffle}{()}随机重排含有{\ttfamily count}个样本值的数组,
每个都有{\ttfamily nDimensions}维(换句话说,
尺寸为{\ttfamily nDimensions}的值构成的块被重排)。
\begin{lstlisting}
`\initcode{Sampling Inline Functions}{=}\initnext{SamplingInlineFunctions}`
template <typename T>
void `\initvar{Shuffle}{}`(T *samp, int count, int nDimensions, `\refvar{RNG}{}` &rng) {
    for (int i = 0; i < count; ++i) {
        int other = i + rng.`\refvar{UniformUInt32}{}`(count - i);
        for (int j = 0; j < nDimensions; ++j)
            std::swap(samp[nDimensions * i + j],
                      samp[nDimensions * other + j]);
    }
}
\end{lstlisting}

样本数组给我们出了个难题:例如若一个积分器
为像素中的每个样本请求样本向量中含64个2D样本值的数组,
则采样器有两个不同的目标要达成:
\begin{enumerate}
    \item 希望数组内的样本本身在2D上分布良好(例如通过使用$8\times8$分层网格)。
          这里的分层法会为每个单独的样本向量提升算出的结果的质量。
    \item 最好保证一个图像样本的数组中的每个样本都不要和图像中相邻样本的任何样本值太相似。
          即我们更希望点相对于其邻居能分布良好,使得在单个像素周围区域上就能很好覆盖整个样本空间。
\end{enumerate}

比起尝试同时解决这里的两个问题,\refvar{StratifiedSampler}{}只解决第一个。
本章后面的其他采样器会以更加精巧的技术回顾该问题并在不同程度上同时解决它们。

第二个复杂性来自于调用者可能会为每个图像样本请求任意数量样本的事实,
所以可能不易应用分层法。(例如,我们要怎么生成七个样本的分层2D模式?)
我们只能生成一个$n\times1$或$1\times n$的分层模式,
但这只能给我们在一个维度上分层的好处但不保证其他维度有好的模式。
方法{\ttfamily StratifiedSampler::RoundSize()}可以将请求进位到
下一个平方数,但我们将换用一种称为\keyindex{拉丁超立方采样}{Latin hypercube sampling}{}(LHS)的方法,
它能生成具有相当好分布的任意数量的任意维数样本。

LHS把每个维度轴均匀划分为$n$个区域并沿对角线在$n$个区域中的
每一个内生成一个扰动的样本,如\reffig{7.20}左边所示。
然后这些样本在每个维度上被随机打乱,生成分布良好的模式。
\begin{figure}[htbp]
    \centering\includegraphics[width=0.8\linewidth]{chap07/LHSshuffle.eps}
    \caption{拉丁超立方采样(有时称为\protect\keyindex{$n$车采样}{$n$-rooks sampling}{})
        选择样本使得网格每行每列只出现单个样本。通过在对角线格子里生成随机样本
        然后随机重排它们的坐标可以做到这点。LHS的一个优点是它能像用分层模式那样
        生成具有良好分布的任意数量的样本,而不仅仅是$m\times n$个样本。}
    \label{fig:7.20}
\end{figure}

LHS的一个优点是当样本投影到样本维度的任意轴时它最小化了样本的聚集。
该性质与分层采样相反,后者2D模式中$n\times n$个样本里的$2n$个可能投影到每个轴上基本相同的点。
\reffig{7.21}展示了对于分层采样模式的这一最坏情况。
\begin{figure}[htbp]
    \centering\includegraphics[width=0.4\linewidth]{chap07/stratified-bad-luck.eps}
    \caption{分层采样的一个最坏情况。在$n\times n$的2D模式中,
        多达$2n$个点可能投影到一个轴上基本相同的点。当生成像这样的“倒霉”模式时,
        用其算出的结果质量通常堪忧(这里,8个样本有几乎一样的$x$值)。}
    \label{fig:7.21}
\end{figure}

尽管解决了聚集问题,LHS对于分层采样并不是必要的改进;
很容易构造样本位置基本共线且大面积采样域没有相邻样本的情形
(例如,当原始样本的排列一致时,即让它们都保持原样)。
特别地,随着$n$增加,拉丁超立方模式比起分层模式越来越低效
\footnote{后续章节我们将回顾该问题,讨论同时是分层的且按拉丁超立方模式分布的样本模式。}。

通用的函数\refvar{LatinHypercube}{()}在任意维度生成任意数量的LHS样本。
因此数组{\ttfamily samples}中的元素数量应为{\ttfamily nSamples*nDim}。

\begin{lstlisting}
`\refcode{Sampling Function Definitions}{+=}\lastnext{SamplingFunctionDefinitions}`
void `\initvar{LatinHypercube}{}`(`\refvar{Float}{}` *samples, int nSamples, int nDim, `\refvar{RNG}{}` &rng) {
    `\refcode{Generate LHS samples along diagonal}{}`
    `\refcode{Permute LHS samples in each dimension}{}`
}
\end{lstlisting}
\begin{lstlisting}
`\initcode{Generate LHS samples along diagonal}{=}`
`\refvar{Float}{}` invNSamples = (`\refvar{Float}{}`)1 / nSamples;
for (int i = 0; i < nSamples; ++i)
    for (int j = 0; j < nDim; ++j) {
        `\refvar{Float}{}` sj = (i + (rng.`\refvar{UniformFloat}{}`())) * invNSamples;
        samples[nDim * i + j] = std::min(sj, `\refvar{OneMinusEpsilon}{}`);
    }
\end{lstlisting}

为了进行重排,该函数在样本上循环,每次在一个维度上随机重排样本点。
注意这和之前的\refvar{Shuffle}{()}例程是不一样的重排:
后者例程做一次重排,让每个样本里的全部{\ttfamily nDim}个样本点在一起,
而这里每次做单个维度的{\ttfamily nDim}次单独重排(\reffig{7.22})
\footnote{尽管不需要重排LHS模式的第一维,但这里的实现还是这样做了,
    因为让第一维的元素变为随机顺序意味着LHS模式可以与来自其他源的采样模式
    结合使用而没有其样本点间存在相关性的危险。}。
\begin{figure}[htbp]
    \centering\includegraphics[width=0.8\linewidth]{chap07/Shufflepermutations.eps}
    \caption{(a)\refvar{Shuffle}{()}例程进行的重排移动附近的整块元素。
        (b)拉丁超立方采样的重排独立地排列每个维度的样本。
        这里展示了打乱三维四元素模式的第二维样本。}
    \label{fig:7.22}
\end{figure}

\begin{lstlisting}
`\initcode{Permute LHS samples in each dimension}{=}`
for (int i = 0; i < nDim; ++i) {
    for (int j = 0; j < nSamples; ++j) {
        int other = j + rng.`\refvar{UniformUInt32}{}`(nSamples - j);
        std::swap(samples[nDim * j + i], samples[nDim * other + i]);
    }
}
\end{lstlisting}

有了函数\refvar{LatinHypercube}{()},
我们选择可以编写代码为当前像素计算样本数组了。
1D样本被分层然后随机打乱,而2D样本用拉丁超立方采样生成。
\begin{lstlisting}
`\initcode{Generate arrays of stratified samples for the pixel}{=}`
for (size_t i = 0; i < `\refvar{samples1DArraySizes}{}`.size(); ++i)
    for (int64_t j = 0; j < `\refvar{samplesPerPixel}{}`; ++j) {
        int count = `\refvar{samples1DArraySizes}{}`[i];
        `\refvar{StratifiedSample1D}{}`(&`\refvar{sampleArray1D}{}`[i][j * count], count, `\refvar[PixelSampler::rng]{rng}{}`,
                           `\refvar{jitterSamples}{}`);
        `\refvar{Shuffle}{}`(&`\refvar{sampleArray1D}{}`[i][j * count], count, 1, `\refvar[PixelSampler::rng]{rng}{}`);
    }
for (size_t i = 0; i < `\refvar{samples2DArraySizes}{}`.size(); ++i)
    for (int64_t j = 0; j < `\refvar{samplesPerPixel}{}`; ++j) {
        int count = `\refvar{samples2DArraySizes}{}`[i];
        `\refvar{LatinHypercube}{}`(&`\refvar{sampleArray2D}{}`[i][j * count].x, count, 2, `\refvar[PixelSampler::rng]{rng}{}`);
    }
\end{lstlisting}

我们将用\reffig{7.23}中的场景来阐述一些\refvar{Sampler}{}实现的性质。
\begin{figure}[htbp]
    \centering\includegraphics[width=0.6\linewidth]{chap07/area-light-example.png}
    \caption{面光源采样示例场景。}
    \label{fig:7.23}
\end{figure}

\reffig{7.24}展示了对于\refvar{DirectLightingIntegrator}{}来自好样本的提升。
第一幅图像每个像素用1个图像样本算得,每个有16个阴影样本。
第二幅每个像素用16个图像样本,每个有1个阴影样本。
因为\refvar{StratifiedSampler}{}能为第一种情况生成良好的LHS模式,
所以即使在取用的阴影样本总数相同时,其阴影质量也好得多。

\begin{figure}[htbp]
    \centering
    \subfloat[1个图像样本,16个阴影样本]{\includegraphics[width=\linewidth]{chap07/shadow-1-16.png}\label{fig:7.24.1}}\\
    \subfloat[16个图像样本,1个阴影样本]{\includegraphics[width=\linewidth]{chap07/shadow-16-1.png}\label{fig:7.24.2}}
    \caption{用来自分层采样器的样本采样面光源。
        (a)展示了每个像素用1个图像样本和16个阴影样本的结果,
        而(b)展示了用16个图像样本且每个只有1个阴影样本的结果。
        两种情况阴影样本的总数是一样的,但因为每个图像样本用16个阴影样本的版本
        可以使用LHS模式,像素区域内所有阴影样本都良好分布,而第二幅图像里
        这里的实现没有办法防止它们分布得很差。差别是惊人的。}
    \label{fig:7.24}
\end{figure}

\section{Halton采样器}\label{sec:Halton采样器}
\begin{remark}
    本节含有高级内容,第一次阅读时可以跳过。
\end{remark}

\refvar{StratifiedSampler}{}的根本目标是生成
分布良好但不均匀的样本点集,使得不存在两个靠得太近的样本点且没有不含样本的过大样本空间区域。
如\reffig{7.18}{}所示,扰动的模式比随机模式做得更好,
不过当相邻层中的样本恰好靠近它们两层的公共边界时其质量可能受影响。

本节介绍\refvar{HaltonSampler}{},它基于直接生成低偏差点集的算法。不像\linebreak
\refvar{StratifiedSampler}{}生成的点集那样,
\refvar{HaltonSampler}{}不仅生成保证不会靠得太近抱团的点,
还生成能同时在样本向量所有维度上都分布良好的点——
而不是像\refvar{StratifiedSampler}{}那样每次只能在一两个维度上做到。

\subsection{Hammersley和Halton序列}\label{sub:Hammersley和Halton序列}
Halton和Hammersley序列是两种紧密相关的低偏差点集。
两者都基于称为\keyindex{倒根}{radical inverse}{}的构造法,
它基于正整数值$a$可以用下式唯一确定的数字序列$d_m(a)\ldots d_2(a)d_1(a)$
表示为$b$进制\sidenote{译者注:这里$b$是大于1的正整数,称为基(数)(base)。}的事实:
\begin{align}
    \label{eq:7.6}
    a=\sum\limits_{i=1}^m{d_i(a)b^{i-1}}\, ,
\end{align}
其中所有数字$d_i(a)$介于0到$b-1$之间。

$b$进制的倒根函数$\varPhi_b$通过将这些数字关于小数点翻转
把非负整数$a$转化为$[0,1)$的小数值:
\begin{align}
    \label{eq:7.7}
    \varPhi_b(a)=0.d_1(a)d_2(a)\ldots d_m(a)\, .
\end{align}
因此,数字$d_i(a)$对倒根的贡献为$\displaystyle\frac{d_i(a)}{b^i}$。

最简单的低偏差序列之一是van der Corput序列
\sidenote{译者注:得名自20世纪荷兰数学家Johannes Gaultherus van der Corput。},
它是由2进制倒根函数给出的1D序列:
\begin{align*}
    x_a=\varPhi_2(a)\, .
\end{align*}

\reftab{7.3}展示了van der Corput序列的前几个值。
注意它怎样递归地将1D直线上的区间对半划分,生成位于每个区间中心的的样本点。
\begin{table}[htbp]
    \centering
    \begin{tabular}{crl}
        \toprule
        $a$      & \textbf{2进制} & $\varPhi_2(a)$      \\
        \midrule
        0        & 0              & $0$                 \\
        1        & 1              & $0.1=\frac{1}{2}$   \\
        2        & 10             & $0.01=\frac{1}{4}$  \\
        3        & 11             & $0.11=\frac{3}{4}$  \\
        4        & 100            & $0.001=\frac{1}{8}$ \\
        5        & 101            & $0.101=\frac{5}{8}$ \\
        $\vdots$ &                &                     \\
        \bottomrule
    \end{tabular}
    \caption{以2进制计算的前几个非负整数的倒根$\varPhi_2(a)$。
        注意$\varPhi_2(a)$的后续值是怎样避开$\varPhi_2(a)$先前的任何值的。
        随着生成该序列越来越多的值,样本必然与之前的样本更加接近,然而仍保证了最小距离足够好。}
    \label{tab:7.3}
\end{table}

该序列的偏差为\sidenote{译者注:这里的大$O$项表明了该偏差的变化趋势。
    事实上它是有界的,相应证明非常复杂,读者可参见\citet{LARCHER2016546}整理的有关结论。}
\begin{align*}
    D^*_N(P)=O\left(\frac{\log N}{N}\right)\, ,
\end{align*}
它与$n$维无限序列取到的最优偏差相匹配,
\begin{align*}
    D^*_N(P)=O\left(\frac{(\log N)^n}{N}\right)\, .
\end{align*}

为了生成$n$维Halton序列,我们使用$b$进制倒根,且模式的每个维度用不同的基。
所用的基必须两两互质,所以自然的选择是使用
前$n$个\keyindex{质数}{prime number}{}$(p_1,\ldots,p_n)$:
\begin{align*}
    x_a=(\varPhi_2(a),\varPhi_3(a),\varPhi_5(a),\ldots,\varPhi_{p_n}(a))\, .
\end{align*}

Halton序列最有用的性质之一是即使不能提前知道所需的样本总量也能用它;
该序列的所有前缀\sidenote{译者注:即序列开头的部分。}都分布良好,
所以向该序列添加额外样本也可保持低偏差(然而对于指数$k_i$,
当样本总数为基的幂之积$\prod(p_i)^{k_i}$时其分布最好)。

$n$维Halton序列的偏差为
\begin{align*}
    D^*_N(x_a)=O\left(\frac{(\log N)^n}{N}\right)\, ,
\end{align*}
它是渐进最优的。

如果固定样本数目$N$,则可以用偏差还低点的Hammersley点集。
Hammersley点集定义为
\begin{align*}
    x_a=\left(\frac{a}{N},\varPhi_{b_1}(a),\varPhi_{b_2}(a),\ldots,\varPhi_{b_n}(a),\right)\, ,
\end{align*}
其中$N$是要取用的样本总数,像之前那样所有基$b_i$都互质。
\reffig{7.25.1}展示了2D Halton序列前216个点的图示。
\reffig{7.25.2}展示了Hammersley序列的前256个点。
\begin{figure}[htbp]
    \centering
    \subfloat[]{\includegraphics[width=0.49\linewidth]{chap07/halton-points.eps}\label{fig:7.25.1}}\,
    \subfloat[]{\includegraphics[width=0.49\linewidth]{chap07/hammersley-points.eps}\label{fig:7.25.2}}
    \caption{两种2D低偏差序列前面的点。(a)Halton(216个点),(b)Hammersley(256个点)。}
    \label{fig:7.25}
\end{figure}

函数\refvar{RadicalInverse}{()}用第{\ttfamily baseIndex}个质数
作为基为给定数字{\ttfamily a}计算倒根。
该函数用庞大的{\ttfamily switch}语句实现,其中{\ttfamily baseIndex}被映射到
合适的质数然后由单独的模板函数\refvar{RadicalInverseSpecialized}{()}实际计算该倒根
(稍后将解释使用少见的基于{\ttfamily switch}的结构的原因)。
\begin{lstlisting}
`\initcode{Low Discrepancy Function Definitions}{=}\initnext{LowDiscrepancyFunctionDefinitions}`
`\refvar{Float}{}` `\initvar{RadicalInverse}{}`(int baseIndex, uint64_t a) {
    switch (baseIndex) {
        case 0:
            `\refcode{Compute base-2 radical inverse}{}`
        case 1: return `\refvar{RadicalInverseSpecialized}{}`<3>(a);
        case 2: return `\refvar{RadicalInverseSpecialized}{}`<5>(a);
        case 3: return `\refvar{RadicalInverseSpecialized}{}`<7>(a);
        `\refcode{Remainder of cases for RadicalInverse()}{}`
    }
}
\end{lstlisting}

对于2进制倒根,我们可以利用数字计算机中的数字已经表示为2进制的事实
来更高效地计算倒根。对于一个64位值$a$,按\refeq{7.6}我们有
\begin{align*}
    a=\sum\limits_{i=1}^{64}{d_i(a)2^{i-1}}\, .
\end{align*}
首先考虑翻转$a$数位的结果,仍将其视作整数值,得到
\begin{align*}
    \sum\limits_{i=1}^{64}{d_i(a)2^{64-i}}\, .
\end{align*}
如果我们再用该值除以$2^{64}$,我们有
\begin{align*}
    \sum\limits_{i=1}^{64}{d_i(a)2^{-i}}\, ,
\end{align*}
它就是$\varPhi_2(a)$。因此,2进制倒根可高效地用数位翻转以及2的幂除法算得。

整数量的数位可以用一系列逻辑位运算高效翻转。
翻转32位整数数位的函数\refvar{ReverseBits32}{()}的第一行
交换了该值的低16位和高16位。下一行同时交换其结果的前8位与第二组8位、第三组8位与第四组。
该过程持续到交换相邻数位的最后一行。为了理解该代码,写出各个十六进制常数的二进制值很有帮助。
例如,{\ttfamily 0xff00ff00}的二进制是{\ttfamily 11111111000000001111111100000000};
很容易看到与该值进行按位或屏蔽了第一、三组8位数量。
\begin{lstlisting}
`\initcode{Low Discrepancy Inline Functions}{=}\initnext{LowDiscrepancyInlineFunctions}`
inline uint32_t `\initvar{ReverseBits32}{}`(uint32_t n) {
    n = (n << 16) | (n >> 16);
    n = ((n & 0x00ff00ff) << 8) | ((n & 0xff00ff00) >> 8);
    n = ((n & 0x0f0f0f0f) << 4) | ((n & 0xf0f0f0f0) >> 4);
    n = ((n & 0x33333333) << 2) | ((n & 0xcccccccc) >> 2);
    n = ((n & 0x55555555) << 1) | ((n & 0xaaaaaaaa) >> 1);
    return n;
}
\end{lstlisting}

然后通过单独翻转两个32位分量再互换它们可以翻转64位值的数位。
\begin{lstlisting}
`\refcode{Low Discrepancy Inline Functions}{+=}\lastnext{LowDiscrepancyInlineFunctions}`
inline uint64_t `\initvar{ReverseBits64}{}`(uint64_t n) {
    uint64_t n0 = `\refvar{ReverseBits32}{}`((uint32_t)n);
    uint64_t n1 = `\refvar{ReverseBits32}{}`((uint32_t)(n >> 32));
    return (n0 << 32) | n1;
}
\end{lstlisting}

然后为了计算2进制倒根,我们翻转数位并乘以$\displaystyle\frac{1}{2^{64}}$,
其中十六进制浮点常数{\ttfamily 0x1p-64}用于值$2^{-64}$。
如\refsub{浮点算术}解释的,通过相应的幂2乘法实现幂2除法
会给出以IEEE浮点数表示的相同结果(且浮点数乘法通常比浮点数除法更高效)。
\begin{lstlisting}
`\initcode{Compute base-2 radical inverse}{=}`
return `\refvar{ReverseBits64}{}`(a) * 0x1p-64;
\end{lstlisting}

对于其他基,模板函数\refvar{RadicalInverseSpecialized}{()}计算倒根
是通过计算起始于$d_1$的数字$d_i$并计算一系列$v_i$,
其中$v_1=d_1$,$v_2=bd_1+d_2$,使得
\begin{align*}
    v_n=b^{n-1}d_1+b^{n-2}d_2+\cdots+d_n\, .
\end{align*}
(例如,十进制中它会把值1234转化为4321。)
该值可完全用整数算法求得,不会积累任何舍入误差。

然后倒根的最终值通过转化为浮点并乘以$\displaystyle\frac{1}{b^n}$求得,
其中$n$是该值的位数,由此得到\refeq{7.7}中的值。
该乘法项是在处理数字时在{\ttfamily invBaseN}中构建的。
\begin{lstlisting}
`\initcode{Low Discrepancy Static Functions}{=}\initnext{LowDiscrepancyStaticFunctions}`
template <int base>
static `\refvar{Float}{}` `\initvar{RadicalInverseSpecialized}{}`(uint64_t a) {
    const `\refvar{Float}{}` invBase = (`\refvar{Float}{}`)1 / (`\refvar{Float}{}`)base;
    uint64_t reversedDigits = 0;
    `\refvar{Float}{}` invBaseN = 1;
    while (a) {
        uint64_t next  = a / base;
        uint64_t digit = a - next * base;
        reversedDigits = reversedDigits * base + digit;
        invBaseN *= invBase;
        a = next;
    }
    return std::min(reversedDigits * invBaseN, `\refvar{OneMinusEpsilon}{}`);
}
\end{lstlisting}

一个自然要问的问题是为什么这里要用对基数做参数化的模板函数
(而不是说调用一个把基数作为参数接收的常规函数,避免为每个基生成单独的代码路径)。
动机是在现代CPU上整数除法非常地慢,利用编译时常数除法的方法则能高效得多。

例如,32位值除以3的整数除法可以通过用该值乘以2863311531得到64位中间值
然后将结果右移33位算得;这俩都是非常高效的运算。
(64位值除以3可以用类似方法,但这个神奇常数大得多;
见\citet{10.5555/2462741}\sidenote{译者注:此处引用改为了新版。}了解关于该技术的更多内容。)
因此这里用模板函数允许编译器意识到在{\ttfamily while}循环中计算{\ttfamily next}值的除法时
实际上是除以常数并给它应用该优化的机会。在2015年代笔记本上用了该优化的代码比
基于整数除法指令的实现运行起来快至5.9倍。

另一个优化是我们避免计算翻转数字与倒数基之积的实时总和;
相反,该乘法会一直推迟到结尾直到循环终止。
这里的主要问题是当前处理器的浮点和整数单元是完全互相独立运算的。
在一个紧密循环里的浮点计算中引用一个整数变量会引入管道暂停
\sidenote{译者注:原文pipeline bubble。},
它与转化这些值并从一个单元移动到另一个单元所需的时间相关。

能计算倒根函数的逆会很有用;函数\refvar{InverseRadicalInverse}{()}接收
某进制翻转过的整数数字,即对应于模板函数\refvar{RadicalInverseSpecialized}{()}中
为了转化为$[0,1)$中的浮点值而乘以因子$\displaystyle\frac{1}{b^n}$之前的{\ttfamily reversedDigits}。
注意为了能正确地计算逆,必须知道原始值中的数字总数:
例如倒根算法中在只含整数的部分之后,1234和123400都能转化为4321;
尾部零变为前导零并丢失了。
\begin{lstlisting}
`\refcode{Low Discrepancy Inline Functions}{+=}\lastnext{LowDiscrepancyInlineFunctions}`
template <int base> inline uint64_t
`\initvar{InverseRadicalInverse}{}`(uint64_t inverse, int nDigits) {
    uint64_t index = 0;
    for (int i = 0; i < nDigits; ++i) {
        uint64_t digit = inverse % base;
        inverse /= base;
        index = index * base + digit;
    }
    return index;
}
\end{lstlisting}

Hammersley和Halton序列的缺点是随着基$b$的增加,样本值会展现出惊人的规律模式。
该问题可以用\keyindex{置乱}{scrambled}{}Halton和Hammersley序列解决,
其计算倒根时对数字施加了重排。
\begin{align}
    \label{eq:7.8}
    \varPsi_b(a)=0.p(d_1(a))p(d_2(a))\ldots p(d_m(a))\, ,
\end{align}
其中$p$是数字$(0,1,\ldots,b-1)$的重排。
注意每个数字用的重排是一样的,在给定基$b$下生成所有样本点用的重排也一样。
\reffig{7.26}展示了对Halton序列置乱后的效果。
\begin{figure}[htbp]
    \centering
    \subfloat[]{\includegraphics[width=0.45\linewidth]{chap07/halton2931.eps}\label{fig:7.26.1}}\,
    \subfloat[]{\includegraphics[width=0.45\linewidth]{chap07/halton2931-permuted.eps}\label{fig:7.26.2}}
    \caption{Halton样本值是否有置乱的图示。(a)在样本向量的高维中,
        样本值的投影开始表现出规律结构。这里展示了来自维度$(\varPhi_{29}(a),\varPhi_{31}(a))$的点。
        (b)置乱序列即\refeq{7.8}通过随机重排样本索引的数字打破了该结构。}
    \label{fig:7.26}
\end{figure}

尽管专门构造的重排能给出稍微更好的结构,但接下来我们将用随机重排;
见“扩展阅读”一节了解更多细节。

函数\refvar{ComputeRadicalInversePermutations}{()}计算这些随机重排表。
它为所有重排初始化单个连续数组,其前两个值是$b=2$时整数零和一的重排,
接下来三个值是$b=3$时0、1、2的重排,后续质数基以此类推。
在进入下面的{\ttfamily for}循环时,{\ttfamily p}指向重排数组的起点来为当前的质数基做初始化。
\begin{lstlisting}
`\refcode{Low Discrepancy Function Definitions}{+=}\lastnext{LowDiscrepancyFunctionDefinitions}`
std::vector<uint16_t> `\initvar{ComputeRadicalInversePermutations}{}`(`\refvar{RNG}{}` &rng) {
    std::vector<uint16_t> perms;
    `\refcode{Allocate space in perms for radical inverse permutations}{}`
    uint16_t *p = &perms[0];
    for (int i = 0; i < `\refvar{PrimeTableSize}{}`; ++i) {
        `\refcode{Generate random permutation for ith prime base}{}`
        p += `\refvar{Primes}{}`[i];
    }
    return perms;
}
\end{lstlisting}

重排数组的总大小由直到预先计算的质数表末尾的质数之和给出。
\begin{lstlisting}
`\initcode{Allocate space in perms for radical inverse permutations}{=}`
int permArraySize = 0;
for (int i = 0; i < `\refvar{PrimeTableSize}{}`; ++i)
    permArraySize += `\refvar{Primes}{}`[i];
perms.resize(permArraySize);
\end{lstlisting}
\begin{lstlisting}
`\initcode{Low Discrepancy Declarations}{=}\initnext{LowDiscrepancyDeclarations}`
static constexpr int `\initvar{PrimeTableSize}{}` = 1000;
extern const int `\initvar{Primes}{}`[`\refvar{PrimeTableSize}{}`];
\end{lstlisting}
\begin{lstlisting}
`\initcode{Low Discrepancy Data Definitions}{=}\initnext{LowDiscrepancyDataDefinitions}`
const int `\refvar{Primes}{}`[`\refvar{PrimeTableSize}{}`] = {
    2, 3, 5, 7, 11,
    `\refcode{Subsequent prime numbers}{}`
};
\end{lstlisting}

生成每次重排很简单:我们只需初始化{\ttfamily p}使之指向
对应当前质数长度的同一重排然后随机打乱它的值。
\begin{lstlisting}
`\initcode{Generate random permutation for ith prime base}{=}`
for (int j = 0; j < `\refvar{Primes}{}`[i]; ++j)
    p[j] = j;
`\refvar{Shuffle}{}`(p, `\refvar{Primes}{}`[i], 1, rng);
\end{lstlisting}

函数\refvar{ScrambledRadicalInverse}{()}本质上和\refvar{RadicalInverse}{()}一样,
除了它是通过重排表为给定基数放入每个数字的。
见\refvar{RadicalInverse}{()}之后的习题\ref{sub:7.11.3}了解关于2进制情况下的更高效实现的讨论。
\begin{lstlisting}
`\refcode{Low Discrepancy Function Definitions}{+=}\lastcode{LowDiscrepancyFunctionDefinitions}`
`\refvar{Float}{}` `\initvar{ScrambledRadicalInverse}{}`(int baseIndex, uint64_t a,
        const uint16_t *perm) {
    switch (baseIndex) {
        case 0: return `\refvar{ScrambledRadicalInverseSpecialized}{}`<2>(perm, a);
        case 1: return `\refvar{ScrambledRadicalInverseSpecialized}{}`<3>(perm, a);
        case 2: return `\refvar{ScrambledRadicalInverseSpecialized}{}`<5>(perm, a);
        case 3: return `\refvar{ScrambledRadicalInverseSpecialized}{}`<7>(perm, a);
        `\refcode{Remainder of cases for ScrambledRadicalInverse()}{}`
    }
}
\end{lstlisting}

下面的实现也考虑了当{\ttfamily perm}将数字0映射为非零值时可能出现的特殊情况。
该情况下,一旦{\ttfamily a}到达0迭代就提前停止,
错误地漏掉了值{\ttfamily perm[0]}构成的无限长后继。
幸运的是,这是一个有简单解析解的几何级数,最后一行加上了其值
\sidenote{译者注:以$b=4$、{\ttfamily perm}数组为$[3,2,1,0]$、$a=9$时为例:
$a$以四进制表示为$21$,翻转得$12$,变为四进制浮点数为$0.12$,准确点说是$0.120000\ldots$。
按照重排表{\ttfamily perm}替换数字后应为$0.213333\ldots$。
但代码只处理到前两位即0.21,故应手动加上后面的$0.003333\ldots$。
一般地,应补充的$b$进制数字为$0.\underbrace{0\ldots0}_{m\text{个}0}p(0)p(0)p(0)\ldots$,
其中$m$为$a$的$b$进制位数,依据等比数列求和公式可得
其值是$\displaystyle\frac{b^{-1}p(0)}{1-b^{-1}}b^{-m}$。}。
\begin{lstlisting}
`\refcode{Low Discrepancy Static Functions}{+=}\lastcode{LowDiscrepancyStaticFunctions}`
template <int base>
static `\refvar{Float}{}` `\initvar{ScrambledRadicalInverseSpecialized}{}`(const uint16_t *perm,
        uint64_t a) {
    const `\refvar{Float}{}` invBase = (`\refvar{Float}{}`)1 / (`\refvar{Float}{}`)base;
    uint64_t reversedDigits = 0;
    `\refvar{Float}{}` invBaseN = 1;
    while (a) {
        uint64_t next  = a / base;
        uint64_t digit = a - next * base;
        reversedDigits = reversedDigits * base + perm[digit];
        invBaseN *= invBase;
        a = next;
    }
    return std::min(invBaseN * (reversedDigits +
                    invBase * perm[0] / (1 - invBase)), `\refvar{OneMinusEpsilon}{}`);
}
\end{lstlisting}

\subsection{Halton采样器实现}\label{sub:Halton采样器实现}
\refvar{HaltonSampler}{}用Halton序列生成样本向量。
不像\refvar{StratifiedSampler}{}那样,它是完全确定性的;
在其运算中它没有使用伪随机数。然而,如果没有充分采样好图像,
则Halton样本可能导致混叠。\reffig{7.27}比较了用基于Halton的采样器
和用上一节的分层采样器来采样棋盘纹理的结果。
注意前景中以及朝向地平线处沿边缘的令人难受的模式。
\begin{figure}[htbp]
    \centering
    \subfloat[1个扰动样本]{\includegraphics[width=\linewidth]{chap07/checkerboard-jitter-1spp.png}\label{fig:7.27.1}}\\
    \subfloat[1个Halton样本]{\includegraphics[width=\linewidth]{chap07/checkerboard-halton-1spp.png}\label{fig:7.27.2}}
    \caption{在图像平面上比较分层采样器和基于Halton点的低偏差采样器。
        (a)每个像素只有单个样本的扰动分层采样器与(b)每个像素只有单个样本的\refvar{HaltonSampler}{}采样器。
        注意尽管Halton模式比分层模式更能复现朝向地平线处的棋盘模式,
        但在低偏差模式中仍有干扰视觉注意力的规律性错误结构;
        它没有像扰动法那样把混叠转化为不那么讨厌的噪声。}
    \label{fig:7.27}
\end{figure}

\begin{lstlisting}
`\initcode{HaltonSampler Declarations}{=}`
class `\initvar{HaltonSampler}{}` : public `\refvar{GlobalSampler}{}` {
public:
    `\refcode{HaltonSampler Public Methods}{}`
private:
    `\refcode{HaltonSampler Private Data}{}`
    `\refcode{HaltonSampler Private Methods}{}`
};
\end{lstlisting}
\begin{lstlisting}
`\initcode{HaltonSampler Method Definitions}{=}\initnext{HaltonSamplerMethodDefinitions}`
`\refvar{HaltonSampler}{}`::`\refvar{HaltonSampler}{}`(int samplesPerPixel,
        const `\refvar{Bounds2i}{}` &sampleBounds)
    : `\refvar{GlobalSampler}{}`(samplesPerPixel) {
    `\refcode{Generate random digit permutations for Halton sampler}{}`
    `\refcode{Find radical inverse base scales and exponents that cover sampling area}{}`
    `\refcode{Compute stride in samples for visiting each pixel area}{}`
    `\refcode{Compute multiplicative inverses for baseScales}{}`
}
\end{lstlisting}
\begin{lstlisting}
`\initcode{HaltonSampler Public Methods}{=}`
`\refvar{HaltonSampler}{}`(int nsamp, const `\refvar{Bounds2i}{}` &sampleBounds);
int64_t `\refvar[HaltonSampler::GetIndexForSample]{GetIndexForSample}{}`(int64_t sampleNum) const;
`\refvar{Float}{}` `\refvar[HaltonSampler::SampleDimension]{SampleDimension}{}`(int64_t index, int dimension) const;
std::unique_ptr<`\refvar{Sampler}{}`> `\refvar{Clone}{}`(int seed);
\end{lstlisting}
\begin{lstlisting}
`\initcode{Compute multiplicative inverses for baseScales}{=}`
`\refvar{multInverse}{}`[0] = `\refvar{multiplicativeInverse}{}`(`\refvar{baseScales}{}`[1], `\refvar{baseScales}{}`[0]);
`\refvar{multInverse}{}`[1] = `\refvar{multiplicativeInverse}{}`(`\refvar{baseScales}{}`[0], `\refvar{baseScales}{}`[1]);
\end{lstlisting}

置乱倒根的重排表在所有\refvar{HaltonSampler}{}实例间
共享并在构造函数首次运行时算出。对pbrt的需求而言,该方法很好:
当前实现只是为不同图块使用不同采样器实例,而那时我们想总是使用相同的重排。
对于其他用途,对何时使用不同的重排有更强的控制是值得的。
\begin{lstlisting}
`\initcode{Generate random digit permutations for Halton sampler}{=}`
if (`\refvar{radicalInversePermutations}{}`.size() == 0) {
    `\refvar{RNG}{}` rng;
    `\refvar{radicalInversePermutations}{}` = `\refvar{ComputeRadicalInversePermutations}{}`(rng);
}
\end{lstlisting}
\begin{lstlisting}
`\initcode{HaltonSampler Private Data}{=}\initnext{HaltonSamplerPrivateData}`
static std::vector<uint16_t> `\initvar{radicalInversePermutations}{}`;
\end{lstlisting}

实用方法\refvar{PermutationForDimension}{()}为给定
维度返回指向重排数组起点的指针。
\begin{lstlisting}
`\initcode{HaltonSampler Private Methods}{=}`
const uint16_t *`\initvar{PermutationForDimension}{}`(int dim) const {
    if (dim >= `\refvar{PrimeTableSize}{}`)
        `\refvar{Severe}{}`("HaltonSampler can only sample %d dimensions.",
               `\refvar{PrimeTableSize}{}`);
    return &`\refvar{radicalInversePermutations}{}`[`\refvar{PrimeSums}{}`[dim]];
}
\end{lstlisting}

为了能为给定维度快速找到偏移量,拥有每个质数之前的所有质数之和是很有帮助的。
\begin{lstlisting}
`\refcode{Low Discrepancy Data Definitions}{+=}\lastcode{LowDiscrepancyDataDefinitions}`
const int `\initvar{PrimeSums}{}`[`\refvar{PrimeTableSize}{}`] = {
    0, 2, 5, 10, 17, 
    `\refcode{Subsequent prime sums}{}`
};
\end{lstlisting}

为了把来自$[0,1)^2$的样本前两维映射为像素坐标,
\refvar{HaltonSampler}{}在每个维度上求得比图像分辨率
和\refvar{kMaxResolution}{}中较小者更大的最小缩放因子$(2^j,3^k)$
(我们稍后将看到这一专门选出的缩放值是怎样让看出一个样本在哪个像素里变简单的)。
缩放之后,任何图像范围之外的样本都被简单忽略掉。

对于分辨率在一或两维上大于\refvar{kMaxResolution}{}的图像,
则在全图上重复一块Halton点。该分辨率限制帮助在算得的样本值中保持足够的浮点精度。
\begin{lstlisting}
`\initcode{Find radical inverse base scales and exponents that cover sampling area}{=}`
`\refvar{Vector2i}{}` res = sampleBounds.`\refvar{pMax}{}` - sampleBounds.`\refvar{pMin}{}`;
for (int i = 0; i < 2; ++i) {
    int base = (i == 0) ? 2 : 3;
    int scale = 1, exp = 0;
    while (scale < std::min(res[i], `\refvar{kMaxResolution}{}`)) {
        scale *= base;
        ++exp;
    }
    `\refvar{baseScales}{}`[i] = scale;
    `\refvar{baseExponents}{}`[i] = exp;
}
\end{lstlisting}

对于每个维度,\refvar{baseScales}{}存有缩放因子$2^j$或$3^k$,
且\refvar{baseExponents}{}存有指数$j$和$k$。
\begin{lstlisting}
`\refcode{HaltonSampler Private Data}{+=}\lastnext{HaltonSamplerPrivateData}`
`\refvar{Point2i}{}` `\initvar{baseScales}{}`, `\initvar{baseExponents}{}`;
\end{lstlisting}
\begin{lstlisting}
`\initcode{HaltonSampler Local Constants}{=}`
static constexpr int `\initvar{kMaxResolution}{}` = 128;
\end{lstlisting}

为了明白为什么\refvar{HaltonSampler}{}使用该方案来
将样本映射到像素坐标,考虑用$b$进制倒根因子$b^n$来缩放算出的值的影响。
如果$a$的数字表示为$b$进制的$d_i(a)$,则回想倒根为$b$进制的值$0.d_1(a)d_2(a)\ldots$。
例如,如果我们用$b^2$乘以该值,则我们有$d_1(a)d_2(a).d_3(a)\ldots$;
前两个数字被移到小数点左边,该值的小数部分从$d_3(a)$开始。

用$b^n$缩放的该运算构建了能够确定哪个样本索引位于哪个像素中的核心。
考虑上面例子中的前两个数字,我们可以看到缩放后的值的整数部分在0到$b^2-1$的范围内,
且随着$a$增加,在该范围内每过$b^2$个值,其$b$进制的最后两个数字就在某特定值上取得一次。

给定值$x$,$0\le x\le b^2-1$,我们可以求得首个整数部分为$x$的值$a$。
根据定义,$b$进制的$x$数字为$d_2(x)d_1(x)$。
因此,如果$d_1(a)=d_2(x)$且$d_2(a)=d_1(x)$,
则$a$的倒根被缩放后的值会有等于$x$的整数部分。

因为\refvar{HaltonSampler}{}中为像素样本用的基$b=2$和$b=3$是互质的,
所以它满足如果样本值由某个$(2^j,3^k)$缩放,则在
范围$(0,0)\rightarrow(2^j-1,3^k-1)$内的任意特定像素
每过$2^j3^k$个样本就会访问一次。该乘积存于\refvar{sampleStride}{}中。
\begin{lstlisting}
`\initcode{Compute stride in samples for visiting each pixel area}{=}`
`\refvar{sampleStride}{}` = `\refvar{baseScales}{}`[0] * `\refvar{baseScales}{}`[1];
\end{lstlisting}
\begin{lstlisting}
`\refcode{HaltonSampler Private Data}{+=}\lastnext{HaltonSamplerPrivateData}`
int `\initvar{sampleStride}{}`;
\end{lstlisting}
\begin{lstlisting}
`\refcode{HaltonSampler Private Data}{+=}\lastnext{HaltonSamplerPrivateData}`
int `\initvar{multInverse}{}`[2];
\end{lstlisting}

位于\refvar{currentPixel}{}内的首个Halton样本的样本索引存于\refvar{offsetForCurrentPixel}{}
中。在为当前像素中的首个样本算出该偏移量后,该像素中的后续样本
可在Halton序列中增量为\refvar{sampleStride}{}的样本处找到
\sidenote{译者注:指索引每增加\refvar{sampleStride}{}就出现一次该像素的样本。}。
\begin{lstlisting}
`\refcode{HaltonSampler Method Definitions}{+=}\lastnext{HaltonSamplerMethodDefinitions}`
int64_t `\refvar{HaltonSampler}{}`::`\initvar[HaltonSampler::GetIndexForSample]{\refvar{GetIndexForSample}{}}{}`(int64_t sampleNum) const {
    if (`\refvar{currentPixel}{}` != `\refvar{pixelForOffset}{}`) {
        `\refcode{Compute Halton sample offset for currentPixel}{}`
        `\refvar{pixelForOffset}{}` = `\refvar{currentPixel}{}`;
    }
    return `\refvar{offsetForCurrentPixel}{}` + sampleNum * `\refvar{sampleStride}{}`;
}
\end{lstlisting}
\begin{lstlisting}
`\refcode{HaltonSampler Private Data}{+=}\lastcode{HaltonSamplerPrivateData}`
mutable `\refvar{Point2i}{}` `\initvar{pixelForOffset}{}` = `\refvar{Point2i}{}`(std::numeric_limits<int>::max(),
                                         std::numeric_limits<int>::max());
mutable int64_t `\initvar{offsetForCurrentPixel}{}`;
\end{lstlisting}

在给定像素$(x,y)$内计算首个已被$(2^j,3^k)$缩放的样本索引
包括计算$x$的2进制最后$j$个数字的逆倒根,我们将其记作$x_r$,
以及$y$的3进制最后$k$个数字的逆倒根$y_r$。这为我们给出了方程组
\begin{align*}
    x_r & \equiv(i \mod 2^j)\, , \\
    y_r & \equiv(i \mod 3^k)\, ,
\end{align*}
其中满足该方程组的索引$i$就是位于给定像素内缩放后的样本索引。
本书中我们没有包含这里解出$i$的代码{\refcode{Compute Halton sample offset for currentPixel}{}}
\sidenote{译者注:我补充回来了。};见Gr\"{u}nschlo\ss{}和Keller \parencite*{10.1007/978-3-642-04107-5_25}
了解用于求解$i$的该算法的细节\sidenote{译者注:参考文献列表中作者名字
    中的德文字母“\ss{}”错误显示为“SS”,目前无法解决,请读者见谅。
    同时欢迎提供解决办法!此外,原文引用的文献将年份误写为2012,已修正。}。
\begin{lstlisting}
`\initcode{Compute Halton sample offset for currentPixel}{=}`
`\refvar{offsetForCurrentPixel}{}` = 0;
if (`\refvar{sampleStride}{}` > 1) {
    `\refvar{Point2i}{}` pm(`\refvar{Mod}{}`(`\refvar{currentPixel}{}`[0], `\refvar{kMaxResolution}{}`),
               `\refvar{Mod}{}`(`\refvar{currentPixel}{}`[1], `\refvar{kMaxResolution}{}`));
    for (int i = 0; i < 2; ++i) {
        uint64_t dimOffset = (i == 0) ?
            `\refvar{InverseRadicalInverse}{}`<2>(pm[i], `\refvar{baseExponents}{}`[i]) :
            `\refvar{InverseRadicalInverse}{}`<3>(pm[i], `\refvar{baseExponents}{}`[i]);
        `\refvar{offsetForCurrentPixel}{}` += dimOffset * (`\refvar{sampleStride}{}` / `\refvar{baseScales}{}`[i]) * `\refvar{multInverse}{}`[i];
    }
    `\refvar{offsetForCurrentPixel}{}` %= `\refvar{sampleStride}{}`;
}
\end{lstlisting}

样本偏移量的计算不考虑随机数字重排,所以在这里算出的样本值中没有包含它们。
此外,因为前两维中低处的\refvar{baseExponents}{[i]}个数字用于选择采样哪个样本,
所以这些数字必须在为样本向量的前两维计算倒根之前就被丢弃掉,
此后方法\refvar[HaltonSampler::SampleDimension]{SampleDimension}{()}应该返回正被采样的像素内的小数偏移量。
更高维则被直接采样,包含随机重排。
\begin{lstlisting}
`\refcode{HaltonSampler Method Definitions}{+=}\lastcode{HaltonSamplerMethodDefinitions}`
`\refvar{Float}{}` `\refvar{HaltonSampler}{}`::`\initvar[HaltonSampler::SampleDimension]{\refvar{SampleDimension}{}}{}`(int64_t index, int dim) const {
    if (dim == 0)
        return `\refvar{RadicalInverse}{}`(dim, index >> `\refvar{baseExponents}{}`[0]);
    else if (dim == 1)
        return `\refvar{RadicalInverse}{}`(dim, index / `\refvar{baseScales}{}`[1]);
    else
        return `\refvar{ScrambledRadicalInverse}{}`(dim, index,
            `\refvar{PermutationForDimension}{}`(dim));
}
\end{lstlisting}

\section{(0,2)序列采样器}\label{sec:(0,2)序列采样器}
\begin{remark}
    本节含有高级内容,第一次阅读时可以跳过。
\end{remark}

另一个生成高质量样本的方法是利用某些低偏差序列的显著性质
即允许我们满足两个想要的样本性质(其中仅一个用\refvar{StratifiedSampler}{}满足了):
它们为图像样本的一个像素值生成样本向量使得每个像素样本的样本值都彼此间分布良好,
同时该像素中所有像素样本的样本值集合也整体上分布良好。

该序列使用Sobol\footnote{\protect\refsec{Sobol采样器}的\refvar{SobolSampler}{}使用了
    Sobol序列的所有维度。}推导出的低偏差序列的前两维。
该序列是一种特殊类型的低偏差序列,称为$(0,2)$序列。
$(0,2)$序列以非常常规的方式分层。例如,$(0,2)$序列中的前16个样本
满足来自\refsec{分层采样}中分层采样的分层约束,
意味着每个范围为$\displaystyle\left(\frac{1}{4},\frac{1}{4}\right)$的矩形中只存在一个样本。
然而它们还满足拉丁超立方约束,即在每个范围为$\displaystyle\left(\frac{1}{16},1\right)$和
$\displaystyle\left(1,\frac{1}{16}\right)$的矩形中只有一个样本。
此外,在每个范围为$\displaystyle\left(\frac{1}{2},\frac{1}{8}\right)$和
$\displaystyle\left(\frac{1}{8},\frac{1}{2}\right)$的矩形中只有一个样本。

\reffig{7.28}展示了划分域的所有可能,其中$(0,2)$序列前16个样本都满足分层性质。
从该模式中获取的每组含16个样本的后续序列也都满足这些分布性质。
\begin{figure}[htbp]
    \centering
    \includegraphics[width=0.49\linewidth]{chap07/elementary1x16.eps}\,
    \includegraphics[width=0.49\linewidth]{chap07/elementary2x8.eps}\\
    \includegraphics[width=0.49\linewidth]{chap07/elementary4x4.eps}\,
    \includegraphics[width=0.49\linewidth]{chap07/elementary8x2.eps}\\
    \includegraphics[width=0.49\linewidth]{chap07/elementary16x1.eps}
    \caption{在所有以2为底数的基本区间中都只有单个样本的采样模式。
        它同时满足$4\times4$分层和拉丁超立方约束以及所示的其他两个分层约束。}
    \label{fig:7.28}
\end{figure}

通常任何来自$(0,2)$序列的长为$2^{l_1+l_2}$的序列(其中$l_i$为非负整数)
都满足该一般分层约束。以2为底的两维的\keyindex{基本区间}{elementary interval}{}定义为
\begin{align*}
    E=\left\{\left[\frac{a_1}{2^{l_1}},\frac{a_1+1}{2^{l_1}}\right)\times\left[\frac{a_2}{2^{l_2}},\frac{a_2+1}{2^{l_2}}\right)\right\}\, ,
\end{align*}
其中整数$a_i=0,1,2\ldots,2^{l_i}-1$.
该序列中前$2^{l_1+l_2}$个值中的每一个样本都在相应基本区间中。
此外,后续每$2^{l_1+l_2}$个值构成的集合也满足同样的性质。

现在为了理解怎样把$(0,2)$序列应用于生成2D样本,
考虑有$2\times2$图像样本的像素,每个含$4\times4$个2D样本构成的数组。
依据相应的基本区间集,$(0,2)$序列前$(2\times2)\times(4\times4)=2^6$个值相互之间分布良好。
此外依据其相应的基本区间,前$4\times4=2^4$个样本自己也分布良好,
其后的$2^4$个也是这样,以此类推。因此,我们可以为一个像素的
首个图像样本的$4\times4$数组样本使用前16个$(0,2)$序列样本,
然后下一个图像样本用接下来的16个,以此类推。
结果是分布非常良好的样本点集。

\subsection{用生成矩阵采样}\label{sub:用生成矩阵采样}
比起\refvar{HaltonSampler}{},Sobol序列基于不同的样本点生成机制,
它在各维度上使用了倒根。即使将倒根函数中的整数除法转化为乘法和位移,
高质量高分辨率渲染所需的计算数十亿样本的计算量也会很大。
大部分计算开销来自于在天生是2进制的计算机上执行不以2为底的计算
(考虑代码片\refcode{Compute base-2 radical inverse}{}和模板函数\refvar{RadicalInverseSpecialized}{()}间的差别)。

假设不以2为底的计算有很高开销,自然要尝试开发完全用2进制运算的样本生成算法。
一个这样高效的方法是用\keyindex{生成矩阵}{generator matrix}{},
它允许在相同基中完成所有计算。不像Halton采样器那样在每个维度使用不同基,
它在每个维度使用不同生成矩阵。为每个采样维度精选好矩阵,
就能生成非常好的低偏差分布点。例如,$(0,2)$序列可用两个特定的2进制生成矩阵定义。

为了看看怎样使用生成矩阵,考虑一个$n$位$b$进制数$a$,
其中$a$的第$i$个数字是$d_i(a)$,且我们有个$n\times n$生成矩阵$\bm C$.
则相应样本点$x_a\in[0,1)$定义为
\begin{align}\label{eq:7.9}
    x_a=[b^{-1}\,b^{-2}\, \cdots\,b^{-n}]\left[
        \begin{array}{cccc}
            c_{1,1} & c_{1,2} & \cdots & c_{1,n} \\
            c_{2,1} & \ddots  &        & c_{2,n} \\
            \vdots  &         & \ddots & \vdots  \\
            c_{n,1} & \cdots  & \cdots & c_{n,n}
        \end{array}
        \right]
    \left[
        \begin{array}{c}
            d_1(a) \\
            d_2(a) \\
            \vdots \\
            d_n(a)
        \end{array}
        \right]\, ,
\end{align}
其中的所有运算都在环$\mathsf{Z}_b$中执行(换句话说,所有运算都按模$b$执行)。
当$a$从0变到$b^n-1$时,这一构造给出一共$b^n$个点。
如果该生成矩阵是单位矩阵,则该定义对应于常规的基$b$倒根
(在继续之前,值得你停下来看看\refeq{7.7}与\refeq{7.9}间的联系)。

本节中,我们将只用$b=2$和$n=32$.尽管把$32\times32$矩阵引入样本生成算法
可能看似不是迈向更优性能的一步,但我们最终将看到采样代码可以映射为
以极为高效的方式用少量位运算执行该计算的实现。

迈向高性能的第一步来自我们按2进制处理的事实;
这样$\bm C$的所有元素不是0就是1,
且因此我们可以把矩阵每行或每列表示为一个无符号32位整数。
我们将选择把矩阵的列表示为{\ttfamily uint32\_t};
该抉择得到一个让$d_i$列向量和$\bm C$相乘的非常高效的算法。

现在考虑计算矩阵-向量乘积${\bm C}[d_i(a)]^{\mathrm{T}}$的任务;
利用矩阵-向量乘积的定义,我们有:
\begin{align}\label{eq:7.10}
    \left[
        \begin{array}{cccc}
            c_{1,1} & c_{1,2} & \cdots & c_{1,n} \\
            c_{2,1} & \ddots  &        & c_{2,n} \\
            \vdots  &         & \ddots & \vdots  \\
            c_{n,1} & \cdots  & \cdots & c_{n,n}
        \end{array}
        \right]\left[
        \begin{array}{c}
            d_1(a) \\
            d_2(a) \\
            \vdots \\
            d_n(a)
        \end{array}
        \right]
    =d_1\left[
        \begin{array}{c}
            c_{1,1} \\
            c_{2,1} \\
            \vdots  \\
            c_{n,1}
        \end{array}
        \right]+\cdots+d_n\left[
        \begin{array}{c}
            c_{1,n} \\
            c_{2,n} \\
            \vdots  \\
            c_{n,n}
        \end{array}
        \right]\, .
\end{align}
换句话说,对于$d_i$每个值为1的数字,$\bm C$的对应列应被求和。
该加法反过来可以在$\mathsf{Z}_2$中高效执行:
在该设置下,加法对应于异或运算(考虑两个运算值可能的组合——
0和1——相加$\mod{2}$的结果,并与同样运算值异或后算出的值比较)。
因此,乘法${\bm C}[d_i(a)]^{\mathrm{T}}$只是在$d_i(a)$位为1处
将$\bm C$的第$i$列异或在一起。该计算在函数\refvar{MultiplyGenerator}{()}中实现。
\begin{lstlisting}
`\refcode{Low Discrepancy Inline Functions}{+=}\lastnext{LowDiscrepancyInlineFunctions}`
inline uint32_t `\initvar{MultiplyGenerator}{}`(const uint32_t *C, uint32_t a) {
    uint32_t v = 0;
    for (int i = 0; a != 0; ++i, a >>= 1)
        if (a & 1)
            v ^= C[i];
    return v;
}
\end{lstlisting}

现在回到\refeq{7.9},若我们把积中的列向量
记为$v={\bm C}[d_i(a)]^{\mathrm{T}}$,则考虑向量积
\begin{align}\label{eq:7.11}
    x_a=[2^{-1}\, 2^{-2}\,\cdots\, 2^{-n}]\left[
        \begin{array}{c}
            v_1    \\
            v_2    \\
            \vdots \\
            v_n
        \end{array}
        \right]=\sum\limits_{i=1}^{32}{2^{-i}v_i}\, .
\end{align}
因为$v$的元素存于单个{\ttfamily uint32\_t},其值理解为{\ttfamily uint32\_t}时是
\begin{align*}
    v=v_1+2v_2+\cdots=\sum\limits_{i=1}^{32}{2^{i-1}v_i}\, .
\end{align*}
如果我们翻转{\ttfamily uint32\_t}内的数位顺序,则我们将有值
\begin{align*}
    v'=\sum\limits_{i=1}^{32}{2^{32-i}v_i}\, .
\end{align*}
这是更有用的值:如果我们将该值除以$2^{32}$,则我们得到\refeq{7.11},
即我们要尝试算出的$x_a$.

因此,如果我们取函数\refvar{MultiplyGenerator}{()}的结果,
颠倒返回值的数位顺序(例如用\refvar{ReverseBits32}{()}),
再用该值除以$2^{32}$以计算$[0,1)$中的浮点数,我们就算出了样本值。

为了节约翻转数位的小小开销,我们可以在传入\refvar{MultiplyGenerator}{()}之前
等价地翻转生成矩阵$\bm C$所有列的数位。接下来我们将遵循该约定。

实践中为了让$(0,2)$序列有用,我们还需要能为
每个图像样本生成多个不同的2D样本值集合,
且我们想为每个像素生成不同的样本值。
该问题的一个方法是为每个像素使用从$(0,2)$序列精选的不重叠子序列
\footnote{\refsec{Sobol采样器}的Sobol采样器采用该方法。}。
另一种方法是随机置乱$(0,2)$序列,通过把随机重排应用于原始序列值$b$进制数字
而构建得到新的$(0,2)$序列。

我们将用的置乱方法归功于\citet{10.1111/1467-8659.00706}。
它反复划分与打乱单位方形$[0,1)^2$.
在这两维的每一个中,它都先将该方形对半分再以50\%的概率交换这两半。
然后它再分别对半划分区间$[0,0.5)$和$[0.5,1)$并随机交换其两半。
该过程递归持续直到2进制表示的所有数位都处理完。
仔细设计该过程使其保留点集的低偏差性质;否则$(0,2)$序列的优势会在置乱时流失。
\reffig{7.29}展示了未置乱的$(0,2)$序列及其两个随机置乱的变种。
\begin{figure}[htbp]
    \centering
    \subfloat[]{\includegraphics[width=0.3\linewidth]{chap07/02-a.eps}\label{fig:7.29.1}}\quad
    \subfloat[]{\includegraphics[width=0.3\linewidth]{chap07/02-b.eps}\label{fig:7.29.2}}\quad
    \subfloat[]{\includegraphics[width=0.3\linewidth]{chap07/02-c.eps}\label{fig:7.29.3}}
    \caption{(a)基于低偏差$(0,2)$序列的采样模式以及(b,c)其两个随机置乱的例子。
        若我们在每个像素中用相同的采样模式,则图像中可能出现伪影,
        而低偏差模式的随机置乱是消除它的高效方法,且仍保留所用点集的低偏差性质。}
    \label{fig:7.29}
\end{figure}

函数\refvar{SampleGeneratorMatrix}{()}将这些片段组合在一起以生成样本值。
\begin{lstlisting}
`\refcode{Low Discrepancy Inline Functions}{+=}\lastnext{LowDiscrepancyInlineFunctions}`
inline `\refvar{Float}{}` `\initvar{SampleGeneratorMatrix}{}`(const uint32_t *C, uint32_t a,
        uint32_t scramble = 0) {
    return (`\refvar{MultiplyGenerator}{}`(C, a) ^ scramble) * 0x1p-32f;
}
\end{lstlisting}

函数\refvar{SampleGeneratorMatrix}{()}已经足够高效,
它每次在运行次数等于值{\ttfamily a}以2为底的对数
的\refvar{MultiplyGenerator}{()}循环中执行少量算数运算。
值得注意的是,通过改变生成样本的顺序、
以\keyindex{格雷码}{Gray code}{}顺序枚举甚至可以做得更好。

用格雷码表示的连续二进制值只相差一位;
\reftab{7.4}的第三列展示了格雷码顺序下的前八个整数。
注意不仅任意一对值之间只有一位改变,
而且在从0起的幂2个即$n$个值中,格雷码枚举了从0到$n-1$的所有值,
只是和平常的顺序不同。
\begin{table}[htbp]
    \centering
    \begin{tabular}{cccc}
        \toprule
        $n$\textbf{(10进制)} & $n$\textbf{(2进制)} & $g(n)$ & \textbf{改变的数位索引} \\
        \midrule
        0                      & 000                   & 000    & n/a                     \\
        1                      & 001                   & 001    & 0                       \\
        2                      & 010                   & 011    & 1                       \\
        3                      & 011                   & 010    & 0                       \\
        4                      & 100                   & 110    & 2                       \\
        5                      & 101                   & 111    & 0                       \\
        6                      & 110                   & 101    & 1                       \\
        7                      & 111                   & 100    & 0                       \\
        \bottomrule
    \end{tabular}
    \caption{格雷码顺序下的前八个整数。每个格雷码值$g(n)$和前一个$g(n-1)$都只有一位不同。
        改变的那位索引由2进制值$n$的尾零个数给出。注意从0起的任意$n$个即幂2个值集合中,
        0到$n-1$间的全部整数都表示了出来,只是和平常的顺序不同。}
    \label{tab:7.4}
\end{table}

可以非常高效地完成计算第$n$个格雷码值。
\begin{lstlisting}
`\refcode{Low Discrepancy Inline Functions}{+=}\lastnext{LowDiscrepancyInlineFunctions}`
inline uint32_t `\initvar{GrayCode}{}`(uint32_t n) {
    return (n >> 1) ^ n;
}
\end{lstlisting}

通过按格雷码顺序枚举样本,我们能极大利用在连续样本间$g(n)$只有一位改变的事实。
假设我们已经为某个索引$a$算出积${\bm C}[d_i(a)]^{\mathrm{T}}=v$:
如果另一个值$a'$和$a$只有一位不同,则我们只需从$v$中加上或减去$\bm C$的一列
以求得$v'={\bm C}[d_i(a')]^{\mathrm{T}}$(回想\refeq{7.10})
\sidenote{译者注:原文将$d_i(a')$误写为$d_I(a')$,已修正。}。
甚至更好的是,加法和减法$\mod{2}$都能用异或执行,
所以需要哪种运算并不重要;我们只需要知道哪一位改变了。
从\reftab{7.4}可以看出,从$g(i)$到$g(i+1)$变化的数位索引由
$i+1$的2进制表示的尾零数目给出。大部分CPU指令集可以在单条指令内数出尾零数量。

综上所述,我们可以用格雷码顺序下的生成矩阵非常高效地生成一系列样本。
\refvar{GrayCodeSample}{()}接收生成矩阵{\ttfamily C},
要生成的样本数量{\ttfamily n},并在{\ttfamily p}所指的内存位置中存储相应样本。
\begin{lstlisting}
`\refcode{Low Discrepancy Inline Functions}{+=}\lastnext{LowDiscrepancyInlineFunctions}`
inline void `\initvar{GrayCodeSample}{}`(const uint32_t *C, uint32_t n,
        uint32_t scramble, Float *p) {
    uint32_t v = scramble;
    for (uint32_t i = 0; i < n; ++i) {
        p[i] = v * 0x1p-32f;  /* 1/2^32 */
        v ^= C[`\refvar{CountTrailingZeros}{}`(i + 1)];
    }
}
\end{lstlisting}

内部循环(省略了循环控制逻辑)的核心x86汇编代码非常简单:
\begin{lstlisting}
    xorps      %xmm1, %xmm1
    cvtsi2ssq  %rax, %xmm1
    mulss      %xmm0, %xmm1
    movss      %xmm1, (%rcx,%rdx,4)
    incq       %rdx
    bsfl       %edx, %eax
    xorl       `\$`31, %eax
    xorl       (%rdi,%rax,4), %esi
\end{lstlisting}

即使不是x86汇编语言的爱好者,也能看出这是生成每个样本值的非常简短的指令序列。

还有生成2D样本的第二版\refvar{GrayCodeSample}{()}(这里没有介绍);
它为每维接收一个生成矩阵并把样本填入{\refvar{Point2f}{}}值的数组。

\subsection{采样器实现}\label{sub:采样器实现02}
\refvar{ZeroTwoSequenceSampler}{}用置乱的$(0,2)$序列
为胶片平面、透镜上的位置以及其他2D样本生成样本,
并用置乱的van der Corput序列生成1D样本。
\begin{lstlisting}
`\initcode{ZeroTwoSequenceSampler Declarations}{=}`
class `\initvar{ZeroTwoSequenceSampler}{}` : public `\refvar{PixelSampler}{}` {
public:
    `\refcode{ZeroTwoSequenceSampler Public Methods}{}`
};
\end{lstlisting}

\section{最大化最小距离采样器}\label{sec:最大化最小距离采样器}


\section{Sobol采样器}\label{sec:Sobol采样器}
\begin{remark}
    本节含有高级内容,第一次阅读时可以跳过。
\end{remark}

本章最后一个\refvar{Sampler}{}基于Sobol确定的一系列生成矩阵。
来自这些矩阵生成的序列的样本的区别在于能非常高效地实现——
因为是完全基于2进制计算的——而且在样本向量的全部$n$个维度上都分布得极好。
\reffig{7.34}展示了前几个Sobol生成矩阵。
\begin{figure}[htbp]
    \centering
    \includegraphics[width=0.24\linewidth]{chap07/sobol0.png}\,
    \includegraphics[width=0.24\linewidth]{chap07/sobol1.png}\,
    \includegraphics[width=0.24\linewidth]{chap07/sobol2.png}\,
    \includegraphics[width=0.24\linewidth]{chap07/sobol3.png}
    \caption{Sobol序列前四维的生成矩阵。注意其规则的结构。}
    \label{fig:7.34}
\end{figure}

\reffig{7.35}用景深测试场景比较了Sobol样本以及分层和Halton点。
\begin{figure}[htbp]
    \subfloat[分层采样]{\includegraphics[width=0.49\linewidth]{chap07/dof-stratified.png}\label{fig:7.35.1}}\,
    \subfloat[Halton采样]{\includegraphics[width=0.49\linewidth]{chap07/dof-halton.png}\label{fig:7.35.2}}\\
    \subfloat[Sobol采样]{\includegraphics[width=0.49\linewidth]{chap07/dof-sobol.png}\label{fig:7.35.3}}
    \caption{为渲染景深比较分层、Halton以及Sobol采样器。
        (a)用\refvar{StratifiedSampler}{}渲染的图像,
        (b)用\refvar{HaltonSampler}{}渲染的图像,以及
        (c)用\refvar{SobolSampler}{}渲染的图像。
        两个低偏差采样器都比分层采样器要好。尽管使用\refvar{SobolSampler}{}的该欠采样图像
        能看到结构化的网格伪影,但Sobol序列经常提供比Halton序列更快的收敛速度。}
    \label{fig:7.35}
\end{figure}

Sobol点的缺点是它们在收敛前容易出现结构化网格伪影;
在\reffig{7.36}展示的图像样本点中可以看出该问题。
\begin{figure}[htbp]
    \centering\includegraphics[width=0.5\linewidth]{chap07/sobol2x2pix.eps}
    \caption{$2\times2$像素的网格,每个用16个Sobol样本采样。
        注意有大量结构,且许多样本互相靠近。该序列在样本向量全部$n$个维度上
        极好的分布性质通常弥补了这些缺点。}
    \label{fig:7.36}
\end{figure}

在\reffig{7.37}的图像中该结构是可见的。
该缺点换来的是,Sobol序列在样本序列全部$n$个维度上分布得极其好。
\begin{figure}[htbp]
    \centering
    \subfloat[Halton采样器]{\includegraphics[width=\linewidth]{chap07/car-halton-undersampled.png}\label{fig:7.37.1}}\\
    \subfloat[Sobol采样器]{\includegraphics[width=\linewidth]{chap07/car-sobol-undersampled.png}\label{fig:7.37.2}}
    \caption{用(a)Halton采样器和(b)Sobol采样器渲染的欠采样图像。
        虽然具有不同的视觉特征,但两个都展现出可见的结构。
        特别是Sobol序列展现出清晰可见的棋盘结构。}
    \label{fig:7.37}
\end{figure}

\begin{lstlisting}
`\initcode{SobolSampler Declarations}{=}`
class `\initvar{SobolSampler}{}` : public `\refvar{GlobalSampler}{}` {
public:
    `\refcode{SobolSampler Public Methods}{}`
private:
    `\refcode{SobolSampler Private Data}{}`
};
\end{lstlisting}

\refvar{SobolSampler}{}用能让样本域$[0,1)^2$覆盖住要采样的图像区域
的最小幂2值来均匀缩放前两维。像\refvar{HaltonSampler}{}那样,
选择该特定缩放方案是为了更容易计算从像素坐标到每个像素内样本索引的逆映射。
\begin{lstlisting}
`\initcode{SobolSampler Public Methods}{=}`
`\refvar{SobolSampler}{}`(int64_t samplesPerPixel, const `\refvar{Bounds2i}{}` &sampleBounds)
    : `\refvar{GlobalSampler}{}`(`\refvar{RoundUpPow2}{}`(samplesPerPixel)),
      `\refvar{sampleBounds}{}`(sampleBounds) {
    `\refvar{resolution}{}` = `\refvar{RoundUpPow2}{}`(std::max(sampleBounds.`\refvar{Diagonal}{}`().x,
                                      sampleBounds.`\refvar{Diagonal}{}`().y));
    `\refvar{log2Resolution}{}` = `\refvar{Log2Int}{}`(`\refvar{resolution}{}`);
}
\end{lstlisting}
\begin{lstlisting}
`\initcode{SobolSampler Private Data}{=}`
const `\refvar{Bounds2i}{}` `\initvar{sampleBounds}{}`;
int `\initvar{resolution}{}`, `\initvar{log2Resolution}{}`;
\end{lstlisting}

如果采样域$[0,1)^2$已被缩放$2^{\text{\refvar{log2Resolution}{}}}$倍
以覆盖像素采样区域,则函数\refvar{SobolIntervalToIndex}{()}返回
像素{\ttfamily p}内第{\ttfamily sampleNum}个样本的索引。
\begin{lstlisting}
`\refcode{Low Discrepancy Declarations}{+=}\lastcode{LowDiscrepancyDeclarations}`
inline uint64_t `\initvar{SobolIntervalToIndex}{}`(const uint32_t log2Resolution,
    uint64_t sampleNum, const `\refvar{Point2i}{}` &p);
\end{lstlisting}

用于推导它所实现的算法的一般方法和Halton采样器在其方法\linebreak
\refvar[HaltonSampler::GetIndexForSample]{GetIndexForSample}{()}中用的一样。
积${\bm C}[d_i(a)]^{\mathrm{T}}$构建了缩放后样本的整数部分,
这里用幂2缩放意味着缩放值以2为底的对数给出了它的位数。
为了求得缩放后能给出特定整数值的$a$值,我们可以计算$\bm C$的逆:设
\begin{align*}
    v={\bm C}[d_i(a)]^{\mathrm{T}}\, ,
\end{align*}
则等价地
\begin{align*}
    {\bm C}^{-1}v=[d_i(a)]^{\mathrm{T}}\, .
\end{align*}
我们这里不会介绍该方法的实现。
\begin{lstlisting}
`\initcode{SobolSampler Method Definitions}{=}\initnext{SobolSamplerMethodDefinitions}`
int64_t `\refvar{SobolSampler}{}`::`\initvar[SobolSampler::GetIndexForSample]{\refvar{GetIndexForSample}{}}{}`(int64_t sampleNum) const {
    return `\refvar{SobolIntervalToIndex}{}`(`\refvar{log2Resolution}{}`, sampleNum,
        `\refvar{Point2i}{}`(`\refvar{currentPixel}{}` - `\refvar{sampleBounds}{}`.`\refvar{pMin}{}`));
}
\end{lstlisting}

有了函数\refvar{SobolSample}{()},为给定样本索引和维度计算样本值很简单。
\begin{lstlisting}
`\refcode{SobolSampler Method Definitions}{+=}\lastcode{SobolSamplerMethodDefinitions}`
`\refvar{Float}{}` `\refvar{SobolSampler}{}`::`\initvar[SobolSampler::SampleDimension]{\refvar{SampleDimension}{}}{}`(int64_t index, int dim) const {
    `\refvar{Float}{}` s = `\refvar{SobolSample}{}`(index, dim);
    `\refcode{Remap Sobol dimensions used for pixel samples}{}`
    return s;
}
\end{lstlisting}

计算Sobol样本值的代码对32和64位浮点值采用不同的路径。
两种情况使用不同的生成矩阵,为64位双精度给出更多精度位数。
\begin{lstlisting}
`\refcode{Low Discrepancy Inline Functions}{+=}\lastnext{LowDiscrepancyInlineFunctions}`
inline `\refvar{Float}{}` `\initvar{SobolSample}{}`(int64_t index, int dimension,
                         uint64_t scramble = 0) {
#ifdef PBRT_FLOAT_AS_DOUBLE
    return `\refvar{SobolSampleDouble}{}`(index, dimension, scramble);
#else
    return `\refvar{SobolSampleFloat}{}`(index, dimension, scramble);
#endif
}
\end{lstlisting}

函数\refvar{SobolSampleFloat}{()}的实现和\refvar{MultiplyGenerator}{()}非常像,
区别是它接收64位索引,所用的矩阵尺寸是$32\times52$.
这些更大的矩阵允许它生成不同的样本值至$a=2^{52}-1$,
而不是之前所用$32\times32$矩阵的$2^{32}-1$。
\begin{lstlisting}
`\refcode{Low Discrepancy Inline Functions}{+=}\lastcode{LowDiscrepancyInlineFunctions}`
inline float `\initvar{SobolSampleFloat}{}`(int64_t a, int dimension,
                              uint32_t scramble) {
    uint32_t v = scramble;
    for (int i = dimension * `\refvar{SobolMatrixSize}{}`; a != 0; a >>= 1, ++i)
        if (a & 1)
            v ^= `\refvar{SobolMatrices32}{}`[i];
    return v * 0x1p-32f; /* 1/2^32 */
}
\end{lstlisting}
\begin{lstlisting}
`\initcode{Sobol Matrix Declarations}{=}`
static constexpr int `\initvar{NumSobolDimensions}{}` = 1024;
static constexpr int `\initvar{SobolMatrixSize}{}` = 52;
extern const uint32_t `\initvar{SobolMatrices32}{}`[`\refvar{NumSobolDimensions}{}` *
                                      `\refvar{SobolMatrixSize}{}`];
\end{lstlisting}

函数{\initvar{SobolSampleDouble}{()}}类似,除了它用64位的Sobol矩阵。
这里没有在文中介绍它。

因为\refvar{SobolSampler}{}是种\refvar{GlobalSampler}{},
所以为前两维返回的值需要被调整使其变为对当前像素的偏移量。
这里用构造函数算出的幂2系数放大样本值,再对样本框的左下角
\sidenote{译者注:原文lower corner,确切翻译的话指坐标值最小的角点。}偏移
以求得对应的栅格样本位置。减去当前整数像素坐标就得到$[0,1)$内的结果。
\begin{lstlisting}
`\initcode{Remap Sobol dimensions used for pixel samples}{=}`
if (dim == 0 || dim == 1)  {
    s = s * `\refvar{resolution}{}` + `\refvar{sampleBounds}{}`.`\refvar{pMin}{}`[dim];
    s = `\refvar{Clamp}{}`(s - `\refvar{currentPixel}{}`[dim], (`\refvar{Float}{}`)0, `\refvar{OneMinusEpsilon}{}`);
}
\end{lstlisting}

\section{图像重建}\label{sec:图像重建}
有了精心选好的图像样本后,我们需要把样本及其算出的辐射值
转化为用于显示或存储的像素值。根据信号处理理论,
我们需要做三件事来为输出图像中的每个像素计算最终值:
\begin{enumerate}
    \item 从图像样本集中重建连续的图像函数$\tilde{L}$.
    \item 对函数$\tilde{L}$预滤波以移除任何超过像素间隔对应的奈奎斯特上限的频率。
    \item 在像素位置采样$\tilde{L}$以计算最终像素值。
\end{enumerate}

因为我们知道我们将只在像素位置处重采样函数$\tilde{L}$,
所以没有必要构建该函数的显式表示。
相反,我们可以用单个滤波函数把前两步结合起来。

回想如果已经用大于奈奎斯特频率的频率对原始函数进行均匀采样
并用sinc滤波器重建,则第一步中的重建函数会完美匹配原始图像函数——
这是个壮举,毕竟我们只有样本点。
但因为图像函数几乎总是有比采样率所能处理的还高的频率(由边缘等造成),
我们选择对其不均匀地采样,把混叠换成噪声。

理想重建背后的理论依赖于均匀间隔的样本。
尽管已有大量将该理论拓展到非均匀采样的尝试,
但目前该问题还没有公认的解决方法。
此外,因为知道采样率不足以刻画函数,所以完美重建是不可能的。
采样理论领域最近的研究重新看待了重建问题,
明确承认实践中完美重建通常是不可能的。
这一观点的微小转变带来了强大的重建新技术。
例如参见\citet{843002}了解关于这些进展的调研。
特别地,重建理论的研究目标已经从完美重建转为
开发能被证明可最小化重建函数与原始函数间差异的重建技术,
\emph{而不管原始函数是否是带限的}。

尽管pbrt中用的重建技术不是直接建立在这些新方法上的,
但它们能解释实践者的经验,即对图像合成所取的样本应用
完美重建技术通常不会得到最高质量的图像。
\begin{figure}[htbp]
    \centering\includegraphics[width=0.5\linewidth]{chap07/2dimagefiltering.eps}
    \caption{2D图像滤波。为了给位于$(x,y)$处标为实心圆的像素
    计算滤波后的像素值,要考虑在$(x,y)$周围范围{\ttfamily radius.x}和
    {\ttfamily radius.y}以内方盒中的所有图像样本。
    每个表示为空心圆的图像样本$(x_i,y_i)$,都被2D滤波函数$f(x-x_i,y-y_i)$赋权。
    所有样本的加权平均即是最终的像素值。}
    \label{fig:7.38}
\end{figure}

为了重建像素值,我们将考虑在一特定像素旁对样本插值的问题。
为了给像素$I(x,y)$计算最终值,插值结果是计算加权平均
\begin{align}\label{eq:7.12}
    I(x,y)=\frac{\sum\limits_i {f(x-x_i,y-y_i)w(x_i,y_i)L(x_i,y_i)}}{\sum\limits_i {f(x-x_i,y-y_i)}}\, ,
\end{align}
其中
\begin{itemize}
    \item $L(x_i,y_i)$是位于$(x_i,y_i)$的第$i$个样本的辐亮度值;
    \item $w(x_i,y_i)$是\refvar{Camera}{}返回的样本贡献权重。
          如\refsub{相机测量方程}和\refsub{采样相机1}所述,
          计算这些权重的方法决定了胶片度量哪个辐射度学量;
    \item $f$是滤波函数。
\end{itemize}

\reffig{7.38}展示了位于$(x,y)$的像素,它有个在$x$和$y$方向
范围分别为{\ttfamily radius.x}和{\ttfamily radius.y}的像素滤波器。
在滤波器范围给出的方盒内的所有样本都可能对该像素值有贡献,
这取决于滤波函数值$f(x-x_i,y-y_i)$.

这里sinc滤波器不是合适的选择:回想当基本函数有超过奈奎斯特上限的频率时
理想sinc滤波器容易振铃(吉布斯现象),这意味着图像中的边缘在附近像素处
有微弱重复的边缘副本。此外,sinc滤波器有\emph{无限支撑}:
距离其中心的有限距离内它不会衰减到零,
所以对于每个输出像素都会需要滤波所有图像样本。实践中没有唯一最好的滤波函数。
为特定场景选择最好的需要结合定量评估与定性判断。

\subsection{滤波函数}\label{sub:滤波函数}
pbrt中的所有滤波器实现都从抽象类\refvar{Filter}{}派生,
它为滤波中用的函数$f(x,y)$提供了接口;见\refeq{7.12}。
(\refsec{胶片与成像管道}描述的)类\refvar{Film}{}存有指向\refvar{Filter}{}的指针
并在把图像样本的贡献累加到最终图像时对其滤波。
(\reffig{7.39}展示比较了用本节各种滤波器来重建像素值而渲染出的图像放大区域。)
基类\refvar{Filter}{}定义在文件\href{https://github.com/mmp/pbrt-v3/blob/master/src/core/filter.h}{\ttfamily core/filter.h}
和\href{https://github.com/mmp/pbrt-v3/blob/master/src/core/filter.cpp}{\ttfamily core/filter.cpp}中。
\begin{figure}[htbp]
    \centering
    \subfloat[矩形滤波器]{\includegraphics[width=0.65\linewidth]{chap07/crown-box.png}\label{fig:7.39.1}}\\
    \subfloat[高斯滤波器]{\includegraphics[width=0.65\linewidth]{chap07/crown-gaussian.png}\label{fig:7.39.2}}\\
    \subfloat[Mitchell-Netravali滤波器]{\includegraphics[width=0.65\linewidth]{chap07/crown-mitchell.png}\label{fig:7.39.3}}\\
    \caption{用来将图像样本转化为像素值的像素重建滤波器对最终图像的质量有明显的影响。
        这里我们看到用(a)矩形滤波器、(b)高斯以及(c) Mitchell-Netravali滤波器滤波的皇冠模型放大区域。
        注意Mitchell滤波器给出了最清晰的图像,而高斯则模糊了它。
        矩形是最不可取的,因为它允许高频混叠泄漏到最终图像中
        (例如注意沿明亮金色边缘的阶梯状模式)。}
    \label{fig:7.39}
\end{figure}

\begin{lstlisting}
`\initcode{Filter Declarations}{=}`
class `\initvar{Filter}{}` {
public:
    `\refcode{Filter Interface}{}`
    `\refcode{Filter Public Data}{}`
};
\end{lstlisting}

\begin{figure}[htbp]
    \centering\includegraphics[width=0.6\linewidth]{chap07/filter-extent-radius.eps}
    \caption{pbrt中滤波器的范围是依据从原点到其截断点的半径来指定的。
        滤波器的支撑域是其整个非零范围,这里等于其半径的两倍。}
    \label{fig:7.40}
\end{figure}

所有滤波器中心都在原点$(0,0)$且定义了半径,超出半径时它们都取值为0;
该宽度也许在$x$和$y$方向是不同的。
构造函数接收半径值并和其倒数一块儿保存,以供滤波器实现使用。
滤波器在每个方向的整个范围(它的支撑域)是其相应半径值的两倍(\reffig{7.40})。
\begin{lstlisting}
`\initcode{Filter Interface}{=}\initnext{FilterInterface}`
`\refvar{Filter}{}`(const `\refvar{Vector2f}{}` &radius)
    : `\refvar[Filter::radius]{radius}{}`(radius),
      `\refvar[Filter::invRadius]{invRadius}{}`(`\refvar{Vector2f}{}`(1 / radius.x, 1 / radius.y)) { }
\end{lstlisting}
\begin{lstlisting}
`\initcode{Filter Public Data}{=}`
const `\refvar{Vector2f}{}` `\initvar[Filter::radius]{radius}{}`, `\initvar[Filter::invRadius]{invRadius}{}`;
\end{lstlisting}

\refvar{Filter}{}实现需要提供的唯一方法是\refvar[Filter::Evaluate]{Evaluate}{()}。
它接收一个给出了样本点相对于滤波器中心位置的2D点参数。
返回滤波器在该点的值。系统中别处的代码永远不会用滤波器范围外的点来调用滤波函数,
所以滤波器实现不需要检查这种情况。
\begin{lstlisting}
`\refcode{Filter Interface}{+=}\lastcode{FilterInterface}`
virtual `\refvar{Float}{}` `\initvar[Filter::Evaluate]{Evaluate}{}`(const `\refvar{Point2f}{}` &p) const = 0;
\end{lstlisting}

\subsubsection*{矩形滤波器}
图形学中最常用的滤波器之一是\keyindex{矩形滤波器}{box filter}{filter滤波器}
(且实际上当没有明确解决滤波和重建时,矩形滤波器就是\emph{事实上}的结果)。
矩形滤波器对图像的一个方形区域内的所有样本等值赋权。
尽管计算很高效,但它可能是最差的滤波器。
从\refsub{理想采样与重建}的讨论中回想矩形滤波器允许高频样本数据
泄漏到重建的值中。这会造成后混叠——
即使原始样本值足够高频以避免混叠,糟糕的滤波还是会引入误差。

\reffig{7.41}(a)展示了矩形滤波器的图像,
\reffig{7.42}展示了用矩形滤波器重建两个1D函数的结果。

对于之前我们用来说明吉布斯现象的阶跃函数,矩形滤波器尚能做得很好。
然而对于频率沿$x$轴增加的正弦函数结果则差得多。
不仅在低频时矩形滤波器把函数重建得很差,即使原始函数是光滑的也给出不连续的结果,
而且它在函数频率接近和超过奈奎斯特上限时也重建得很差。
\begin{lstlisting}
`\initcode{BoxFilter Declarations}{=}`
class `\initvar{BoxFilter}{}` : public `\refvar{Filter}{}` {
public:
    `\refvar{BoxFilter}{}`(const `\refvar{Vector2f}{}` &radius) : `\refvar{Filter}{}`(radius) { }
    `\refvar{Float}{}` `\refvar[BoxFilter::Evaluate]{Evaluate}{}`(const `\refvar{Point2f}{}` &p) const;
};
\end{lstlisting}

\begin{figure}[htbp]
    \centering
    \includegraphics[width=0.45\linewidth]{chap07/box-filter.eps}\,
    \includegraphics[width=0.45\linewidth]{chap07/triangle-filter.eps}
    \caption{(a)矩形滤波器和(b)三角滤波器的图示。尽管两者都不是特别好的滤波器,
        但它们的计算都很高效,易于实现,且对于评估其他滤波器是很好的基准线。}
    \label{fig:7.41}
\end{figure}
\begin{figure}[htbp]
    \centering
    \subfloat[]{\includegraphics[width=0.49\linewidth]{chap07/box-recon-a.eps}\label{fig:7.42.1}}\,
    \subfloat[]{\includegraphics[width=0.49\linewidth]{chap07/box-recon-b.eps}\label{fig:7.42.2}}
    \caption{矩形滤波器重建(a)阶跃函数和(b)频率随着$x$增加而增加的正弦函数。
        该滤波器对阶跃函数如料想那样工作得很好,但对于正弦函数则工作得极其糟糕。}
    \label{fig:7.42}
\end{figure}

因为不会在$(x,y)$值超出滤波器范围时调用求值函数,
所以对于滤波函数值它可以总是返回1.
\begin{lstlisting}
`\initcode{BoxFilter Method Definitions}{=}`
`\refvar{Float}{}` `\refvar{BoxFilter}{}`::`\initvar[BoxFilter::Evaluate]{\refvar[Filter::Evaluate]{Evaluate}{}}{}`(const `\refvar{Point2f}{}` &p) const {
    return 1.;
}
\end{lstlisting}

\subsubsection*{三角滤波器}
\keyindex{三角滤波器}{triangle filter}{filter滤波器}给出了比矩形稍好的结果:在滤波器的方形范围上权重从滤波器中心起线性下降。
参见\reffig{7.41}(b)的三角滤波器图示。
\begin{lstlisting}
`\initcode{TriangleFilter Declarations}{=}`
class `\initvar{TriangleFilter}{}` : public `\refvar{Filter}{}` {
public:
    `\refvar{TriangleFilter}{}`(const `\refvar{Vector2f}{}` &radius) : `\refvar{Filter}{}`(radius) { }
    `\refvar{Float}{}` `\refvar[TriangleFilter::Evaluate]{Evaluate}{}`(const `\refvar{Point2f}{}` &p) const;
};
\end{lstlisting}

三角滤波器求值很简单:该实现只需计算同时基于滤波器在$x$和$y$方向上宽度的线性函数。
\begin{lstlisting}
`\initcode{TriangleFilter Method Definitions}{=}`
`\refvar{Float}{}` `\refvar{TriangleFilter}{}`::`\initvar[TriangleFilter::Evaluate]{\refvar[Filter::Evaluate]{Evaluate}{}}{}`(const `\refvar{Point2f}{}` &p) const {
    return std::max((`\refvar{Float}{}`)0, `\refvar[Filter::radius]{radius}{}`.x - std::abs(p.x)) *
           std::max((`\refvar{Float}{}`)0, `\refvar[Filter::radius]{radius}{}`.y - std::abs(p.y));
}
\end{lstlisting}

\subsubsection*{高斯滤波器}
不像矩形和三角滤波器那样,\keyindex{高斯滤波器}{Gaussian filter}{filter滤波器}在实践中给出了较好的结果。
该滤波器应用了中心在像素处且绕其径向对称的高斯凸块。
高斯从滤波器值中减去了它在范围末端的值,
好让滤波器在其界限处变为0(\reffig{7.43})。
比起其他一些滤波器,高斯确实倾向于造成最终图像轻微模糊,
但该模糊实际上能帮助遮住图像中留下的任何混叠。

\begin{figure}[htbp]
    \centering
    \subfloat[]{\includegraphics[width=0.49\linewidth]{chap07/gaussian-filter.eps}\label{fig:7.43.1}}\,
    \subfloat[]{\includegraphics[width=0.49\linewidth]{chap07/mitchell-filter.eps}\label{fig:7.43.2}}
    \caption{(a)高斯滤波器和(b) $\displaystyle B=\frac{1}{3}$
        且$\displaystyle C=\frac{1}{3}$的Mitchell滤波器的图示,每个的宽度是2.
        高斯给出倾向于有点模糊的图像,而Mitchell滤波器的负波瓣有助于强调和
        锐化最终图像中的边界。}
    \label{fig:7.43}
\end{figure}

\begin{lstlisting}
`\initcode{GaussianFilter Declarations}{=}`
class `\initvar{GaussianFilter}{}` : public `\refvar{Filter}{}` {
public:
    `\refcode{GaussianFilter Public Methods}{}`
private:
    `\refcode{GaussianFilter Private Data}{}`
    `\refcode{GaussianFilter Utility Functions}{}`
};
\end{lstlisting}

半径为$r$的1D高斯滤波函数是
\begin{align*}
    f(x)={\mathrm{e}}^{-\alpha x^2}-{\mathrm{e}}^{-\alpha r^2}\, ,
\end{align*}
其中$\alpha$控制滤波器的衰减速率。更小的值使衰减更慢,给出更模糊的图像。
这里第二项保证高斯在其范围末端变为0而不是有个陡崖。
为了效率,构造函数在每个方向上预先计算了常数项${\mathrm{e}}^{-\alpha r^2}$.
\begin{lstlisting}
`\initcode{GaussianFilter Public Methods}{=}`
`\refvar{GaussianFilter}{}`(const `\refvar{Vector2f}{}` &radius, `\refvar{Float}{}` alpha)
    : `\refvar{Filter}{}`(radius), `\refvar[GaussianFilter::alpha]{alpha}{}`(alpha),
      `\refvar{expX}{}`(std::exp(-alpha * radius.x * radius.x)),
      `\refvar{expY}{}`(std::exp(-alpha * radius.y * radius.y)) { }
\end{lstlisting}
\begin{lstlisting}
`\initcode{GaussianFilter Private Data}{=}`
const `\refvar{Float}{}` `\initvar[GaussianFilter::alpha]{alpha}{}`;
const `\refvar{Float}{}` `\initvar{expX}{}`, `\initvar{expY}{}`;
\end{lstlisting}

因为2D高斯函数可分解为两个1D高斯的乘积,
所以实现调用函数\refvar{Gaussian}{()}
两次并把结果相乘。
\begin{lstlisting}
`\initcode{GaussianFilter Method Definitions}{=}`
`\refvar{Float}{}` `\refvar{GaussianFilter}{}`::`\initvar[GaussianFilter::Evaluate]{\refvar[Filter::Evaluate]{Evaluate}{}}{}`(const `\refvar{Point2f}{}` &p) const {
    return `\refvar{Gaussian}{}`(p.x, `\refvar{expX}{}`) * `\refvar{Gaussian}{}`(p.y, `\refvar{expY}{}`);
}
\end{lstlisting}
\begin{lstlisting}
`\initcode{GaussianFilter Utility Functions}{=}`
`\refvar{Float}{}` `\initvar{Gaussian}{}`(`\refvar{Float}{}` d, `\refvar{Float}{}` expv) const {
    return std::max((`\refvar{Float}{}`)0, `\refvar{Float}{}`(std::exp(-`\refvar[GaussianFilter::alpha]{alpha}{}` * d * d) - expv));
}
\end{lstlisting}

\subsubsection*{Mitchell滤波器}
设计滤波器是出了名的难,混合了数学分析与感知实验。
\citet{10.1145/54852.378514}为了能以系统的方式
探索该空间而开发了一系列参数化滤波函数。
在分析了受试者对用各种参数值滤波后的图像的主观反应后,
它们开发的滤波器能很好地权衡\emph{振铃}(图像中的幻影边缘挨着实际边缘)
与\emph{模糊}(过于模糊的结果)——两种来自糟糕重建滤波器的常见伪影。

从\reffig{7.43.2}中注意到该滤波函数在其边缘处取负值;
它有\keyindex{负波瓣}{negative lobe}{}。
实践中这些负区域提升了边缘的清晰度,给出更保真的
\sidenote{译者注:原文crisper。}图像(降低模糊)。
然而如果它们变得太大,则振铃就开始进入图像。
此外,因为最终像素值可能因此变为负数,
所以最终需要将它们截断到合法输出范围。

\reffig{7.44}展示了该滤波器重建的两个测试函数。
它对于两者都做得非常好:对于阶跃函数有最小的振铃,
对于正弦函数也做得非常好,直到采样率不足以刻画函数细节。
\begin{figure}[htbp]
    \centering
    \subfloat[]{\includegraphics[width=0.49\linewidth]{chap07/mitchell-recon-a.eps}\label{fig:7.44.1}}\,
    \subfloat[]{\includegraphics[width=0.49\linewidth]{chap07/mitchell-recon-b.eps}\label{fig:7.44.2}}
    \caption{Mitchell-Netravali滤波器用于重建示例函数。
        它对这两个函数都工作得很好,(a)对阶跃函数引入了最小振铃
        且(b)准确地表示了正弦曲线直到来自欠采样的混叠开始占主要地位。}
    \label{fig:7.44}
\end{figure}

\begin{lstlisting}
`\initcode{MitchellFilter Declarations}{=}`
class `\initvar{MitchellFilter}{}` : public `\refvar{Filter}{}` {
public:
    `\refcode{MitchellFilter Public Methods}{}`
private:
    const `\refvar{Float}{}` `\initvar[MitchellFilter::B]{B}{}`, `\initvar[MitchellFilter::C]{C}{}`;
};
\end{lstlisting}

Mitchell滤波器有两个参数叫$B$和$C$.
尽管这些参数可用任意值,但\citeauthor{10.1145/54852.378514}
建议它们位于直线$B+2C=1$上。
\begin{lstlisting}
`\initcode{MitchellFilter Public Methods}{=}\initnext{MitchellFilterPublicMethods}`
`\refvar{MitchellFilter}{}`(const `\refvar{Vector2f}{}` &radius, `\refvar{Float}{}` B, `\refvar{Float}{}` C)
    : `\refvar{Filter}{}`(radius), `\refvar[MitchellFilter::B]{B}{}`(B), `\refvar[MitchellFilter::C]{C}{}`(C) {
}
\end{lstlisting}

Mitchell-Netravali滤波器是在$x$和$y$方向上的1D滤波函数的乘积,
因此像高斯滤波器那样是可分离的(事实上pbrt中提供的所有滤波器都是可分离的)。
然而接口\refvar{Filter::Evaluate}{()}并不这样强制要求,
给将来实现新的滤波器提供更多灵活性。
\begin{lstlisting}
`\initcode{MitchellFilter Method Definitions}{=}`
`\refvar{Float}{}` `\refvar{MitchellFilter}{}`::`\initvar[MitchellFilter::Evaluate]{\refvar[Filter::Evaluate]{Evaluate}{}}{}`(const `\refvar{Point2f}{}` &p) const {
    return `\refvar{Mitchell1D}{}`(p.x * `\refvar[Filter::invRadius]{invRadius}{}`.x) * `\refvar{Mitchell1D}{}`(p.y * `\refvar[Filter::invRadius]{invRadius}{}`.y);
}
\end{lstlisting}

Mitchell滤波器中用的1D函数是个定义在范围$[-2,2]$上的偶函数。
通过联合定义在$[0,1]$上的三次多项式和另一个定义在$[1,2]$上的三次多项式得出了该函数。
这个结合的多项式还被平面$x=0$反射以得到完整的函数。
这些多项式由参数$B$和$C$控制,且被精心挑选以保证在$x=0$,$x=1$和$x=2$处
的$C^0$和$C^1$连续性。该多项式是
\begin{align*}
    f(x)=\frac{1}{6}\left\{
    \begin{array}{ll}
        (12-9B-6C)|x|^3+(-18+12B+6C)|x|^2+(6-2B),          & |x|<1\, ,      \\
        (-B-6C)|x|^3+(6B+30C)|x|^2+(-12B-48C)|x|+(8B+24C), & 1\le |x|<2\, , \\
        0,                                                 & \text{其他。}
    \end{array}
    \right.
\end{align*}

\begin{lstlisting}
`\refcode{MitchellFilter Public Methods}{+=}\lastcode{MitchellFilterPublicMethods}`
`\refvar{Float}{}` `\initvar{Mitchell1D}{}`(`\refvar{Float}{}` x) const {
    x = std::abs(2 * x);
    if (x > 1)
        return ((-`\refvar[MitchellFilter::B]{B}{}` - 6*`\refvar[MitchellFilter::C]{C}{}`) * x*x*x + (6*`\refvar[MitchellFilter::B]{B}{}` + 30*`\refvar[MitchellFilter::C]{C}{}`) * x*x +
                (-12*`\refvar[MitchellFilter::B]{B}{}` - 48*`\refvar[MitchellFilter::C]{C}{}`) * x + (8*`\refvar[MitchellFilter::B]{B}{}` + 24*`\refvar[MitchellFilter::C]{C}{}`)) * (1.f/6.f);
    else
        return ((12 - 9*`\refvar[MitchellFilter::B]{B}{}` - 6*`\refvar[MitchellFilter::C]{C}{}`) * x*x*x + 
                (-18 + 12*`\refvar[MitchellFilter::B]{B}{}` + 6*`\refvar[MitchellFilter::C]{C}{}`) * x*x +
                (6 - 2*`\refvar[MitchellFilter::B]{B}{}`)) * (1.f/6.f);
}
\end{lstlisting}

\subsubsection*{窗sinc滤波器}
最后,类\refvar{LanczosSincFilter}{}实现了基于sinc函数的滤波器。
实践中,sinc滤波器常常乘以另一个在一定距离后变为0的函数。
这得到了有限范围的滤波函数,它对具有合理性能的实现是必要的。
一个额外的参数$\tau$控制sinc函数在把值截断为0前要经过多少个周期
\sidenote{译者注:原文cycle,此处翻译为“周期”是折中做法,不是指严格意义上周期函数的周期。}。
\reffig{7.45}展示了三个周期的sinc函数图示,以及我们用的窗函数图示,
它由Lanczos开发。Lanczos窗就是sinc函数的中心波瓣,它被缩放以覆盖$\tau$个周期:
\begin{align*}
    w(x)=\frac{\sin\left(\displaystyle\frac{\pi x}{\tau}\right)}{\displaystyle\frac{\pi x}{\tau}}\, .
\end{align*}

\reffig{7.45}也展示了我们这里实现的滤波器,它是sinc函数和窗函数的乘积。
\begin{figure}[htbp]
    \centering
    \subfloat[]{\includegraphics[width=0.49\linewidth]{chap07/sinc-and-window.eps}\label{fig:7.45.1}}\,
    \subfloat[]{\includegraphics[width=0.49\linewidth]{chap07/sinc-times-window.eps}\label{fig:7.45.2}}
    \caption{sinc滤波器的图示。(a)三个周期后被截断了的sinc函数(蓝线)
        以及Lanczos窗函数(橙线)。(b)这两个函数的乘积,正如在\refvar{LanczosSincFilter}{}中的实现。}
    \label{fig:7.45}
\end{figure}

\reffig{7.46}为均匀1D样本展示了加窗的sinc重建结果。
因为加窗,重建的阶跃函数比用无限范围sinc函数重建的展现出小得多的振铃
(和\reffig{7.11}比较)。加窗的sinc滤波器在重建正弦函数时也做得极好,直到预混叠开始。
\begin{figure}[htbp]
    \centering
    \subfloat[]{\includegraphics[width=0.49\linewidth]{chap07/sinc-recon-a.eps}\label{fig:7.46.1}}\,
    \subfloat[]{\includegraphics[width=0.49\linewidth]{chap07/sinc-recon-b.eps}}
    \caption{用加窗sinc滤波器重建示例函数的结果。这里$\tau=3$.
        (a)像无限sinc那样,对于阶跃函数它也遭受了振铃,尽管在加窗版本中的振铃要少得多。
        (b)然而滤波器对于正弦函数做得很好。}
    \label{fig:7.46}
\end{figure}

\begin{lstlisting}
`\initcode{Sinc Filter Declarations}{=}`
class `\initvar{LanczosSincFilter}{}` : public `\refvar{Filter}{}` {
public:
    `\refcode{LanczosSincFilter Public Methods}{}`
private:
    const `\refvar{Float}{}` `\initvar[LanczosSincFilter::tau]{tau}{}`;
};
\end{lstlisting}
\begin{lstlisting}
`\initcode{LanczosSincFilter Public Methods}{=}\initnext{LanczosSincFilterPublicMethods}`
`\refvar{LanczosSincFilter}{}`(const `\refvar{Vector2f}{}` &radius, `\refvar{Float}{}` tau)
    : `\refvar{Filter}{}`(radius), `\refvar[LanczosSincFilter::tau]{tau}{}`(tau) { }
\end{lstlisting}
\begin{lstlisting}
`\initcode{Sinc Filter Method Definitions}{=}`
`\refvar{Float}{}` `\refvar{LanczosSincFilter}{}`::`\initvar[LanczosSincFilter::Evaluate]{\refvar[Filter::Evaluate]{Evaluate}{}}{}`(const `\refvar{Point2f}{}` &p) const {
    return `\refvar{WindowedSinc}{}`(p.x, `\refvar[Filter::radius]{radius}{}`.x) * `\refvar{WindowedSinc}{}`(p.y, `\refvar[Filter::radius]{radius}{}`.y);
}
\end{lstlisting}

该实现计算sinc函数值然后乘以Lanczos窗函数值。
\begin{lstlisting}
`\refcode{LanczosSincFilter Public Methods}{+=}\lastnext{LanczosSincFilterPublicMethods}`
`\refvar{Float}{}` `\initvar{Sinc}{}`(`\refvar{Float}{}` x) const {
    x = std::abs(x);
    if (x < 1e-5)  return 1;
    return std::sin(`\refvar{Pi}{}` * x) / (`\refvar{Pi}{}` * x);
}
\end{lstlisting}
\begin{lstlisting}
`\refcode{LanczosSincFilter Public Methods}{+=}\lastcode{LanczosSincFilterPublicMethods}`
`\refvar{Float}{}` `\initvar{WindowedSinc}{}`(`\refvar{Float}{}` x, `\refvar{Float}{}` radius) const {
    x = std::abs(x);
    if (x > radius) return 0;
    `\refvar{Float}{}` lanczos = `\refvar{Sinc}{}`(x / `\refvar[LanczosSincFilter::tau]{tau}{}`);
    return `\refvar{Sinc}{}`(x) * lanczos;
}
\end{lstlisting}

\section{胶片与成像管道}\label{sec:胶片与成像管道}
相机中胶片或传感器类型对入射光转换为图像中颜色的方式具有戏剧性影响。
在pbrt中,类\refvar{Film}{}在模拟相机中对传感设备建模。
在为每条相机光线求得辐亮度后,\refvar{Film}{}的实现
决定了样本对胶片平面上的相机光线起始点周围像素的贡献并更新其图像表示。
当主渲染循环退出时,\refvar{Film}{}将最终图像写入文件。

对于真实相机模型,\refsub{相机测量方程}介绍了测量方程,
它描述了相机中的传感器怎样度量一段时间内到达传感器区域上的能量大小。
对于更简单的相机模型,我们可将传感器视作度量某段时间内一小片区域上的平均辐亮度。
选择采用哪种度量的影响被封装在\refvar[GenerateRayDifferential]{Camera::GenerateRayDifferential}{()}为
光线返回的权重中。因此,\refvar{Film}{}的实现
可在不考虑这些变化的情况下处理,只需用这些权重缩放提供的辐亮度。

本节介绍了单个\refvar{Film}{}实现,它将像素重建方程应用于计算最终像素值。
对于基于物理的渲染器,通常最好是把结果图像存于浮点图像格式。
这样做在如何使用输出方面比起用8位无符号整数值的传统图像格式提供了更多的灵活性;
浮点格式避免了将图像量化为8位时造成的大量信息损失。

为了在现代显示设备上显示这样的图像,有必要将这些浮点像素值映射为离散值。
例如,计算机监视器通常希望每个像素的颜色由一个RGB颜色三元组描述,
而不是用任意的光谱功率分布。因此通用基函数系数描述的光谱在能显示之前必须转化为RGB表示。
一个相关问题是,比起许多真实世界场景中出现的范围,
显示器具有小得多的可显示辐亮度值范围。因此,像素值必须
以让最终显示的图像看起来尽可能接近其在无限制的理想显示设备上的样子的方式映射到可显示的范围。
这些问题是通过研究\keyindex{色调映射}{tone mapping}{}来解决的;
“扩展阅读”一节有关于该话题的更多信息。

\subsection{胶片类}\label{sub:胶片类}
\refvar{Film}{}定义在文件\href{https://github.com/mmp/pbrt-v3/blob/master/src/core/film.h}{\ttfamily core/film.h}
和\href{https://github.com/mmp/pbrt-v3/blob/master/src/core/film.cpp}{\ttfamily core/film.cpp}中。
\begin{lstlisting}
`\initcode{Film Declarations}{=}\initnext{FilmDeclarations}`
class `\initvar{Film}{}` {
public:
    `\refcode{Film Public Methods}{}`
    `\refcode{Film Public Data}{}`
private:
    `\refcode{Film Private Data}{}`
    `\refcode{Film Private Methods}{}`
};
\end{lstlisting}
\begin{lstlisting}
`\initcode{Film Public Methods}{=}`
`\refvar{Film}{}`(const `\refvar{Point2i}{}` &resolution, const `\refvar{Bounds2f}{}` &cropWindow,
    std::unique_ptr<`\refvar{Filter}{}`> filter,
    `\refvar{Float}{}` diagonal, const std::string &filename, `\refvar{Float}{}` scale);
`\refvar{Bounds2i}{}` `\refvar{GetSampleBounds}{}`() const;
`\refvar{Bounds2f}{}` `\refvar{GetPhysicalExtent}{}`() const;
std::unique_ptr<`\refvar{FilmTile}{}`> `\refvar{GetFilmTile}{}`(const `\refvar{Bounds2i}{}` &sampleBounds);
void `\refvar{MergeFilmTile}{}`(std::unique_ptr<`\refvar{FilmTile}{}`> tile);
void `\refvar{SetImage}{}`(const `\refvar{Spectrum}{}` *img) const;
void `\refvar{AddSplat}{}`(const `\refvar{Point2f}{}` &p, const `\refvar{Spectrum}{}` &v);
void `\refvar[Film::WriteImage]{WriteImage}{}`(`\refvar{Float}{}` splatScale = 1);
void `\refvar[Film::Clear]{Clear}{}`();
\end{lstlisting}

许多值传给了构造函数:图像以像素为单位的整个分辨率;
可能指定了要渲染的图像子集的裁剪窗口;
胶片物理区域的对角线长度,它在构造函数中单位是毫米,但这里转换为了米;
一个滤波函数;输出图像的文件名以及控制图像像素值如何存于文件的参数。
\begin{lstlisting}
`\initcode{Film Method Definitions}{=}\initnext{FilmMethodDefinitions}`
`\refvar{Film}{}`::`\refvar{Film}{}`(const `\refvar{Point2i}{}` &resolution, const `\refvar{Bounds2f}{}` &cropWindow,
        std::unique_ptr<`\refvar{Filter}{}`> filt, `\refvar{Float}{}` diagonal,
        const std::string &filename, `\refvar{Float}{}` scale)
    : `\refvar{fullResolution}{}`(resolution), `\refvar{diagonal}{}`(diagonal * .001),
    `\refvar{filter}{}`(std::move(filt)), `\refvar{filename}{}`(filename), `\refvar{scale}{}`(scale) {
    `\refcode{Compute film image bounds}{}`
    `\refcode{Allocate film image storage}{}`
    `\refcode{Precompute filter weight table}{}`
}
\end{lstlisting}

\begin{lstlisting}
`\initcode{Film Public Data}{=}\initnext{FilmPublicData}`
const `\refvar{Point2i}{}` `\initvar{fullResolution}{}`;
const `\refvar{Float}{}` `\initvar{diagonal}{}`;
std::unique_ptr<`\refvar{Filter}{}`> `\initvar{filter}{}`;
const std::string `\initvar{filename}{}`;
\end{lstlisting}

裁剪窗口与整体图像分辨率结合起来给出了实际需要存储和写出的像素边界。
裁剪窗口对于调试或将大图像分解为可在不同电脑上渲染的小块然后重新组装很有用。
裁剪窗口是在NDC空间中指定的,每个坐标范围是0到1(\reffig{7.47})。
\begin{figure}[htbp]
    \centering\includegraphics[width=0.4\linewidth]{chap07/Cropwindow.eps}
    \caption{图像裁剪窗口指定要渲染的图像子集。
    它在NDC空间中给出,坐标范围为从$(0,0)$到$(1,1)$.
    类\refvar{Film}{}仅为裁剪窗口内的区域分配空间存储像素值。}
    \label{fig:7.47}
\end{figure}

\refvar[croppedPixelBounds]{Film::croppedPixelBounds}{}保存了
裁剪窗口从左上角到右下角的像素边界。小数像素坐标被舍入;
这保证了如果图像按邻接的裁剪窗口分块渲染,则最终像素只在一个子图像内出现。
\begin{lstlisting}
`\initcode{Compute film image bounds}{=}`
`\refvar{croppedPixelBounds}{}` =
    `\refvar{Bounds2i}{}`(`\refvar{Point2i}{}`(std::ceil(`\refvar{fullResolution}{}`.x * cropWindow.`\refvar{pMin}{}`.x),
                     std::ceil(`\refvar{fullResolution}{}`.y * cropWindow.`\refvar{pMin}{}`.y)),
             `\refvar{Point2i}{}`(std::ceil(`\refvar{fullResolution}{}`.x * cropWindow.`\refvar{pMax}{}`.x),
                     std::ceil(`\refvar{fullResolution}{}`.y * cropWindow.`\refvar{pMax}{}`.y)));
\end{lstlisting}
\begin{lstlisting}
`\refcode{Film Public Data}{+=}\lastcode{FilmPublicData}`
`\refvar{Bounds2i}{}` `\initvar{croppedPixelBounds}{}`;
\end{lstlisting}

有了(可能被裁的)图像像素分辨率后,构造函数
为每个像素分配一个\refvar{Pixel}{}结构体数组。
光谱像素贡献值的运行加权和用XYZ颜色(\refsub{XYZ颜色})表示
并保存于成员变量\refvar[Pixel:xyz]{xyz}{}中。
\refvar[Pixel:filterWeightSum]{filterWeightSum}{}
持有表示样本对像素的贡献的滤波权重值之和。
\refvar[Pixel:splatXYZ]{splatXYZ}{}持有样本背板
\sidenote{译者注:此短语翻译不确定,原文sample splats。欢迎读者提供帮助。}(不加权)的和。
成员\refvar[Pixel:pad]{pad}{}是没用的;
它唯一的目的是保证结构体\refvar{Pixel}{}是32字节大小,而不是28
(这里假设\refvar{Float}{}是4字节;否则它保证是64字节结构体)。
该填充保证了\refvar{Pixel}{}不会跨越缓存行,
这样当获取\refvar{Pixel}{}时引发的缓存缺失不超过一次
(只要数组的首个\refvar{Pixel}{}是在缓存行开头分配的)。
\begin{lstlisting}
`\initcode{Film Private Data}{=}\initnext{FilmPrivateData}`
struct `\initvar{Pixel}{}` {
    `\refvar{Float}{}` `\initvar[Pixel:xyz]{xyz}{}`[3] = { 0, 0, 0 };
    `\refvar{Float}{}` `\initvar[Pixel:filterWeightSum]{filterWeightSum}{}` = 0;
    `\refvar{AtomicFloat}{}` `\initvar[Pixel:splatXYZ]{splatXYZ}{}`[3];
    `\refvar{Float}{}` `\initvar[Pixel:pad]{pad}{}`;
};
std::unique_ptr<`\refvar{Pixel}{}`[]> `\initvar[Pixel::pixels]{pixels}{}`;
\end{lstlisting}
\begin{lstlisting}
`\initcode{Allocate film image storage}{=}`
`\refvar[Pixel::pixels]{pixels}{}` = std::unique_ptr<`\refvar{Pixel}{}`[]>(new `\refvar{Pixel}{}`[`\refvar{croppedPixelBounds}{}`.Area()]);
\end{lstlisting}

用XYZ颜色存储像素值的两个自然选择是使用\refvar{Spectrum}{}值或存储RGB值。
这里即使是在进行全光谱渲染时也不值得保存完整的\refvar{Spectrum}{}值。
因为写入到输出文件的最终颜色不包括\refvar{Spectrum}{}样本全集,
所以这里转换为三刺激值和先保存\refvar{Spectrum}{}再转换
为图像输出上的三刺激值相比并不代表损失了信息。
该情况下如果\refvar{Spectrum}{}有大量样本,
则不保存完整的\refvar{Spectrum}{}值能节约大量内存
(如果pbrt支持把\refvar{SampledSpectrum}{}值保存到文件,则需要重新考虑该设计选择)。

我们已选择用XYZ颜色而不是RGB以强调XYZ是独立于显示器的颜色表示,
而RGB需要假定显示器响应曲线的特定集合(\refsub{RGB颜色})。
(然而最后我们还是不得不转换为RGB,因为很少有图像文件格式保存XYZ颜色。)

有了典型的滤波器设置,每个图像样本都可能对最终图像中的16个或更多像素作出贡献。
特别是对于在光线相交测试和着色计算上花费相对较少时间的简单场景,
花在为每个样本更新图像上的时间会很多。
因此,\refvar{Film}{}预先计算滤波值的表使得
我们可以免去对方法\refvar{Filter::Evaluate}{()}的虚函数调用开销
以及推算滤波器的开销,而可以为滤波使用来自表中的值。
实践中不在每个样本的精确位置上推算滤波而引入的误差并不明显。

这里的实现做了合理假设即滤波器定义满足$f(x,y)=f(|x|,|y|)$,
所以表格只需要为滤波偏移量正象限存储值。
该假设对于目前pbrt中所有可用的\refvar{Filter}{}
都成立且在实践中对大多数滤波器成立。
这让表的大小变为四分之一并改进了内存访问的一致性,
使缓存性能更好\footnote{这里该实现可进一步利用目前pbrt中所有滤波器都是可分离的事实,
只分配两个1D表。然而,为了允许更容易地添加不同滤波函数,我们这里没有假定可分离性。}。
\begin{lstlisting}
`\initcode{Precompute filter weight table}{=}`
int offset = 0;
for (int y = 0; y < `\refvar{filterTableWidth}{}`; ++y) {
    for (int x = 0; x < `\refvar{filterTableWidth}{}`; ++x, ++offset) {
        `\refvar{Point2f}{}` p;
        p.x = (x + 0.5f) * `\refvar{filter}{}`->`\refvar[Filter::radius]{radius}{}`.x / `\refvar{filterTableWidth}{}`;
        p.y = (y + 0.5f) * `\refvar{filter}{}`->`\refvar[Filter::radius]{radius}{}`.y / `\refvar{filterTableWidth}{}`;
        `\refvar{filterTable}{}`[offset] = `\refvar{filter}{}`->`\refvar[Filter::Evaluate]{Evaluate}{}`(p);
    }
}
\end{lstlisting}
\begin{lstlisting}
`\refcode{Film Private Data}{+=}\lastnext{FilmPrivateData}`
static constexpr int `\initvar{filterTableWidth}{}` = 16;
`\refvar{Float}{}` `\initvar{filterTable}{}`[`\refvar{filterTableWidth}{}` * `\refvar{filterTableWidth}{}`];
\end{lstlisting}

\begin{lstlisting}
`\refcode{Film Method Definitions}{+=}\lastnext{FilmMethodDefinitions}`
`\refvar{Bounds2i}{}` `\refvar{Film}{}`::`\initvar{GetSampleBounds}{()}` const {
    `\refvar{Bounds2f}{}` floatBounds(
        `\refvar{Floor}{}`(`\refvar{Point2f}{}`(`\refvar{croppedPixelBounds}{}`.pMin) + `\refvar{Vector2f}{}`(0.5f, 0.5f) -
              `\refvar{filter}{}`->`\refvar[Filter::radius]{radius}{}`),
        `\refvar{Ceil}{}`( `\refvar{Point2f}{}`(`\refvar{croppedPixelBounds}{}`.pMax) - `\refvar{Vector2f}{}`(0.5f, 0.5f) +
              `\refvar{filter}{}`->`\refvar[Filter::radius]{radius}{}`));
    return (`\refvar{Bounds2i}{}`)floatBounds;
}
\end{lstlisting}
\begin{lstlisting}
`\refcode{Film Method Definitions}{+=}\lastnext{FilmMethodDefinitions}`
`\refvar{Bounds2f}{}` `\refvar{Film}{}`::`\initvar{GetPhysicalExtent}{}`() const {
    `\refvar{Float}{}` aspect = (`\refvar{Float}{}`)`\refvar{fullResolution}{}`.y / (`\refvar{Float}{}`)`\refvar{fullResolution}{}`.x;
    `\refvar{Float}{}` x = std::sqrt(`\refvar{diagonal}{}` * `\refvar{diagonal}{}` / (1 + aspect * aspect));
    `\refvar{Float}{}` y = aspect * x;
    return `\refvar{Bounds2f}{}`(`\refvar{Point2f}{}`(-x / 2, -y / 2), `\refvar{Point2f}{}`(x / 2, y / 2));
}
\end{lstlisting}
\subsection{为胶片提供像素值}\label{sub:为胶片提供像素值}
\begin{lstlisting}
`\refcode{Film Method Definitions}{+=}\lastnext{FilmMethodDefinitions}`
std::unique_ptr<`\refvar{FilmTile}{}`> `\refvar{Film}{}`::`\initvar{GetFilmTile}{}`(
        const `\refvar{Bounds2i}{}` &sampleBounds) {
    `\refcode{Bound image pixels that samples in sampleBounds contribute to}{}`
    return std::unique_ptr<`\refvar{FilmTile}{}`>(new FilmTile(tilePixelBounds,
        filter->radius, filterTable, filterTableWidth));
}
\end{lstlisting}

\begin{lstlisting}
`\refcode{Film Declarations}{+=}\lastcode{FilmDeclarations}`
class `\initvar{FilmTile}{}` {
public:
    `\refcode{FilmTile Public Methods}{}`
private:
    `\refcode{FilmTile Private Data}{}`
};
\end{lstlisting}

\begin{lstlisting}
`\initcode{FilmTile Public Methods}{=}\initnext{FilmTilePublicMethods}`
`\refvar{FilmTile}{}`(const `\refvar{Bounds2i}{}` &pixelBounds, const `\refvar{Vector2f}{}` &filterRadius,
    const `\refvar{Float}{}` *filterTable, int filterTableSize)
    : `\refvar{pixelBounds}{}`(pixelBounds), `\refvar{filterRadius}{}`(filterRadius),
    `\refvar{invFilterRadius}{}`(1 / filterRadius.x, 1 / filterRadius.y),
    `\refvar{filterTable}{}`(filterTable), `\refvar{filterTableSize}{}`(filterTableSize) {
    `\refvar[FilmTile::pixels]{pixels}{}` = std::vector<`\refvar{FilmTilePixel}{}`>(std::max(0, pixelBounds.Area()));
}
\end{lstlisting}

\begin{lstlisting}
`\initcode{FilmTile Private Data}{=}`
const `\refvar{Bounds2i}{}` `\initvar{pixelBounds}{}`;
const `\refvar{Vector2f}{}` `\initvar{filterRadius}{}`, `\initvar{invFilterRadius}{}`;
const `\refvar{Float}{}` *`\initvar{filterTable}{}`;
const int `\initvar{filterTableSize}{}`;
std::vector<`\refvar{FilmTilePixel}{}`> `\initvar[FilmTile::pixels]{pixels}{}`;
\end{lstlisting}

\begin{lstlisting}
`\refcode{FilmTile Public Methods}{+=}\lastnext{FilmTilePublicMethods}`
void `\initvar{AddSample}{}`(const `\refvar{Point2f}{}` &pFilm, const `\refvar{Spectrum}{}` &L,
    `\refvar{Float}{}` sampleWeight = 1.) {
    `\refcode{Compute sample's raster bounds}{}`
    `\refcode{Loop over filter support and add sample to pixel arrays}{}`
}
\end{lstlisting}

\begin{lstlisting}
`\initcode{Compute sample's raster bounds}{=}`
`\refvar{Point2f}{}` pFilmDiscrete = pFilm - `\refvar{Vector2f}{}`(0.5f, 0.5f);
`\refvar{Point2i}{}` p0 = (`\refvar{Point2i}{}`)`\refvar{Ceil}{}`(pFilmDiscrete - filterRadius);
`\refvar{Point2i}{}` p1 = (`\refvar{Point2i}{}`)`\refvar{Floor}{}`(pFilmDiscrete + filterRadius) + `\refvar{Point2i}{}`(1, 1);
p0 = `\refvar[Point3::Max]{Max}{}`(p0, pixelBounds.pMin);
p1 = `\refvar[Point3::Min]{Min}{}`(p1, pixelBounds.pMax);
\end{lstlisting}

\begin{lstlisting}
`\refcode{Film Method Definitions}{+=}\lastnext{FilmMethodDefinitions}`
void `\refvar{Film}{}`::`\initvar{MergeFilmTile}{}`(std::unique_ptr<`\refvar{FilmTile}{}`> tile) {
    std::lock_guard<std::mutex> lock(`\refvar{mutex}{}`);
    for (`\refvar{Point2i}{}` pixel : tile->`\refvar{GetPixelBounds}{}`()) {
        `\refcode{Merge pixel into Film::pixels}{}`
    }
}
\end{lstlisting}
\begin{lstlisting}
`\refcode{Film Private Data}{+=}\lastnext{FilmPrivateData}`
std::mutex `\initvar{mutex}{}`;
\end{lstlisting}
\begin{lstlisting}
`\refcode{FilmTile Public Methods}{+=}\lastcode{FilmTilePublicMethods}`
`\refvar{Bounds2i}{}` `\initvar{GetPixelBounds}{}`() const { return `\refvar{pixelBounds}{}`; }
\end{lstlisting}
\begin{lstlisting}
`\refcode{Film Private Data}{+=}\lastcode{FilmPrivateData}`
const `\refvar{Float}{}` `\initvar{scale}{}`;
\end{lstlisting}

\subsection{图像输出}\label{sub:图像输出}
\begin{lstlisting}
`\refcode{Film Method Definitions}{+=}\lastcode{FilmMethodDefinitions}`
void `\refvar{Film}{}`::`\initvar[Film::WriteImage]{WriteImage}{}`(`\refvar{Float}{}` splatScale) {
    `\refcode{Convert image to RGB and compute final pixel values}{}`
    `\refcode{Write RGB image}{}`
}
\end{lstlisting}

\section{扩展阅读}\label{sec:扩展阅读07}

\section{扩展阅读}\label{sec:扩展阅读07}
\begin{enumerate}
    \item xxx
    \item xxx
    \item \label{sub:7.11.3}\circletwo
\end{enumerate}

\section{译者补充:信号处理}\label{sec:译者补充:信号处理}
\begin{remark}
    本节内容不是原书内容,而是译者根据\citet{DigitalSignalProcessing}、
    \citet{enwiki:1115652231}补充的,请酌情参考和斧正。
\end{remark}

\subsection{单位冲激函数}\label{sub:单位冲激函数}
\begin{definition}
    数学中,\keyindex{狄拉克$\delta$分布}{Dirac delta distribution}{}是定义在实数域上的广义分布或函数。
    它在除零以外的点上都取零,且在整个实数域上的积分等于一。通常记作$\delta(\cdot)$.
\end{definition}

狄拉克$\delta$分布也称\keyindex{狄拉克$\delta$函数}{Dirac delta function}{},
简称\keyindex{$\delta$分布}{delta distribution}{}或
\keyindex{$\delta$函数}{delta function}{},
它最早由英国理论物理学家保罗·狄拉克(Paul Adrien Maurice Dirac)提出,
在物理和工程界有广泛应用,也称作\keyindex{单位冲激函数}{unit impulse function}{}。

单位冲激函数不是严格意义上的函数,但形式上遵守微积分运算法则。
可以将其视作在非零处取零,在零处取无穷大,即
\begin{align}
    \delta(t)\approx\left\{
    \begin{array}{ll}
        +\infty, & \text{当}t=0,     \\
        0,       & \text{当}t\neq 0.
    \end{array}
    \right.
\end{align}
且满足如下积分约束的函数:
\begin{align}
    \int_{-\infty}^{\infty}\delta(t)\mathrm{d}t=1\, .
\end{align}

依据单位冲激函数的定义,可推导出以下性质:
\begin{theorem}
    单位冲激函数具有缩放性质:对任意实数$\alpha\neq0$,有
    \begin{align}
        \delta(\alpha t)=\frac{\delta(t)}{|\alpha|}\, .
    \end{align}
\end{theorem}
\begin{corollary}
    单位冲激函数具有对称性,即
    \begin{align}
        \delta(t)=\delta(-t)\, .
    \end{align}
\end{corollary}
\begin{theorem}
    单位冲激函数具有时延性质,也称平移性质或筛选性质,
    即对于可积函数$f$,它可以采样出$t=\tau$处的值:
    \begin{align}
        \int_{-\infty}^{\infty}\delta(t-\tau)f(t)\mathrm{d}t=f(\tau)\, .
    \end{align}
\end{theorem}
\subsection{关于傅里叶变换的推导}\label{sub:关于傅里叶变换的推导}
\subsubsection*{矩形函数}
对于矩形函数
\begin{align}
    f(t)=\left\{\begin{array}{ll}
        1, & \displaystyle\text{若}|t|<\frac{1}{2}, \\
        0, & \text{其他}.
    \end{array}\right.
\end{align}
其频率表示为
\begin{align}
    F(\omega) & =\int_{-\infty}^{\infty}f(t)\mathrm{e}^{-\mathrm{i}2\pi\omega t}\mathrm{d}t
    =\int_{-\frac{1}{2}}^{\frac{1}{2}}\mathrm{e}^{-\mathrm{i}2\pi\omega t}\mathrm{d}t
    =-\frac{\mathrm{e}^{-\mathrm{i}2\pi\omega t}}{\mathrm{i}2\pi\omega}\bigg|_{t=-\frac{1}{2}}^{\frac{1}{2}}\nonumber \\
              & =-\frac{\mathrm{e}^{-\mathrm{i}\pi\omega}-\mathrm{e}^{\mathrm{i}\pi\omega}}{\mathrm{i}2\pi\omega}
    =\frac{\mathrm{i}2\sin(\pi\omega)}{\mathrm{i}2\pi\omega}
    =\frac{\sin(\pi\omega)}{\pi\omega}\, .
\end{align}

\subsubsection*{高斯函数}
对于高斯函数
\begin{align}
    f(t)=\mathrm{e}^{-\pi t^2}\, ,
\end{align}
其频率表示为
\begin{align}
    F(\omega) & =\int_{-\infty}^{\infty}f(t)\mathrm{e}^{-\mathrm{i}2\pi\omega t}\mathrm{d}t
    =\int_{-\infty}^{\infty}\mathrm{e}^{-\pi t^2}\mathrm{e}^{-\mathrm{i}2\pi\omega t}\mathrm{d}t
    =\int_{-\infty}^{\infty}\mathrm{e}^{-\pi((t+\mathrm{i}\omega)^2+\omega^2)}\mathrm{d}t\nonumber                  \\
              & =\mathrm{e}^{-\pi\omega^2}\int_{-\infty}^{\infty}\mathrm{e}^{-\pi(t+\mathrm{i}\omega)^2}\mathrm{d}t
    =\mathrm{e}^{-\pi\omega^2}\int_{-\infty}^{\infty}\mathrm{e}^{-\pi t^2}\mathrm{d}t
    =\mathrm{e}^{-\pi\omega^2}\, .
\end{align}
\subsubsection*{单位冲激函数}
对于单位冲激函数
\begin{align}
    f(t)=\delta(t)\, ,
\end{align}
其频率表示为
\begin{align}
    F(\omega)=\int_{-\infty}^{\infty}f(t)\mathrm{e}^{-\mathrm{i}2\pi\omega t}\mathrm{d}t
    =\int_{-\infty}^{\infty}\delta(t)\mathrm{e}^{-\mathrm{i}2\pi\omega t}\mathrm{d}t
    =\mathrm{e}^{-\mathrm{i}2\pi\omega\cdot0}
    =1\, .
\end{align}
\subsection*{单位常函数}
定义指数衰减函数为
\begin{align}
    f_a(t)=\mathrm{e}^{-a|t|},\quad (a>0)\, .
\end{align}
则单位常函数可视作指数衰减函数的极限,即
\begin{align}
    f(t)=\lim\limits_{a\rightarrow0^+}f_a(t)=1\, .
\end{align}
于是常函数的频率表示满足
\begin{align}
    F(\omega) & =\int_{-\infty}^{\infty}f(t)\mathrm{e}^{-\mathrm{i}2\pi\omega t}\mathrm{d}t
    =\int_{-\infty}^{\infty}\lim\limits_{a\rightarrow0^+}\mathrm{e}^{-a|t|}\mathrm{e}^{-\mathrm{i}2\pi\omega t}\mathrm{d}t
    =\lim\limits_{a\rightarrow0^+}\int_{-\infty}^{\infty}\mathrm{e}^{-a|t|-\mathrm{i}2\pi\omega t}\mathrm{d}t\nonumber                                                                                  \\
              & =\lim\limits_{a\rightarrow0^+}\left(\int_{-\infty}^0\mathrm{e}^{(a-\mathrm{i}2\pi\omega)t}\mathrm{d}t+\int_0^{\infty}\mathrm{e}^{-(a+\mathrm{i}2\pi\omega)t}\mathrm{d}t\right)\nonumber \\
              & =\lim\limits_{a\rightarrow0^+}\left(\frac{\mathrm{e}^{(a-\mathrm{i}2\pi\omega)t}}{a-\mathrm{i}2\pi\omega}\bigg|_{t=-\infty}^0
    +\frac{\mathrm{e}^{-(a+\mathrm{i}2\pi\omega)t}}{-(a+\mathrm{i}2\pi\omega)}\bigg|_{t=0}^{\infty}\right)\nonumber                                                                                     \\
              & =\lim\limits_{a\rightarrow0^+}\left(\frac{1}{a-\mathrm{i}2\pi\omega}+\frac{1}{a+\mathrm{i}2\pi\omega}\right)=\lim\limits_{a\rightarrow0^+}\frac{2a}{a^2+4\pi^2\omega^2}\nonumber        \\
              & =\left\{\begin{array}{ll}
        0,      & \text{若}\omega\neq0, \\
        \infty, & \text{若}\omega=0.
    \end{array}\right.
\end{align}
注意到该频率表示取极限的部分在实数域上积分与$a$无关且为
\begin{align}
    \int_{-\infty}^{\infty}\frac{2a}{a^2+4\pi^2\omega^2}\mathrm{d}\omega
    =\frac{1}{\pi}\int_{-\infty}^{\infty}\frac{1}{1+\left(\frac{2\pi\omega}{a}\right)^2}\mathrm{d}\frac{2\pi\omega}{a}
    =\frac{1}{\pi}\arctan\frac{2\pi\omega}{a}\bigg|_{\omega=-\infty}^{\infty}=1\, .
\end{align}
于是该频率表示实际上就是单位冲激函数,即
\begin{align}
    F(\omega)=\delta(\omega)\, .
\end{align}

\subsubsection*{傅里叶变换的性质}
\begin{theorem}
    傅里叶变换具有频移与时移性质,即对于傅里叶变换对$f(t)\leftrightarrow F(\omega)$,
    给定任意常数$\tau$和$\omega_0$,则有相应的变换对
    \begin{align}
        f(t)\mathrm{e}^{\mathrm{i}2\pi\omega_0 t} & \leftrightarrow F(\omega-\omega_0)\, ,                              \\
        f(t-\tau)                                 & \leftrightarrow F(\omega)\mathrm{e}^{-\mathrm{i}2\pi\omega\tau}\, .
    \end{align}
\end{theorem}
\begin{prove}
    对于时域表示$f(t)\mathrm{e}^{\mathrm{i}2\pi\omega_0 t}$,其傅里叶变换为
    \begin{align}
        \int_{-\infty}^{\infty}f(t)\mathrm{e}^{\mathrm{i}2\pi\omega_0 t}\mathrm{e}^{-\mathrm{i}2\pi\omega t}\mathrm{d}t
        =\int_{-\infty}^{\infty}f(t)\mathrm{e}^{-\mathrm{i}2\pi(\omega-\omega_0) t}\mathrm{d}t=F(\omega-\omega_0)\, .
    \end{align}
    对于频率表示$F(\omega)\mathrm{e}^{-\mathrm{i}2\pi\omega\tau}$,其傅里叶逆变换为
    \begin{align}
        \int_{-\infty}^{\infty}F(\omega)\mathrm{e}^{-\mathrm{i}2\pi\omega\tau}\mathrm{e}^{\mathrm{i}2\pi\omega t}\mathrm{d}\omega
        =\int_{-\infty}^{\infty}F(\omega)\mathrm{e}^{\mathrm{i}2\pi\omega(t-\tau)}\mathrm{d}\omega
        =f(t-\tau)\, .
    \end{align}
\end{prove}
\begin{theorem}
    傅里叶变换和逆变换互为逆运算,即
    \begin{align}
        \mathcal{F}^{-1}\{\mathcal{F}\{f(t)\}\}      & =f(t)\, ,      \\
        \mathcal{F}\{\mathcal{F}^{-1}\{F(\omega)\}\} & =F(\omega)\, .
    \end{align}
\end{theorem}

\begin{prove}
    利用时延性质和单位冲激函数的傅里叶变换对可得
    \begin{align}
        \mathcal{F}^{-1}\{\mathcal{F}\{f(t)\}\}= & \int_{-\infty}^{\infty}\left(\int_{-\infty}^{\infty}f(\tau)\mathrm{e}^{-\mathrm{i}2\pi\omega\tau}\mathrm{d}\tau\right)\mathrm{e}^{\mathrm{i}2\pi\omega t}\mathrm{d}\omega\nonumber \\
        =                                        & \int_{-\infty}^{\infty}f(\tau)\left(\int_{-\infty}^{\infty}\mathrm{e}^{\mathrm{i}2\pi\omega(t-\tau)}\mathrm{d}\omega\right)\mathrm{d}\tau\nonumber                                 \\
        =                                        & \int_{-\infty}^{\infty}f(\tau)\delta(t-\tau)\mathrm{d}\tau\nonumber                                                                                                                \\
        =                                        & f(t)\, .
    \end{align}
    第二个式子同理运用频移性质和单位常函数的傅里叶变换对可证。
\end{prove}

\subsubsection*{余弦函数}
对于余弦函数
\begin{align}
    f(t)=\cos t\, ,
\end{align}
其频率表示为
\begin{align}
    F(\omega) & =\int_{-\infty}^{\infty}f(t)\mathrm{e}^{-\mathrm{i}2\pi\omega t}\mathrm{d}t\nonumber                                                                                                \\
              & =\int_{-\infty}^{\infty}\mathrm{e}^{-\mathrm{i}2\pi\omega t}\cos t\mathrm{d}t\nonumber                                                                                              \\
              & =\int_{-\infty}^{\infty}\frac{1}{2}(\mathrm{e}^{\mathrm{i}t}+\mathrm{e}^{-\mathrm{i}t})\mathrm{e}^{-\mathrm{i}2\pi\omega t}\mathrm{d}t\nonumber                                     \\
              & =\frac{1}{2}\int_{-\infty}^{\infty}(\mathrm{e}^{\mathrm{i}2\pi\frac{1}{2\pi}t}+\mathrm{e}^{\mathrm{i}2\pi\frac{-1}{2\pi}t})\mathrm{e}^{-\mathrm{i}2\pi\omega t}\mathrm{d}t\nonumber \\
              & =\frac{1}{2}(\delta(\omega-\frac{1}{2\pi})+\delta(\omega+\frac{1}{2\pi}))\nonumber                                                                                                  \\
              & =\pi(\delta(1-2\pi\omega)+\delta(1+2\pi\omega))\, .
\end{align}

\subsubsection*{shah函数}
\begin{theorem}
    周期为$T$的函数$f(t)$可被展开为唯一的\keyindex{傅里叶级数}{Fourier series}{},其指数形式为
    \begin{align}
        f(t)=\sum\limits_{n=-\infty}^{\infty}a_n\mathrm{e}^{\mathrm{i}2\pi\frac{n}{T}t}\, ,
    \end{align}
    其中系数
    \begin{align}
        a_n=\frac{1}{T}\int\limits_T f(t)\mathrm{e}^{-\mathrm{i}2\pi\frac{n}{T}t}\mathrm{d}t\, .
    \end{align}
\end{theorem}

对于周期为$T$的shah函数
\begin{align}
    f(t)=T\sum\limits_{k=-\infty}^{\infty}\delta(t-kT)\, ,
\end{align}
其傅里叶展开中的系数为
\begin{align}
    a_n=\frac{1}{T}\int_{-\frac{T}{2}}^{\frac{T}{2}}f(t)\mathrm{e}^{-\mathrm{i}2\pi\frac{n}{T}t}\mathrm{d}t
    =\frac{1}{T}\int_{-\frac{T}{2}}^{\frac{T}{2}}T\delta(t)\mathrm{e}^{-\mathrm{i}2\pi\frac{n}{T}t}\mathrm{d}t
    =1\, .
\end{align}
于是shah函数可展开为
\begin{align}
    f(t)=\sum\limits_{n=-\infty}^{\infty}\mathrm{e}^{\mathrm{i}2\pi\frac{n}{T}t}\, .
\end{align}
因此其频域表示为
\begin{align}
    F(\omega) & =\int_{-\infty}^{\infty}f(t)\mathrm{e}^{-\mathrm{i}2\pi\omega t}\mathrm{d}t\nonumber                                                                    \\
              & =\int_{-\infty}^{\infty}\sum\limits_{n=-\infty}^{\infty}\mathrm{e}^{\mathrm{i}2\pi\frac{n}{T}t}\mathrm{e}^{-\mathrm{i}2\pi\omega t}\mathrm{d}t\nonumber \\
              & =\sum\limits_{n=-\infty}^{\infty}\int_{-\infty}^{\infty}\mathrm{e}^{-\mathrm{i}2\pi(\omega-\frac{n}{T})t}\mathrm{d}t\nonumber                           \\
              & =\sum\limits_{n=-\infty}^{\infty}\delta(\omega-\frac{n}{T})\, .
\end{align}

\section{译者补充:初等数论基础}\label{sec:译者补充:初等数论基础}

\begin{remark}
    本节内容不是原书内容,而是译者根据\citet{ElementaryNumberTheory}
    所著著作补充的,请酌情参考和斧正。
\end{remark}

\begin{notation}
    本节我们重申以下记号:
    \begin{itemize}
        \item 用$\mathbb{N}$表示全体正整数构成的集合;$\mathbb{Z}$表示全体整数构成的集合。
        \item 若命题$p$能推出命题$q$,则记为$p\Rightarrow q$;若$p$与$q$等价,则记为$p\Leftrightarrow q$.
    \end{itemize}
\end{notation}

\subsection*{基本原理}
\begin{theorem}[\protect\keyindex{最小自然数原理}{least number principle}{}]\label{theorem:7.ex02.1}
    设$T$是$\mathbb{N}$的一非空子集,则必有$t_0\in T$,
    使对任意的$t\in T$有$t_0\le t$,即$t_0$是$T$中最小的自然数。
\end{theorem}

% \begin{theorem}[最大自然数原理]
%     设$M$是$\mathbb{N}$的一非空子集,若$M$有上界(即存在$a\in \mathbb{N}$使
%     对任意的$m\in M$有$m\le a$),则必有$m_0\in M$,使对任意的$m\in M$有$m\le m_0$,
%     即$m_0$是$M$中最大的自然数。
% \end{theorem}

% \begin{theorem}[\protect\keyindex{归纳原理}{principle of induction}{}]
%     设$S\subseteq \mathbb{N}$,且满足
%     \begin{enumerate}
%         \item 有$1\in S$;
%         \item 对任意$n\in S$都有$n+1\in S$;
%     \end{enumerate}
%     则$S=\mathbb{N}$.
% \end{theorem}

% \begin{theorem}[\protect\keyindex{数学归纳法}{mathematical induction}{}]
%     设$P(n)$是关于自然数$n$的命题,若
%     \begin{enumerate}
%         \item 当$n=1$时,$P(1)$成立;
%         \item $P(n)$成立时必能推出$P(n+1)$成立;
%     \end{enumerate}
%     则$P(n)$对所有自然数$n$均成立。
% \end{theorem}

% \begin{theorem}[\protect 第二种数学归纳法]
%     设$P(n)$是关于自然数$n$的命题,若
%     \begin{enumerate}
%         \item 当$n=1$时,$P(1)$成立;
%         \item 设$n>1$,对所有自然数$m<n$都有$P(m)$成立时必能推出$P(n)$成立;
%     \end{enumerate}
%     则$P(n)$对所有自然数$n$均成立。
% \end{theorem}

\begin{theorem}[\protect\keyindex{鸽巢原理}{pigeonhole principle}{}]\label{theorem:7.ex02.2}
    对于某$n\in\mathbb{N}$,现有$n$个笼子和$n+1$只鸽子,
    所有的鸽子都被关在鸽笼里,那么至少有一个笼子有至少2只鸽子。
    也称\keyindex{狄利克雷抽屉原理}{Dirichlet's drawer principle}{}。
\end{theorem}

\subsection*{整除}
\begin{definition}
    设$a,b\in\mathbb{Z}$且$a\neq0$,若存在$q\in\mathbb{Z}$使得$b=aq$,
    则称$a$\keyindex{整除}{divide evenly}{}$b$,或说$b$能被$a$整除,记作$a|b$,
    并称$a$是$b$的\keyindex{因数}{divisor}{},也称{\sffamily 约数}、{\sffamily 除数},
    $b$是$a$的\keyindex{倍数}{multiple}{}。$a$不能整除$b$时记作$a\nmid b$.
\end{definition}

\begin{example}
    6能整除18,记作$6|18$,6是18的因数,18是6的倍数。
\end{example}

\begin{theorem}\label{theorem:7.ex02.3}
    整除具有以下性质:
    \begin{enumerate}
        \item $a|b\Leftrightarrow -a|b \Leftrightarrow a|-b \Leftrightarrow |a|||b|$;
        \item $a|b$且$b|c \Rightarrow a|c$;
        \item $a|b$且$a|c \Leftrightarrow$对任意的$x,y\in\mathbb{Z}$有$a|bx+cy$;
        \item 设$m\neq0$,则$a|b\Leftrightarrow ma|mb$;
        \item $a|b$且$b|a\Rightarrow b=\pm a$;
        \item 设$b\neq0$,则$a|b\Rightarrow |a|\le|b|$.
    \end{enumerate}
\end{theorem}
% \begin{corollary}
%     非零整数的因数只有有限个。
% \end{corollary}
% \begin{theorem}
%     设整数$b\neq0$,而$d_1,d_2,\ldots,d_k$是$b$的全体因数,
%     则$\displaystyle\frac{b}{d_1},\frac{b}{d_2},\ldots,\frac{b}{d_k}$也是
%     $b$的全体因数。此外,若$b>0$,则当$d$遍历$b$的全体正因数时,
%     $\displaystyle\frac{b}{d}$也遍历$b$的全体正因数。
% \end{theorem}
\begin{definition}
    设整数$p\neq0,\pm1$,若$p$除了$\pm1,\pm p$外没有其他因数,
    则称$p$为\keyindex{质数}{prime number}{},也称{\sffamily 素数}、{\sffamily 不可约数}。
    若$a\neq0,\pm1$且$a$不是质数,则称$a$是\keyindex{合数}{composite number}{}。
\end{definition}
\begin{example}
    3、5、11是质数,4、6、12是合数。0和1既不是质数也不是合数。
\end{example}
\begin{notation}
    下文若无特别说明,所指的质数总是正的。
\end{notation}
% \begin{theorem}
%     \begin{enumerate}
%         \item $a>1$是合数$\Leftrightarrow$$a=de,1<d<a,1<e<a$;
%         \item 若$d>1$,$p$是质数且$d|p$,则$d=p$.
%     \end{enumerate}
% \end{theorem}
% \begin{theorem}
%     若$a$是合数,则必存在质数$p|a$.
% \end{theorem}
% \begin{definition}
%     若一个整数的因数是质数时,称该因数为\keyindex{质因数}{prime factor}{}。
% \end{definition}
% \begin{theorem}
%     设整数$a\ge2$,则$a$一定可表示为质数的乘积(包括$a$本身是质数),即
%     \begin{align}\label{eq:7.ex02.primefactor}
%         a=p_1p_2\cdots p_s\, ,
%     \end{align}
%     其中$p_j(1\le j\le s)$是质数。
% \end{theorem}
% \begin{example}
%     1260共有6个质因数(包括相同的),其中不相同的有4个,即
%     $1260=2\times2\times3\times3\times5\times7=2^2\times3^2\times5\times7$.
% \end{example}
% \begin{corollary}
%     设整数$a\ge2$,
%     \begin{enumerate}
%         \item 若$a$是合数,则必有质数$p|a$且$p\le\sqrt{a}$;
%         \item 若$a$有表示\refeq{7.ex02.primefactor},则必有质数$p|a$且$p\le a^{\frac{1}{s}}$.
%     \end{enumerate}
% \end{corollary}
% \begin{theorem}
%     质数有无穷多个。
% \end{theorem}
% \begin{theorem}
%     设全体质数按大小排序成
%     \begin{align}
%         p_1=2,\quad p2=3,\quad p_3=5,\ldots\, .
%     \end{align}
%     则有
%     \begin{align}
%         p_n\le2^{2^{n-1}},\quad n=1,2,\ldots\, ,
%     \end{align}
%     及
%     \begin{align}
%         \pi(x)>\log_2{\log_2{x}},\quad x\ge2\, ,
%     \end{align}
%     其中$\pi(x)$表示不超过$x$的质数个数。
% \end{theorem}

\subsection*{带余除法}
初等数论的证明中最重要、最基本、最直接的工具是下面的
\keyindex{带余除法}{division with remainder}{},
也称\keyindex{欧几里德除法}{Euclidean division}{}。
\begin{theorem}\label{theorem:7.ex02.4}
    % \label{theorem:7.ex02.EuclideanDivision}
    对于给定的$a,b\in\mathbb{Z}$且$a\neq0$,必存在唯一一对$q,r\in\mathbb{Z}$,满足
    \begin{align}\label{eq:7.ex02.EuclideanDivision}
        b=qa+r,\quad 0\le r<|a|\, .
    \end{align}
    此外,$a|b \Leftrightarrow r=0$.
\end{theorem}
\begin{prove}
    {\sffamily 唯一性}\quad 若还有整数$q'$与$r'$满足
    \begin{align}\label{eq:7.ex02.prove-theorem4-01}
        b=q'a+r',\quad 0\le r'<|a|\, ,
    \end{align}
    不妨设$r'\ge r$.则由\refeq{7.ex02.EuclideanDivision}和
    \refeq{7.ex02.prove-theorem4-01}得$0\le r'-r<|a|$,及
    \begin{align}
        r'-r=(q-q')a\, .
    \end{align}
    若$r'-r>0$,则由上式及定理\ref{theorem:7.ex02.3}(6)
    推出$|a|\le r'-r$.这和$r'-r<|a|$矛盾。所以必有$r'=r$,进而得$q'=q$.

        {\sffamily 存在性}\quad 当$a|b$时,可取$q=\displaystyle\frac{b}{a}$,$r=0$.
    当$a\nmid b$时。考虑集合
    \begin{align}
        T=\{b-ka:k=0,\pm1,\pm2,\ldots\}\, .
    \end{align}
    容易看出,集合$T$中必有正整数,所以由定理\ref{theorem:7.ex02.1}知,
    $T$中必有一个最小正整数,设为
    \begin{align}
        t_0=b-k_0a>0\, .
    \end{align}
    现在来证明必有$t_0<|a|$.因$a\nmid b$,所以$t_0\neq |a|$.
    若$t_0>|a|$,则$t_1=t_0-|a|>0$,显然$t_1\in T$,$t_1<t_0$.
    这和$t_0$的最小性矛盾。取$q=k_0$,$r=t_0$就满足要求。

    最后,显然当$b=qa+r$时,$a|b \Leftrightarrow a|r$.
    当满足$0\le r<|a|$时,由定理\ref{theorem:7.ex02.3}(6)就
    推出$a|r \Leftrightarrow r=0$.这就证明了定理的最后一部分。
\end{prove}

上述定理还有更灵活的形式。
\begin{theorem}\label{theorem:7.ex02.5}
    对于给定的$a,b,d\in\mathbb{Z}$且$a\neq0$,必存在唯一一对$q_1,r_1\in\mathbb{Z}$,满足
    \begin{align}\label{eq:7.7.ex02.remainder}
        b=q_1a+r_1,\quad d\le r_1<|a|+d\, .
    \end{align}
    此外,$a|b \Leftrightarrow a|r_1$.
\end{theorem}

只要对$a$和$b-d$用定理\ref{theorem:7.ex02.4}即可推出定理\ref{theorem:7.ex02.5}。
适当选取$d$可令\refeq{7.7.ex02.remainder}变形为下面的形式:
\begin{align}
    b & =q_1a+r_1, & -\frac{|a|}{2}< r_1\le\frac{|a|}{2}\, ,\label{eq:7.ex02.remainder02} \\
    b & =q_1a+r_1, & -\frac{|a|}{2}\le r_1<\frac{|a|}{2}\, ,\label{eq:7.ex02.remainder03} \\
    b & =q_1a+r_1, & 1\le r_1\le |a|\, .\label{eq:7.ex02.remainder04}
\end{align}
通常称\refeq{7.ex02.EuclideanDivision}中的$r$为$b$被$a$除后的\keyindex{最小非负余数}{least non-negative remainder}{remainder余数},
\refeq{7.ex02.remainder02}和\refeq{7.ex02.remainder03}中的$r_1$都称为\keyindex{绝对最小余数}{least absolute remainder}{remainder余数},
\refeq{7.ex02.remainder04}中的$r_1$称为\keyindex{最小正余数}{least positive remainder}{remainder余数},
\refeq{7.7.ex02.remainder}中的$r_1$统称为\keyindex{余数}{remainder}{}。

% \begin{corollary}
%     设$a>0$,任意整数被$a$除后所得的最小非负余数是且仅是$0,1,\ldots,a-1$这$a$个数中的一个。
% \end{corollary}
\begin{corollary}
    给定正整数$a\ge2$,则任一正整数$n$必可唯一表示为
    \begin{align}
        n=r_ka^k+r_{k-1}a^{k-1}+\cdots+r_1a+r_0\, ,
    \end{align}
    其中整数$k\ge0,0\le r_j\le a-1(0\le j\le k),r_k\neq0$.
    这即正整数的$a$进制表示。
\end{corollary}
\begin{prove}
    对正整数$n$必有唯一的$k\ge 0$,使得$a^k\le n<a^{k+1}$.
    由带余除法知,必有唯一的$q_0,r_0$满足
    \begin{align}
        n=q_0a+r_0,\quad 0\le r_0<a\, .
    \end{align}
    若$k=0$,则必有$q_0=0$,$1\le r_0<a$,所以结论成立。
    设结论对$k=m\ge0$成立,则当$k=m+1$时,上式中的$q_0$必满足
    \begin{align}
        a^m\le q_0<a^{m+1}\, .
    \end{align}
    由假设知
    \begin{align}
        q_0=s_ma^m+\cdots+s_0\, ,
    \end{align}
    其中$0\le s_j\le a-1(0\le j\le m-1)$,$1\le s_m\le a-1$.因而有
    \begin{align}
        n=s_ma^{m+1}+\cdots+s_0a+r_0\, ,
    \end{align}
    即结论对$m+1$也成立。由数学归纳法,推论得证。
\end{prove}

\subsection*{最大公因数与最小公倍数}
\begin{definition}
    设$a_1,a_2\in\mathbb{Z}$,若$d|a_1$且$d|a_2$,则称$d$是
    $a_1$与$a_2$的\keyindex{公因数}{common divisor}{divisor因数}。
    一般地,设$a_1,\ldots,a_k$是$k$个整数,若$d|a_1,\cdots,d|a_k$,
    则称$d$是$a_1,\ldots,a_k$的公因数。
\end{definition}
\begin{example}
    12和18的公因数是$\pm1,\pm2,\pm3,\pm6$.$n$和$n+1$的公因数是$\pm1$.
    当$a_1,\ldots,a_k$中有一个不为零时,它们的公因数个数有限。
\end{example}
\begin{definition}
    设$a_1,a_2\in\mathbb{Z}$不全为零,称$a_1$和$a_2$的公因数中
    最大的为$a_1$和$a_2$的\keyindex{最大公因数}{greatest common divisor}{divisor因数}(GCD),
    记作$(a_1,a_2)$.一般地,设$a_1,\ldots,a_k$是$k$个不全为零的整数,
    称$a_1,\ldots,a_k$的公因数中最大的为$a_1,\ldots,a_k$的最大公因数,
    记作$(a_1,\ldots,a_k)$.用$\mathcal{D}(a_1,\ldots,a_k)$表示$a_1,\ldots,a_k$的
    所有公因数组成的集合。于是
    \begin{align}
        (a_1,a_2)        & =\max\limits_{d\in\mathcal{D}(a_1,a_2)}{d}\, ,        \\
        (a_1,\ldots,a_k) & =\max\limits_{d\in\mathcal{D}(a_1,\ldots,a_k)}{d}\, .
    \end{align}
\end{definition}
\begin{example}
    $\mathcal{D}(12,16)=\{\pm1,\pm2,\pm3,\pm6\}$,$(12,18)=6$;
    $\mathcal{D}(6,10,-15)=\{\pm1\}$,$(6,10,-15)=1$;
    $(n,n+1)=1$.
\end{example}
\begin{theorem}\label{theorem:7.ex02.6}
    最大公因数满足以下性质:
    \begin{enumerate}
        \item $(a_1,a_2)=(a_2,a_1)=(-a_1,a_2)$;一般地,\\
              $(a_1,a_2,\ldots,a_i,\ldots,a_k)=(a_i,a_2,\ldots,a_1,\ldots,a_k)=(-a_1,a_2,\ldots,a_i,\ldots,a_k)$;
        \item $a_1|a_j(j=2,\ldots,k)\Rightarrow (a_1,a_2)=(a_1,a_2,\ldots,a_k)=|a_1|$;
        \item 对任意整数$x$,$(a_1,a_2)=(a_1,a_2,a_1x)$;$(a_1,\ldots,a_k)=(a_1,\ldots,a_k,a_1x)$;
        \item 对任意整数$x$,$(a_1,a_2)=(a_1,a_2+a_1x)$;\\
              $(a_1,a_2,a_3,\ldots,a_k)=(a_1,a_2+a_1x,a_3,\ldots,a_k)$;
        \item 若$p$是质数,则
              \begin{align}
                  (p,a_1)=\left\{
                  \begin{array}{ll}
                      p, & \text{若}p|a_1\, ,      \\
                      1, & \text{若}p\nmid a_1\, ;
                  \end{array}
                  \right.
              \end{align}
              一般地
              \begin{align}
                  (p,a_1,\ldots,a_k)=\left\{
                  \begin{array}{ll}
                      p, & \text{若}p|a_j,\quad j=1,2,\ldots,k, \\
                      1, & \text{其他。}
                  \end{array}
                  \right.
              \end{align}
    \end{enumerate}
\end{theorem}
\begin{definition}
    若$(a_1,a_2)=1$,则称$a_1$和$a_2$是\keyindex{互质}{coprime}{}
    (或relatively prime、mutually prime)的,也称{\sffamily 互素}、{\sffamily 既约}。
    一般地,若$(a_1,\ldots,a_k)=1$,则称$a_1,\ldots,a_k$是互质的。
\end{definition}
\begin{theorem}\label{theorem:7.ex02.7}
    若存在整数$x_1,\ldots,x_k$使得$a_1x_1+\cdots+a_kx_k=1$,则$a_1,\ldots,a_k$是互质的。
\end{theorem}
\begin{prove}
    因为$a_1,\ldots,a_k$的任意公因数$d$一定要整除1,所以必有$d=\pm1$.定理得证。
\end{prove}
\begin{theorem}\label{theorem:7.ex02.8}
    设正整数$m|(a_1,\ldots,a_k)$,则
    \begin{align}\label{eq:7.ex02.prove-theorem8-01}
        m\left(\frac{a_1}{m},\cdots,\frac{a_k}{m}\right)=(a_1,\ldots,a_k)\, .
    \end{align}
    特别地有
    \begin{align}\label{eq:7.ex02.prove-theorem8-02}
        \left(\frac{a_1}{(a_1,\cdots,a_k)},\ldots,\frac{a_k}{(a_1,\cdots,a_k)}\right)=1\, .
    \end{align}
\end{theorem}
\begin{prove}
    记$D=(a_1,\ldots,a_k)$.由$m|D$,$D|a_j(1\le j \le k)$知
    $m|a_j(1\le j \le k)$,故
    \begin{align}
        \frac{D}{m}\bigg|\frac{a_j}{m},\quad j=1,\ldots,k\, ,
    \end{align}
    即$\displaystyle\frac{D}{m}$是$\displaystyle\frac{a_1}{m},\ldots,\frac{a_k}{m}$的公因数
    且为正,所以由定义知
    \begin{align}\label{eq:7.ex02.prove-theorem8-03}
        \frac{D}{m}\le\left(\frac{a_1}{m},\ldots,\frac{a_k}{m}\right)\, .
    \end{align}
    另一方面,若$\displaystyle d\bigg|\frac{a_j}{m}(1\le j\le k)$,
    则$md|a_j(j=1,\ldots,k)$,由定义知
    \begin{align}
        md\le D,\quad \text{即}d\le\frac{D}{m}\, .
    \end{align}
    取$d=\displaystyle\left(\frac{a_1}{m},\ldots,\frac{a_k}{m}\right)$,
    由此及\refeq{7.ex02.prove-theorem8-03}即得\refeq{7.ex02.prove-theorem8-01}。
    在\refeq{7.ex02.prove-theorem8-01}中取$m=(a_1,\ldots,a_k)$即得\refeq{7.ex02.prove-theorem8-02}。
\end{prove}
\begin{definition}
    设$a_1,a_2\in\mathbb{Z}$均不为零,若$a_1|l$且$a_2|l$,
    则称$l$是$a_1$和$a_2$的\keyindex{公倍数}{common multiple}{multiple倍数}。
    一般地,设$a_1,\ldots,a_k$是$k$个均不为零的整数,
    若$a_1|l,\ldots,a_k|l$,则称$l$是$a_1,\ldots,a_k$的公倍数。
    此外,以$\mathcal{L}(a_1,\ldots,a_k)$表示$a_1,\ldots,a_k$的所有公倍数构成的集合。
\end{definition}
\begin{example}
    $\mathcal{L}(2,3)=\{0,\pm6,\pm12,\ldots,\pm6k,\ldots\}$.
\end{example}
\begin{definition}
    设$a_1,a_2\in\mathbb{Z}$均不为零,我们把$a_1$和$a_2$公倍数中的最小正数
    称为$a_1$和$a_2$的\keyindex{最小公倍数}{least common multiple}{multiple倍数},记作$[a_1,a_2]$,即
    \begin{align}
        [a_1,a_2]=\min\limits_{l\in\mathcal{L}(a_1,a_2),l>0}{l}\, .
    \end{align}
    一般地,设$a_1,\ldots,a_k\in\mathbb{Z}$均不为零,我们把
    $a_1,\ldots,a_k$公倍数中的最小正数称为$a_1,\ldots,a_k$的最小公倍数,
    记作$[a_1,\ldots,a_k]$,即
    \begin{align}
        [a_1,\ldots,a_k]=\min\limits_{l\in\mathcal{L}(a_1,\ldots,a_k),l>0}{l}\, .
    \end{align}
\end{definition}
\begin{example}
    $[2,3]=6$;$[2,3,4]=12$.
\end{example}
\begin{theorem}\label{theorem:7.ex02.9}
    最小公倍数满足以下性质:
    \begin{enumerate}
        \item $[a_1,a_2]=[a_2,a_1]=[-a_1,a_2]$;一般有\\
              $[a_1,a_2,\ldots,a_i,\ldots,a_k]=[a_i,a_2,\ldots,a_1,\ldots,a_k]=[-a_1,a_2,\ldots,a_i,\ldots,a_k]$;
        \item $a_2|a_1\Rightarrow [a_1,a_2]=|a_1|$;\\
              $a_j|a_1(2\le j\le k)\Rightarrow [a_1,\ldots,a_k]=|a_1|$;
        \item 对任意的$d|a_1$,有$[a_1,a_2]=[a_1,a_2,d]$;$[a_1,\ldots,a_k]=[a_1,\ldots,a_k,d]$.
    \end{enumerate}
\end{theorem}
\begin{theorem}\label{theorem:7.ex02.10}
    设$m>0$,则$[ma_1,\ldots,ma_k]=m[a_1,\ldots,a_k]$.
\end{theorem}
\begin{prove}
    设$L=[ma_1,\ldots,ma_k], L'=[a_1,\ldots,a_k]$.
    由$ma_j|L(1\le j\le k)$推出$\displaystyle a_j\bigg|\frac{L}{m}(1\le j\le k)$,
    进而由最小公倍数定义知$L'\le\displaystyle\frac{L}{m}$.
    另一方面,由$a_j|L'(1\le j\le k)$推出$ma_j|mL'(1\le j\le k)$,
    进而由最小公倍数定义得$L\le mL'$.由此定理得证。
\end{prove}
\begin{theorem}\label{theorem:7.ex02.11}
    $a_j|c(1\le j\le k)\Leftrightarrow [a_1,\ldots,a_k]|c$.
\end{theorem}
\begin{prove}
    $[a_1,\ldots,a_k]|c\Rightarrow a_j|c(1\le j\le k)$是显然的。
    下面证$a_j|c(1\le j\le k)\Rightarrow [a_1,\ldots,a_k]|c$.
    设$L=[a_1,\ldots,a_k]$.由定理\ref{theorem:7.ex02.4}知,有$q,r$使得
    \begin{align}
        c=qL+r,\quad 0\le r<L\, .
    \end{align}
    由此及$a_j|c$推出$a_j|r(1\le j\le k)$,所以$r$是公倍数。
    进而由最小公倍数的定义及$0\le r<L$可得$r=0$,即$L|c$.
    结论表明:公倍数一定是最小公倍数的倍数。
\end{prove}
\begin{theorem}\label{theorem:7.ex02.12}
    设$D$为正整数,则$D=(a_1,\ldots,a_k)$的充要条件是
    \begin{enumerate}
        \item $D|a_j(1\le j\le k)$;
        \item 若$d|a_j(1\le j\le k)$,则$d|D$.
    \end{enumerate}
\end{theorem}
\begin{prove}
    {\sffamily 充分性}\quad 由第一个条件知$D$是$a_j(1\le j\le k)$的公因数,
    由第二个条件、定理\ref{theorem:7.ex02.3}(6)及$D\ge1$知,
    $a_j(1\le j\le k)$的任一公因数$d$有$|d|\le D$.
    因而由定义知$D=(a_1,\ldots,a_k)$.

        {\sffamily 必要性}\quad 设$d_1,\ldots,d_s$是$a_1,\ldots,a_k$的
    全体公因数,$L=[d_1,\ldots,d_s]$.由定理\ref{theorem:7.ex02.11}
    知$L|a_j(1\le j\le k)$,因此$L$满足了两个条件。
    由上面充分性的证明知$L=(a_1,\ldots,a_k)=D$.必要性得证。
    结论表明:公因数一定是最大公因数的因数。
\end{prove}
\begin{theorem}\label{theorem:7.ex02.13}
    设$m>0$,则$m(b_1,\ldots,b_k)=(mb_1,\ldots,mb_k)$.
\end{theorem}
\begin{prove}
    在定理\ref{theorem:7.ex02.8}中取$a_j=mb_j(1\le j\le k)$,
    由定理\ref{theorem:7.ex02.12}可得$m|(a_1,\ldots,a_k)$.
    因此\refeq{7.ex02.prove-theorem8-01}成立,即本定理结论成立。
\end{prove}
\begin{theorem}\label{theorem:7.ex02.14}
    \begin{enumerate}
        \item $(a_1,a_2,a_3,\ldots,a_k)=((a_1,a_2),a_3,\ldots,a_k)$;
        \item $(a_1,\ldots,a_{k+r})=((a_1,\ldots,a_k),(a_{k+1},\ldots,a_{k+r}))$.
    \end{enumerate}
\end{theorem}
\begin{prove}
    对于第一个结论:若$d|a_j(1\le j\le k)$,则由定理\ref{theorem:7.ex02.12}知,
    $d|(a_1,a_2)$,$d|a_j(3\le j\le k)$;反过来,若$d|(a_1,a_2)$,$d|a_j(3\le j\le k)$,
    则由定义知,$d|a_j(1\le j\le k)$.这就证明了
    \begin{align}
        \mathcal{D}(a_1,a_2,a_3,\ldots,a_k)=\mathcal{D}((a_1,a_2),a_3,\ldots,a_k)\, .
    \end{align}
    故第一个结论成立。由它可立即推出第二个结论。
\end{prove}
\begin{theorem}\label{theorem:7.ex02.15}
    设$(m,a)=1$,则$(m,ab)=(m,b)$.
\end{theorem}
\begin{prove}
    $m=0$时$a=\pm1$,结论显然成立。当$m\neq0$时,
    由定理\ref{theorem:7.ex02.6}、定理\ref{theorem:7.ex02.13}和定理\ref{theorem:7.ex02.14}可得
    \begin{align}
        (m,b)=(m,b(m,a))=(m,(mb,ab))=(m,mb,ab)=(m,ab)\, .
    \end{align}
    得证。
\end{prove}
\begin{theorem}\label{theorem:7.ex02.16}
    设$(m,a)=1$,那么,若$m|ab$,则$m|b$.
\end{theorem}
\begin{prove}
    由定理\ref{theorem:7.ex02.6}和定理\ref{theorem:7.ex02.15}得
    $|m|=(m,ab)=(m,b)$,于是$m|b$.
\end{prove}
% \begin{theorem}
%     $[a_1,a_2](a_1,a_2)=|a_1a_2|$.
% \end{theorem}
\begin{theorem}\label{theorem:7.ex02.17}
    设$a_1,\ldots,a_k\in\mathbb{Z}$不全为零,则有
    \begin{enumerate}
        \item $(a_1,\ldots,a_k)=\min\{s=a_1x_1+\cdots+a_kx_k:x_j\in\mathbb{Z}(1\le j\le k),s>0\}$,即
              $a_1,\ldots,a_k$的最大公因数等于$a_1,\ldots,a_k$的所有整系数线性组合
              构成的集合$S$中的最小正整数。
        \item 一定存在一组整数$x'_1,\ldots,x'_k$使得
              \begin{align}\label{eq:7.ex02.theorem17-02}
                  (a_1,\ldots,a_k)=a_1x'_1+\cdots+a_kx'_k\, .
              \end{align}
    \end{enumerate}
\end{theorem}
\begin{prove}
    由于$0<a_1^2+\cdots+a_k^2\in S$,所以集合$S$中有正整数,
    由定理\ref{theorem:7.ex02.1}知$S$中必有最小正整数,记为$s_0$.
    显然对任一公因数$d|a_j(1\le j \le k)$必有$d|s_0$,所以$|d|\le s_0$.
    另一方面,对任一$a_j$由定理\ref{theorem:7.ex02.4}知存在$q_j,r_j$满足
    \begin{align}
        a_j=q_js_0+r_j,\quad 0\le r_j<s_0\, .
    \end{align}
    显然$r_j\in S$.若$r_j>0$,则和$s_0$的最小性矛盾,所以$r_j=0$,
    即$s_0|a_j(1\le j \le k)$.所以$s_0$是最大公因数。$s_0$当然是
    \refeq{7.ex02.theorem17-02}右边的形式。
\end{prove}

% \subsection*{算术基本定理}
% \begin{theorem}
%     设$p$是质数,$p|a_1a_2$,则$p|a_1$或$p|a_2$至少有一个成立。
%     一般地,若$p|a_1\cdots a_k$,则$p|a_1,\ldots,p|a_k$至少有一个成立。
% \end{theorem}
% \begin{theorem}[\protect\keyindex{算术基本定理}{fundamental theorem of arithmetic}{}]
%     设$a>1$,则必有
%     \begin{align}\label{eq:7.ex02.arithmeticfundamental}
%         a=p_1p_2\cdots p_s\, ,
%     \end{align}
%     其中$p_j(1\le j\le s)$是质数,且在不计次序的意义下,
%     表示\refeq{7.ex02.arithmeticfundamental}是唯一的。
% \end{theorem}

\subsection*{辗转相除法}
\keyindex{辗转相除法}{Euclidean algorithm}{},
也称{\sffamily 欧几里得算法},是指下面求取最大公因数的算法。
它最早出现于欧几里得的《几何原本》中,我国则可追溯至约东汉出现的《九章算术》。
\begin{theorem}\label{theorem:7.ex02.EuclideanAlgorithm}
    给定$u_0,u_1\in\mathbb{Z}$,且$u_1\neq0,u_1\nmid u_0$.
    我们一定可以反复应用定理\ref{theorem:7.ex02.EuclideanDivision}得到下面$k+1$个等式:
    \begin{align}
        u_0     & =q_0u_1+u_2,             &  & 0<u_2<|u_1|,\nonumber    \\
        u_1     & =q_1u_2+u_3,             &  & 0<u_3<u_2,\nonumber      \\
        u_2     & =q_2u_3+u_4,             &  & 0<u_4<u_3,\nonumber      \\
        \cdots  & \cdots\cdots\cdots\cdots &  & \cdots\cdots\cdots\cdots \\
        u_{k-2} & =q_{k-2}u_{k-1}+u_k,     &  & 0<u_k<u_{k-1},\nonumber  \\
        u_{k-1} & =q_{k-1}u_k+u_{k+1},     &  & 0<u_{k+1}<u_k,\nonumber  \\
        u_k     & =q_ku_{k+1}.             &  & \nonumber
    \end{align}
\end{theorem}
% \begin{theorem}
%     在定理\ref{theorem:7.ex02.EuclideanAlgorithm}的条件和符号下,我们有
%     \begin{enumerate}
%         \item $u_{k+1}=(u_0,u_1)$;
%         \item $d|u_0$且$d|u_1$的充要条件是$d|u_{k+1}$;
%         \item 存在整数$x_0,x_1$,使$u_{k+1}=x_0u_0+x_1u_1$.
%     \end{enumerate}
% \end{theorem}
% \begin{example}
%     利用辗转相除法求198和252的最大公因数,并将其表示为198和252的整系数线性组合。因为
%     \begin{align*}
%         252 & =1\times198+54\, , \\
%         198 & =3\times54+36\, ,  \\
%         54  & =1\times36+18\, ,  \\
%         36  & =2\times18\, ,
%     \end{align*}
%     于是$(252,198)=(198,54)=(54,36)=(36,18)=18$,且得
%     \begin{align*}
%         18 & =54-1\times36                 \\
%            & =54-(198-3\times54)           \\
%            & =-198+4\times54               \\
%            & =-198+4\times(252-1\times198) \\
%            & =4\times252-5\times198\, .
%     \end{align*}
% \end{example}

% \subsection*{同余}
% \begin{definition}
%     设$a,b,m\in\mathbb{Z}$且$m\neq0$,若$m|a-b$,则称$a$与$b$\keyindex{模$m$同余}{congruent modulo $m$}{},
%     也称$a$同余于$b$模$m$、$b$是$a$对模$m$的剩余,记作
%     \begin{align}\label{eq:7.ex02.congruent}
%         a\equiv b\pmod{m}\, ,
%     \end{align}
%     其中$m$称为\keyindex{模}{modulus}{},称\refeq{7.ex02.congruent}为模$m$的同余式;
%     否则称$a$不同余于$b$模$m$、$b$不是$a$对模$m$的剩余,记作
%     \begin{align}
%         a\not\equiv b\pmod{m}\, .
%     \end{align}
% \end{definition}

% 因为$m|a-b\Leftrightarrow -m|a-b$,所以\refeq{7.ex02.congruent}等价于$a\equiv b\pmod{-m}$.
% 由此,下文均假定模$m\ge1$.\refeq{7.ex02.congruent}中,
% 若$0\le b<m$,则称$b$是$a$对模$m$的最小非负剩余;
% 若$1\le b\le m$,则称$b$是$a$对模$m$的最小正剩余;
% 若$\displaystyle -\frac{m}{2}<b\le\frac{m}{2}$(或$\displaystyle -\frac{m}{2}\le b<\frac{m}{2}$),
% 则称$b$是$a$对模$m$的绝对最小剩余。

% \begin{example}
%     $m|a$可记为$a\equiv 0\pmod{m}$;偶数可记为$a\equiv 0\pmod{2}$;
%     奇数可记为$a\equiv 1\pmod{2}$.
% \end{example}

% \begin{theorem}
%     $a$与$b$模$m$同余的充要条件是$a$和$b$被$m$除后的最小非负余数相等,即若
%     \begin{align}
%         a & =q_1m+r_1, & 0\le r_1<m\, , \\
%         b & =q_2m+r_2, & 0\le r_2<m\, ,
%     \end{align}
%     则$r_1=r_2$.
% \end{theorem}

% 容易证明,$a$对模$m$的最小非负剩余、最小正剩余、绝对最小剩余
% 正好分别是$a$被$m$除后的最小非负余数、最小正余数、绝对最小余数。

% \begin{theorem}\label{theorem:7.ex02.congruentequivalence}
%     同余是一种等价关系,即有
%     \begin{enumerate}
%         \item $a\equiv a\pmod{m}$;
%         \item $a\equiv b\pmod{m} \Leftrightarrow b\equiv a\pmod{m}$;
%         \item $a\equiv b\pmod{m}, b\equiv c\pmod{m} \Rightarrow a\equiv c\pmod{m}$.
%     \end{enumerate}
% \end{theorem}
% \begin{theorem}
%     同余式可以相加,即若
%     \begin{align}\label{eq:7.ex02.addcongruent}
%         a\equiv b\pmod{m},\qquad c\equiv d\pmod{m}\, ,
%     \end{align}
%     则
%     \begin{align}
%         a+c\equiv b+d\pmod{m}\, .
%     \end{align}
% \end{theorem}
% \begin{theorem}
%     同余式可以相乘,即若\refeq{7.ex02.addcongruent}成立,则有
%     \begin{align}
%         ac\equiv bd\pmod{m}\, .
%     \end{align}
% \end{theorem}
% \begin{theorem}
%     设$f(x)=a_nx^n+\cdots+a_0$,$g(x)=b_nx^n+\cdots+b_0$是
%     两个整系数多项式,满足
%     \begin{align}\label{eq:7.ex02.polynomialcongruent}
%         a_j\equiv b_j\pmod{m},\quad 0\le j\le n\, .
%     \end{align}
%     那么若$a\equiv b\pmod{m}$,则
%     \begin{align}
%         f(a)\equiv g(b)\pmod{m}\, .
%     \end{align}
% \end{theorem}
% \begin{definition}
%     把满足\refeq{7.ex02.polynomialcongruent}的这两个多项式
%     称作多项式$f(x)$与$g(x)$模$m$同余,记作
%     \begin{align}
%         f(x)\Equiv g(x)\pmod{m}\, .
%     \end{align}
% \end{definition}

% \begin{theorem}
%     设$d\ge1$, $d|m$,则$a\equiv b\pmod{m} \Rightarrow a\equiv b\pmod{d}$.
% \end{theorem}
% \begin{theorem}
%     设$d\neq0$,则$a\equiv b\pmod{m} \Leftrightarrow da\equiv db\pmod{|d|m}$.
% \end{theorem}

% 注意在模不变的条件下,同余式两边不能相约。
% \begin{example}
%     $6\times3\equiv6\times8\pmod{10}$,但是$3\not\equiv8\pmod{10}$.
% \end{example}

% \begin{theorem}\label{theorem:7.ex02.congruentreduce}
%     同余式$\displaystyle ca\equiv cb\pmod{m}\Leftrightarrow a\equiv b\pmod{\frac{m}{(c,m)}}$.
%     特别地,当$(c,m)=1$时可得$a\equiv b\pmod{m}$,即此时可两边约去$c$.
% \end{theorem}

% \begin{theorem}\label{theorem:7.ex02.modularinverse}
%     若$m\ge1$,$(a,m)=1$,则存在$c$使得
%     \begin{align}\label{eq:7.ex02.modularinverse}
%         ca\equiv1\pmod{m}\, .
%     \end{align}
%     我们把$c$称作$a$对模$m$的逆,或\keyindex{模逆元}{modular multiplicative inverse}{},
%     记作$a^{-1}\pmod{m}$或$a^{-1}$.
% \end{theorem}

% $a$对模$m$的逆不是唯一的。若$c$是$a$对模$m$的逆,
% 则任一$\bar{c}\equiv c\pmod{m}$也必是$a$对模$m$的逆;
% $a$对模$m$的任意两个逆$c_1,c_2$必有$c_1\equiv c_2\pmod{m}$;
% 若$(a,m)=1$,则$(a^{-1},m)=1$,及$(a^{-1})^{-1}\equiv a\pmod{m}$.
% \begin{notation}
%     下文中约定$a^{-1}\pmod{m}$或$a^{-1}$指
%     任一取定的满足\refeq{7.ex02.modularinverse}的$c$.
% \end{notation}

% \begin{example}
%     $a$对模7的逆(只列出了一个值):
%     \begin{table}[htbp]
%         \centering
%         \begin{tabular}{c|cccccc}
%             \toprule
%             $a$              & 1 & 2 & 3 & 4 & 5 & 6 \\
%             \midrule
%             $a^{-1}\pmod{7}$ & 1 & 4 & 5 & 2 & 3 & 6 \\
%             \bottomrule
%         \end{tabular}
%         \caption{$a$对模7的逆(只列出了一个值)。}
%         \label{tab:7.ex02.modularinverse}
%     \end{table}
% \end{example}

% \begin{theorem}\label{theorem:7.ex02.conditioncongruentegroup}
%     同余式组
%     \begin{align}
%         a\equiv b\pmod{m_j}\, \quad j=1,2,\ldots,k
%     \end{align}
%     同时成立的充要条件是
%     \begin{align}
%         a\equiv b\pmod{[m_1,\ldots,m_k]}\, .
%     \end{align}
% \end{theorem}
% \begin{definition}[同余类(剩余类)]
%     由定理\ref{theorem:7.ex02.congruentequivalence}知,对于给定的模$m$,
%     整数的同余关系是一个等价关系,因此全体整数可按对模$m$是否同余
%     分为若干个两两不相交的集合,使得在同一个集合中的任意两个数
%     对模$m$一定同余,而属于不同集合中的两个数对模$m$一定不同余。
%     每一个这样的集合称为是模$m$的\keyindex{同余类}{congruence class}{},
%     或模$m$的\keyindex{剩余类}{residue class}{}。
%     我们把$r$所属的模$m$的同余类表示为$r\mod{m}$.
% \end{definition}
% \begin{theorem}
%     \begin{enumerate}
%         \item $r\mod{m}=\{r+km:k\in\mathbb{Z}\}$;
%         \item $r\mod{m}=s\mod{m}\Leftrightarrow r\equiv s\pmod{m}$;
%         \item 对任意的$r,s$,要么$r\mod{m}=s\mod{m}$,要么$r\mod{m}$与$s\mod{m}$的交集为空集。
%     \end{enumerate}
% \end{theorem}
% \begin{theorem}
%     对于给定的模$m$,有且恰有$m$个不同的模$m$的同余类,即
%     \begin{align}
%         0\mod{m},\quad 1\mod{m},\quad \ldots,\quad (m-1)\mod{m}\, .
%     \end{align}
% \end{theorem}
% \begin{theorem}
%     \begin{enumerate}
%         \item 在任意取定的$m+1$个整数中,必有两个数对模$m$同余;
%         \item 存在$m$个数两两对模$m$不同余。
%     \end{enumerate}
% \end{theorem}
% \begin{definition}
%     一组数$y_1,\ldots,y_s$称为是模$m$的\keyindex{完全剩余系}{complete residue system}{},
%     如果对任意的$a$有且仅有一个$y_j$满足$a\equiv y_j\pmod{m}$.
% \end{definition}

% \subsection*{同余方程}
% \begin{definition}
%     设整系数多项式
%     \begin{align}
%         f(x)=a_nx^n+\cdots+a_1x+a_0\, ,
%     \end{align}
%     我们称含有变量$x$的同余式
%     \begin{align}\label{eq:7.ex02.congruenceequation}
%         f(x)\equiv0\pmod{m}
%     \end{align}
%     为模$m$的\keyindex{同余方程}{congruence equation}{equation方程}。
%     若整数$c$满足
%     \begin{align}
%         f(c)\equiv0\pmod{m}\, ,
%     \end{align}
%     则称$c$是同余方程\refeq{7.ex02.congruenceequation}的\keyindex{解}{solution}{}。
% \end{definition}

% 在上述定义中,显然同余类$c\mod{m}$中的任一整数也是
% 同余方程\refeq{7.ex02.congruenceequation}的解。
% 我们把这些解都看作是相同的,也常说同余类$c\mod{m}$是该方程的解,
% 写为$x\equiv c\pmod{m}$.当$c_1,c_2$均为该同余方程的解且对模$m$不同余时
% 才把它们看作是不同的解。我们把所有对模$m$两两不同余的解的个数
% 称为是同余方程\refeq{7.ex02.congruenceequation}
% 的\keyindex{解数}{number of solutions}{}。
% 因此我们只需要在模$m$的一组完全剩余系中来解模$m$的同余方程。
% 显然模$m$的同余方程的解数至多为$m$.
% \begin{example}
%     对于同余方程$4x^2+27x-12\equiv0\pmod{15}$,
%     取模15的一个完全剩余系$-7,-6,\ldots,-1,0,1,\ldots,6,7$,
%     直接代入验算知$x=-6,3$是解,所以改同余方程的解
%     是$x\equiv -6,3\pmod{15}$,解数为2.
% \end{example}
% \begin{definition}
%     设$m\nmid a$,称
%     \begin{align}\label{eq:7.ex02.linearcongruence}
%         ax\equiv b\pmod{m}
%     \end{align}
%     为模$m$的{\sffamily 一次同余方程}。
% \end{definition}
% \begin{theorem}
%     \refeq{7.ex02.linearcongruence}有解的必要条件是
%     \begin{align}\label{eq:7.ex02.conditionsolution}
%         (a,m)|b\, .
%     \end{align}
% \end{theorem}
% \begin{theorem}\label{theorem:7.ex02.hassolutionlinear}
%     当$(a,m)=1$时,同余方程\refeq{7.ex02.linearcongruence}必有解,且其解数为1.
% \end{theorem}
% \begin{prove}
%     当$(a,m)=1$时,由定理\ref{theorem:7.ex02.modularinverse}知,
%     $a$对模$m$有逆$a^{-1}$(任取一个)满足
%     \begin{align}
%         aa^{-1}\equiv1\pmod{m}\, .
%     \end{align}
%     容易看出
%     \begin{align}
%         x_1=a^{-1}b
%     \end{align}
%     就满足同余方程\refeq{7.ex02.linearcongruence}。若还有解$x_2$,则有
%     \begin{align}
%         ax_2\equiv ax_1\pmod{m}\, ,
%     \end{align}
%     则从定理\ref{theorem:7.ex02.congruentreduce}推出
%     \begin{align}
%         x_2\equiv x_1\pmod{m}\, .
%     \end{align}
%     这就证明了解数为1.
% \end{prove}
% \begin{theorem}
%     同余方程\refeq{7.ex02.linearcongruence}有解的充要条件
%     是\refeq{7.ex02.conditionsolution}成立。在有解时,
%     它的解数等于$(a,m)$,以及若$x_0$是\refeq{7.ex02.linearcongruence}的解,
%     则它的$(a,m)$个解是
%     \begin{align}
%         x\equiv x_0+\frac{m}{(a,m)}t\pmod{m},\quad t=0,1,\ldots,(a,m)-1\, .
%     \end{align}
% \end{theorem}

% \begin{definition}
%     设$f_j(x), j=1,2,\ldots,k$是整系数多项式,我们把含有变量$x$的一组同余式
%     \begin{align}\label{eq:7.ex02.congruencegroup}
%         f_j(x)\equiv0\pmod{m_j},\quad 1\le j\le k\, ,
%     \end{align}
%     称为{\sffamily 同余方程组}。若整数$c$同时满足
%     \begin{align}
%         f_j(c)\equiv0\pmod{m_j},\quad 1\le j\le k\, ,
%     \end{align}
%     则称$c$是同余方程组\refeq{7.ex02.congruencegroup}的\keyindex{解}{solution}{}。
% \end{definition}

% 显然在上述定义中,同余类
% \begin{align}\label{eq:7.ex02.groupsolution}
%     c\mod{m},\quad m=[m_1,\ldots,m_k]
% \end{align}
% 中任一整数也是同余方程组\refeq{7.ex02.congruencegroup}的解,
% 我们把它们看作是相同的,也常说同余类\refeq{7.ex02.groupsolution}是
% 该同余方程组的一个解,写作$x\equiv c\pmod{m}$.
% 当$c_1,c_2$均为该同余方程组的解且对模$m$不同余时
% 才把它们看作是不同的解。我们把所有对模$m$两两不同余的解的个数
% 称为是同余方程组\refeq{7.ex02.congruencegroup}的\keyindex{解数}{number of solutions}{}。
% 因此我们只需要在模$m$的一组完全剩余系中来解该同余方程组,
% 它的解数至多为$m$.此外,只要同余方程组中任一一个方程无解,
% 则\refeq{7.ex02.congruencegroup}一定无解。
% \begin{theorem}[\protect\keyindex{中国剩余定理}{Chinese remainder theorem}{}(CRT)]
%     也称{\sffamily 孙子定理}:设$m_1,\ldots,m_k$是两两互质的正整数,
%     则对任意整数$a_1,\ldots,a_k$,一次同余方程组
%     \begin{align}\label{eq:7.ex02.CRT}
%         x\equiv a_j\pmod{m_j},\quad 1\le j\le k\, ,
%     \end{align}
%     必有解,且解数为1.事实上,该同余方程组的解是
%     \begin{align}
%         x\equiv M_1M_1^{-1}a_1+\ldots+M_kM_k^{-1}a_k\pmod{m}\, ,
%     \end{align}
%     这里$m=m_1\cdots m_k$,$m=m_jM_j(1\le j\le k)$,以及$M_j^{-1}$是满足
%     \begin{align}
%         M_jM_j^{-1}\equiv1\pmod{m_j},\quad 1\le j\le k
%     \end{align}
%     的一个整数(即是$M_j$对模$m_j$的逆)。
% \end{theorem}
% \begin{prove}
%     首先指出一个事实:若$x_0$满足同余方程组\refeq{7.ex02.CRT},
%     且$x_0'$满足下面的另一同余方程组
%     \begin{align}
%         x\equiv a_j'\pmod{m_j},\quad 1\le j\le k\, ,
%     \end{align}
%     则$x_0+x_0'$一定是同余方程组
%     \begin{align}
%         x\equiv a_j+a_j'\pmod{m_j},\quad 1\le j\le k
%     \end{align}
%     的解。因此,我们可用下面的叠加方法来求同余方程组\refeq{7.ex02.CRT}的解。设
%     \begin{align}\label{eq:7.ex02.proveCRT03}
%         a_j^{(i)}=\left\{\begin{array}{ll}
%             a_j, & \text{若}i=j\, ,     \\
%             0,   & \text{若}i\neq j\, .
%         \end{array}\right.
%     \end{align}
%     对每个固定的$i(1\le j\le k)$考虑同余方程组
%     \begin{align}\label{eq:7.ex02.proveCRT01}
%         x\equiv a_j^{(i)}\pmod{m_j},\quad 1\le j\le k\, .
%     \end{align}
%     注意到$j\neq i$时$a_j^{(i)}=0$,结合$m_j$两两互质,
%     由这个方程组的第$1,\ldots,i-1,i+1,\ldots,k$个方程知
%     \begin{align}
%         x\equiv0\pmod{M_i}\, ,
%     \end{align}
%     即
%     \begin{align}\label{eq:7.ex02.proveCRT02}
%         x=M_iy\, .
%     \end{align}
%     代入第$i$个方程得
%     \begin{align}
%         M_iy\equiv a_i\pmod{m_i}\, .
%     \end{align}
%     由定理\ref{theorem:7.ex02.hassolutionlinear}的证法知
%     \begin{align}
%         y\equiv M_i^{-1}a_i\pmod{m_i}\, ,
%     \end{align}
%     即
%     \begin{align}
%         M_iy\equiv M_iM_i^{-1}a_i\pmod{m}\, .
%     \end{align}
%     由此及\refeq{7.ex02.proveCRT02}得
%     \begin{align}
%         x\equiv M_iM_i^{-1}a_i\pmod{m}\, .
%     \end{align}
%     容易验证,$M_iM_i^{-1}a_i$确是同余方程组\refeq{7.ex02.proveCRT01}的解
%     (这就证明了它有解且解数为1)。注意到由\refeq{7.ex02.proveCRT03}可得
%     \begin{align}
%         a_j^{(1)}+a_j^{(2)}+\cdots+a_j^{(k)}=a_j\, ,
%     \end{align}
%     所以$M_1M_1^{-1}a_1+\cdots+M_kM_k^{-1}a_k$一定是同余方程组\refeq{7.ex02.CRT}的解。
%     若$c_1,c_2$均是同余方程组\refeq{7.ex02.CRT}的解,
%     则必有
%     \begin{align}
%         c_1\equiv c_2\pmod{m_j},\quad 1\le j\le k\, .
%     \end{align}
%     又因为$m_1,\ldots,m_k$两两互质,所以
%     \begin{align}
%         m=m_1\cdots m_k=[m_1,\ldots,m_k]\, .
%     \end{align}
%     利用\ref{theorem:7.ex02.conditioncongruentegroup}结合上两式可得
%     \begin{align}
%         c_1\equiv c_2\pmod{m}\, ,
%     \end{align}
%     即同余方程组\refeq{7.ex02.CRT}的解数必为1.
% \end{prove}


\part{光的散射}
\chapterimage{Pictures/chap08/dragons-fourier-600x1200.png}
\chapter{反射模型}\label{chap:反射模型}
\setcounter{sidenote}{1}
本章定义一组类来描述光在表面上散射的方式。回想\refsub{BRDF}中
我们介绍了双向反射分布函数(BRDF)抽象来描述表面的光反射,
双向透射分布函数(BTDF)来描述表面的透射,以及
双向散射分布函数(BSFD)来统合这两种效应。
本章中,我们将从为这些表面反射和透射函数定义通用接口开始。

许多来自表面的散射通常最好描述为多个BRDF和BTDF随空间变化的混合体;
在第\refchap{材质},我们将介绍结合了多个BRDF和BTDF的BSDF对象
以表示来自表面的整体散射。本章回避了反射和折射性质随表面变化的问题;
第\refchap{纹理}的纹理类将解决该问题。
BRDF和BTDF只显式建模了在表面上同一点入射和出射的光的散射。
对于展现出有意义的次表面光传输的曲面,我们将引入类\refvar{BSSRDF}{},
在第\refchap{体积散射}介绍一些相关理论后,它将在\refsec{BSSRDF}对次表面散射建模。

表面反射模型有以下几个来源:
\begin{itemize}
    \item \emph{测量的数据}:许多真实世界表面的反射分布性质已在实验室中测定。
          这样的数据可直接以表格形式使用或用来为一组基函数计算系数。
    \item \emph{现象模型}\sidenote{译者注:原文phenomenological models。}:
          试图描述真实世界表面定性性质的方程在仿真时可能很有效。
          这类BSDF可能很容易使用,因为它们常常有直观的参数来修改其表现(例如“粗糙度”)。
    \item \emph{模拟}:有时关于表面组成的底层信息是已知的。
          例如,我们可能知道涂料是悬浮在介质中的平均大小彩色颗粒组成的,
          或者某种布料是两种织线组成的,且知道每种的反射性质。
          在这些情况下,可以模拟来自微观几何体的光散射来生成反射数据。
          该模拟可在渲染时进行,或作为预处理完成后去适配一组基函数供渲染时使用。
    \item \emph{物理(波动)光学}:一些反射模型是用详细的光模型推导出的,
          将其视作波并计算麦克斯韦方程组的解以求解光是怎么从已知性质的表面散射的。
          这些模型常常计算量很大,然而对于渲染应用而言它们通常并不比基于几何光学的模型精确多少。
    \item \emph{几何光学}:像模拟方法那样,如果表面的底层散射和几何性质已知,
          则有时能直接从这些描述中推出解析式的反射模型。几何光学让建模光与表面的交互
          更加容易处理,因为可以忽略像偏振那样的复杂波动效应。
\end{itemize}
本章末的“扩展阅读”一节给出了许多这样的反射模型索引。

在我们定义相关接口前,简要回顾下它们是怎么嵌入整个系统的。
如果用了\refvar{SamplerIntegrator}{},则会为每条光线
调用方法\refvar[Li]{SamplerIntegrator::Li}{()}的实现。
在找到与几何图元最近的相交处后,它调用与该图元关联的表面着色器。
表面着色器实现为\refvar{Material}{}子类的方法并负责决定表面上特定点的BSDF是什么;
它返回的BSDF对象持有BRDF和BTDF且已分配内存和初始化来表示该点的散射。
然后积分器基于该点的入射光照用BSDF计算该点的散射光
(使用\refvar{BDPTIntegrator}{}、\refvar{MLTIntegrator}{}或\refvar{SPPMIntegrator}{}而
不是\refvar{SamplerIntegrator}{}的过程大致相同)。

\subsection{基本术语}\label{sub:基本术语}
为了能比较不同反射模型的视觉表现,我们将介绍一些基本术语以描述来自表面的反射。

来自表面的反射可分为四大类:\keyindex{漫反射}{diffuse}{}、\keyindex{光泽镜面}{glossy specular}{}、
\keyindex{完美镜面}{perfect specular}{}和\keyindex{逆反射}{retro-reflective}{}(\reffig{8.1})。
大多数真实表面展现的反射都是这四种的混合。漫反射表面在所有方向均等地散射光。
尽管完美的漫反射表面是不可物理实现的,但几乎是漫反射表面的例子包括暗沉的黑板和哑光的油漆。
光泽镜面表面例如塑料或高光泽涂料优先在一组反射方向上散射光——
\begin{lstlisting}
`\initcode{BSDF Inline Functions}{=}\initnext{BSDFInlineFunctions}`
inline `\refvar{Float}{}` `\initvar{CosTheta}{}`(const `\refvar{Vector3f}{}` &w) { return w.z; }
inline `\refvar{Float}{}` `\initvar{Cos2Theta}{}`(const `\refvar{Vector3f}{}` &w) { return w.z * w.z; }
inline `\refvar{Float}{}` `\initvar{AbsCosTheta}{}`(const `\refvar{Vector3f}{}` &w) { return std::abs(w.z); }
\end{lstlisting}
\section{基本接口}\label{sec:基本接口}
我们将首先定义单个BRDF和BTDF函数的接口。
BRDF和BTDF共享共同的基类\refvar{BxDF}{}。
因为两者都有一样的接口,共享相同的基类减少了重复代码并
允许系统的一些部分和一般的\refvar{BxDF}{}配合而不用区分BRDF和BTDF。
\begin{lstlisting}
`\initcode{BxDF Declarations}{=}\initnext{BxDFDeclarations}`
class `\initvar{BxDF}{}` {
public:
    `\refcode{BxDF Interface}{}`
    `\refcode{BxDF Public Data}{}`
};
\end{lstlisting}

\refsec{BSDF}将要介绍的类\refvar{BSDF}{}持有一系列\refvar{BxDF}{}对象
来一起描述表面上一点的散射。尽管我们把\refvar{BxDF}{}的实现细节隐藏到
反射和透射材质的公共接口后,第\refchap{光传输I:表面反射}到\refchap{光传输III:双向方法}的
一些光传输算法还是需要区分这两个类型。因此,所有\refvar{BxDF}{}都
有成员\refvar{BxDF::type}{}持有来自\refvar{BxDFType}{}的标志。
对于每个\refvar{BxDF}{},该标志应至少有一个置为\refvar[BSDFREFLECTION]{BSDF\_REFLECTION}{}
或\refvar[BSDFTRANSMISSION]{BSDF\_TRANSMISSION}{},且恰有一个漫反射、光泽或镜面标志。
注意没有逆反射标志;这里的分类中逆反射被当作光泽反射。

\begin{lstlisting}
`\initcode{BSDF Declarations}{=}\initnext{BSDFDeclarations}`
enum `\initvar{BxDFType}{}` {
    `\initvar[BSDFREFLECTION]{BSDF\_REFLECTION}{}` = 1 << 0,
    `\initvar[BSDFTRANSMISSION]{BSDF\_TRANSMISSION}{}` = 1 << 1,
    `\initvar[BSDFDIFFUSE]{BSDF\_DIFFUSE}{}` = 1 << 2,
    `\initvar[BSDFGLOSSY]{BSDF\_GLOSSY}{}` = 1 << 3,
    `\initvar[BSDFSPECULAR]{BSDF\_SPECULAR}{}` = 1 << 4,
    `\initvar[BSDFALL]{BSDF\_ALL}{}` = BSDF_DIFFUSE | BSDF_GLOSSY | BSDF_SPECULAR |
                        BSDF_REFLECTION | BSDF_TRANSMISSION,
};
\end{lstlisting}

\begin{lstlisting}
`\initcode{BxDF Interface}{=}\initnext{BxDFInterface}`
`\refvar{BxDF}{}`(`\refvar{BxDFType}{}` type) : `\refvar[BxDF::type]{type}{}`(type) { }
\end{lstlisting}

\begin{lstlisting}
`\initcode{BxDF Public Data}{=}`
const `\refvar{BxDFType}{}` `\initvar[BxDF::type]{type}{}`;
\end{lstlisting}

实用方法\refvar{MatchesFlags}{()}确定\refvar{BxDF}{}是否匹配用户提供的类型标志:
\begin{lstlisting}
`\refcode{BxDF Interface}{+=}\lastnext{BxDFInterface}`
bool `\initvar{MatchesFlags}{}`(`\refvar{BxDFType}{}` t) const {
    return (`\refvar[BxDF::type]{type}{}` & t) == `\refvar[BxDF::type]{type}{}`;
}
\end{lstlisting}

\refvar{BxDF}{}提供的关键方法是\refvar{BxDF::f}{()}。
它为给定的方向对返回分布函数的值。该接口隐式假设了不同波长的光是解耦的——
某一波长的能量不会反射成不同波长。通过作出该假设,反射函数的效应可以直接用\refvar{Spectrum}{}表示。
支持该假设不成立的荧光材料则要求该方法返回一个$n\times n$矩阵以编码光谱样本间的能量转化
(其中$n$是\refvar{Spectrum}{}表示中的样本数量)。
\begin{lstlisting}
`\refcode{BxDF Interface}{+=}\lastnext{BxDFInterface}`
virtual `\refvar{Spectrum}{}` `\initvar[BxDF::f]{f}{}`(const `\refvar{Vector3f}{}` &wo, const `\refvar{Vector3f}{}` &wi) const = 0;
\end{lstlisting}

不是所有\refvar{BxDF}{}都能用方法\refvar[BxDF::f]{f}{()}求值。
例如,像镜子、玻璃或水那样的完美镜面物体只把来自单个入射方向的光朝单个出射方向散射。
这样的\refvar{BxDF}{}最好用$\delta$分布描述,即除了光散射的单个方向外都取零。
pbrt中这些\refvar{BxDF}{}需要特殊处理,所以我们也会提供方法\refvar[BxDF::Samplef]{BxDF::Sample\_f}{()}。
该方法既能用于处理由$\delta$分布描述的散射,
也能从散射光有多个方向的\refvar{BxDF}{}中随机采样方向;
第二种应用将在\refsec{采样反射函数}中讨论蒙特卡罗BSDF采样时解释。

\refvar[BxDF::Samplef]{BxDF::Sample\_f}{()}计算给定出射方向${\bm\omega}_{\mathrm{o}}$的
入射光方向${\bm\omega}_{\mathrm{i}}$并为这对方向返回\refvar{BxDF}{}的值。
对于$\delta$分布,\refvar{BxDF}{}有必要这样选择入射光方向,因为调用者
无法生成合适的方向${\bm\omega}_{\mathrm{i}}$
\footnote{反射函数中的$\delta$分布对于光传输算法有一些额外微妙的影响。
    \refsub{镜面反射与透射}和\refsub{被积函数中的delta分布}详细描述了该问题。}。
$\delta$分布的\refvar{BxDF}{}不需要参数{\ttfamily sample}和{\ttfamily pdf},
所以它们会在后面的\refsec{采样反射函数}解释,到时我们将为非镜面反射函数提供该方法的实现。
\begin{lstlisting}
`\refcode{BxDF Interface}{+=}\lastnext{BxDFInterface}`
virtual `\refvar{Spectrum}{}` `\initvar[BxDF::Samplef]{Sample\_f}{}`(const `\refvar{Vector3f}{}` &wo, `\refvar{Vector3f}{}` *wi,
    const `\refvar{Point2f}{}` &sample, `\refvar{Float}{}` *pdf,
    `\refvar{BxDFType}{}` *sampledType = nullptr) const;
\end{lstlisting}

\subsection{反射}\label{sub:反射}
将4D的BRDF或BTDF的表现聚合起来定义为一对方向上的函数,
并将其简化为单个方向上的2D函数甚至是描述其整体散射表现的常数值很有用。

\keyindex{半球定向反射率}{hemispherical-directional reflectance}{}是
一个2D函数,它给出了半球上常量照明于给定方向上的反射率,
或者等价地,因来自给定方向的光而在半球上的总反射率
\footnote{这两个量相等的事实源自反射函数的互易性。BTDF通常不互易;见\refsub{非对称散射}。}。
它定义为
\begin{align}
    \label{eq:8.1}
    \rho_{\mathrm{hd}}({\bm\omega}_{\mathrm{o}})=\int_{H^2({\bm n})}{f_{\mathrm{r}}({\bm p},{\bm \omega}_\mathrm{o},{\bm \omega}_\mathrm{i})|\cos{\theta_{\mathrm{i}}}|\mathrm{d}{\bm \omega}_\mathrm{i}}\, .
\end{align}

方法\refvar{BxDF::rho}{()}计算反射函数$\rho_{\mathrm{hd}}$.
一些\refvar{BxDF}{}能解析地计算该值,然而大部分用蒙特卡罗积分来计算其近似值。
对于那些\refvar{BxDF}{},参数{\ttfamily nSamples}和{\ttfamily samples}供
蒙特卡罗算法的实现使用;它们将在\refsub{应用:估计反射率}解释。
\begin{lstlisting}
`\refcode{BxDF Interface}{+=}\lastnext{BxDFInterface}`
virtual `\refvar{Spectrum}{}` `\initvar[BxDF::rho]{rho}{}`(const `\refvar{Vector3f}{}` &wo, int nSamples,
                     const `\refvar{Point2f}{}` *samples) const;
\end{lstlisting}

表面的\keyindex{半球半球反射率}{hemispherical-hemispherical reflectance}{}记为$\rho_{\mathrm{hh}}$,
该光谱值给出了当各方向入射光相同时表面反射的入射光比例。它是
\begin{align*}
    \rho_{\mathrm{hh}}=\frac{1}{\pi}\int_{H^2({\bm n})}\int_{H^2({\bm n})}f_{\mathrm{r}}({\bm p},{\bm \omega}_\mathrm{o},{\bm \omega}_\mathrm{i})|\cos{\theta_{\mathrm{o}}}\cos{\theta_{\mathrm{i}}}|\mathrm{d}{\bm \omega}_\mathrm{o}\mathrm{d}{\bm \omega}_\mathrm{i}\, .
\end{align*}

如果不提供方向${\bm\omega}_\mathrm{o}$,则方法\refvar[BxDF::rho2]{BxDF::rho}{()}计算$\rho_{\mathrm{hh}}$.
剩下的参数又是在需要时用于计算$\rho_{\mathrm{hh}}$值的蒙特卡罗估计。
\begin{lstlisting}
`\refcode{BxDF Interface}{+=}\lastcode{BxDFInterface}`
virtual `\refvar{Spectrum}{}` `\initvar[BxDF::rho2]{rho}{}`(int nSamples, const `\refvar{Point2f}{}` *samples1,
                     const `\refvar{Point2f}{}` *samples2) const;
\end{lstlisting}

\subsection{BxDF缩放适配器}\label{sub:BxDF缩放适配器}
取一个给定的\refvar{BxDF}{}并用一个\refvar{Spectrum}{}值
缩放它的作用也很有用。\refvar{ScaledBxDF}{}
封装器持有一个\refvar{BxDF}{*}和\refvar{Spectrum}{}并实现其功能。
该类由\refvar{MixMaterial}{}(定义于\refsub{混合材料})使用,
它基于另两种材料的加权和创建\refvar{BSDF}{}。
\begin{lstlisting}
`\refcode{BxDF Declarations}{+=}\lastnext{BxDFDeclarations}`
class `\initvar{ScaledBxDF}{}` : public `\refvar{BxDF}{}` {
public:
    `\refcode{ScaledBxDF Public Methods}{}`
private:
    `\refvar{BxDF}{}` *`\initvar[ScaledBxDF::bxdf]{bxdf}{}`;
    `\refvar{Spectrum}{}` `\initvar[ScaledBxDF::scale]{scale}{}`;
};
\end{lstlisting}
\begin{lstlisting}
`\initcode{ScaledBxDF Public Methods}{=}`
`\refvar{ScaledBxDF}{}`(`\refvar{BxDF}{}` *bxdf, const `\refvar{Spectrum}{}` &scale)
    : `\refvar{BxDF}{}`(`\refvar{BxDFType}{}`(bxdf->`\refvar[BxDF::type]{type}{}`)), `\refvar[ScaledBxDF::bxdf]{bxdf}{}`(bxdf), `\refvar[ScaledBxDF::scale]{scale}{}`(scale) {
}
\end{lstlisting}

\refvar{ScaledBxDF}{}的方法实现很简单;我们这里只介绍\refvar[ScaledBxDF::f]{f}{()}。
\begin{lstlisting}
`\initcode{BxDF Method Definitions}{=}\initnext{BxDFMethodDefinitions}`
`\refvar{Spectrum}{}` `\refvar{ScaledBxDF}{}`::`\initvar[ScaledBxDF::f]{f}{}`(const `\refvar{Vector3f}{}` &wo, const `\refvar{Vector3f}{}` &wi) const {
    return `\refvar[ScaledBxDF::scale]{scale}{}` * `\refvar[ScaledBxDF::bxdf]{bxdf}{}`->`\refvar[BxDF::f]{f}{}`(wo, wi);
}
\end{lstlisting}

\section{镜面反射与透射}\label{sec:镜面反射与透射}

绝对光滑表面上光的特性较容易用物理和几何光学模型分析刻画。
这些表面展现出入射光的完美镜像反射和透射;
对于给定的方向${\bm\omega}_{\mathrm{i}}$,
所有光都散射到单个出射方向${\bm\omega}_{\mathrm{o}}$.
对于镜面反射,该出射方向和法线所成角与入射方向相同:
\begin{align*}
    \theta_{\mathrm{i}}=\theta_{\mathrm{o}}\, ,
\end{align*}
且其中$\varphi_{\mathrm{o}}=\varphi_{\mathrm{i}}+\pi$.
对于透射,我们也有$\varphi_{\mathrm{o}}=\varphi_{\mathrm{i}}+\pi$,
且出射方向$\theta_{\mathrm{t}}$由斯涅尔定律给出,
它将折射方向与曲面法线$\bm n$的夹角$\theta_{\mathrm{t}}$与
入射光线与曲面法线$\bm n$的夹角$\theta_{\mathrm{i}}$联系起来
(本章末的习题之一是用光学的费马原理推导斯涅尔定律)。
斯涅尔定律基于入射光线所在介质的\keyindex{折射率}{index of refraction}{}和
要进入的介质的折射率。折射率描述了光在特定介质中相比在真空中传播要慢多少。
我们用希腊字母$\eta$表示折射率,读作“eta”。斯涅尔定律是
\begin{align}
    \label{eq:8.2}
    \eta_{\mathrm{i}}\sin\theta_{\mathrm{i}}=\eta_{\mathrm{t}}\sin\theta_{\mathrm{t}}\, .
\end{align}

通常,折射率随波长变化。因此,在两种不同介质界面上入射光通常散射到多个方向,
该效应称为\keyindex{色散}{dispersion}{}。
当入射白光被棱镜分出光谱成分时可以观察到该效应。
图形学中的通行做法是忽略该波长依赖性,
因为该效应通常对视觉准确性并不关键且忽略它能极大简化光传输计算。
可选地,可以在有色散物体的环境中追踪多束光路(例如一系列离散波长)。
第\refchap{光传输I:表面反射}末的“扩展阅读”一节有关于该话题的更多信息指引。
\begin{figure}[htbp]
    \centering
    \subfloat[镜面反射]{\includegraphics[width=0.75\linewidth]{chap08/dragon-specular-reflect.png}\label{fig:8.4.1}}\\
    \subfloat[镜面透射]{\includegraphics[width=0.75\linewidth]{chap08/dragon-specular-transmit.png}\label{fig:8.4.2}}
    \caption{用(1)完美镜面反射和(2)完美镜面折射渲染的龙模型。图像(2)排除了
        内外反射的影响;导致的能量损失产生了显眼的暗区(感谢Christian Schüller提供模型)。}
    \label{fig:8.4}
\end{figure}

\reffig{8.4}展示了完美镜面反射和透射的效果。

\subsection{菲涅尔反射率}\label{sub:菲涅尔反射率}
除了反射和透射方向,还有必要计算反射或透射的入射光占比。
为了物理上准确反射或折射,该项依赖于方向,而不能用每个表面的缩放常数表征。
\keyindex{菲涅耳方程}{Fresnel equations}{}
\sidenote{译者注:得名于法国物理学家奥古斯丁·菲涅耳(Augustin-Jean Fresnel)。}描述了
表面上反射光的量;它们是麦克斯韦方程在光滑表面上的解。

给定折射率和入射光与曲面法线所成角度,菲涅耳方程
指定了材料对两种不同偏振状态的入射照明相应的反射率。
因为偏振的视觉效果在大多数环境下是受限的,
所以在pbrt中我们将作出光是无偏振的常用假设;
即光波是随机朝向的。有了该简化假设,菲涅尔反射率就是
平行和垂直偏振项的均方。

此刻有必要指出几个重要材料类别的差异:
\begin{enumerate}
    \item 第一类是\keyindex{介电质}{dielectric}{},
          是不会导电的材料。它们有实数值的折射率(通常在范围1-3内)且
          透射\footnote{注意介电质可能充满能吸收大部分或所有透射光的粒子(例如石油)。
              像水那样的介电质也能通过添加离子使之导电而变成电解质溶液。
              这两方面都和材料本身划分为介电质或导体无关。}一部分入射照明。
          介电质的例子有玻璃、矿油、水和空气。
    \item 第二类组成是\keyindex{导体}{conductor}{}例如金属。
          价电子可以自由地在原子晶格中移动,允许电流从一个地方流到另一处。
          当导体受到电磁辐射例如可见光时,这一基本的原子属性就会转化为完全不同的特性:
          该材料是不透明的并反射回大部分照明。一部分光也透射进导体内部并被迅速吸收:
          总吸收通常发生在材料表面0.1$\mu\mathrm{m}$内,因此只有极薄的金属膜才能透射足够的光量。
          我们在pbrt中忽略该效应而只建模导体的反射部分。与介电质相反,
          导体有复数值的折射率$\bar{\eta}=\eta+\mathrm{i}k$.
    \item \keyindex{半导体}{semiconductor}{}例如硅或锗是本书中我们不予考虑的第三类。
\end{enumerate}

导体和介电质都由同一组菲涅尔方程表征。
尽管如此,我们更喜欢为介电质创建特殊的求值函数,
这样当折射率保证为实数值时,这些方程会取特别简单的形式。

\begin{table}[htbp]
    \centering
    \begin{tabular}{ll}
        \toprule
        \textbf{介质}  & \textbf{折射率}$\eta$ \\
        \midrule
        真空           & 1.0                   \\
        海平面上的空气 & 1.00029               \\
        冰             & 1.31                  \\
        水(20℃)      & 1.333                 \\
        熔融石英       & 1.46                  \\
        玻璃           & 1.5-1.6               \\
        蓝宝石         & 1.77                  \\
        钻石           & 2.42                  \\
        \bottomrule
    \end{tabular}
    \caption{各种物体的折射率,给出了光在真空中的速度与
        光在介质中的速度的比值。它们通常是与波长相关的量;
        这些值是在可见波长上的均值。}
    \label{tab:8.1}
\end{table}

为了计算两种介电质界面处的菲涅尔反射率,我们
需要知道这两种介质的折射率。\reftab{8.1}有许多介电质的折射率。
介电质的菲涅尔反射率公式是
\begin{align*}
    r_{\parallel} & =\frac{\eta_{\mathrm{t}}\cos\theta_{\mathrm{i}}-\eta_{\mathrm{i}}\cos\theta_{\mathrm{t}}}{\eta_{\mathrm{t}}\cos\theta_{\mathrm{i}}+\eta_{\mathrm{i}}\cos\theta_{\mathrm{t}}}\, , \\
    r_{\perp}     & =\frac{\eta_{\mathrm{i}}\cos\theta_{\mathrm{i}}-\eta_{\mathrm{t}}\cos\theta_{\mathrm{t}}}{\eta_{\mathrm{i}}\cos\theta_{\mathrm{i}}+\eta_{\mathrm{t}}\cos\theta_{\mathrm{t}}}\, ,
\end{align*}
其中$r_{\parallel}$是平行偏振光的菲涅尔反射率,
$r_{\perp}$是垂直偏振光的反射率,$eta_{\mathrm{i}}$和$\eta_{\mathrm{t}}$是
入射和透射介质的折射率,$\bm\omega_{\mathrm{i}}$和$\bm\omega_{\mathrm{t}}$是
入射和透射方向。$\bm\omega_{\mathrm{t}}$由斯涅尔定律算出(见\refsub{镜面透射})。

余弦项应该大于或等于零;出于计算这些值的目的,
当计算$\cos\theta_{\mathrm{i}}$和$\cos\theta_{\mathrm{t}}$时,
几何法线应该分别翻转到和$\bm\omega_{\mathrm{i}}$或$\bm\omega_{\mathrm{t}}$同侧。

对于非偏振光,菲涅尔反射率是
\begin{align*}
    F_{\mathrm{r}}=\frac{1}{2}(r_{\parallel}^2+r_{\perp}^2)\, .
\end{align*}

因为能量守恒,介电质传输的能量为$1-F_{\mathrm{r}}$.

函数\refvar{FrDielectric}{()}为介电质材料和非偏振光计算菲涅尔反射率公式。
量$\cos\theta_{\mathrm{i}}$作为参数{\ttfamily cosThetaI}传入。
\begin{lstlisting}
`\initcode{BxDF Utility Functions}{=}`
`\refvar{Float}{}` `\initvar{FrDielectric}{}`(`\refvar{Float}{}` cosThetaI, `\refvar{Float}{}` etaI, `\refvar{Float}{}` etaT) {
    cosThetaI = `\refvar{Clamp}{}`(cosThetaI, -1, 1);
    `\refcode{Potentially swap indices of refraction}{}`
    `\refcode{Compute cosThetaT using Snell's law}{}`
    `\refvar{Float}{}` Rparl = ((etaT * cosThetaI) - (etaI * cosThetaT)) /
                  ((etaT * cosThetaI) + (etaI * cosThetaT));
    `\refvar{Float}{}` Rperp = ((etaI * cosThetaI) - (etaT * cosThetaT)) /
                  ((etaI * cosThetaI) + (etaT * cosThetaT));
    return (Rparl * Rparl + Rperp * Rperp) / 2;
}
\end{lstlisting}

为了求得折射角的余弦{\ttfamily cosThetaT},
首先需要确定入射方向是在介质的外面还是里面,这样才能恰当解释两个折射率。

入射角余弦的符号表明了入射光在介质的哪一侧(\reffig{8.5})。
如果余弦在0到1间,则光线在外侧,如果在-1到0间,则光线在内侧。
调整参数{\ttfamily etaI}和{\ttfamily etaT}使得{\ttfamily etaI}
是入射介质的折射率,这样保证了{\ttfamily cosThetaI}非负。

\begin{figure}[htbp]
    \centering
    \includegraphics[width=0.7\linewidth]{chap08/BSDFanglegivesinout.eps}
    \caption{方向$\bm\omega$和几何曲面法线间夹角$\theta$的余弦
        表明方向是指向表面外侧(和法线在同一半球)还是表面内侧。
        在标准反射坐标系中,该测试只要求检查方向向量的$z$分量。
        这里,$\bm\omega$在上半球,取正值余弦,而${\bm\omega}'$在下半球。}
    \label{fig:8.5}
\end{figure}

\begin{lstlisting}
`\initcode{Potentially swap indices of refraction}{=}`
bool entering = cosThetaI > 0.f;
if (!entering) {
    std::swap(etaI, etaT);
    cosThetaI = std::abs(cosThetaI);
}
\end{lstlisting}

一旦确定折射率,我们就能用斯涅尔定律(\refeq{8.2})计算
透射方向和曲面法线夹角的正弦$\sin\theta_{\mathrm{t}}$.
最后,用恒等式$\sin^2\theta+\cos^2\theta=1$求得该角的余弦。
\begin{lstlisting}
`\initcode{Compute cosThetaT using Snell's law}{=}`
`\refvar{Float}{}` sinThetaI = std::sqrt(std::max((`\refvar{Float}{}`)0,
                                     1 - cosThetaI * cosThetaI));
`\refvar{Float}{}` sinThetaT = etaI / etaT * sinThetaI;
`\refcode{Handle total internal reflection}{}`
`\refvar{Float}{}` cosThetaT = std::sqrt(std::max((`\refvar{Float}{}`)0,
                                     1 - sinThetaT * sinThetaT));
\end{lstlisting}

当从一种介质传播到另一种折射率更低的介质时,入射角接近掠角的光不能进入另一介质。
发生该现象的最大角称为\keyindex{临界角}{critical angle}{};
当$\theta_{\mathrm{i}}$大于临界角时,发生\keyindex{全内反射}{total internal reflection}{reflection反射},
所有光都被反射。这里通过$\sin\theta_{\mathrm{t}}$大于1检测到该情况;
此时不需要菲涅尔方程。
\begin{lstlisting}
`\initcode{Handle total internal reflection}{=}`
if (sinThetaT >= 1)
    return 1;
\end{lstlisting}

我们现在聚焦一般情况下的复数折射率$\bar{\eta}=\eta+\mathrm{i}k$,
其中一些入射光被材料部分吸收并变为热量。
除了实部,一般菲涅尔公式现在也依赖于虚部$k$,
称为\keyindex{吸收系数}{absorption coefficient}{}。

\reffig{8.6}展示了金的折射率和吸收系数图示。两者都是与波长相关的量。
pbrt发行版中目录{\ttfamily scenes/spds/metals}下有各种金属的$\eta$与$k$与波长相关的数据。
下章的\reffig{9.4}展示了用金属材料渲染的模型。
\begin{figure}[htbp]
    \centering
    \includegraphics[width=0.7\linewidth]{chap08/au-k-eta.eps}
    \caption{金的吸收系数和折射率。该图展示了金的吸收系数$k$(实线)
        和折射率$\eta$(虚线)随光谱变化的值,横轴是波长,单位纳米。}
    \label{fig:8.6}
\end{figure}

导体和介电质界面处的菲涅尔反射率是
\begin{align}
    \label{eq:8.3}
    r_{\perp}     & =\frac{a^2+b^2-2a\cos\theta+\cos^2\theta}{a^2+b^2+2a\cos\theta+\cos^2\theta}\, ,\nonumber                                                     \\
    r_{\parallel} & =r_{\perp}\frac{(a^2+b^2)\cos^2\theta-2a\cos\theta\sin^2\theta+\sin^4\theta}{(a^2+b^2)\cos^2\theta+2a\cos\theta\sin^2\theta+\sin^4\theta}\, ,
\end{align}
其中
\begin{align*}
    a^2+b^2=\sqrt{(\eta^2-k^2-\sin^2\theta)^2+4\eta^2k^2}\, ,
\end{align*}
且$\displaystyle\eta+\mathrm{i}k=\frac{\bar{\eta_\mathrm{t}}}{\bar{\eta_\mathrm{i}}}$是
用复数除法算出的相对折射率。然而,通常$\bar{\eta_\mathrm{i}}$是介电质的
所以可以替代使用普通的实数除法。

该计算由函数\refvar{FrConductor}{()}实现
\footnote{注意这稍微用词不当,因为函数在技术上包含了介电质$k=0$的情况。
    也就是说,我们选这个名称是为了表明该函数应只用于处理导体,
    因为它比求\refvar{FrDielectric}{()}的开销更大。};
该实现直接对应\refeq{8.3}所以这里不介绍了。
\begin{lstlisting}
`\initcode{Reflection Declarations}{=}`
`\refvar{Spectrum}{}` `\initvar{FrConductor}{}`(`\refvar{Float}{}` cosThetaI, const `\refvar{Spectrum}{}` &etaI,
    const `\refvar{Spectrum}{}` &etaT, const `\refvar{Spectrum}{}` &k);
\end{lstlisting}

为了方便,我们定义抽象类\refvar{Fresnel}{}以提供接口计算菲涅尔反射系数。
用该接口的实现帮助简化后续可能需要支持两种形式的BRDF实现。
\begin{lstlisting}
`\refcode{BxDF Declarations}{+=}\lastnext{BxDFDeclarations}`
class `\initvar{Fresnel}{}` {
public:
    `\refcode{Fresnel Interface}{}`
};
\end{lstlisting}

\refvar{Fresnel}{}接口提供的唯一函数是\refvar{Fresnel::Evaluate}{()}。
给定入射方向和曲面法线夹角的余弦,它返回表面反射的光量。
\begin{lstlisting}
`\initcode{Fresnel Interface}{=}`
virtual `\refvar{Spectrum}{}` `\initvar[Fresnel::Evaluate]{Evaluate}{}`(`\refvar{Float}{}` cosI) const = 0;
\end{lstlisting}

\subsubsection*{菲涅尔导体}
\refvar{FresnelConductor}{}为导体实现该接口。
\begin{lstlisting}
`\refcode{BxDF Declarations}{+=}\lastnext{BxDFDeclarations}`
class `\initvar{FresnelConductor}{}` : public `\refvar{Fresnel}{}` {
public:
    `\refcode{FresnelConductor Public Methods}{}`
private:
    `\refvar{Spectrum}{}` `\initvar[FresnelConductor::etaI]{etaI}{}`, `\initvar[FresnelConductor::etaT]{etaT}{}`, `\initvar[FresnelConductor::k]{k}{}`;
};
\end{lstlisting}

其构造函数存有给定的折射率$\eta$和吸收系数$k$.
\begin{lstlisting}
`\initcode{FresnelConductor Public Methods}{=}`
`\refvar{FresnelConductor}{}`(const `\refvar{Spectrum}{}` &etaI, const `\refvar{Spectrum}{}` &etaT,
    const `\refvar{Spectrum}{}` &k) : `\refvar[FresnelConductor::etaI]{etaI}{}`(etaI), `\refvar[FresnelConductor::etaT]{etaT}{}`(), `\refvar[FresnelConductor::k]{k}{}`(k) { }
\end{lstlisting}

\refvar{FresnelConductor}{}的求值例程也很简单;它只需调用之前定义的函数
\refvar{FrConductor}{()}。注意在调用\refvar{FrConductor}{()}
前{\ttfamily cosThetaI}要取绝对值,因为\refvar{FrConductor}{()}
要求该余弦是在法线和${\bm\omega}_{\mathrm{i}}$在表面的同一侧时测出的,
或者等价地,应该用$\cos\theta_{\mathrm{i}}$的绝对值。
\begin{lstlisting}
`\refcode{BxDF Method Definitions}{+=}\lastnext{BxDFMethodDefinitions}`
`\refvar{Spectrum}{}` `\refvar{FresnelConductor}{}`::`\initvar[FresnelConductor::Evaluate]{Evaluate}{}`(`\refvar{Float}{}` cosThetaI) const {
    return `\refvar{FrConductor}{}`(std::abs(cosThetaI), `\refvar[FresnelConductor::etaI]{etaI}{}`, `\refvar[FresnelConductor::etaT]{etaT}{}`, `\refvar[FresnelConductor::k]{k}{}`);
}
\end{lstlisting}

\subsubsection*{菲涅尔介电质}
\refvar{FresnelDielectric}{}类似地为介电质材料实现了\refvar{Fresnel}{}接口。
\begin{lstlisting}
`\refcode{BxDF Declarations}{+=}\lastnext{BxDFDeclarations}`
class `\initvar{FresnelDielectric}{}` : public `\refvar{Fresnel}{}` {
public:
    `\refcode{FresnelDielectric Public Methods}{}`
private:
    `\refvar{Float}{}` `\initvar[FresnelDielectric::etaI]{etaI}{}`, `\initvar[FresnelDielectric::etaT]{etaT}{}`;
};
\end{lstlisting}

其构造函数存有表面内外侧的折射率。
\begin{lstlisting}
`\initcode{FresnelDielectric Public Methods}{=}`
`\refvar{FresnelDielectric}{}`(`\refvar{Float}{}` etaI, `\refvar{Float}{}` etaT) : `\refvar[FresnelDielectric::etaI]{etaI}{}`(etaI), `\refvar[FresnelDielectric::etaT]{etaT}{}`(etaT) { }
\end{lstlisting}

\refvar{FresnelDielectric}{}的求值例程类似地调用\refvar{FrDielectric}{()}。
\begin{lstlisting}
`\refcode{BxDF Method Definitions}{+=}\lastnext{BxDFMethodDefinitions}`
`\refvar{Spectrum}{}` `\refvar{FresnelDielectric}{}`::`\initvar[FresnelDielectric::Evaluate]{Evaluate}{}`(`\refvar{Float}{}` cosThetaI) const {
    return `\refvar{FrDielectric}{}`(cosThetaI, `\refvar[FresnelDielectric::etaI]{etaI}{}`, `\refvar[FresnelDielectric::etaT]{etaT}{}`);
}
\end{lstlisting}

\subsubsection*{特殊菲涅尔接口}
\refvar{Fresnel}{}接口的实现\refvar{FresnelNoOp}{}对所有入射方向返回100\%反射率。
尽管这在物理上不可实现,但这是可用的方便能力。
\begin{lstlisting}
`\refcode{BxDF Declarations}{+=}\lastnext{BxDFDeclarations}`
class `\initvar{FresnelNoOp}{}` : public `\refvar{Fresnel}{}` {
public:
    `\refvar{Spectrum}{}` `\initvar[FresnelNoOp::Evaluate]{Evaluate}{}`(`\refvar{Float}{}`) const { return `\refvar{Spectrum}{}`(1.); }
};
\end{lstlisting}

\subsection{镜面反射}\label{sub:镜面反射}
我们现在可以实现类\refvar{SpecularReflection}{},
它用菲涅尔接口计算反射光的占比,描述了物理可实现的镜面反射。
首先,我们将推导描述镜面反射的BRDF。
既然菲涅尔方程给出了反射光的比例$F_{\mathrm{r}}({\bm\omega})$,
那么我们需要这样的BRDF
\begin{align*}
    L_{\mathrm{o}}({\bm\omega}_{\mathrm{o}})=\int{f_{\mathrm{r}}({\bm\omega}_{\mathrm{o}},{\bm\omega}_{\mathrm{i}})L_{\mathrm{i}}({\bm\omega}_{\mathrm{i}})|\cos\theta_{\mathrm{i}}|\mathrm{d}{\bm\omega}_{\mathrm{i}}}=F_{\mathrm{r}}({\bm\omega}_{\mathrm{r}})L_{\mathrm{i}}({\bm\omega}_{\mathrm{r}})\, ,
\end{align*}
其中${\bm\omega}_{\mathrm{r}}=R({\bm\omega}_{\mathrm{o}},{\bm n})$是由${\bm\omega}_{\mathrm{o}}$关于
曲面法线$\bm n$反射的镜面反射向量(回想对于镜面反射有$\theta_{\mathrm{r}}=\theta_{\mathrm{o}}$,
因此$F_{\mathrm{r}}({\bm\omega}_{\mathrm{o}})=F_{\mathrm{r}}({\bm\omega}_{\mathrm{r}})$)。

此类BRDF可用狄拉克$\delta$分布构造。
回顾\refsec{采样理论}中$\delta$分布有个好用的性质
\begin{align}\label{eq:8.4}
    \int{f(x)\delta(x-x_0)\mathrm{d}x}=f(x_0)\, .
\end{align}
然而相比于标准函数,$\delta$分布需要特殊处理。
特别地,对有$\delta$分布的积分求数值积分必须显式考虑$\delta$分布。
考虑\refeq{8.4}中的积分:如果我们尝试用梯形法则或
其他一些数值积分技术计算它,则按$\delta$分布的定义,
在任意取值点$x_i$处$\delta(x_i)$为非零值的概率都为零。
确切地说,我们必须允许$\delta$分布自己确定取值点。
我们将在来自特殊\refvar{BxDF}{}的光传输积分以及
第\refchap{光源}的一些光源中遇到$\delta$分布。

直觉上,我们想让镜面反射BRDF在完美反射方向以外任何地方都取零,
这暗示了要用$\delta$分布。首先可能想到的是用$\delta$函数
把入射方向限制到镜面反射方向${\bm\omega}_{\mathrm{r}}$.
这样得到BRDF
\begin{align*}
    f_{\mathrm{r}}({\bm\omega}_{\mathrm{o}},{\bm\omega}_{\mathrm{i}})=\delta_{\mathrm{r}}({\bm\omega}_{\mathrm{i}}-{\bm\omega}_{\mathrm{r}})F_{\mathrm{r}}({\bm\omega}_{\mathrm{i}})\, .
\end{align*}

尽管这看起来很诱人,但把它代入散射方程\refeq{5.9}就暴露了问题:
\begin{align*}
    L_{\mathrm{o}}({\bm\omega}_{\mathrm{o}}) & =\int{\delta_{\mathrm{r}}({\bm\omega}_{\mathrm{i}}-{\bm\omega}_{\mathrm{r}})F_{\mathrm{r}}({\bm\omega}_{\mathrm{i}})}L_{\mathrm{i}}({\bm\omega}_{\mathrm{i}})|\cos\theta_{\mathrm{i}}|\mathrm{d}{\bm\omega}_{\mathrm{i}} \\
                                             & =F_{\mathrm{r}}({\bm\omega}_{\mathrm{r}})L_{\mathrm{i}}({\bm\omega}_{\mathrm{r}})|\cos\theta_{\mathrm{r}}|\, .
\end{align*}
这是错的,因为它含有额外因子$\cos\theta_{\mathrm{r}}$.
然而,我们可以把该因子分解以求得完美镜面反射正确的BRDF:
\begin{align*}
    f_\mathrm{r}({\bm p},{\bm \omega}_\mathrm{o},{\bm \omega}_\mathrm{i})=F_{\mathrm{r}}({\bm\omega}_{\mathrm{r}})\frac{\delta_{\mathrm{r}}({\bm\omega}_{\mathrm{i}}-{\bm\omega}_{\mathrm{r}})}{|\cos\theta_{\mathrm{r}|}}\, ,
\end{align*}
\begin{lstlisting}
`\refcode{BxDF Declarations}{+=}\lastnext{BxDFDeclarations}`
class `\initvar{SpecularReflection}{}` : public `\refvar{BxDF}{}` {
public:
    `\refcode{SpecularReflection Public Methods}{}`
private:
    `\refcode{SpecularReflection Private Data}{}`
};
\end{lstlisting}

\refvar{SpecularReflection}{}的构造函数接收用于缩放反射颜色的\refvar{Spectrum}{}和
描述介电质或导体菲涅尔性质的\refvar{Fresnel}{}对象指针。
\begin{lstlisting}
`\initcode{SpecularReflection Public Methods}{=}\initnext{SpecularReflectionPublicMethods}`
`\refvar{SpecularReflection}{}`(const `\refvar{Spectrum}{}` &R, `\refvar{Fresnel}{}` *fresnel) 
    : `\refvar{BxDF}{}`(`\refvar{BxDFType}{}`(`\refvar[BSDFREFLECTION]{BSDF\_REFLECTION}{}` | `\refvar[BSDFSPECULAR]{BSDF\_SPECULAR}{}`)), `\refvar[SpecularReflection::R]{R}{}`(R),
      `\refvar[SpecularReflection::fresnel]{fresnel}{}`(fresnel) { }
\end{lstlisting}
\begin{lstlisting}
`\initcode{SpecularReflection Private Data}{=}`
const `\refvar{Spectrum}{}` `\initvar[SpecularReflection::R]{R}{}`;
const `\refvar{Fresnel}{}` *`\initvar[SpecularReflection::fresnel]{fresnel}{}`;
\end{lstlisting}

剩下的实现就简单了。没有散射从\refvar{SpecularReflection::f}{()}返回,
因为对于任意一对方向,$\delta$函数不返回散射
\footnote{如果调用者碰巧传入一个向量及其完美镜像方向,该函数仍然返回零。
    尽管这些反射函数的接口有点奇怪,我们最终仍然能得到正确结果,
    因为带有$\delta$分布奇点的反射函数将得到光传输例程的特殊处理(见第\refchap{光传输I:表面反射})。}。
\begin{lstlisting}
`\refcode{SpecularReflection Public Methods}{+=}\lastnext{SpecularReflectionPublicMethods}`
`\refvar{Spectrum}{}` `\initvar[SpecularReflection::f]{f}{}`(const `\refvar{Vector3f}{}` &wo, const `\refvar{Vector3f}{}` &wi) const { 
    return `\refvar{Spectrum}{}`(0.f); 
}
\end{lstlisting}
然而,我们确实实现了方法\refvar[SpecularReflection::Samplef]{Sample\_f}{()},
它根据$\delta$分布选择合适的方向。
它把输出变量{\ttfamily wi}设为提供的方向{\ttfamily wo}关于
曲面法线的反射。值{\ttfamily *pdf}设为一;
\refsub{镜面反射与透射}讨论了关于该数值一所表示的数学量的一些细节。
\begin{lstlisting}
`\refcode{BxDF Method Definitions}{+=}\lastnext{BxDFMethodDefinitions}`
`\refvar{Spectrum}{}` `\refvar{SpecularReflection}{}`::`\initvar[SpecularReflection::Samplef]{Sample\_f}{}`(const `\refvar{Vector3f}{}` &wo,
        `\refvar{Vector3f}{}` *wi, const `\refvar{Point2f}{}` &sample, `\refvar{Float}{}` *pdf,
        `\refvar{BxDFType}{}` *sampledType) const {
    `\refcode{Compute perfect specular reflection direction}{}`
    *pdf = 1;
    return `\refvar[SpecularReflection::fresnel]{fresnel}{}`->`\refvar[Fresnel::Evaluate]{Evaluate}{}`(`\refvar{CosTheta}{}`(*wi)) * `\refvar[SpecularReflection::R]{R}{}` / `\refvar{AbsCosTheta}{}`(*wi);
}
\end{lstlisting}

期望的入射方向是${\bm\omega}_{\mathrm{o}}$关于曲面法线的反射$R({\bm\omega}_{\mathrm{o}},{\bm n})$.
用向量几何学可以非常简单地计算该方向。
首先,注意到入射方向、反射方向和曲面法线均在同一平面内。
我们可以把平面内的向量$\bm\omega$分解为两个分量之和:
一个平行于$\bm n$,我们记作${\bm\omega}_{\parallel}$,另一个垂直即${\bm\omega}_{\perp}$.

这些向量很容易计算:如果$\bm n$和$\bm\omega$规范化了,
则${\bm\omega}_{\parallel}$是$(\cos\theta){\bm n}=({\bm n}\cdot{\bm\omega}){\bm n}$(\reffig{8.7})。
因为${\bm\omega}_{\parallel}+{\bm\omega}_{\perp}={\bm\omega}$,
\begin{align*}
    {\bm\omega}_{\perp}={\bm\omega}-{\bm\omega}_{\parallel}={\bm\omega}-({\bm n}\cdot{\bm\omega}){\bm n}\, .
\end{align*}
\begin{figure}[htbp]
    \centering
    \includegraphics[width=0.4\linewidth]{Pictures/chap08/Parallelprojectionomeganormal.eps}
    \caption{向量$\bm\omega$在法线$\bm n$上的平行投影由${\bm\omega}_{\parallel}=(\cos\theta){\bm n}=({\bm n}\cdot{\bm\omega}){\bm n}$给出。
    垂直分量由${\bm\omega}_{\perp}=(\sin\theta){\bm n}$给出但用${\bm\omega}_{\perp}={\bm\omega}-{\bm\omega}_{\parallel}$计算更简单。}
    \label{fig:8.7}
\end{figure}

\reffig{8.8}展示了计算反射方向${\bm\omega}_{\mathrm{r}}$的设置。
我们可以看到两个向量都有相同的${\bm\omega}_{\parallel}$分量,
且${\bm\omega}_{\mathrm{r}\perp}$的值是${\bm\omega}_{\mathrm{o}\perp}$取反。
因此,我们有
\begin{align}
    \label{eq:8.5}
    {\bm\omega}_{\mathrm{r}}={\bm\omega}_{\mathrm{r}\perp}+{\bm\omega}_{\mathrm{r}\parallel} & =-{\bm\omega}_{\mathrm{o}\perp}+{\bm\omega}_{\mathrm{o}\parallel}\nonumber                                                        \\
                                                                                             & =-({\bm\omega}_{\mathrm{o}}-({\bm n}\cdot{\bm\omega}_{\mathrm{o}}){\bm n})+({\bm n}\cdot{\bm\omega}_{\mathrm{o}}){\bm n}\nonumber \\
                                                                                             & =-{\bm\omega}_{\mathrm{o}}+2({\bm n}\cdot{\bm\omega}_{\mathrm{o}}){\bm n}\, .
\end{align}
\begin{figure}[htbp]
    \centering
    \includegraphics[width=0.5\linewidth]{Pictures/chap08/Perfectreflectioncomponents.eps}
    \caption{因为角$\theta_{\mathrm{o}}$和$\theta_{\mathrm{r}}$相等,
    所以完美反射方向的平行分量${\bm\omega}_{\mathrm{r}\parallel}$和
    入射方向的相同:${\bm\omega}_{\mathrm{r}\parallel}={\bm\omega}_{\mathrm{o}\parallel}$.
    其垂直分量就是入射方向垂直分量取反。}
    \label{fig:8.8}
\end{figure}

函数\refvar{Reflect}{()}实现了该计算。
\begin{lstlisting}
`\refcode{BSDF Inline Functions}{+=}\lastnext{BSDFInlineFunctions}`
inline `\refvar{Vector3f}{}` `\initvar{Reflect}{}`(const `\refvar{Vector3f}{}` &wo, const `\refvar{Vector3f}{}` &n) {
    return -wo + 2 * `\refvar{Dot}{}`(wo, n) * n;
}
\end{lstlisting}

在BRDF坐标系中,${\bm n}=(0,0,1)$,该表达式可极大简化。
\begin{lstlisting}
`\initcode{Compute perfect specular reflection direction}{=}`
*wi = `\refvar{Vector3f}{}`(-wo.x, -wo.y, wo.z);
\end{lstlisting}

\subsection{镜面透射}\label{sub:镜面透射}
\begin{lstlisting}
`\refcode{BSDF Inline Functions}{+=}\lastnext{BSDFInlineFunctions}`
inline bool `\initvar{Refract}{}`(const `\refvar{Vector3f}{}` &wi, const `\refvar{Normal3f}{}` &n, `\refvar{Float}{}` eta,
        `\refvar{Vector3f}{}` *wt) {
    `\refcode{Compute cos $\theta_t$ using Snell's law}{}`
    *wt = eta * -wi + (eta * cosThetaI - cosThetaT) * `\refvar{Vector3f}{}`(n);
    return true;
}
\end{lstlisting}

\input{content/chap09.tex}

\input{content/chap10.tex}

\input{content/chap11.tex}

\input{content/chap12.tex}

\part{光传输算法}
\input{content/chap13.tex}

\input{content/chap14.tex}

\input{content/chap15.tex}

\chapter{光传输III:双向方法}\label{chap:光传输III:双向方法}

\input{content/chap1601.tex}

\input{content/chap1602.tex}

\section{双向路径追踪}\label{sec:双向路径追踪}

\input{content/chap1604.tex}

\part{回顾与未来}
\input{content/chap17.tex}

\appendix
\part{附录}
\input{content/chapextraA.tex}

\input{content/chapextraB.tex}

% \chapter{In-text Elements}

% \section{Theorems}\index{Theorems}

% This is an example of theorems.

% \subsection{Several equations}\index{Theorems!Several Equations}
% This is a theorem consisting of several equations.

% \begin{theorem}[Name of the theorem]
%     In $E=\mathbb{R}^n$ all norms are equivalent. It has the properties:
%     \begin{align}
%          & \big| ||\mathbf{x}|| - ||\mathbf{y}|| \big|\leq || \mathbf{x}- \mathbf{y}||                            \\
%          & ||\sum_{i=1}^n\mathbf{x}_i||\leq \sum_{i=1}^n||\mathbf{x}_i||\quad\text{where $n$ is a finite integer}
%     \end{align}
% \end{theorem}

% \subsection{Single Line}\index{Theorems!Single Line}
% This is a theorem consisting of just one line.

% \begin{theorem}
%     A set $\mathcal{D}(G)$ in dense in $L^2(G)$, $|\cdot|_0$.
% \end{theorem}

% %------------------------------------------------

% \section{Definitions}

% This is an example of a definition.

% \begin{definition}[Definition name]
%     Given a vector space $E$, a norm on $E$ is an application, denoted $||\cdot||$, $E$ in $\mathbb{R}^+=[0,+\infty[$ such that:
%     \begin{align}
%          & ||\mathbf{x}||=0\ \Rightarrow\ \mathbf{x}=\mathbf{0}        \\
%          & ||\lambda \mathbf{x}||=|\lambda|\cdot ||\mathbf{x}||        \\
%          & ||\mathbf{x}+\mathbf{y}||\leq ||\mathbf{x}||+||\mathbf{y}||
%     \end{align}
% \end{definition}

% %------------------------------------------------

% \section{Notations}

% \begin{notation}
%     Given an open subset $G$ of $\mathbb{R}^n$, the set of functions $\varphi$ are:
%     \begin{enumerate}
%         \item Bounded support $G$;
%         \item Infinitely differentiable;
%     \end{enumerate}
%     a vector space is denoted by $\mathcal{D}(G)$.
% \end{notation}

% %------------------------------------------------

% \section{Remarks}

% This is an example of a remark.

% \begin{remark}
%     The concepts presented here are now in conventional employment in mathematics. Vector spaces are taken over the field $\mathbb{K}=\mathbb{R}$, however, established properties are easily extended to $\mathbb{K}=\mathbb{C}$.
% \end{remark}

% %------------------------------------------------

% \section{Corollaries}

% This is an example of a corollary.

% \begin{corollary}[Corollary name]
%     The concepts presented here are now in conventional employment in mathematics. Vector spaces are taken over the field $\mathbb{K}=\mathbb{R}$, however, established properties are easily extended to $\mathbb{K}=\mathbb{C}$.
% \end{corollary}

% %------------------------------------------------

% \section{Propositions}

% This is an example of propositions.

% \subsection{Several equations}\index{Propositions!Several Equations}

% \begin{proposition}[Proposition name]
%     It has the properties:
%     \begin{align}
%          & \big| ||\mathbf{x}|| - ||\mathbf{y}|| \big|\leq || \mathbf{x}- \mathbf{y}||                            \\
%          & ||\sum_{i=1}^n\mathbf{x}_i||\leq \sum_{i=1}^n||\mathbf{x}_i||\quad\text{where $n$ is a finite integer}
%     \end{align}
% \end{proposition}

% \subsection{Single Line}\index{Propositions!Single Line}

% \begin{proposition}
%     Let $f,g\in L^2(G)$; if $\forall \varphi\in\mathcal{D}(G)$, $(f,\varphi)_0=(g,\varphi)_0$ then $f = g$.
% \end{proposition}

% %------------------------------------------------

% \section{Examples}

% This is an example of examples.

% \subsection{Equation and Text}

% \begin{example}
%     Let $G=\{x\in\mathbb{R}^2:|x|<3\}$ and denoted by: $x^0=(1,1)$; consider the function:
%     \begin{equation}
%         f(x)=\left\{\begin{aligned}& \mathrm{e}^{|x|} &  & \text{si $|x-x^0|\leq 1/2$} \\ & 0  &  & \text{si $|x-x^0|> 1/2$}\end{aligned}\right.
%     \end{equation}
%     The function $f$ has bounded support, we can take $A=\{x\in\mathbb{R}^2:|x-x^0|\leq 1/2+\epsilon\}$ for all $\epsilon\in\intoo{0}{5/2-\sqrt{2}}$.
% \end{example}

% \subsection{Paragraph of Text}

% \begin{example}[Example name]
%     rrr
% \end{example}

% %------------------------------------------------

% \section{Exercises}

% This is an example of an exercise.

% \begin{exercise}
%     This is a good place to ask a question to test learning progress or further cement ideas into students' minds.
% \end{exercise}

% %------------------------------------------------

% \section{Problems}

% \begin{problem}
% What is the average airspeed velocity of an unladen swallow?
% \end{problem}

% %------------------------------------------------

% \section{Vocabulary}

% Define a word to improve a students' vocabulary.

% \begin{vocabulary}[Word]
%     Definition of word.
% \end{vocabulary}


% \chapterimage{chapter_head_1.pdf} % Chapter heading image

% \chapter{Presenting Information}


% \begin{enumerate}
%     \item The first item
%     \item The second item
%     \item The third item
% \end{enumerate}


% \begin{itemize}
%     \item The first item
%     \item The second item
%     \item The third item
% \end{itemize}

% \begin{description}
%     \item[Name] Description
%     \item[Word] Definition
%     \item[Comment] Elaboration
% \end{description}

% \section{Table}

% \begin{table}[h]
%     \centering
%     \begin{tabular}{l l l}
%         \toprule
%         \textbf{Treatments} & \textbf{Response 1} & \textbf{Response 2} \\
%         \midrule
%         Treatment 1         & 0.0003262           & 0.562               \\
%         Treatment 2         & 0.0015681           & 0.910               \\
%         Treatment 3         & 0.0009271           & 0.296               \\
%         \bottomrule
%     \end{tabular}
%     \caption{Table caption}
%     \label{tab:example} % Unique label used for referencing the table in-text
%     %\addcontentsline{toc}{table}{Table \ref{tab:example}} % Uncomment to add the table to the table of contents
% \end{table}

% Referencing Table \ref{tab:example} in-text automatically.

% %------------------------------------------------

% \section{Figure}

% \begin{figure}[h]
%     \centering\includegraphics[scale=0.5]{placeholder.jpg}
%     \caption{Figure caption}
%     \label{fig:placeholder} % Unique label used for referencing the figure in-text
%     % \addcontentsline{toc}{figure}{Figure \ref{fig:placeholder}} % Uncomment to add the figure to the table of contents
% \end{figure}

% Referencing Figure \ref{fig:placeholder} in-text automatically.

%----------------------------------------------------------------------------------------
%	BIBLIOGRAPHY
%----------------------------------------------------------------------------------------
\renewcommand{\bibname}{参考文献}
\chapter*{参考文献}
\markboth{\sffamily\normalsize\bfseries 参考文献}{\sffamily\normalsize\bfseries 参考文献}
\addcontentsline{toc}{chapter}{\textcolor{ocre}{参考文献}} % Add a Bibliography heading to the table of contents
\printbibliography[heading=bibempty]

%------------------------------------------------

% \section*{论文}
% \addcontentsline{toc}{section}{论文}
% \printbibliography[heading=bibempty,type=article]

%------------------------------------------------

% \section*{书籍}
% \addcontentsline{toc}{section}{书籍}
% \printbibliography[heading=bibempty,type=book]

%----------------------------------------------------------------------------------------
%	INDEX
%----------------------------------------------------------------------------------------
\renewcommand{\indexname}{\sffamily\bfseries 索引}
\cleardoublepage % Make sure the index starts on an odd (right side) page
\phantomsection
\setlength{\columnsep}{0.75cm} % Space between the 2 columns of the index
\addcontentsline{toc}{chapter}{\textcolor{ocre}{索引}} % Add an Index heading to the table of contents
\printindex % Output the index

%----------------------------------------------------------------------------------------

\end{document}
